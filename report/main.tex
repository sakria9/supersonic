\documentclass{article}
\usepackage{amsmath}
\usepackage{amssymb}
\usepackage{algorithm}
\usepackage{algpseudocode}
\usepackage{graphicx}
\usepackage{hyperref}
\usepackage{float}

\usepackage{tikz}
\usepackage{tikz-cd}
\usepackage{pgfplots}
\pgfplotsset{compat=1.14}
\def\mathdefault#1{\displaystyle #1}

\title{Acoustic Network with Coroutines}
\author{}
\date{Fall 2024}

\begin{document}

\maketitle

\begin{abstract}
This report presents the development and implementation of an acoustic communication network using sound waves as the physical medium.
The project involves designing and implementing the physical, link, and network layers of the communication stack.
Key techniques such as modulation schemes, error correction, clock synchronization, CSMA were employed to ensure reliable data transmission.
Challenges such as phase inversion, clock drift, and latency were addressed, and solutions were implemented to mitigate their effects.
\end{abstract}

% table of contents
\tableofcontents

\section{Introduction}

Acoustic networks offer a unique alternative to traditional wireless communication systems.
This project explores the implementation of an acoustic communication network using sound waves as the medium.
The goal is to develop a complete communication stack, from the physical layer to the network layer, and to evaluate its performance.

\section{Physical Layer}

Physical layer is the lowest layer of the network stack.
It is responsible for transmitting raw bits over the physical medium.
In our project, the physical medium is the air (or wire), and the raw bits are modulated into sound waves.

Breifly, our physical receives bits from the above layer, modulates them into sound waves, and call the audio framework to play the sound waves.
There are several important techniques in the physical layer, including modulation schemes, frame structure, forward error correction, clock synchronization, etc. We will discuss them in the following sections.
We will also discuss some hardware issues, which critically affect the performance of the physical layer.

\subsection{Sound Representation}

Sound is a wave that propagates through the air.
The sound wave can be represented by a sequence of samples.
More specifically, the sound wave can be represented by a sequence of floating point numbers, which are the amplitude of the sound wave at different time points.

\subsection{Audio Framework}

The audio framework is responsible for playing and recording sound waves.
It provides APIs for the physical layer to play and record sound waves.
The physical layer play sound waves by calling the audio framework's play API, and record sound waves by calling the audio framework's record API.

There are many audio frameworks available on the market. We have tried several of them, including JACK, miniaudio. Both of them are cross-platform audio frameworks, which can be used on both Windows and Linux.

We find that the different audio frameworks share similar API.
Basically, we need to provide a callback function, which will be called by the audio framework.
The type signature of the callback function is usually like this:
\begin{verbatim}
void callback(float* in, float* out, int frames, void* userData);
\end{verbatim}

In every second, the audio framework will call the callback function multiple times.
The time interval between two calls is determined by the frame size and the sample rate.

We didn't tune the frame size in our project and we keep the sample rate to be 48000 Hz.

\subsection{Hardware Issues}

In this section, we will discuss various hardware issues that we encountered during the development of our acoustic communication system. We will describe the nature of these issues, their causes, and the solutions we implemented to mitigate their effects.

\subsubsection{Phase Inversion}

One of the issues encountered in our project was phase inversion.
When a sinusoidal signal \(\sin(t)\) is played and subsequently recorded, the recorded waveform appears as \(-\sin(t)\), indicating a phase inversion.

\begin{figure}[H]
    \noindent\makebox[\textwidth]{%% Creator: Matplotlib, PGF backend
%%
%% To include the figure in your LaTeX document, write
%%   \input{<filename>.pgf}
%%
%% Make sure the required packages are loaded in your preamble
%%   \usepackage{pgf}
%%
%% Also ensure that all the required font packages are loaded; for instance,
%% the lmodern package is sometimes necessary when using math font.
%%   \usepackage{lmodern}
%%
%% Figures using additional raster images can only be included by \input if
%% they are in the same directory as the main LaTeX file. For loading figures
%% from other directories you can use the `import` package
%%   \usepackage{import}
%%
%% and then include the figures with
%%   \import{<path to file>}{<filename>.pgf}
%%
%% Matplotlib used the following preamble
%%   \def\mathdefault#1{#1}
%%   \everymath=\expandafter{\the\everymath\displaystyle}
%%   \IfFileExists{scrextend.sty}{
%%     \usepackage[fontsize=10.000000pt]{scrextend}
%%   }{
%%     \renewcommand{\normalsize}{\fontsize{10.000000}{12.000000}\selectfont}
%%     \normalsize
%%   }
%%   
%%   \makeatletter\@ifpackageloaded{underscore}{}{\usepackage[strings]{underscore}}\makeatother
%%
\begingroup%
\makeatletter%
\begin{pgfpicture}%
\pgfpathrectangle{\pgfpointorigin}{\pgfqpoint{7.162050in}{2.000000in}}%
\pgfusepath{use as bounding box, clip}%
\begin{pgfscope}%
\pgfsetbuttcap%
\pgfsetmiterjoin%
\definecolor{currentfill}{rgb}{1.000000,1.000000,1.000000}%
\pgfsetfillcolor{currentfill}%
\pgfsetlinewidth{0.000000pt}%
\definecolor{currentstroke}{rgb}{1.000000,1.000000,1.000000}%
\pgfsetstrokecolor{currentstroke}%
\pgfsetdash{}{0pt}%
\pgfpathmoveto{\pgfqpoint{0.000000in}{0.000000in}}%
\pgfpathlineto{\pgfqpoint{7.162050in}{0.000000in}}%
\pgfpathlineto{\pgfqpoint{7.162050in}{2.000000in}}%
\pgfpathlineto{\pgfqpoint{0.000000in}{2.000000in}}%
\pgfpathlineto{\pgfqpoint{0.000000in}{0.000000in}}%
\pgfpathclose%
\pgfusepath{fill}%
\end{pgfscope}%
\begin{pgfscope}%
\pgfsetbuttcap%
\pgfsetmiterjoin%
\definecolor{currentfill}{rgb}{1.000000,1.000000,1.000000}%
\pgfsetfillcolor{currentfill}%
\pgfsetlinewidth{0.000000pt}%
\definecolor{currentstroke}{rgb}{0.000000,0.000000,0.000000}%
\pgfsetstrokecolor{currentstroke}%
\pgfsetstrokeopacity{0.000000}%
\pgfsetdash{}{0pt}%
\pgfpathmoveto{\pgfqpoint{0.895256in}{0.220000in}}%
\pgfpathlineto{\pgfqpoint{6.445845in}{0.220000in}}%
\pgfpathlineto{\pgfqpoint{6.445845in}{1.760000in}}%
\pgfpathlineto{\pgfqpoint{0.895256in}{1.760000in}}%
\pgfpathlineto{\pgfqpoint{0.895256in}{0.220000in}}%
\pgfpathclose%
\pgfusepath{fill}%
\end{pgfscope}%
\begin{pgfscope}%
\pgfsetbuttcap%
\pgfsetroundjoin%
\definecolor{currentfill}{rgb}{0.000000,0.000000,0.000000}%
\pgfsetfillcolor{currentfill}%
\pgfsetlinewidth{0.803000pt}%
\definecolor{currentstroke}{rgb}{0.000000,0.000000,0.000000}%
\pgfsetstrokecolor{currentstroke}%
\pgfsetdash{}{0pt}%
\pgfsys@defobject{currentmarker}{\pgfqpoint{0.000000in}{-0.048611in}}{\pgfqpoint{0.000000in}{0.000000in}}{%
\pgfpathmoveto{\pgfqpoint{0.000000in}{0.000000in}}%
\pgfpathlineto{\pgfqpoint{0.000000in}{-0.048611in}}%
\pgfusepath{stroke,fill}%
}%
\begin{pgfscope}%
\pgfsys@transformshift{1.108740in}{0.220000in}%
\pgfsys@useobject{currentmarker}{}%
\end{pgfscope}%
\end{pgfscope}%
\begin{pgfscope}%
\definecolor{textcolor}{rgb}{0.000000,0.000000,0.000000}%
\pgfsetstrokecolor{textcolor}%
\pgfsetfillcolor{textcolor}%
\pgftext[x=1.108740in,y=0.122778in,,top]{\color{textcolor}{\rmfamily\fontsize{10.000000}{12.000000}\selectfont\catcode`\^=\active\def^{\ifmmode\sp\else\^{}\fi}\catcode`\%=\active\def%{\%}$\mathdefault{0.000}$}}%
\end{pgfscope}%
\begin{pgfscope}%
\pgfsetbuttcap%
\pgfsetroundjoin%
\definecolor{currentfill}{rgb}{0.000000,0.000000,0.000000}%
\pgfsetfillcolor{currentfill}%
\pgfsetlinewidth{0.803000pt}%
\definecolor{currentstroke}{rgb}{0.000000,0.000000,0.000000}%
\pgfsetstrokecolor{currentstroke}%
\pgfsetdash{}{0pt}%
\pgfsys@defobject{currentmarker}{\pgfqpoint{0.000000in}{-0.048611in}}{\pgfqpoint{0.000000in}{0.000000in}}{%
\pgfpathmoveto{\pgfqpoint{0.000000in}{0.000000in}}%
\pgfpathlineto{\pgfqpoint{0.000000in}{-0.048611in}}%
\pgfusepath{stroke,fill}%
}%
\begin{pgfscope}%
\pgfsys@transformshift{2.176161in}{0.220000in}%
\pgfsys@useobject{currentmarker}{}%
\end{pgfscope}%
\end{pgfscope}%
\begin{pgfscope}%
\definecolor{textcolor}{rgb}{0.000000,0.000000,0.000000}%
\pgfsetstrokecolor{textcolor}%
\pgfsetfillcolor{textcolor}%
\pgftext[x=2.176161in,y=0.122778in,,top]{\color{textcolor}{\rmfamily\fontsize{10.000000}{12.000000}\selectfont\catcode`\^=\active\def^{\ifmmode\sp\else\^{}\fi}\catcode`\%=\active\def%{\%}$\mathdefault{0.001}$}}%
\end{pgfscope}%
\begin{pgfscope}%
\pgfsetbuttcap%
\pgfsetroundjoin%
\definecolor{currentfill}{rgb}{0.000000,0.000000,0.000000}%
\pgfsetfillcolor{currentfill}%
\pgfsetlinewidth{0.803000pt}%
\definecolor{currentstroke}{rgb}{0.000000,0.000000,0.000000}%
\pgfsetstrokecolor{currentstroke}%
\pgfsetdash{}{0pt}%
\pgfsys@defobject{currentmarker}{\pgfqpoint{0.000000in}{-0.048611in}}{\pgfqpoint{0.000000in}{0.000000in}}{%
\pgfpathmoveto{\pgfqpoint{0.000000in}{0.000000in}}%
\pgfpathlineto{\pgfqpoint{0.000000in}{-0.048611in}}%
\pgfusepath{stroke,fill}%
}%
\begin{pgfscope}%
\pgfsys@transformshift{3.243582in}{0.220000in}%
\pgfsys@useobject{currentmarker}{}%
\end{pgfscope}%
\end{pgfscope}%
\begin{pgfscope}%
\definecolor{textcolor}{rgb}{0.000000,0.000000,0.000000}%
\pgfsetstrokecolor{textcolor}%
\pgfsetfillcolor{textcolor}%
\pgftext[x=3.243582in,y=0.122778in,,top]{\color{textcolor}{\rmfamily\fontsize{10.000000}{12.000000}\selectfont\catcode`\^=\active\def^{\ifmmode\sp\else\^{}\fi}\catcode`\%=\active\def%{\%}$\mathdefault{0.002}$}}%
\end{pgfscope}%
\begin{pgfscope}%
\pgfsetbuttcap%
\pgfsetroundjoin%
\definecolor{currentfill}{rgb}{0.000000,0.000000,0.000000}%
\pgfsetfillcolor{currentfill}%
\pgfsetlinewidth{0.803000pt}%
\definecolor{currentstroke}{rgb}{0.000000,0.000000,0.000000}%
\pgfsetstrokecolor{currentstroke}%
\pgfsetdash{}{0pt}%
\pgfsys@defobject{currentmarker}{\pgfqpoint{0.000000in}{-0.048611in}}{\pgfqpoint{0.000000in}{0.000000in}}{%
\pgfpathmoveto{\pgfqpoint{0.000000in}{0.000000in}}%
\pgfpathlineto{\pgfqpoint{0.000000in}{-0.048611in}}%
\pgfusepath{stroke,fill}%
}%
\begin{pgfscope}%
\pgfsys@transformshift{4.311003in}{0.220000in}%
\pgfsys@useobject{currentmarker}{}%
\end{pgfscope}%
\end{pgfscope}%
\begin{pgfscope}%
\definecolor{textcolor}{rgb}{0.000000,0.000000,0.000000}%
\pgfsetstrokecolor{textcolor}%
\pgfsetfillcolor{textcolor}%
\pgftext[x=4.311003in,y=0.122778in,,top]{\color{textcolor}{\rmfamily\fontsize{10.000000}{12.000000}\selectfont\catcode`\^=\active\def^{\ifmmode\sp\else\^{}\fi}\catcode`\%=\active\def%{\%}$\mathdefault{0.003}$}}%
\end{pgfscope}%
\begin{pgfscope}%
\pgfsetbuttcap%
\pgfsetroundjoin%
\definecolor{currentfill}{rgb}{0.000000,0.000000,0.000000}%
\pgfsetfillcolor{currentfill}%
\pgfsetlinewidth{0.803000pt}%
\definecolor{currentstroke}{rgb}{0.000000,0.000000,0.000000}%
\pgfsetstrokecolor{currentstroke}%
\pgfsetdash{}{0pt}%
\pgfsys@defobject{currentmarker}{\pgfqpoint{0.000000in}{-0.048611in}}{\pgfqpoint{0.000000in}{0.000000in}}{%
\pgfpathmoveto{\pgfqpoint{0.000000in}{0.000000in}}%
\pgfpathlineto{\pgfqpoint{0.000000in}{-0.048611in}}%
\pgfusepath{stroke,fill}%
}%
\begin{pgfscope}%
\pgfsys@transformshift{5.378424in}{0.220000in}%
\pgfsys@useobject{currentmarker}{}%
\end{pgfscope}%
\end{pgfscope}%
\begin{pgfscope}%
\definecolor{textcolor}{rgb}{0.000000,0.000000,0.000000}%
\pgfsetstrokecolor{textcolor}%
\pgfsetfillcolor{textcolor}%
\pgftext[x=5.378424in,y=0.122778in,,top]{\color{textcolor}{\rmfamily\fontsize{10.000000}{12.000000}\selectfont\catcode`\^=\active\def^{\ifmmode\sp\else\^{}\fi}\catcode`\%=\active\def%{\%}$\mathdefault{0.004}$}}%
\end{pgfscope}%
\begin{pgfscope}%
\pgfsetbuttcap%
\pgfsetroundjoin%
\definecolor{currentfill}{rgb}{0.000000,0.000000,0.000000}%
\pgfsetfillcolor{currentfill}%
\pgfsetlinewidth{0.803000pt}%
\definecolor{currentstroke}{rgb}{0.000000,0.000000,0.000000}%
\pgfsetstrokecolor{currentstroke}%
\pgfsetdash{}{0pt}%
\pgfsys@defobject{currentmarker}{\pgfqpoint{0.000000in}{-0.048611in}}{\pgfqpoint{0.000000in}{0.000000in}}{%
\pgfpathmoveto{\pgfqpoint{0.000000in}{0.000000in}}%
\pgfpathlineto{\pgfqpoint{0.000000in}{-0.048611in}}%
\pgfusepath{stroke,fill}%
}%
\begin{pgfscope}%
\pgfsys@transformshift{6.445845in}{0.220000in}%
\pgfsys@useobject{currentmarker}{}%
\end{pgfscope}%
\end{pgfscope}%
\begin{pgfscope}%
\definecolor{textcolor}{rgb}{0.000000,0.000000,0.000000}%
\pgfsetstrokecolor{textcolor}%
\pgfsetfillcolor{textcolor}%
\pgftext[x=6.445845in,y=0.122778in,,top]{\color{textcolor}{\rmfamily\fontsize{10.000000}{12.000000}\selectfont\catcode`\^=\active\def^{\ifmmode\sp\else\^{}\fi}\catcode`\%=\active\def%{\%}$\mathdefault{0.005}$}}%
\end{pgfscope}%
\begin{pgfscope}%
\definecolor{textcolor}{rgb}{0.000000,0.000000,0.000000}%
\pgfsetstrokecolor{textcolor}%
\pgfsetfillcolor{textcolor}%
\pgftext[x=3.670551in,y=-0.056234in,,top]{\color{textcolor}{\rmfamily\fontsize{10.000000}{12.000000}\selectfont\catcode`\^=\active\def^{\ifmmode\sp\else\^{}\fi}\catcode`\%=\active\def%{\%}Time (s)}}%
\end{pgfscope}%
\begin{pgfscope}%
\pgfsetbuttcap%
\pgfsetroundjoin%
\definecolor{currentfill}{rgb}{0.000000,0.000000,0.000000}%
\pgfsetfillcolor{currentfill}%
\pgfsetlinewidth{0.803000pt}%
\definecolor{currentstroke}{rgb}{0.000000,0.000000,0.000000}%
\pgfsetstrokecolor{currentstroke}%
\pgfsetdash{}{0pt}%
\pgfsys@defobject{currentmarker}{\pgfqpoint{-0.048611in}{0.000000in}}{\pgfqpoint{-0.000000in}{0.000000in}}{%
\pgfpathmoveto{\pgfqpoint{-0.000000in}{0.000000in}}%
\pgfpathlineto{\pgfqpoint{-0.048611in}{0.000000in}}%
\pgfusepath{stroke,fill}%
}%
\begin{pgfscope}%
\pgfsys@transformshift{0.895256in}{0.290000in}%
\pgfsys@useobject{currentmarker}{}%
\end{pgfscope}%
\end{pgfscope}%
\begin{pgfscope}%
\definecolor{textcolor}{rgb}{0.000000,0.000000,0.000000}%
\pgfsetstrokecolor{textcolor}%
\pgfsetfillcolor{textcolor}%
\pgftext[x=0.512539in, y=0.241775in, left, base]{\color{textcolor}{\rmfamily\fontsize{10.000000}{12.000000}\selectfont\catcode`\^=\active\def^{\ifmmode\sp\else\^{}\fi}\catcode`\%=\active\def%{\%}$\mathdefault{\ensuremath{-}1.0}$}}%
\end{pgfscope}%
\begin{pgfscope}%
\pgfsetbuttcap%
\pgfsetroundjoin%
\definecolor{currentfill}{rgb}{0.000000,0.000000,0.000000}%
\pgfsetfillcolor{currentfill}%
\pgfsetlinewidth{0.803000pt}%
\definecolor{currentstroke}{rgb}{0.000000,0.000000,0.000000}%
\pgfsetstrokecolor{currentstroke}%
\pgfsetdash{}{0pt}%
\pgfsys@defobject{currentmarker}{\pgfqpoint{-0.048611in}{0.000000in}}{\pgfqpoint{-0.000000in}{0.000000in}}{%
\pgfpathmoveto{\pgfqpoint{-0.000000in}{0.000000in}}%
\pgfpathlineto{\pgfqpoint{-0.048611in}{0.000000in}}%
\pgfusepath{stroke,fill}%
}%
\begin{pgfscope}%
\pgfsys@transformshift{0.895256in}{0.640000in}%
\pgfsys@useobject{currentmarker}{}%
\end{pgfscope}%
\end{pgfscope}%
\begin{pgfscope}%
\definecolor{textcolor}{rgb}{0.000000,0.000000,0.000000}%
\pgfsetstrokecolor{textcolor}%
\pgfsetfillcolor{textcolor}%
\pgftext[x=0.512539in, y=0.591775in, left, base]{\color{textcolor}{\rmfamily\fontsize{10.000000}{12.000000}\selectfont\catcode`\^=\active\def^{\ifmmode\sp\else\^{}\fi}\catcode`\%=\active\def%{\%}$\mathdefault{\ensuremath{-}0.5}$}}%
\end{pgfscope}%
\begin{pgfscope}%
\pgfsetbuttcap%
\pgfsetroundjoin%
\definecolor{currentfill}{rgb}{0.000000,0.000000,0.000000}%
\pgfsetfillcolor{currentfill}%
\pgfsetlinewidth{0.803000pt}%
\definecolor{currentstroke}{rgb}{0.000000,0.000000,0.000000}%
\pgfsetstrokecolor{currentstroke}%
\pgfsetdash{}{0pt}%
\pgfsys@defobject{currentmarker}{\pgfqpoint{-0.048611in}{0.000000in}}{\pgfqpoint{-0.000000in}{0.000000in}}{%
\pgfpathmoveto{\pgfqpoint{-0.000000in}{0.000000in}}%
\pgfpathlineto{\pgfqpoint{-0.048611in}{0.000000in}}%
\pgfusepath{stroke,fill}%
}%
\begin{pgfscope}%
\pgfsys@transformshift{0.895256in}{0.990000in}%
\pgfsys@useobject{currentmarker}{}%
\end{pgfscope}%
\end{pgfscope}%
\begin{pgfscope}%
\definecolor{textcolor}{rgb}{0.000000,0.000000,0.000000}%
\pgfsetstrokecolor{textcolor}%
\pgfsetfillcolor{textcolor}%
\pgftext[x=0.620564in, y=0.941775in, left, base]{\color{textcolor}{\rmfamily\fontsize{10.000000}{12.000000}\selectfont\catcode`\^=\active\def^{\ifmmode\sp\else\^{}\fi}\catcode`\%=\active\def%{\%}$\mathdefault{0.0}$}}%
\end{pgfscope}%
\begin{pgfscope}%
\pgfsetbuttcap%
\pgfsetroundjoin%
\definecolor{currentfill}{rgb}{0.000000,0.000000,0.000000}%
\pgfsetfillcolor{currentfill}%
\pgfsetlinewidth{0.803000pt}%
\definecolor{currentstroke}{rgb}{0.000000,0.000000,0.000000}%
\pgfsetstrokecolor{currentstroke}%
\pgfsetdash{}{0pt}%
\pgfsys@defobject{currentmarker}{\pgfqpoint{-0.048611in}{0.000000in}}{\pgfqpoint{-0.000000in}{0.000000in}}{%
\pgfpathmoveto{\pgfqpoint{-0.000000in}{0.000000in}}%
\pgfpathlineto{\pgfqpoint{-0.048611in}{0.000000in}}%
\pgfusepath{stroke,fill}%
}%
\begin{pgfscope}%
\pgfsys@transformshift{0.895256in}{1.340000in}%
\pgfsys@useobject{currentmarker}{}%
\end{pgfscope}%
\end{pgfscope}%
\begin{pgfscope}%
\definecolor{textcolor}{rgb}{0.000000,0.000000,0.000000}%
\pgfsetstrokecolor{textcolor}%
\pgfsetfillcolor{textcolor}%
\pgftext[x=0.620564in, y=1.291775in, left, base]{\color{textcolor}{\rmfamily\fontsize{10.000000}{12.000000}\selectfont\catcode`\^=\active\def^{\ifmmode\sp\else\^{}\fi}\catcode`\%=\active\def%{\%}$\mathdefault{0.5}$}}%
\end{pgfscope}%
\begin{pgfscope}%
\pgfsetbuttcap%
\pgfsetroundjoin%
\definecolor{currentfill}{rgb}{0.000000,0.000000,0.000000}%
\pgfsetfillcolor{currentfill}%
\pgfsetlinewidth{0.803000pt}%
\definecolor{currentstroke}{rgb}{0.000000,0.000000,0.000000}%
\pgfsetstrokecolor{currentstroke}%
\pgfsetdash{}{0pt}%
\pgfsys@defobject{currentmarker}{\pgfqpoint{-0.048611in}{0.000000in}}{\pgfqpoint{-0.000000in}{0.000000in}}{%
\pgfpathmoveto{\pgfqpoint{-0.000000in}{0.000000in}}%
\pgfpathlineto{\pgfqpoint{-0.048611in}{0.000000in}}%
\pgfusepath{stroke,fill}%
}%
\begin{pgfscope}%
\pgfsys@transformshift{0.895256in}{1.690000in}%
\pgfsys@useobject{currentmarker}{}%
\end{pgfscope}%
\end{pgfscope}%
\begin{pgfscope}%
\definecolor{textcolor}{rgb}{0.000000,0.000000,0.000000}%
\pgfsetstrokecolor{textcolor}%
\pgfsetfillcolor{textcolor}%
\pgftext[x=0.620564in, y=1.641775in, left, base]{\color{textcolor}{\rmfamily\fontsize{10.000000}{12.000000}\selectfont\catcode`\^=\active\def^{\ifmmode\sp\else\^{}\fi}\catcode`\%=\active\def%{\%}$\mathdefault{1.0}$}}%
\end{pgfscope}%
\begin{pgfscope}%
\definecolor{textcolor}{rgb}{0.000000,0.000000,0.000000}%
\pgfsetstrokecolor{textcolor}%
\pgfsetfillcolor{textcolor}%
\pgftext[x=0.456984in,y=0.990000in,,bottom,rotate=90.000000]{\color{textcolor}{\rmfamily\fontsize{10.000000}{12.000000}\selectfont\catcode`\^=\active\def^{\ifmmode\sp\else\^{}\fi}\catcode`\%=\active\def%{\%}Amplitude}}%
\end{pgfscope}%
\begin{pgfscope}%
\pgfpathrectangle{\pgfqpoint{0.895256in}{0.220000in}}{\pgfqpoint{5.550589in}{1.540000in}}%
\pgfusepath{clip}%
\pgfsetrectcap%
\pgfsetroundjoin%
\pgfsetlinewidth{1.505625pt}%
\definecolor{currentstroke}{rgb}{0.121569,0.466667,0.705882}%
\pgfsetstrokecolor{currentstroke}%
\pgfsetdash{}{0pt}%
\pgfpathmoveto{\pgfqpoint{1.108740in}{0.990000in}}%
\pgfpathlineto{\pgfqpoint{1.130978in}{1.419909in}}%
\pgfpathlineto{\pgfqpoint{1.153216in}{1.670837in}}%
\pgfpathlineto{\pgfqpoint{1.175454in}{1.619072in}}%
\pgfpathlineto{\pgfqpoint{1.197692in}{1.271923in}}%
\pgfpathlineto{\pgfqpoint{1.219930in}{0.782239in}}%
\pgfpathlineto{\pgfqpoint{1.242168in}{0.389590in}}%
\pgfpathlineto{\pgfqpoint{1.264406in}{0.304397in}}%
\pgfpathlineto{\pgfqpoint{1.286644in}{0.588496in}}%
\pgfpathlineto{\pgfqpoint{1.308882in}{1.101201in}}%
\pgfpathlineto{\pgfqpoint{1.331120in}{1.556760in}}%
\pgfpathlineto{\pgfqpoint{1.353358in}{1.679857in}}%
\pgfpathlineto{\pgfqpoint{1.375596in}{1.378899in}}%
\pgfpathlineto{\pgfqpoint{1.397834in}{0.826403in}}%
\pgfpathlineto{\pgfqpoint{1.420072in}{0.374464in}}%
\pgfpathlineto{\pgfqpoint{1.442309in}{0.333863in}}%
\pgfpathlineto{\pgfqpoint{1.464547in}{0.750586in}}%
\pgfpathlineto{\pgfqpoint{1.486785in}{1.344125in}}%
\pgfpathlineto{\pgfqpoint{1.509023in}{1.682424in}}%
\pgfpathlineto{\pgfqpoint{1.531261in}{1.497631in}}%
\pgfpathlineto{\pgfqpoint{1.553499in}{0.913793in}}%
\pgfpathlineto{\pgfqpoint{1.575737in}{0.385219in}}%
\pgfpathlineto{\pgfqpoint{1.597975in}{0.350829in}}%
\pgfpathlineto{\pgfqpoint{1.620213in}{0.858121in}}%
\pgfpathlineto{\pgfqpoint{1.642451in}{1.484975in}}%
\pgfpathlineto{\pgfqpoint{1.664689in}{1.675603in}}%
\pgfpathlineto{\pgfqpoint{1.686927in}{1.240145in}}%
\pgfpathlineto{\pgfqpoint{1.709165in}{0.566142in}}%
\pgfpathlineto{\pgfqpoint{1.731403in}{0.294161in}}%
\pgfpathlineto{\pgfqpoint{1.753641in}{0.705459in}}%
\pgfpathlineto{\pgfqpoint{1.775879in}{1.406990in}}%
\pgfpathlineto{\pgfqpoint{1.798116in}{1.683372in}}%
\pgfpathlineto{\pgfqpoint{1.820354in}{1.229414in}}%
\pgfpathlineto{\pgfqpoint{1.842592in}{0.513579in}}%
\pgfpathlineto{\pgfqpoint{1.864830in}{0.319043in}}%
\pgfpathlineto{\pgfqpoint{1.887068in}{0.880684in}}%
\pgfpathlineto{\pgfqpoint{1.909306in}{1.572029in}}%
\pgfpathlineto{\pgfqpoint{1.931544in}{1.581908in}}%
\pgfpathlineto{\pgfqpoint{1.953782in}{0.879742in}}%
\pgfpathlineto{\pgfqpoint{1.976020in}{0.308942in}}%
\pgfpathlineto{\pgfqpoint{1.998258in}{0.588496in}}%
\pgfpathlineto{\pgfqpoint{2.020496in}{1.382858in}}%
\pgfpathlineto{\pgfqpoint{2.042734in}{1.669022in}}%
\pgfpathlineto{\pgfqpoint{2.064972in}{1.053861in}}%
\pgfpathlineto{\pgfqpoint{2.087210in}{0.349282in}}%
\pgfpathlineto{\pgfqpoint{2.109448in}{0.526309in}}%
\pgfpathlineto{\pgfqpoint{2.131685in}{1.356400in}}%
\pgfpathlineto{\pgfqpoint{2.153923in}{1.664897in}}%
\pgfpathlineto{\pgfqpoint{2.176161in}{0.315103in}}%
\pgfpathlineto{\pgfqpoint{2.198399in}{0.623600in}}%
\pgfpathlineto{\pgfqpoint{2.220637in}{1.453691in}}%
\pgfpathlineto{\pgfqpoint{2.242875in}{1.630718in}}%
\pgfpathlineto{\pgfqpoint{2.265113in}{0.926139in}}%
\pgfpathlineto{\pgfqpoint{2.287351in}{0.310978in}}%
\pgfpathlineto{\pgfqpoint{2.309589in}{0.597142in}}%
\pgfpathlineto{\pgfqpoint{2.331827in}{1.391504in}}%
\pgfpathlineto{\pgfqpoint{2.354065in}{1.671058in}}%
\pgfpathlineto{\pgfqpoint{2.376303in}{1.100258in}}%
\pgfpathlineto{\pgfqpoint{2.398541in}{0.398092in}}%
\pgfpathlineto{\pgfqpoint{2.420779in}{0.407971in}}%
\pgfpathlineto{\pgfqpoint{2.443017in}{1.099316in}}%
\pgfpathlineto{\pgfqpoint{2.465255in}{1.660957in}}%
\pgfpathlineto{\pgfqpoint{2.487492in}{1.466421in}}%
\pgfpathlineto{\pgfqpoint{2.509730in}{0.750586in}}%
\pgfpathlineto{\pgfqpoint{2.531968in}{0.296628in}}%
\pgfpathlineto{\pgfqpoint{2.554206in}{0.573010in}}%
\pgfpathlineto{\pgfqpoint{2.576444in}{1.274541in}}%
\pgfpathlineto{\pgfqpoint{2.598682in}{1.685839in}}%
\pgfpathlineto{\pgfqpoint{2.620920in}{1.413858in}}%
\pgfpathlineto{\pgfqpoint{2.643158in}{0.739855in}}%
\pgfpathlineto{\pgfqpoint{2.665396in}{0.304397in}}%
\pgfpathlineto{\pgfqpoint{2.687634in}{0.495025in}}%
\pgfpathlineto{\pgfqpoint{2.709872in}{1.121879in}}%
\pgfpathlineto{\pgfqpoint{2.732110in}{1.629171in}}%
\pgfpathlineto{\pgfqpoint{2.754348in}{1.594781in}}%
\pgfpathlineto{\pgfqpoint{2.776586in}{1.066207in}}%
\pgfpathlineto{\pgfqpoint{2.798824in}{0.482369in}}%
\pgfpathlineto{\pgfqpoint{2.821061in}{0.297576in}}%
\pgfpathlineto{\pgfqpoint{2.843299in}{0.635875in}}%
\pgfpathlineto{\pgfqpoint{2.865537in}{1.229414in}}%
\pgfpathlineto{\pgfqpoint{2.887775in}{1.646137in}}%
\pgfpathlineto{\pgfqpoint{2.910013in}{1.605536in}}%
\pgfpathlineto{\pgfqpoint{2.932251in}{1.153597in}}%
\pgfpathlineto{\pgfqpoint{2.954489in}{0.601101in}}%
\pgfpathlineto{\pgfqpoint{2.976727in}{0.300143in}}%
\pgfpathlineto{\pgfqpoint{2.998965in}{0.423240in}}%
\pgfpathlineto{\pgfqpoint{3.021203in}{0.878799in}}%
\pgfpathlineto{\pgfqpoint{3.043441in}{1.391504in}}%
\pgfpathlineto{\pgfqpoint{3.065679in}{1.675603in}}%
\pgfpathlineto{\pgfqpoint{3.087917in}{1.590410in}}%
\pgfpathlineto{\pgfqpoint{3.110155in}{1.197761in}}%
\pgfpathlineto{\pgfqpoint{3.132393in}{0.708077in}}%
\pgfpathlineto{\pgfqpoint{3.154631in}{0.360928in}}%
\pgfpathlineto{\pgfqpoint{3.176868in}{0.309163in}}%
\pgfpathlineto{\pgfqpoint{3.199106in}{0.560091in}}%
\pgfpathlineto{\pgfqpoint{3.221344in}{0.990000in}}%
\pgfpathlineto{\pgfqpoint{4.311003in}{0.990000in}}%
\pgfpathlineto{\pgfqpoint{4.355479in}{1.171173in}}%
\pgfpathlineto{\pgfqpoint{4.377717in}{1.257878in}}%
\pgfpathlineto{\pgfqpoint{4.399955in}{1.340000in}}%
\pgfpathlineto{\pgfqpoint{4.422193in}{1.416133in}}%
\pgfpathlineto{\pgfqpoint{4.444431in}{1.484975in}}%
\pgfpathlineto{\pgfqpoint{4.466669in}{1.545347in}}%
\pgfpathlineto{\pgfqpoint{4.488907in}{1.596218in}}%
\pgfpathlineto{\pgfqpoint{4.511145in}{1.636716in}}%
\pgfpathlineto{\pgfqpoint{4.533383in}{1.666148in}}%
\pgfpathlineto{\pgfqpoint{4.555620in}{1.684011in}}%
\pgfpathlineto{\pgfqpoint{4.577858in}{1.690000in}}%
\pgfpathlineto{\pgfqpoint{4.600096in}{1.684011in}}%
\pgfpathlineto{\pgfqpoint{4.622334in}{1.666148in}}%
\pgfpathlineto{\pgfqpoint{4.644572in}{1.636716in}}%
\pgfpathlineto{\pgfqpoint{4.666810in}{1.596218in}}%
\pgfpathlineto{\pgfqpoint{4.689048in}{1.545347in}}%
\pgfpathlineto{\pgfqpoint{4.711286in}{1.484975in}}%
\pgfpathlineto{\pgfqpoint{4.733524in}{1.416133in}}%
\pgfpathlineto{\pgfqpoint{4.755762in}{1.340000in}}%
\pgfpathlineto{\pgfqpoint{4.778000in}{1.257878in}}%
\pgfpathlineto{\pgfqpoint{4.822476in}{1.081368in}}%
\pgfpathlineto{\pgfqpoint{4.889190in}{0.808827in}}%
\pgfpathlineto{\pgfqpoint{4.911427in}{0.722122in}}%
\pgfpathlineto{\pgfqpoint{4.933665in}{0.640000in}}%
\pgfpathlineto{\pgfqpoint{4.955903in}{0.563867in}}%
\pgfpathlineto{\pgfqpoint{4.978141in}{0.495025in}}%
\pgfpathlineto{\pgfqpoint{5.000379in}{0.434653in}}%
\pgfpathlineto{\pgfqpoint{5.022617in}{0.383782in}}%
\pgfpathlineto{\pgfqpoint{5.044855in}{0.343284in}}%
\pgfpathlineto{\pgfqpoint{5.067093in}{0.313852in}}%
\pgfpathlineto{\pgfqpoint{5.089331in}{0.295989in}}%
\pgfpathlineto{\pgfqpoint{5.111569in}{0.290000in}}%
\pgfpathlineto{\pgfqpoint{5.133807in}{0.295989in}}%
\pgfpathlineto{\pgfqpoint{5.156045in}{0.313852in}}%
\pgfpathlineto{\pgfqpoint{5.178283in}{0.343284in}}%
\pgfpathlineto{\pgfqpoint{5.200521in}{0.383782in}}%
\pgfpathlineto{\pgfqpoint{5.222759in}{0.434653in}}%
\pgfpathlineto{\pgfqpoint{5.244996in}{0.495025in}}%
\pgfpathlineto{\pgfqpoint{5.267234in}{0.563867in}}%
\pgfpathlineto{\pgfqpoint{5.289472in}{0.640000in}}%
\pgfpathlineto{\pgfqpoint{5.311710in}{0.722122in}}%
\pgfpathlineto{\pgfqpoint{5.356186in}{0.898632in}}%
\pgfpathlineto{\pgfqpoint{5.422900in}{1.171173in}}%
\pgfpathlineto{\pgfqpoint{5.445138in}{1.257878in}}%
\pgfpathlineto{\pgfqpoint{5.467376in}{1.340000in}}%
\pgfpathlineto{\pgfqpoint{5.489614in}{1.416133in}}%
\pgfpathlineto{\pgfqpoint{5.511852in}{1.484975in}}%
\pgfpathlineto{\pgfqpoint{5.534090in}{1.545347in}}%
\pgfpathlineto{\pgfqpoint{5.556328in}{1.596218in}}%
\pgfpathlineto{\pgfqpoint{5.578566in}{1.636716in}}%
\pgfpathlineto{\pgfqpoint{5.600803in}{1.666148in}}%
\pgfpathlineto{\pgfqpoint{5.623041in}{1.684011in}}%
\pgfpathlineto{\pgfqpoint{5.645279in}{1.690000in}}%
\pgfpathlineto{\pgfqpoint{5.667517in}{1.684011in}}%
\pgfpathlineto{\pgfqpoint{5.689755in}{1.666148in}}%
\pgfpathlineto{\pgfqpoint{5.711993in}{1.636716in}}%
\pgfpathlineto{\pgfqpoint{5.734231in}{1.596218in}}%
\pgfpathlineto{\pgfqpoint{5.756469in}{1.545347in}}%
\pgfpathlineto{\pgfqpoint{5.778707in}{1.484975in}}%
\pgfpathlineto{\pgfqpoint{5.800945in}{1.416133in}}%
\pgfpathlineto{\pgfqpoint{5.823183in}{1.340000in}}%
\pgfpathlineto{\pgfqpoint{5.845421in}{1.257878in}}%
\pgfpathlineto{\pgfqpoint{5.889897in}{1.081368in}}%
\pgfpathlineto{\pgfqpoint{5.956610in}{0.808827in}}%
\pgfpathlineto{\pgfqpoint{5.978848in}{0.722122in}}%
\pgfpathlineto{\pgfqpoint{6.001086in}{0.640000in}}%
\pgfpathlineto{\pgfqpoint{6.023324in}{0.563867in}}%
\pgfpathlineto{\pgfqpoint{6.045562in}{0.495025in}}%
\pgfpathlineto{\pgfqpoint{6.067800in}{0.434653in}}%
\pgfpathlineto{\pgfqpoint{6.090038in}{0.383782in}}%
\pgfpathlineto{\pgfqpoint{6.112276in}{0.343284in}}%
\pgfpathlineto{\pgfqpoint{6.134514in}{0.313852in}}%
\pgfpathlineto{\pgfqpoint{6.156752in}{0.295989in}}%
\pgfpathlineto{\pgfqpoint{6.178990in}{0.290000in}}%
\pgfpathlineto{\pgfqpoint{6.201228in}{0.295989in}}%
\pgfpathlineto{\pgfqpoint{6.223466in}{0.313852in}}%
\pgfpathlineto{\pgfqpoint{6.245704in}{0.343284in}}%
\pgfpathlineto{\pgfqpoint{6.267942in}{0.383782in}}%
\pgfpathlineto{\pgfqpoint{6.290179in}{0.434653in}}%
\pgfpathlineto{\pgfqpoint{6.312417in}{0.495025in}}%
\pgfpathlineto{\pgfqpoint{6.334655in}{0.563867in}}%
\pgfpathlineto{\pgfqpoint{6.356893in}{0.640000in}}%
\pgfpathlineto{\pgfqpoint{6.379131in}{0.722122in}}%
\pgfpathlineto{\pgfqpoint{6.423607in}{0.898632in}}%
\pgfpathlineto{\pgfqpoint{6.455845in}{1.031087in}}%
\pgfpathlineto{\pgfqpoint{6.455845in}{1.031087in}}%
\pgfusepath{stroke}%
\end{pgfscope}%
\begin{pgfscope}%
\pgfpathrectangle{\pgfqpoint{0.895256in}{0.220000in}}{\pgfqpoint{5.550589in}{1.540000in}}%
\pgfusepath{clip}%
\pgfsetrectcap%
\pgfsetroundjoin%
\pgfsetlinewidth{1.505625pt}%
\definecolor{currentstroke}{rgb}{1.000000,0.498039,0.054902}%
\pgfsetstrokecolor{currentstroke}%
\pgfsetdash{}{0pt}%
\pgfpathmoveto{\pgfqpoint{1.108740in}{0.982374in}}%
\pgfpathlineto{\pgfqpoint{1.130978in}{0.960798in}}%
\pgfpathlineto{\pgfqpoint{1.153216in}{0.926554in}}%
\pgfpathlineto{\pgfqpoint{1.175454in}{0.914548in}}%
\pgfpathlineto{\pgfqpoint{1.197692in}{0.958469in}}%
\pgfpathlineto{\pgfqpoint{1.219930in}{1.030738in}}%
\pgfpathlineto{\pgfqpoint{1.242168in}{1.079893in}}%
\pgfpathlineto{\pgfqpoint{1.264406in}{1.094782in}}%
\pgfpathlineto{\pgfqpoint{1.286644in}{1.073484in}}%
\pgfpathlineto{\pgfqpoint{1.308882in}{1.009846in}}%
\pgfpathlineto{\pgfqpoint{1.331120in}{0.934949in}}%
\pgfpathlineto{\pgfqpoint{1.353358in}{0.900577in}}%
\pgfpathlineto{\pgfqpoint{1.375596in}{0.927985in}}%
\pgfpathlineto{\pgfqpoint{1.397834in}{0.991794in}}%
\pgfpathlineto{\pgfqpoint{1.420072in}{1.044880in}}%
\pgfpathlineto{\pgfqpoint{1.442309in}{1.054087in}}%
\pgfpathlineto{\pgfqpoint{1.464547in}{1.014289in}}%
\pgfpathlineto{\pgfqpoint{1.486785in}{0.950522in}}%
\pgfpathlineto{\pgfqpoint{1.509023in}{0.909891in}}%
\pgfpathlineto{\pgfqpoint{1.531261in}{0.928647in}}%
\pgfpathlineto{\pgfqpoint{1.575737in}{1.069382in}}%
\pgfpathlineto{\pgfqpoint{1.597975in}{1.085276in}}%
\pgfpathlineto{\pgfqpoint{1.620213in}{1.025056in}}%
\pgfpathlineto{\pgfqpoint{1.642451in}{0.932599in}}%
\pgfpathlineto{\pgfqpoint{1.664689in}{0.885923in}}%
\pgfpathlineto{\pgfqpoint{1.686927in}{0.927878in}}%
\pgfpathlineto{\pgfqpoint{1.709165in}{1.023005in}}%
\pgfpathlineto{\pgfqpoint{1.731403in}{1.086173in}}%
\pgfpathlineto{\pgfqpoint{1.753641in}{1.060261in}}%
\pgfpathlineto{\pgfqpoint{1.775879in}{0.968018in}}%
\pgfpathlineto{\pgfqpoint{1.798116in}{0.897437in}}%
\pgfpathlineto{\pgfqpoint{1.820354in}{0.920231in}}%
\pgfpathlineto{\pgfqpoint{1.842592in}{1.014054in}}%
\pgfpathlineto{\pgfqpoint{1.864830in}{1.082307in}}%
\pgfpathlineto{\pgfqpoint{1.887068in}{1.052164in}}%
\pgfpathlineto{\pgfqpoint{1.909306in}{0.954709in}}%
\pgfpathlineto{\pgfqpoint{1.931544in}{0.897843in}}%
\pgfpathlineto{\pgfqpoint{1.953782in}{0.945823in}}%
\pgfpathlineto{\pgfqpoint{1.976020in}{1.040415in}}%
\pgfpathlineto{\pgfqpoint{1.998258in}{1.068827in}}%
\pgfpathlineto{\pgfqpoint{2.020496in}{1.000233in}}%
\pgfpathlineto{\pgfqpoint{2.042734in}{0.923948in}}%
\pgfpathlineto{\pgfqpoint{2.064972in}{0.935804in}}%
\pgfpathlineto{\pgfqpoint{2.087210in}{1.015122in}}%
\pgfpathlineto{\pgfqpoint{2.109448in}{1.053254in}}%
\pgfpathlineto{\pgfqpoint{2.153923in}{0.951975in}}%
\pgfpathlineto{\pgfqpoint{2.176161in}{0.983933in}}%
\pgfpathlineto{\pgfqpoint{2.198399in}{1.033814in}}%
\pgfpathlineto{\pgfqpoint{2.220637in}{0.975324in}}%
\pgfpathlineto{\pgfqpoint{2.242875in}{0.884470in}}%
\pgfpathlineto{\pgfqpoint{2.265113in}{0.959516in}}%
\pgfpathlineto{\pgfqpoint{2.287351in}{1.105057in}}%
\pgfpathlineto{\pgfqpoint{2.309589in}{1.062910in}}%
\pgfpathlineto{\pgfqpoint{2.331827in}{0.909763in}}%
\pgfpathlineto{\pgfqpoint{2.354065in}{0.891648in}}%
\pgfpathlineto{\pgfqpoint{2.376303in}{1.003116in}}%
\pgfpathlineto{\pgfqpoint{2.398541in}{1.077201in}}%
\pgfpathlineto{\pgfqpoint{2.420779in}{1.046012in}}%
\pgfpathlineto{\pgfqpoint{2.443017in}{0.978336in}}%
\pgfpathlineto{\pgfqpoint{2.465255in}{0.946912in}}%
\pgfpathlineto{\pgfqpoint{2.487492in}{0.951462in}}%
\pgfpathlineto{\pgfqpoint{2.509730in}{0.981306in}}%
\pgfpathlineto{\pgfqpoint{2.531968in}{1.027897in}}%
\pgfpathlineto{\pgfqpoint{2.554206in}{1.045841in}}%
\pgfpathlineto{\pgfqpoint{2.576444in}{1.004441in}}%
\pgfpathlineto{\pgfqpoint{2.598682in}{0.936744in}}%
\pgfpathlineto{\pgfqpoint{2.620920in}{0.907371in}}%
\pgfpathlineto{\pgfqpoint{2.643158in}{0.961011in}}%
\pgfpathlineto{\pgfqpoint{2.665396in}{1.061521in}}%
\pgfpathlineto{\pgfqpoint{2.687634in}{1.099589in}}%
\pgfpathlineto{\pgfqpoint{2.709872in}{1.022941in}}%
\pgfpathlineto{\pgfqpoint{2.732110in}{0.908546in}}%
\pgfpathlineto{\pgfqpoint{2.754348in}{0.873661in}}%
\pgfpathlineto{\pgfqpoint{2.776586in}{0.944990in}}%
\pgfpathlineto{\pgfqpoint{2.798824in}{1.045905in}}%
\pgfpathlineto{\pgfqpoint{2.821061in}{1.095316in}}%
\pgfpathlineto{\pgfqpoint{2.843299in}{1.065601in}}%
\pgfpathlineto{\pgfqpoint{2.887775in}{0.899381in}}%
\pgfpathlineto{\pgfqpoint{2.910013in}{0.881608in}}%
\pgfpathlineto{\pgfqpoint{2.932251in}{0.946848in}}%
\pgfpathlineto{\pgfqpoint{2.954489in}{1.046738in}}%
\pgfpathlineto{\pgfqpoint{2.976727in}{1.099589in}}%
\pgfpathlineto{\pgfqpoint{2.998965in}{1.071775in}}%
\pgfpathlineto{\pgfqpoint{3.021203in}{0.996985in}}%
\pgfpathlineto{\pgfqpoint{3.043441in}{0.927045in}}%
\pgfpathlineto{\pgfqpoint{3.065679in}{0.893763in}}%
\pgfpathlineto{\pgfqpoint{3.087917in}{0.909891in}}%
\pgfpathlineto{\pgfqpoint{3.110155in}{0.968873in}}%
\pgfpathlineto{\pgfqpoint{3.132393in}{1.039796in}}%
\pgfpathlineto{\pgfqpoint{3.154631in}{1.083140in}}%
\pgfpathlineto{\pgfqpoint{3.176868in}{1.076432in}}%
\pgfpathlineto{\pgfqpoint{3.221344in}{0.988248in}}%
\pgfpathlineto{\pgfqpoint{3.243582in}{0.971393in}}%
\pgfpathlineto{\pgfqpoint{3.265820in}{0.981006in}}%
\pgfpathlineto{\pgfqpoint{3.288058in}{0.987479in}}%
\pgfpathlineto{\pgfqpoint{3.332534in}{0.958512in}}%
\pgfpathlineto{\pgfqpoint{3.377010in}{0.955350in}}%
\pgfpathlineto{\pgfqpoint{3.399248in}{0.955222in}}%
\pgfpathlineto{\pgfqpoint{3.421486in}{0.974171in}}%
\pgfpathlineto{\pgfqpoint{3.443724in}{1.006449in}}%
\pgfpathlineto{\pgfqpoint{3.465962in}{1.021253in}}%
\pgfpathlineto{\pgfqpoint{3.488200in}{1.005573in}}%
\pgfpathlineto{\pgfqpoint{3.510437in}{0.986326in}}%
\pgfpathlineto{\pgfqpoint{3.532675in}{0.990491in}}%
\pgfpathlineto{\pgfqpoint{3.554913in}{1.003031in}}%
\pgfpathlineto{\pgfqpoint{3.577151in}{0.999506in}}%
\pgfpathlineto{\pgfqpoint{3.599389in}{0.987714in}}%
\pgfpathlineto{\pgfqpoint{3.621627in}{0.981370in}}%
\pgfpathlineto{\pgfqpoint{3.643865in}{0.978614in}}%
\pgfpathlineto{\pgfqpoint{3.666103in}{0.981498in}}%
\pgfpathlineto{\pgfqpoint{3.688341in}{0.992393in}}%
\pgfpathlineto{\pgfqpoint{3.710579in}{0.998844in}}%
\pgfpathlineto{\pgfqpoint{3.755055in}{0.986689in}}%
\pgfpathlineto{\pgfqpoint{3.777293in}{0.988868in}}%
\pgfpathlineto{\pgfqpoint{3.799531in}{0.992371in}}%
\pgfpathlineto{\pgfqpoint{3.821769in}{0.992008in}}%
\pgfpathlineto{\pgfqpoint{3.844007in}{0.990470in}}%
\pgfpathlineto{\pgfqpoint{3.888482in}{0.982523in}}%
\pgfpathlineto{\pgfqpoint{3.910720in}{0.987607in}}%
\pgfpathlineto{\pgfqpoint{3.932958in}{0.997819in}}%
\pgfpathlineto{\pgfqpoint{3.955196in}{0.998994in}}%
\pgfpathlineto{\pgfqpoint{3.977434in}{0.991282in}}%
\pgfpathlineto{\pgfqpoint{3.999672in}{0.987757in}}%
\pgfpathlineto{\pgfqpoint{4.021910in}{0.987906in}}%
\pgfpathlineto{\pgfqpoint{4.044148in}{0.984040in}}%
\pgfpathlineto{\pgfqpoint{4.066386in}{0.982801in}}%
\pgfpathlineto{\pgfqpoint{4.088624in}{0.991111in}}%
\pgfpathlineto{\pgfqpoint{4.110862in}{1.000767in}}%
\pgfpathlineto{\pgfqpoint{4.133100in}{1.002561in}}%
\pgfpathlineto{\pgfqpoint{4.155338in}{0.995170in}}%
\pgfpathlineto{\pgfqpoint{4.177576in}{0.982908in}}%
\pgfpathlineto{\pgfqpoint{4.199813in}{0.976649in}}%
\pgfpathlineto{\pgfqpoint{4.222051in}{0.983976in}}%
\pgfpathlineto{\pgfqpoint{4.244289in}{0.997819in}}%
\pgfpathlineto{\pgfqpoint{4.266527in}{1.004270in}}%
\pgfpathlineto{\pgfqpoint{4.288765in}{0.997562in}}%
\pgfpathlineto{\pgfqpoint{4.311003in}{0.985172in}}%
\pgfpathlineto{\pgfqpoint{4.333241in}{0.975901in}}%
\pgfpathlineto{\pgfqpoint{4.355479in}{0.970753in}}%
\pgfpathlineto{\pgfqpoint{4.377717in}{0.968488in}}%
\pgfpathlineto{\pgfqpoint{4.399955in}{0.964878in}}%
\pgfpathlineto{\pgfqpoint{4.444431in}{0.948130in}}%
\pgfpathlineto{\pgfqpoint{4.466669in}{0.946912in}}%
\pgfpathlineto{\pgfqpoint{4.488907in}{0.950245in}}%
\pgfpathlineto{\pgfqpoint{4.511145in}{0.954859in}}%
\pgfpathlineto{\pgfqpoint{4.533383in}{0.961161in}}%
\pgfpathlineto{\pgfqpoint{4.555620in}{0.969428in}}%
\pgfpathlineto{\pgfqpoint{4.577858in}{0.979981in}}%
\pgfpathlineto{\pgfqpoint{4.622334in}{1.010721in}}%
\pgfpathlineto{\pgfqpoint{4.644572in}{1.015293in}}%
\pgfpathlineto{\pgfqpoint{4.666810in}{1.007111in}}%
\pgfpathlineto{\pgfqpoint{4.689048in}{1.001108in}}%
\pgfpathlineto{\pgfqpoint{4.711286in}{1.006556in}}%
\pgfpathlineto{\pgfqpoint{4.755762in}{1.032276in}}%
\pgfpathlineto{\pgfqpoint{4.778000in}{1.038599in}}%
\pgfpathlineto{\pgfqpoint{4.822476in}{1.035544in}}%
\pgfpathlineto{\pgfqpoint{4.844714in}{1.039454in}}%
\pgfpathlineto{\pgfqpoint{4.866952in}{1.041077in}}%
\pgfpathlineto{\pgfqpoint{4.889190in}{1.036036in}}%
\pgfpathlineto{\pgfqpoint{4.911427in}{1.032639in}}%
\pgfpathlineto{\pgfqpoint{4.955903in}{1.039902in}}%
\pgfpathlineto{\pgfqpoint{4.978141in}{1.038129in}}%
\pgfpathlineto{\pgfqpoint{5.044855in}{1.022065in}}%
\pgfpathlineto{\pgfqpoint{5.089331in}{1.014908in}}%
\pgfpathlineto{\pgfqpoint{5.133807in}{0.997861in}}%
\pgfpathlineto{\pgfqpoint{5.156045in}{0.996494in}}%
\pgfpathlineto{\pgfqpoint{5.178283in}{0.998759in}}%
\pgfpathlineto{\pgfqpoint{5.200521in}{0.994807in}}%
\pgfpathlineto{\pgfqpoint{5.222759in}{0.980985in}}%
\pgfpathlineto{\pgfqpoint{5.244996in}{0.963318in}}%
\pgfpathlineto{\pgfqpoint{5.267234in}{0.951911in}}%
\pgfpathlineto{\pgfqpoint{5.289472in}{0.951847in}}%
\pgfpathlineto{\pgfqpoint{5.311710in}{0.960178in}}%
\pgfpathlineto{\pgfqpoint{5.333948in}{0.967164in}}%
\pgfpathlineto{\pgfqpoint{5.356186in}{0.963511in}}%
\pgfpathlineto{\pgfqpoint{5.378424in}{0.951591in}}%
\pgfpathlineto{\pgfqpoint{5.400662in}{0.941038in}}%
\pgfpathlineto{\pgfqpoint{5.422900in}{0.940055in}}%
\pgfpathlineto{\pgfqpoint{5.467376in}{0.956718in}}%
\pgfpathlineto{\pgfqpoint{5.489614in}{0.959153in}}%
\pgfpathlineto{\pgfqpoint{5.511852in}{0.957508in}}%
\pgfpathlineto{\pgfqpoint{5.534090in}{0.953385in}}%
\pgfpathlineto{\pgfqpoint{5.556328in}{0.952936in}}%
\pgfpathlineto{\pgfqpoint{5.600803in}{0.965348in}}%
\pgfpathlineto{\pgfqpoint{5.645279in}{0.979938in}}%
\pgfpathlineto{\pgfqpoint{5.667517in}{0.982694in}}%
\pgfpathlineto{\pgfqpoint{5.689755in}{0.984296in}}%
\pgfpathlineto{\pgfqpoint{5.711993in}{0.991004in}}%
\pgfpathlineto{\pgfqpoint{5.734231in}{1.001194in}}%
\pgfpathlineto{\pgfqpoint{5.756469in}{1.009952in}}%
\pgfpathlineto{\pgfqpoint{5.778707in}{1.016212in}}%
\pgfpathlineto{\pgfqpoint{5.823183in}{1.025013in}}%
\pgfpathlineto{\pgfqpoint{5.845421in}{1.028239in}}%
\pgfpathlineto{\pgfqpoint{5.867659in}{1.033558in}}%
\pgfpathlineto{\pgfqpoint{5.889897in}{1.041376in}}%
\pgfpathlineto{\pgfqpoint{5.912135in}{1.046845in}}%
\pgfpathlineto{\pgfqpoint{5.934372in}{1.045521in}}%
\pgfpathlineto{\pgfqpoint{5.956610in}{1.037510in}}%
\pgfpathlineto{\pgfqpoint{5.978848in}{1.028174in}}%
\pgfpathlineto{\pgfqpoint{6.001086in}{1.027576in}}%
\pgfpathlineto{\pgfqpoint{6.045562in}{1.047593in}}%
\pgfpathlineto{\pgfqpoint{6.067800in}{1.043726in}}%
\pgfpathlineto{\pgfqpoint{6.090038in}{1.026594in}}%
\pgfpathlineto{\pgfqpoint{6.112276in}{1.007752in}}%
\pgfpathlineto{\pgfqpoint{6.134514in}{1.000638in}}%
\pgfpathlineto{\pgfqpoint{6.178990in}{1.011683in}}%
\pgfpathlineto{\pgfqpoint{6.201228in}{1.007303in}}%
\pgfpathlineto{\pgfqpoint{6.223466in}{0.993824in}}%
\pgfpathlineto{\pgfqpoint{6.245704in}{0.976777in}}%
\pgfpathlineto{\pgfqpoint{6.267942in}{0.962891in}}%
\pgfpathlineto{\pgfqpoint{6.290179in}{0.957401in}}%
\pgfpathlineto{\pgfqpoint{6.334655in}{0.960776in}}%
\pgfpathlineto{\pgfqpoint{6.356893in}{0.956782in}}%
\pgfpathlineto{\pgfqpoint{6.401369in}{0.943558in}}%
\pgfpathlineto{\pgfqpoint{6.455845in}{0.933776in}}%
\pgfpathlineto{\pgfqpoint{6.455845in}{0.933776in}}%
\pgfusepath{stroke}%
\end{pgfscope}%
\begin{pgfscope}%
\pgfsetrectcap%
\pgfsetmiterjoin%
\pgfsetlinewidth{0.803000pt}%
\definecolor{currentstroke}{rgb}{0.000000,0.000000,0.000000}%
\pgfsetstrokecolor{currentstroke}%
\pgfsetdash{}{0pt}%
\pgfpathmoveto{\pgfqpoint{0.895256in}{0.220000in}}%
\pgfpathlineto{\pgfqpoint{0.895256in}{1.760000in}}%
\pgfusepath{stroke}%
\end{pgfscope}%
\begin{pgfscope}%
\pgfsetrectcap%
\pgfsetmiterjoin%
\pgfsetlinewidth{0.803000pt}%
\definecolor{currentstroke}{rgb}{0.000000,0.000000,0.000000}%
\pgfsetstrokecolor{currentstroke}%
\pgfsetdash{}{0pt}%
\pgfpathmoveto{\pgfqpoint{6.445845in}{0.220000in}}%
\pgfpathlineto{\pgfqpoint{6.445845in}{1.760000in}}%
\pgfusepath{stroke}%
\end{pgfscope}%
\begin{pgfscope}%
\pgfsetrectcap%
\pgfsetmiterjoin%
\pgfsetlinewidth{0.803000pt}%
\definecolor{currentstroke}{rgb}{0.000000,0.000000,0.000000}%
\pgfsetstrokecolor{currentstroke}%
\pgfsetdash{}{0pt}%
\pgfpathmoveto{\pgfqpoint{0.895256in}{0.220000in}}%
\pgfpathlineto{\pgfqpoint{6.445845in}{0.220000in}}%
\pgfusepath{stroke}%
\end{pgfscope}%
\begin{pgfscope}%
\pgfsetrectcap%
\pgfsetmiterjoin%
\pgfsetlinewidth{0.803000pt}%
\definecolor{currentstroke}{rgb}{0.000000,0.000000,0.000000}%
\pgfsetstrokecolor{currentstroke}%
\pgfsetdash{}{0pt}%
\pgfpathmoveto{\pgfqpoint{0.895256in}{1.760000in}}%
\pgfpathlineto{\pgfqpoint{6.445845in}{1.760000in}}%
\pgfusepath{stroke}%
\end{pgfscope}%
\begin{pgfscope}%
\definecolor{textcolor}{rgb}{0.000000,0.000000,0.000000}%
\pgfsetstrokecolor{textcolor}%
\pgfsetfillcolor{textcolor}%
\pgftext[x=3.670551in,y=1.843333in,,base]{\color{textcolor}{\rmfamily\fontsize{12.000000}{14.400000}\selectfont\catcode`\^=\active\def^{\ifmmode\sp\else\^{}\fi}\catcode`\%=\active\def%{\%}Phase Flip}}%
\end{pgfscope}%
\begin{pgfscope}%
\pgfsetbuttcap%
\pgfsetmiterjoin%
\definecolor{currentfill}{rgb}{1.000000,1.000000,1.000000}%
\pgfsetfillcolor{currentfill}%
\pgfsetfillopacity{0.800000}%
\pgfsetlinewidth{1.003750pt}%
\definecolor{currentstroke}{rgb}{0.800000,0.800000,0.800000}%
\pgfsetstrokecolor{currentstroke}%
\pgfsetstrokeopacity{0.800000}%
\pgfsetdash{}{0pt}%
\pgfpathmoveto{\pgfqpoint{3.173443in}{1.261543in}}%
\pgfpathlineto{\pgfqpoint{4.167658in}{1.261543in}}%
\pgfpathquadraticcurveto{\pgfqpoint{4.195436in}{1.261543in}}{\pgfqpoint{4.195436in}{1.289321in}}%
\pgfpathlineto{\pgfqpoint{4.195436in}{1.662778in}}%
\pgfpathquadraticcurveto{\pgfqpoint{4.195436in}{1.690556in}}{\pgfqpoint{4.167658in}{1.690556in}}%
\pgfpathlineto{\pgfqpoint{3.173443in}{1.690556in}}%
\pgfpathquadraticcurveto{\pgfqpoint{3.145666in}{1.690556in}}{\pgfqpoint{3.145666in}{1.662778in}}%
\pgfpathlineto{\pgfqpoint{3.145666in}{1.289321in}}%
\pgfpathquadraticcurveto{\pgfqpoint{3.145666in}{1.261543in}}{\pgfqpoint{3.173443in}{1.261543in}}%
\pgfpathlineto{\pgfqpoint{3.173443in}{1.261543in}}%
\pgfpathclose%
\pgfusepath{stroke,fill}%
\end{pgfscope}%
\begin{pgfscope}%
\pgfsetrectcap%
\pgfsetroundjoin%
\pgfsetlinewidth{1.505625pt}%
\definecolor{currentstroke}{rgb}{0.121569,0.466667,0.705882}%
\pgfsetstrokecolor{currentstroke}%
\pgfsetdash{}{0pt}%
\pgfpathmoveto{\pgfqpoint{3.201221in}{1.586389in}}%
\pgfpathlineto{\pgfqpoint{3.340110in}{1.586389in}}%
\pgfpathlineto{\pgfqpoint{3.478999in}{1.586389in}}%
\pgfusepath{stroke}%
\end{pgfscope}%
\begin{pgfscope}%
\definecolor{textcolor}{rgb}{0.000000,0.000000,0.000000}%
\pgfsetstrokecolor{textcolor}%
\pgfsetfillcolor{textcolor}%
\pgftext[x=3.590110in,y=1.537778in,left,base]{\color{textcolor}{\rmfamily\fontsize{10.000000}{12.000000}\selectfont\catcode`\^=\active\def^{\ifmmode\sp\else\^{}\fi}\catcode`\%=\active\def%{\%}Playback}}%
\end{pgfscope}%
\begin{pgfscope}%
\pgfsetrectcap%
\pgfsetroundjoin%
\pgfsetlinewidth{1.505625pt}%
\definecolor{currentstroke}{rgb}{1.000000,0.498039,0.054902}%
\pgfsetstrokecolor{currentstroke}%
\pgfsetdash{}{0pt}%
\pgfpathmoveto{\pgfqpoint{3.201221in}{1.392716in}}%
\pgfpathlineto{\pgfqpoint{3.340110in}{1.392716in}}%
\pgfpathlineto{\pgfqpoint{3.478999in}{1.392716in}}%
\pgfusepath{stroke}%
\end{pgfscope}%
\begin{pgfscope}%
\definecolor{textcolor}{rgb}{0.000000,0.000000,0.000000}%
\pgfsetstrokecolor{textcolor}%
\pgfsetfillcolor{textcolor}%
\pgftext[x=3.590110in,y=1.344105in,left,base]{\color{textcolor}{\rmfamily\fontsize{10.000000}{12.000000}\selectfont\catcode`\^=\active\def^{\ifmmode\sp\else\^{}\fi}\catcode`\%=\active\def%{\%}Capture}}%
\end{pgfscope}%
\end{pgfpicture}%
\makeatother%
\endgroup%
}
    \caption{Illustration of phase inversion in recorded sound wave.}
    \label{fig:flip}
\end{figure}

As illustrated in Figure \ref{fig:flip}, the amplitude of the recorded sound wave is the negative of the original played sound wave.

The underlying cause of this phase inversion was not extensively investigated.
However, this issue can be readily corrected by multiplying the recorded sound wave by \(-1\).

\subsubsection{Clock Drift}

Clock drift occurs when the clock of the recording device is not perfectly synchronized with the clock of the playback device.
As a result, the recorded sound wave will gradually drift out of sync with the played sound wave over time.
This can cause significant problems in our communication systems, as the timing of the received signal is crucial for correct demodulation and decoding.

\begin{figure}[H]
    \noindent\makebox[\textwidth]{%% Creator: Matplotlib, PGF backend
%%
%% To include the figure in your LaTeX document, write
%%   \input{<filename>.pgf}
%%
%% Make sure the required packages are loaded in your preamble
%%   \usepackage{pgf}
%%
%% Also ensure that all the required font packages are loaded; for instance,
%% the lmodern package is sometimes necessary when using math font.
%%   \usepackage{lmodern}
%%
%% Figures using additional raster images can only be included by \input if
%% they are in the same directory as the main LaTeX file. For loading figures
%% from other directories you can use the `import` package
%%   \usepackage{import}
%%
%% and then include the figures with
%%   \import{<path to file>}{<filename>.pgf}
%%
%% Matplotlib used the following preamble
%%   \def\mathdefault#1{#1}
%%   \everymath=\expandafter{\the\everymath\displaystyle}
%%   \IfFileExists{scrextend.sty}{
%%     \usepackage[fontsize=10.000000pt]{scrextend}
%%   }{
%%     \renewcommand{\normalsize}{\fontsize{10.000000}{12.000000}\selectfont}
%%     \normalsize
%%   }
%%   
%%   \makeatletter\@ifpackageloaded{underscore}{}{\usepackage[strings]{underscore}}\makeatother
%%
\begingroup%
\makeatletter%
\begin{pgfpicture}%
\pgfpathrectangle{\pgfpointorigin}{\pgfqpoint{7.162050in}{2.000000in}}%
\pgfusepath{use as bounding box, clip}%
\begin{pgfscope}%
\pgfsetbuttcap%
\pgfsetmiterjoin%
\definecolor{currentfill}{rgb}{1.000000,1.000000,1.000000}%
\pgfsetfillcolor{currentfill}%
\pgfsetlinewidth{0.000000pt}%
\definecolor{currentstroke}{rgb}{1.000000,1.000000,1.000000}%
\pgfsetstrokecolor{currentstroke}%
\pgfsetdash{}{0pt}%
\pgfpathmoveto{\pgfqpoint{0.000000in}{0.000000in}}%
\pgfpathlineto{\pgfqpoint{7.162050in}{0.000000in}}%
\pgfpathlineto{\pgfqpoint{7.162050in}{2.000000in}}%
\pgfpathlineto{\pgfqpoint{0.000000in}{2.000000in}}%
\pgfpathlineto{\pgfqpoint{0.000000in}{0.000000in}}%
\pgfpathclose%
\pgfusepath{fill}%
\end{pgfscope}%
\begin{pgfscope}%
\pgfsetbuttcap%
\pgfsetmiterjoin%
\definecolor{currentfill}{rgb}{1.000000,1.000000,1.000000}%
\pgfsetfillcolor{currentfill}%
\pgfsetlinewidth{0.000000pt}%
\definecolor{currentstroke}{rgb}{0.000000,0.000000,0.000000}%
\pgfsetstrokecolor{currentstroke}%
\pgfsetstrokeopacity{0.000000}%
\pgfsetdash{}{0pt}%
\pgfpathmoveto{\pgfqpoint{0.895256in}{0.220000in}}%
\pgfpathlineto{\pgfqpoint{6.445845in}{0.220000in}}%
\pgfpathlineto{\pgfqpoint{6.445845in}{1.760000in}}%
\pgfpathlineto{\pgfqpoint{0.895256in}{1.760000in}}%
\pgfpathlineto{\pgfqpoint{0.895256in}{0.220000in}}%
\pgfpathclose%
\pgfusepath{fill}%
\end{pgfscope}%
\begin{pgfscope}%
\pgfsetbuttcap%
\pgfsetroundjoin%
\definecolor{currentfill}{rgb}{0.000000,0.000000,0.000000}%
\pgfsetfillcolor{currentfill}%
\pgfsetlinewidth{0.803000pt}%
\definecolor{currentstroke}{rgb}{0.000000,0.000000,0.000000}%
\pgfsetstrokecolor{currentstroke}%
\pgfsetdash{}{0pt}%
\pgfsys@defobject{currentmarker}{\pgfqpoint{0.000000in}{-0.048611in}}{\pgfqpoint{0.000000in}{0.000000in}}{%
\pgfpathmoveto{\pgfqpoint{0.000000in}{0.000000in}}%
\pgfpathlineto{\pgfqpoint{0.000000in}{-0.048611in}}%
\pgfusepath{stroke,fill}%
}%
\begin{pgfscope}%
\pgfsys@transformshift{1.436777in}{0.220000in}%
\pgfsys@useobject{currentmarker}{}%
\end{pgfscope}%
\end{pgfscope}%
\begin{pgfscope}%
\definecolor{textcolor}{rgb}{0.000000,0.000000,0.000000}%
\pgfsetstrokecolor{textcolor}%
\pgfsetfillcolor{textcolor}%
\pgftext[x=1.436777in,y=0.122778in,,top]{\color{textcolor}{\rmfamily\fontsize{10.000000}{12.000000}\selectfont\catcode`\^=\active\def^{\ifmmode\sp\else\^{}\fi}\catcode`\%=\active\def%{\%}$\mathdefault{0.005}$}}%
\end{pgfscope}%
\begin{pgfscope}%
\pgfsetbuttcap%
\pgfsetroundjoin%
\definecolor{currentfill}{rgb}{0.000000,0.000000,0.000000}%
\pgfsetfillcolor{currentfill}%
\pgfsetlinewidth{0.803000pt}%
\definecolor{currentstroke}{rgb}{0.000000,0.000000,0.000000}%
\pgfsetstrokecolor{currentstroke}%
\pgfsetdash{}{0pt}%
\pgfsys@defobject{currentmarker}{\pgfqpoint{0.000000in}{-0.048611in}}{\pgfqpoint{0.000000in}{0.000000in}}{%
\pgfpathmoveto{\pgfqpoint{0.000000in}{0.000000in}}%
\pgfpathlineto{\pgfqpoint{0.000000in}{-0.048611in}}%
\pgfusepath{stroke,fill}%
}%
\begin{pgfscope}%
\pgfsys@transformshift{2.339312in}{0.220000in}%
\pgfsys@useobject{currentmarker}{}%
\end{pgfscope}%
\end{pgfscope}%
\begin{pgfscope}%
\definecolor{textcolor}{rgb}{0.000000,0.000000,0.000000}%
\pgfsetstrokecolor{textcolor}%
\pgfsetfillcolor{textcolor}%
\pgftext[x=2.339312in,y=0.122778in,,top]{\color{textcolor}{\rmfamily\fontsize{10.000000}{12.000000}\selectfont\catcode`\^=\active\def^{\ifmmode\sp\else\^{}\fi}\catcode`\%=\active\def%{\%}$\mathdefault{0.006}$}}%
\end{pgfscope}%
\begin{pgfscope}%
\pgfsetbuttcap%
\pgfsetroundjoin%
\definecolor{currentfill}{rgb}{0.000000,0.000000,0.000000}%
\pgfsetfillcolor{currentfill}%
\pgfsetlinewidth{0.803000pt}%
\definecolor{currentstroke}{rgb}{0.000000,0.000000,0.000000}%
\pgfsetstrokecolor{currentstroke}%
\pgfsetdash{}{0pt}%
\pgfsys@defobject{currentmarker}{\pgfqpoint{0.000000in}{-0.048611in}}{\pgfqpoint{0.000000in}{0.000000in}}{%
\pgfpathmoveto{\pgfqpoint{0.000000in}{0.000000in}}%
\pgfpathlineto{\pgfqpoint{0.000000in}{-0.048611in}}%
\pgfusepath{stroke,fill}%
}%
\begin{pgfscope}%
\pgfsys@transformshift{3.241847in}{0.220000in}%
\pgfsys@useobject{currentmarker}{}%
\end{pgfscope}%
\end{pgfscope}%
\begin{pgfscope}%
\definecolor{textcolor}{rgb}{0.000000,0.000000,0.000000}%
\pgfsetstrokecolor{textcolor}%
\pgfsetfillcolor{textcolor}%
\pgftext[x=3.241847in,y=0.122778in,,top]{\color{textcolor}{\rmfamily\fontsize{10.000000}{12.000000}\selectfont\catcode`\^=\active\def^{\ifmmode\sp\else\^{}\fi}\catcode`\%=\active\def%{\%}$\mathdefault{0.007}$}}%
\end{pgfscope}%
\begin{pgfscope}%
\pgfsetbuttcap%
\pgfsetroundjoin%
\definecolor{currentfill}{rgb}{0.000000,0.000000,0.000000}%
\pgfsetfillcolor{currentfill}%
\pgfsetlinewidth{0.803000pt}%
\definecolor{currentstroke}{rgb}{0.000000,0.000000,0.000000}%
\pgfsetstrokecolor{currentstroke}%
\pgfsetdash{}{0pt}%
\pgfsys@defobject{currentmarker}{\pgfqpoint{0.000000in}{-0.048611in}}{\pgfqpoint{0.000000in}{0.000000in}}{%
\pgfpathmoveto{\pgfqpoint{0.000000in}{0.000000in}}%
\pgfpathlineto{\pgfqpoint{0.000000in}{-0.048611in}}%
\pgfusepath{stroke,fill}%
}%
\begin{pgfscope}%
\pgfsys@transformshift{4.144381in}{0.220000in}%
\pgfsys@useobject{currentmarker}{}%
\end{pgfscope}%
\end{pgfscope}%
\begin{pgfscope}%
\definecolor{textcolor}{rgb}{0.000000,0.000000,0.000000}%
\pgfsetstrokecolor{textcolor}%
\pgfsetfillcolor{textcolor}%
\pgftext[x=4.144381in,y=0.122778in,,top]{\color{textcolor}{\rmfamily\fontsize{10.000000}{12.000000}\selectfont\catcode`\^=\active\def^{\ifmmode\sp\else\^{}\fi}\catcode`\%=\active\def%{\%}$\mathdefault{0.008}$}}%
\end{pgfscope}%
\begin{pgfscope}%
\pgfsetbuttcap%
\pgfsetroundjoin%
\definecolor{currentfill}{rgb}{0.000000,0.000000,0.000000}%
\pgfsetfillcolor{currentfill}%
\pgfsetlinewidth{0.803000pt}%
\definecolor{currentstroke}{rgb}{0.000000,0.000000,0.000000}%
\pgfsetstrokecolor{currentstroke}%
\pgfsetdash{}{0pt}%
\pgfsys@defobject{currentmarker}{\pgfqpoint{0.000000in}{-0.048611in}}{\pgfqpoint{0.000000in}{0.000000in}}{%
\pgfpathmoveto{\pgfqpoint{0.000000in}{0.000000in}}%
\pgfpathlineto{\pgfqpoint{0.000000in}{-0.048611in}}%
\pgfusepath{stroke,fill}%
}%
\begin{pgfscope}%
\pgfsys@transformshift{5.046916in}{0.220000in}%
\pgfsys@useobject{currentmarker}{}%
\end{pgfscope}%
\end{pgfscope}%
\begin{pgfscope}%
\definecolor{textcolor}{rgb}{0.000000,0.000000,0.000000}%
\pgfsetstrokecolor{textcolor}%
\pgfsetfillcolor{textcolor}%
\pgftext[x=5.046916in,y=0.122778in,,top]{\color{textcolor}{\rmfamily\fontsize{10.000000}{12.000000}\selectfont\catcode`\^=\active\def^{\ifmmode\sp\else\^{}\fi}\catcode`\%=\active\def%{\%}$\mathdefault{0.009}$}}%
\end{pgfscope}%
\begin{pgfscope}%
\pgfsetbuttcap%
\pgfsetroundjoin%
\definecolor{currentfill}{rgb}{0.000000,0.000000,0.000000}%
\pgfsetfillcolor{currentfill}%
\pgfsetlinewidth{0.803000pt}%
\definecolor{currentstroke}{rgb}{0.000000,0.000000,0.000000}%
\pgfsetstrokecolor{currentstroke}%
\pgfsetdash{}{0pt}%
\pgfsys@defobject{currentmarker}{\pgfqpoint{0.000000in}{-0.048611in}}{\pgfqpoint{0.000000in}{0.000000in}}{%
\pgfpathmoveto{\pgfqpoint{0.000000in}{0.000000in}}%
\pgfpathlineto{\pgfqpoint{0.000000in}{-0.048611in}}%
\pgfusepath{stroke,fill}%
}%
\begin{pgfscope}%
\pgfsys@transformshift{5.949451in}{0.220000in}%
\pgfsys@useobject{currentmarker}{}%
\end{pgfscope}%
\end{pgfscope}%
\begin{pgfscope}%
\definecolor{textcolor}{rgb}{0.000000,0.000000,0.000000}%
\pgfsetstrokecolor{textcolor}%
\pgfsetfillcolor{textcolor}%
\pgftext[x=5.949451in,y=0.122778in,,top]{\color{textcolor}{\rmfamily\fontsize{10.000000}{12.000000}\selectfont\catcode`\^=\active\def^{\ifmmode\sp\else\^{}\fi}\catcode`\%=\active\def%{\%}$\mathdefault{0.010}$}}%
\end{pgfscope}%
\begin{pgfscope}%
\definecolor{textcolor}{rgb}{0.000000,0.000000,0.000000}%
\pgfsetstrokecolor{textcolor}%
\pgfsetfillcolor{textcolor}%
\pgftext[x=3.670551in,y=-0.056234in,,top]{\color{textcolor}{\rmfamily\fontsize{10.000000}{12.000000}\selectfont\catcode`\^=\active\def^{\ifmmode\sp\else\^{}\fi}\catcode`\%=\active\def%{\%}Time (s)}}%
\end{pgfscope}%
\begin{pgfscope}%
\pgfsetbuttcap%
\pgfsetroundjoin%
\definecolor{currentfill}{rgb}{0.000000,0.000000,0.000000}%
\pgfsetfillcolor{currentfill}%
\pgfsetlinewidth{0.803000pt}%
\definecolor{currentstroke}{rgb}{0.000000,0.000000,0.000000}%
\pgfsetstrokecolor{currentstroke}%
\pgfsetdash{}{0pt}%
\pgfsys@defobject{currentmarker}{\pgfqpoint{-0.048611in}{0.000000in}}{\pgfqpoint{-0.000000in}{0.000000in}}{%
\pgfpathmoveto{\pgfqpoint{-0.000000in}{0.000000in}}%
\pgfpathlineto{\pgfqpoint{-0.048611in}{0.000000in}}%
\pgfusepath{stroke,fill}%
}%
\begin{pgfscope}%
\pgfsys@transformshift{0.895256in}{0.290000in}%
\pgfsys@useobject{currentmarker}{}%
\end{pgfscope}%
\end{pgfscope}%
\begin{pgfscope}%
\definecolor{textcolor}{rgb}{0.000000,0.000000,0.000000}%
\pgfsetstrokecolor{textcolor}%
\pgfsetfillcolor{textcolor}%
\pgftext[x=0.512539in, y=0.241775in, left, base]{\color{textcolor}{\rmfamily\fontsize{10.000000}{12.000000}\selectfont\catcode`\^=\active\def^{\ifmmode\sp\else\^{}\fi}\catcode`\%=\active\def%{\%}$\mathdefault{\ensuremath{-}1.0}$}}%
\end{pgfscope}%
\begin{pgfscope}%
\pgfsetbuttcap%
\pgfsetroundjoin%
\definecolor{currentfill}{rgb}{0.000000,0.000000,0.000000}%
\pgfsetfillcolor{currentfill}%
\pgfsetlinewidth{0.803000pt}%
\definecolor{currentstroke}{rgb}{0.000000,0.000000,0.000000}%
\pgfsetstrokecolor{currentstroke}%
\pgfsetdash{}{0pt}%
\pgfsys@defobject{currentmarker}{\pgfqpoint{-0.048611in}{0.000000in}}{\pgfqpoint{-0.000000in}{0.000000in}}{%
\pgfpathmoveto{\pgfqpoint{-0.000000in}{0.000000in}}%
\pgfpathlineto{\pgfqpoint{-0.048611in}{0.000000in}}%
\pgfusepath{stroke,fill}%
}%
\begin{pgfscope}%
\pgfsys@transformshift{0.895256in}{0.640000in}%
\pgfsys@useobject{currentmarker}{}%
\end{pgfscope}%
\end{pgfscope}%
\begin{pgfscope}%
\definecolor{textcolor}{rgb}{0.000000,0.000000,0.000000}%
\pgfsetstrokecolor{textcolor}%
\pgfsetfillcolor{textcolor}%
\pgftext[x=0.512539in, y=0.591775in, left, base]{\color{textcolor}{\rmfamily\fontsize{10.000000}{12.000000}\selectfont\catcode`\^=\active\def^{\ifmmode\sp\else\^{}\fi}\catcode`\%=\active\def%{\%}$\mathdefault{\ensuremath{-}0.5}$}}%
\end{pgfscope}%
\begin{pgfscope}%
\pgfsetbuttcap%
\pgfsetroundjoin%
\definecolor{currentfill}{rgb}{0.000000,0.000000,0.000000}%
\pgfsetfillcolor{currentfill}%
\pgfsetlinewidth{0.803000pt}%
\definecolor{currentstroke}{rgb}{0.000000,0.000000,0.000000}%
\pgfsetstrokecolor{currentstroke}%
\pgfsetdash{}{0pt}%
\pgfsys@defobject{currentmarker}{\pgfqpoint{-0.048611in}{0.000000in}}{\pgfqpoint{-0.000000in}{0.000000in}}{%
\pgfpathmoveto{\pgfqpoint{-0.000000in}{0.000000in}}%
\pgfpathlineto{\pgfqpoint{-0.048611in}{0.000000in}}%
\pgfusepath{stroke,fill}%
}%
\begin{pgfscope}%
\pgfsys@transformshift{0.895256in}{0.990000in}%
\pgfsys@useobject{currentmarker}{}%
\end{pgfscope}%
\end{pgfscope}%
\begin{pgfscope}%
\definecolor{textcolor}{rgb}{0.000000,0.000000,0.000000}%
\pgfsetstrokecolor{textcolor}%
\pgfsetfillcolor{textcolor}%
\pgftext[x=0.620564in, y=0.941775in, left, base]{\color{textcolor}{\rmfamily\fontsize{10.000000}{12.000000}\selectfont\catcode`\^=\active\def^{\ifmmode\sp\else\^{}\fi}\catcode`\%=\active\def%{\%}$\mathdefault{0.0}$}}%
\end{pgfscope}%
\begin{pgfscope}%
\pgfsetbuttcap%
\pgfsetroundjoin%
\definecolor{currentfill}{rgb}{0.000000,0.000000,0.000000}%
\pgfsetfillcolor{currentfill}%
\pgfsetlinewidth{0.803000pt}%
\definecolor{currentstroke}{rgb}{0.000000,0.000000,0.000000}%
\pgfsetstrokecolor{currentstroke}%
\pgfsetdash{}{0pt}%
\pgfsys@defobject{currentmarker}{\pgfqpoint{-0.048611in}{0.000000in}}{\pgfqpoint{-0.000000in}{0.000000in}}{%
\pgfpathmoveto{\pgfqpoint{-0.000000in}{0.000000in}}%
\pgfpathlineto{\pgfqpoint{-0.048611in}{0.000000in}}%
\pgfusepath{stroke,fill}%
}%
\begin{pgfscope}%
\pgfsys@transformshift{0.895256in}{1.340000in}%
\pgfsys@useobject{currentmarker}{}%
\end{pgfscope}%
\end{pgfscope}%
\begin{pgfscope}%
\definecolor{textcolor}{rgb}{0.000000,0.000000,0.000000}%
\pgfsetstrokecolor{textcolor}%
\pgfsetfillcolor{textcolor}%
\pgftext[x=0.620564in, y=1.291775in, left, base]{\color{textcolor}{\rmfamily\fontsize{10.000000}{12.000000}\selectfont\catcode`\^=\active\def^{\ifmmode\sp\else\^{}\fi}\catcode`\%=\active\def%{\%}$\mathdefault{0.5}$}}%
\end{pgfscope}%
\begin{pgfscope}%
\pgfsetbuttcap%
\pgfsetroundjoin%
\definecolor{currentfill}{rgb}{0.000000,0.000000,0.000000}%
\pgfsetfillcolor{currentfill}%
\pgfsetlinewidth{0.803000pt}%
\definecolor{currentstroke}{rgb}{0.000000,0.000000,0.000000}%
\pgfsetstrokecolor{currentstroke}%
\pgfsetdash{}{0pt}%
\pgfsys@defobject{currentmarker}{\pgfqpoint{-0.048611in}{0.000000in}}{\pgfqpoint{-0.000000in}{0.000000in}}{%
\pgfpathmoveto{\pgfqpoint{-0.000000in}{0.000000in}}%
\pgfpathlineto{\pgfqpoint{-0.048611in}{0.000000in}}%
\pgfusepath{stroke,fill}%
}%
\begin{pgfscope}%
\pgfsys@transformshift{0.895256in}{1.690000in}%
\pgfsys@useobject{currentmarker}{}%
\end{pgfscope}%
\end{pgfscope}%
\begin{pgfscope}%
\definecolor{textcolor}{rgb}{0.000000,0.000000,0.000000}%
\pgfsetstrokecolor{textcolor}%
\pgfsetfillcolor{textcolor}%
\pgftext[x=0.620564in, y=1.641775in, left, base]{\color{textcolor}{\rmfamily\fontsize{10.000000}{12.000000}\selectfont\catcode`\^=\active\def^{\ifmmode\sp\else\^{}\fi}\catcode`\%=\active\def%{\%}$\mathdefault{1.0}$}}%
\end{pgfscope}%
\begin{pgfscope}%
\definecolor{textcolor}{rgb}{0.000000,0.000000,0.000000}%
\pgfsetstrokecolor{textcolor}%
\pgfsetfillcolor{textcolor}%
\pgftext[x=0.456984in,y=0.990000in,,bottom,rotate=90.000000]{\color{textcolor}{\rmfamily\fontsize{10.000000}{12.000000}\selectfont\catcode`\^=\active\def^{\ifmmode\sp\else\^{}\fi}\catcode`\%=\active\def%{\%}Amplitude}}%
\end{pgfscope}%
\begin{pgfscope}%
\pgfpathrectangle{\pgfqpoint{0.895256in}{0.220000in}}{\pgfqpoint{5.550589in}{1.540000in}}%
\pgfusepath{clip}%
\pgfsetrectcap%
\pgfsetroundjoin%
\pgfsetlinewidth{1.505625pt}%
\definecolor{currentstroke}{rgb}{0.121569,0.466667,0.705882}%
\pgfsetstrokecolor{currentstroke}%
\pgfsetdash{}{0pt}%
\pgfpathmoveto{\pgfqpoint{0.891496in}{0.313852in}}%
\pgfpathlineto{\pgfqpoint{0.910298in}{0.383782in}}%
\pgfpathlineto{\pgfqpoint{0.929101in}{0.495025in}}%
\pgfpathlineto{\pgfqpoint{0.947904in}{0.640000in}}%
\pgfpathlineto{\pgfqpoint{0.966707in}{0.808827in}}%
\pgfpathlineto{\pgfqpoint{1.004313in}{1.171173in}}%
\pgfpathlineto{\pgfqpoint{1.023115in}{1.340000in}}%
\pgfpathlineto{\pgfqpoint{1.041918in}{1.484975in}}%
\pgfpathlineto{\pgfqpoint{1.060721in}{1.596218in}}%
\pgfpathlineto{\pgfqpoint{1.079524in}{1.666148in}}%
\pgfpathlineto{\pgfqpoint{1.098327in}{1.690000in}}%
\pgfpathlineto{\pgfqpoint{1.117129in}{1.666148in}}%
\pgfpathlineto{\pgfqpoint{1.135932in}{1.596218in}}%
\pgfpathlineto{\pgfqpoint{1.154735in}{1.484975in}}%
\pgfpathlineto{\pgfqpoint{1.173538in}{1.340000in}}%
\pgfpathlineto{\pgfqpoint{1.192341in}{1.171173in}}%
\pgfpathlineto{\pgfqpoint{1.229946in}{0.808827in}}%
\pgfpathlineto{\pgfqpoint{1.248749in}{0.640000in}}%
\pgfpathlineto{\pgfqpoint{1.267552in}{0.495025in}}%
\pgfpathlineto{\pgfqpoint{1.286355in}{0.383782in}}%
\pgfpathlineto{\pgfqpoint{1.305157in}{0.313852in}}%
\pgfpathlineto{\pgfqpoint{1.323960in}{0.290000in}}%
\pgfpathlineto{\pgfqpoint{1.342763in}{0.313852in}}%
\pgfpathlineto{\pgfqpoint{1.361566in}{0.383782in}}%
\pgfpathlineto{\pgfqpoint{1.380369in}{0.495025in}}%
\pgfpathlineto{\pgfqpoint{1.399171in}{0.640000in}}%
\pgfpathlineto{\pgfqpoint{1.417974in}{0.808827in}}%
\pgfpathlineto{\pgfqpoint{1.455580in}{1.171173in}}%
\pgfpathlineto{\pgfqpoint{1.474383in}{1.340000in}}%
\pgfpathlineto{\pgfqpoint{1.493186in}{1.484975in}}%
\pgfpathlineto{\pgfqpoint{1.511988in}{1.596218in}}%
\pgfpathlineto{\pgfqpoint{1.530791in}{1.666148in}}%
\pgfpathlineto{\pgfqpoint{1.549594in}{1.690000in}}%
\pgfpathlineto{\pgfqpoint{1.568397in}{1.666148in}}%
\pgfpathlineto{\pgfqpoint{1.587200in}{1.596218in}}%
\pgfpathlineto{\pgfqpoint{1.606002in}{1.484975in}}%
\pgfpathlineto{\pgfqpoint{1.624805in}{1.340000in}}%
\pgfpathlineto{\pgfqpoint{1.643608in}{1.171173in}}%
\pgfpathlineto{\pgfqpoint{1.681214in}{0.808827in}}%
\pgfpathlineto{\pgfqpoint{1.700016in}{0.640000in}}%
\pgfpathlineto{\pgfqpoint{1.718819in}{0.495025in}}%
\pgfpathlineto{\pgfqpoint{1.737622in}{0.383782in}}%
\pgfpathlineto{\pgfqpoint{1.756425in}{0.313852in}}%
\pgfpathlineto{\pgfqpoint{1.775228in}{0.290000in}}%
\pgfpathlineto{\pgfqpoint{1.794030in}{0.313852in}}%
\pgfpathlineto{\pgfqpoint{1.812833in}{0.383782in}}%
\pgfpathlineto{\pgfqpoint{1.831636in}{0.495025in}}%
\pgfpathlineto{\pgfqpoint{1.850439in}{0.640000in}}%
\pgfpathlineto{\pgfqpoint{1.869242in}{0.808827in}}%
\pgfpathlineto{\pgfqpoint{1.906847in}{1.171173in}}%
\pgfpathlineto{\pgfqpoint{1.925650in}{1.340000in}}%
\pgfpathlineto{\pgfqpoint{1.944453in}{1.484975in}}%
\pgfpathlineto{\pgfqpoint{1.963256in}{1.596218in}}%
\pgfpathlineto{\pgfqpoint{1.982059in}{1.666148in}}%
\pgfpathlineto{\pgfqpoint{2.000861in}{1.690000in}}%
\pgfpathlineto{\pgfqpoint{2.019664in}{1.666148in}}%
\pgfpathlineto{\pgfqpoint{2.038467in}{1.596218in}}%
\pgfpathlineto{\pgfqpoint{2.057270in}{1.484975in}}%
\pgfpathlineto{\pgfqpoint{2.076073in}{1.340000in}}%
\pgfpathlineto{\pgfqpoint{2.094875in}{1.171173in}}%
\pgfpathlineto{\pgfqpoint{2.132481in}{0.808827in}}%
\pgfpathlineto{\pgfqpoint{2.151284in}{0.640000in}}%
\pgfpathlineto{\pgfqpoint{2.170087in}{0.495025in}}%
\pgfpathlineto{\pgfqpoint{2.188889in}{0.383782in}}%
\pgfpathlineto{\pgfqpoint{2.207692in}{0.313852in}}%
\pgfpathlineto{\pgfqpoint{2.226495in}{0.290000in}}%
\pgfpathlineto{\pgfqpoint{2.245298in}{0.313852in}}%
\pgfpathlineto{\pgfqpoint{2.264101in}{0.383782in}}%
\pgfpathlineto{\pgfqpoint{2.282903in}{0.495025in}}%
\pgfpathlineto{\pgfqpoint{2.301706in}{0.640000in}}%
\pgfpathlineto{\pgfqpoint{2.320509in}{0.808827in}}%
\pgfpathlineto{\pgfqpoint{2.358115in}{1.171173in}}%
\pgfpathlineto{\pgfqpoint{2.376917in}{1.340000in}}%
\pgfpathlineto{\pgfqpoint{2.395720in}{1.484975in}}%
\pgfpathlineto{\pgfqpoint{2.414523in}{1.596218in}}%
\pgfpathlineto{\pgfqpoint{2.433326in}{1.666148in}}%
\pgfpathlineto{\pgfqpoint{2.452129in}{1.690000in}}%
\pgfpathlineto{\pgfqpoint{2.470932in}{1.666148in}}%
\pgfpathlineto{\pgfqpoint{2.489734in}{1.596218in}}%
\pgfpathlineto{\pgfqpoint{2.508537in}{1.484975in}}%
\pgfpathlineto{\pgfqpoint{2.527340in}{1.340000in}}%
\pgfpathlineto{\pgfqpoint{2.546143in}{1.171173in}}%
\pgfpathlineto{\pgfqpoint{2.583748in}{0.808827in}}%
\pgfpathlineto{\pgfqpoint{2.602551in}{0.640000in}}%
\pgfpathlineto{\pgfqpoint{2.621354in}{0.495025in}}%
\pgfpathlineto{\pgfqpoint{2.640157in}{0.383782in}}%
\pgfpathlineto{\pgfqpoint{2.658960in}{0.313852in}}%
\pgfpathlineto{\pgfqpoint{2.677762in}{0.290000in}}%
\pgfpathlineto{\pgfqpoint{2.696565in}{0.313852in}}%
\pgfpathlineto{\pgfqpoint{2.715368in}{0.383782in}}%
\pgfpathlineto{\pgfqpoint{2.734171in}{0.495025in}}%
\pgfpathlineto{\pgfqpoint{2.752974in}{0.640000in}}%
\pgfpathlineto{\pgfqpoint{2.771776in}{0.808827in}}%
\pgfpathlineto{\pgfqpoint{2.809382in}{1.171173in}}%
\pgfpathlineto{\pgfqpoint{2.828185in}{1.340000in}}%
\pgfpathlineto{\pgfqpoint{2.846988in}{1.484975in}}%
\pgfpathlineto{\pgfqpoint{2.865790in}{1.596218in}}%
\pgfpathlineto{\pgfqpoint{2.884593in}{1.666148in}}%
\pgfpathlineto{\pgfqpoint{2.903396in}{1.690000in}}%
\pgfpathlineto{\pgfqpoint{2.922199in}{1.666148in}}%
\pgfpathlineto{\pgfqpoint{2.941002in}{1.596218in}}%
\pgfpathlineto{\pgfqpoint{2.959805in}{1.484975in}}%
\pgfpathlineto{\pgfqpoint{2.978607in}{1.340000in}}%
\pgfpathlineto{\pgfqpoint{2.997410in}{1.171173in}}%
\pgfpathlineto{\pgfqpoint{3.035016in}{0.808827in}}%
\pgfpathlineto{\pgfqpoint{3.053819in}{0.640000in}}%
\pgfpathlineto{\pgfqpoint{3.072621in}{0.495025in}}%
\pgfpathlineto{\pgfqpoint{3.091424in}{0.383782in}}%
\pgfpathlineto{\pgfqpoint{3.110227in}{0.313852in}}%
\pgfpathlineto{\pgfqpoint{3.129030in}{0.290000in}}%
\pgfpathlineto{\pgfqpoint{3.147833in}{0.313852in}}%
\pgfpathlineto{\pgfqpoint{3.166635in}{0.383782in}}%
\pgfpathlineto{\pgfqpoint{3.185438in}{0.495025in}}%
\pgfpathlineto{\pgfqpoint{3.204241in}{0.640000in}}%
\pgfpathlineto{\pgfqpoint{3.223044in}{0.808827in}}%
\pgfpathlineto{\pgfqpoint{3.260649in}{1.171173in}}%
\pgfpathlineto{\pgfqpoint{3.279452in}{1.340000in}}%
\pgfpathlineto{\pgfqpoint{3.298255in}{1.484975in}}%
\pgfpathlineto{\pgfqpoint{3.317058in}{1.596218in}}%
\pgfpathlineto{\pgfqpoint{3.335861in}{1.666148in}}%
\pgfpathlineto{\pgfqpoint{3.354663in}{1.690000in}}%
\pgfpathlineto{\pgfqpoint{3.373466in}{1.666148in}}%
\pgfpathlineto{\pgfqpoint{3.392269in}{1.596218in}}%
\pgfpathlineto{\pgfqpoint{3.411072in}{1.484975in}}%
\pgfpathlineto{\pgfqpoint{3.429875in}{1.340000in}}%
\pgfpathlineto{\pgfqpoint{3.448677in}{1.171173in}}%
\pgfpathlineto{\pgfqpoint{3.486283in}{0.808827in}}%
\pgfpathlineto{\pgfqpoint{3.505086in}{0.640000in}}%
\pgfpathlineto{\pgfqpoint{3.523889in}{0.495025in}}%
\pgfpathlineto{\pgfqpoint{3.542692in}{0.383782in}}%
\pgfpathlineto{\pgfqpoint{3.561494in}{0.313852in}}%
\pgfpathlineto{\pgfqpoint{3.580297in}{0.290000in}}%
\pgfpathlineto{\pgfqpoint{3.599100in}{0.313852in}}%
\pgfpathlineto{\pgfqpoint{3.617903in}{0.383782in}}%
\pgfpathlineto{\pgfqpoint{3.636706in}{0.495025in}}%
\pgfpathlineto{\pgfqpoint{3.655508in}{0.640000in}}%
\pgfpathlineto{\pgfqpoint{3.674311in}{0.808827in}}%
\pgfpathlineto{\pgfqpoint{3.711917in}{1.171173in}}%
\pgfpathlineto{\pgfqpoint{3.730720in}{1.340000in}}%
\pgfpathlineto{\pgfqpoint{3.749522in}{1.484975in}}%
\pgfpathlineto{\pgfqpoint{3.768325in}{1.596218in}}%
\pgfpathlineto{\pgfqpoint{3.787128in}{1.666148in}}%
\pgfpathlineto{\pgfqpoint{3.805931in}{1.690000in}}%
\pgfpathlineto{\pgfqpoint{3.824734in}{1.666148in}}%
\pgfpathlineto{\pgfqpoint{3.843536in}{1.596218in}}%
\pgfpathlineto{\pgfqpoint{3.862339in}{1.484975in}}%
\pgfpathlineto{\pgfqpoint{3.881142in}{1.340000in}}%
\pgfpathlineto{\pgfqpoint{3.899945in}{1.171173in}}%
\pgfpathlineto{\pgfqpoint{3.937550in}{0.808827in}}%
\pgfpathlineto{\pgfqpoint{3.956353in}{0.640000in}}%
\pgfpathlineto{\pgfqpoint{3.975156in}{0.495025in}}%
\pgfpathlineto{\pgfqpoint{3.993959in}{0.383782in}}%
\pgfpathlineto{\pgfqpoint{4.012762in}{0.313852in}}%
\pgfpathlineto{\pgfqpoint{4.031565in}{0.290000in}}%
\pgfpathlineto{\pgfqpoint{4.050367in}{0.313852in}}%
\pgfpathlineto{\pgfqpoint{4.069170in}{0.383782in}}%
\pgfpathlineto{\pgfqpoint{4.087973in}{0.495025in}}%
\pgfpathlineto{\pgfqpoint{4.106776in}{0.640000in}}%
\pgfpathlineto{\pgfqpoint{4.125579in}{0.808827in}}%
\pgfpathlineto{\pgfqpoint{4.163184in}{1.171173in}}%
\pgfpathlineto{\pgfqpoint{4.181987in}{1.340000in}}%
\pgfpathlineto{\pgfqpoint{4.200790in}{1.484975in}}%
\pgfpathlineto{\pgfqpoint{4.219593in}{1.596218in}}%
\pgfpathlineto{\pgfqpoint{4.238395in}{1.666148in}}%
\pgfpathlineto{\pgfqpoint{4.257198in}{1.690000in}}%
\pgfpathlineto{\pgfqpoint{4.276001in}{1.666148in}}%
\pgfpathlineto{\pgfqpoint{4.294804in}{1.596218in}}%
\pgfpathlineto{\pgfqpoint{4.313607in}{1.484975in}}%
\pgfpathlineto{\pgfqpoint{4.332409in}{1.340000in}}%
\pgfpathlineto{\pgfqpoint{4.351212in}{1.171173in}}%
\pgfpathlineto{\pgfqpoint{4.388818in}{0.808827in}}%
\pgfpathlineto{\pgfqpoint{4.407621in}{0.640000in}}%
\pgfpathlineto{\pgfqpoint{4.426423in}{0.495025in}}%
\pgfpathlineto{\pgfqpoint{4.445226in}{0.383782in}}%
\pgfpathlineto{\pgfqpoint{4.464029in}{0.313852in}}%
\pgfpathlineto{\pgfqpoint{4.482832in}{0.290000in}}%
\pgfpathlineto{\pgfqpoint{4.501635in}{0.313852in}}%
\pgfpathlineto{\pgfqpoint{4.520438in}{0.383782in}}%
\pgfpathlineto{\pgfqpoint{4.539240in}{0.495025in}}%
\pgfpathlineto{\pgfqpoint{4.558043in}{0.640000in}}%
\pgfpathlineto{\pgfqpoint{4.576846in}{0.808827in}}%
\pgfpathlineto{\pgfqpoint{4.614452in}{1.171173in}}%
\pgfpathlineto{\pgfqpoint{4.633254in}{1.340000in}}%
\pgfpathlineto{\pgfqpoint{4.652057in}{1.484975in}}%
\pgfpathlineto{\pgfqpoint{4.670860in}{1.596218in}}%
\pgfpathlineto{\pgfqpoint{4.689663in}{1.666148in}}%
\pgfpathlineto{\pgfqpoint{4.708466in}{1.690000in}}%
\pgfpathlineto{\pgfqpoint{4.727268in}{1.666148in}}%
\pgfpathlineto{\pgfqpoint{4.746071in}{1.596218in}}%
\pgfpathlineto{\pgfqpoint{4.764874in}{1.484975in}}%
\pgfpathlineto{\pgfqpoint{4.783677in}{1.340000in}}%
\pgfpathlineto{\pgfqpoint{4.802480in}{1.171173in}}%
\pgfpathlineto{\pgfqpoint{4.840085in}{0.808827in}}%
\pgfpathlineto{\pgfqpoint{4.858888in}{0.640000in}}%
\pgfpathlineto{\pgfqpoint{4.877691in}{0.495025in}}%
\pgfpathlineto{\pgfqpoint{4.896494in}{0.383782in}}%
\pgfpathlineto{\pgfqpoint{4.915296in}{0.313852in}}%
\pgfpathlineto{\pgfqpoint{4.934099in}{0.290000in}}%
\pgfpathlineto{\pgfqpoint{4.952902in}{0.313852in}}%
\pgfpathlineto{\pgfqpoint{4.971705in}{0.383782in}}%
\pgfpathlineto{\pgfqpoint{4.990508in}{0.495025in}}%
\pgfpathlineto{\pgfqpoint{5.009311in}{0.640000in}}%
\pgfpathlineto{\pgfqpoint{5.028113in}{0.808827in}}%
\pgfpathlineto{\pgfqpoint{5.065719in}{1.171173in}}%
\pgfpathlineto{\pgfqpoint{5.084522in}{1.340000in}}%
\pgfpathlineto{\pgfqpoint{5.103325in}{1.484975in}}%
\pgfpathlineto{\pgfqpoint{5.122127in}{1.596218in}}%
\pgfpathlineto{\pgfqpoint{5.140930in}{1.666148in}}%
\pgfpathlineto{\pgfqpoint{5.159733in}{1.690000in}}%
\pgfpathlineto{\pgfqpoint{5.178536in}{1.666148in}}%
\pgfpathlineto{\pgfqpoint{5.197339in}{1.596218in}}%
\pgfpathlineto{\pgfqpoint{5.216141in}{1.484975in}}%
\pgfpathlineto{\pgfqpoint{5.234944in}{1.340000in}}%
\pgfpathlineto{\pgfqpoint{5.253747in}{1.171173in}}%
\pgfpathlineto{\pgfqpoint{5.291353in}{0.808827in}}%
\pgfpathlineto{\pgfqpoint{5.310155in}{0.640000in}}%
\pgfpathlineto{\pgfqpoint{5.328958in}{0.495025in}}%
\pgfpathlineto{\pgfqpoint{5.347761in}{0.383782in}}%
\pgfpathlineto{\pgfqpoint{5.366564in}{0.313852in}}%
\pgfpathlineto{\pgfqpoint{5.385367in}{0.290000in}}%
\pgfpathlineto{\pgfqpoint{5.404169in}{0.313852in}}%
\pgfpathlineto{\pgfqpoint{5.422972in}{0.383782in}}%
\pgfpathlineto{\pgfqpoint{5.441775in}{0.495025in}}%
\pgfpathlineto{\pgfqpoint{5.460578in}{0.640000in}}%
\pgfpathlineto{\pgfqpoint{5.479381in}{0.808827in}}%
\pgfpathlineto{\pgfqpoint{5.516986in}{1.171173in}}%
\pgfpathlineto{\pgfqpoint{5.535789in}{1.340000in}}%
\pgfpathlineto{\pgfqpoint{5.554592in}{1.484975in}}%
\pgfpathlineto{\pgfqpoint{5.573395in}{1.596218in}}%
\pgfpathlineto{\pgfqpoint{5.592198in}{1.666148in}}%
\pgfpathlineto{\pgfqpoint{5.611000in}{1.690000in}}%
\pgfpathlineto{\pgfqpoint{5.629803in}{1.666148in}}%
\pgfpathlineto{\pgfqpoint{5.648606in}{1.596218in}}%
\pgfpathlineto{\pgfqpoint{5.667409in}{1.484975in}}%
\pgfpathlineto{\pgfqpoint{5.686212in}{1.340000in}}%
\pgfpathlineto{\pgfqpoint{5.705014in}{1.171173in}}%
\pgfpathlineto{\pgfqpoint{5.742620in}{0.808827in}}%
\pgfpathlineto{\pgfqpoint{5.761423in}{0.640000in}}%
\pgfpathlineto{\pgfqpoint{5.780226in}{0.495025in}}%
\pgfpathlineto{\pgfqpoint{5.799028in}{0.383782in}}%
\pgfpathlineto{\pgfqpoint{5.817831in}{0.313852in}}%
\pgfpathlineto{\pgfqpoint{5.836634in}{0.290000in}}%
\pgfpathlineto{\pgfqpoint{5.855437in}{0.313852in}}%
\pgfpathlineto{\pgfqpoint{5.874240in}{0.383782in}}%
\pgfpathlineto{\pgfqpoint{5.893042in}{0.495025in}}%
\pgfpathlineto{\pgfqpoint{5.911845in}{0.640000in}}%
\pgfpathlineto{\pgfqpoint{5.930648in}{0.808827in}}%
\pgfpathlineto{\pgfqpoint{5.968254in}{1.171173in}}%
\pgfpathlineto{\pgfqpoint{5.987056in}{1.340000in}}%
\pgfpathlineto{\pgfqpoint{6.005859in}{1.484975in}}%
\pgfpathlineto{\pgfqpoint{6.024662in}{1.596218in}}%
\pgfpathlineto{\pgfqpoint{6.043465in}{1.666148in}}%
\pgfpathlineto{\pgfqpoint{6.062268in}{1.690000in}}%
\pgfpathlineto{\pgfqpoint{6.081071in}{1.666148in}}%
\pgfpathlineto{\pgfqpoint{6.099873in}{1.596218in}}%
\pgfpathlineto{\pgfqpoint{6.118676in}{1.484975in}}%
\pgfpathlineto{\pgfqpoint{6.137479in}{1.340000in}}%
\pgfpathlineto{\pgfqpoint{6.156282in}{1.171173in}}%
\pgfpathlineto{\pgfqpoint{6.193887in}{0.808827in}}%
\pgfpathlineto{\pgfqpoint{6.212690in}{0.640000in}}%
\pgfpathlineto{\pgfqpoint{6.231493in}{0.495025in}}%
\pgfpathlineto{\pgfqpoint{6.250296in}{0.383782in}}%
\pgfpathlineto{\pgfqpoint{6.269099in}{0.313852in}}%
\pgfpathlineto{\pgfqpoint{6.287901in}{0.290000in}}%
\pgfpathlineto{\pgfqpoint{6.306704in}{0.313852in}}%
\pgfpathlineto{\pgfqpoint{6.325507in}{0.383782in}}%
\pgfpathlineto{\pgfqpoint{6.344310in}{0.495025in}}%
\pgfpathlineto{\pgfqpoint{6.363113in}{0.640000in}}%
\pgfpathlineto{\pgfqpoint{6.381915in}{0.808827in}}%
\pgfpathlineto{\pgfqpoint{6.419521in}{1.171173in}}%
\pgfpathlineto{\pgfqpoint{6.438324in}{1.340000in}}%
\pgfpathlineto{\pgfqpoint{6.455845in}{1.475093in}}%
\pgfpathlineto{\pgfqpoint{6.455845in}{1.475093in}}%
\pgfusepath{stroke}%
\end{pgfscope}%
\begin{pgfscope}%
\pgfpathrectangle{\pgfqpoint{0.895256in}{0.220000in}}{\pgfqpoint{5.550589in}{1.540000in}}%
\pgfusepath{clip}%
\pgfsetrectcap%
\pgfsetroundjoin%
\pgfsetlinewidth{1.505625pt}%
\definecolor{currentstroke}{rgb}{1.000000,0.498039,0.054902}%
\pgfsetstrokecolor{currentstroke}%
\pgfsetdash{}{0pt}%
\pgfpathmoveto{\pgfqpoint{0.891496in}{0.897181in}}%
\pgfpathlineto{\pgfqpoint{0.910298in}{0.904017in}}%
\pgfpathlineto{\pgfqpoint{0.929101in}{0.920807in}}%
\pgfpathlineto{\pgfqpoint{0.985510in}{0.990555in}}%
\pgfpathlineto{\pgfqpoint{1.023115in}{1.030353in}}%
\pgfpathlineto{\pgfqpoint{1.041918in}{1.049109in}}%
\pgfpathlineto{\pgfqpoint{1.060721in}{1.065879in}}%
\pgfpathlineto{\pgfqpoint{1.079524in}{1.076710in}}%
\pgfpathlineto{\pgfqpoint{1.098327in}{1.079978in}}%
\pgfpathlineto{\pgfqpoint{1.117129in}{1.076795in}}%
\pgfpathlineto{\pgfqpoint{1.135932in}{1.068656in}}%
\pgfpathlineto{\pgfqpoint{1.154735in}{1.057227in}}%
\pgfpathlineto{\pgfqpoint{1.173538in}{1.042295in}}%
\pgfpathlineto{\pgfqpoint{1.192341in}{1.022428in}}%
\pgfpathlineto{\pgfqpoint{1.211143in}{0.998310in}}%
\pgfpathlineto{\pgfqpoint{1.248749in}{0.947916in}}%
\pgfpathlineto{\pgfqpoint{1.267552in}{0.928135in}}%
\pgfpathlineto{\pgfqpoint{1.286355in}{0.914185in}}%
\pgfpathlineto{\pgfqpoint{1.305157in}{0.905790in}}%
\pgfpathlineto{\pgfqpoint{1.323960in}{0.901646in}}%
\pgfpathlineto{\pgfqpoint{1.342763in}{0.901710in}}%
\pgfpathlineto{\pgfqpoint{1.361566in}{0.907328in}}%
\pgfpathlineto{\pgfqpoint{1.380369in}{0.919034in}}%
\pgfpathlineto{\pgfqpoint{1.399171in}{0.936274in}}%
\pgfpathlineto{\pgfqpoint{1.417974in}{0.958341in}}%
\pgfpathlineto{\pgfqpoint{1.455580in}{1.006470in}}%
\pgfpathlineto{\pgfqpoint{1.474383in}{1.027790in}}%
\pgfpathlineto{\pgfqpoint{1.493186in}{1.046802in}}%
\pgfpathlineto{\pgfqpoint{1.511988in}{1.063166in}}%
\pgfpathlineto{\pgfqpoint{1.530791in}{1.075385in}}%
\pgfpathlineto{\pgfqpoint{1.549594in}{1.081537in}}%
\pgfpathlineto{\pgfqpoint{1.568397in}{1.079935in}}%
\pgfpathlineto{\pgfqpoint{1.587200in}{1.070750in}}%
\pgfpathlineto{\pgfqpoint{1.606002in}{1.056629in}}%
\pgfpathlineto{\pgfqpoint{1.624805in}{1.040479in}}%
\pgfpathlineto{\pgfqpoint{1.643608in}{1.022535in}}%
\pgfpathlineto{\pgfqpoint{1.662411in}{1.001386in}}%
\pgfpathlineto{\pgfqpoint{1.700016in}{0.953470in}}%
\pgfpathlineto{\pgfqpoint{1.718819in}{0.931937in}}%
\pgfpathlineto{\pgfqpoint{1.737622in}{0.915253in}}%
\pgfpathlineto{\pgfqpoint{1.756425in}{0.906281in}}%
\pgfpathlineto{\pgfqpoint{1.775228in}{0.905298in}}%
\pgfpathlineto{\pgfqpoint{1.812833in}{0.911622in}}%
\pgfpathlineto{\pgfqpoint{1.831636in}{0.917112in}}%
\pgfpathlineto{\pgfqpoint{1.850439in}{0.929288in}}%
\pgfpathlineto{\pgfqpoint{1.869242in}{0.950117in}}%
\pgfpathlineto{\pgfqpoint{1.906847in}{1.004206in}}%
\pgfpathlineto{\pgfqpoint{1.925650in}{1.026594in}}%
\pgfpathlineto{\pgfqpoint{1.944453in}{1.043214in}}%
\pgfpathlineto{\pgfqpoint{1.963256in}{1.055561in}}%
\pgfpathlineto{\pgfqpoint{1.982059in}{1.064362in}}%
\pgfpathlineto{\pgfqpoint{2.000861in}{1.070771in}}%
\pgfpathlineto{\pgfqpoint{2.019664in}{1.075107in}}%
\pgfpathlineto{\pgfqpoint{2.038467in}{1.073997in}}%
\pgfpathlineto{\pgfqpoint{2.057270in}{1.064149in}}%
\pgfpathlineto{\pgfqpoint{2.076073in}{1.046546in}}%
\pgfpathlineto{\pgfqpoint{2.151284in}{0.960969in}}%
\pgfpathlineto{\pgfqpoint{2.170087in}{0.940845in}}%
\pgfpathlineto{\pgfqpoint{2.188889in}{0.923115in}}%
\pgfpathlineto{\pgfqpoint{2.207692in}{0.909742in}}%
\pgfpathlineto{\pgfqpoint{2.226495in}{0.902564in}}%
\pgfpathlineto{\pgfqpoint{2.245298in}{0.901218in}}%
\pgfpathlineto{\pgfqpoint{2.264101in}{0.904508in}}%
\pgfpathlineto{\pgfqpoint{2.282903in}{0.912583in}}%
\pgfpathlineto{\pgfqpoint{2.301706in}{0.926234in}}%
\pgfpathlineto{\pgfqpoint{2.320509in}{0.944797in}}%
\pgfpathlineto{\pgfqpoint{2.339312in}{0.966950in}}%
\pgfpathlineto{\pgfqpoint{2.376917in}{1.015143in}}%
\pgfpathlineto{\pgfqpoint{2.395720in}{1.036292in}}%
\pgfpathlineto{\pgfqpoint{2.414523in}{1.053831in}}%
\pgfpathlineto{\pgfqpoint{2.433326in}{1.066819in}}%
\pgfpathlineto{\pgfqpoint{2.452129in}{1.075086in}}%
\pgfpathlineto{\pgfqpoint{2.470932in}{1.078782in}}%
\pgfpathlineto{\pgfqpoint{2.489734in}{1.077329in}}%
\pgfpathlineto{\pgfqpoint{2.508537in}{1.069190in}}%
\pgfpathlineto{\pgfqpoint{2.527340in}{1.053596in}}%
\pgfpathlineto{\pgfqpoint{2.564946in}{1.011683in}}%
\pgfpathlineto{\pgfqpoint{2.640157in}{0.927131in}}%
\pgfpathlineto{\pgfqpoint{2.658960in}{0.911643in}}%
\pgfpathlineto{\pgfqpoint{2.677762in}{0.902479in}}%
\pgfpathlineto{\pgfqpoint{2.696565in}{0.900812in}}%
\pgfpathlineto{\pgfqpoint{2.715368in}{0.905469in}}%
\pgfpathlineto{\pgfqpoint{2.734171in}{0.913779in}}%
\pgfpathlineto{\pgfqpoint{2.752974in}{0.925251in}}%
\pgfpathlineto{\pgfqpoint{2.771776in}{0.940589in}}%
\pgfpathlineto{\pgfqpoint{2.790579in}{0.959473in}}%
\pgfpathlineto{\pgfqpoint{2.809382in}{0.982545in}}%
\pgfpathlineto{\pgfqpoint{2.846988in}{1.033515in}}%
\pgfpathlineto{\pgfqpoint{2.865790in}{1.053831in}}%
\pgfpathlineto{\pgfqpoint{2.884593in}{1.067225in}}%
\pgfpathlineto{\pgfqpoint{2.903396in}{1.073420in}}%
\pgfpathlineto{\pgfqpoint{2.922199in}{1.074424in}}%
\pgfpathlineto{\pgfqpoint{2.941002in}{1.072693in}}%
\pgfpathlineto{\pgfqpoint{2.959805in}{1.068976in}}%
\pgfpathlineto{\pgfqpoint{2.978607in}{1.060816in}}%
\pgfpathlineto{\pgfqpoint{2.997410in}{1.044602in}}%
\pgfpathlineto{\pgfqpoint{3.016213in}{1.021445in}}%
\pgfpathlineto{\pgfqpoint{3.053819in}{0.972312in}}%
\pgfpathlineto{\pgfqpoint{3.072621in}{0.951890in}}%
\pgfpathlineto{\pgfqpoint{3.091424in}{0.934437in}}%
\pgfpathlineto{\pgfqpoint{3.110227in}{0.919077in}}%
\pgfpathlineto{\pgfqpoint{3.129030in}{0.907969in}}%
\pgfpathlineto{\pgfqpoint{3.147833in}{0.902906in}}%
\pgfpathlineto{\pgfqpoint{3.166635in}{0.903034in}}%
\pgfpathlineto{\pgfqpoint{3.185438in}{0.907307in}}%
\pgfpathlineto{\pgfqpoint{3.204241in}{0.917325in}}%
\pgfpathlineto{\pgfqpoint{3.223044in}{0.935035in}}%
\pgfpathlineto{\pgfqpoint{3.279452in}{1.005295in}}%
\pgfpathlineto{\pgfqpoint{3.298255in}{1.025440in}}%
\pgfpathlineto{\pgfqpoint{3.317058in}{1.043278in}}%
\pgfpathlineto{\pgfqpoint{3.335861in}{1.058658in}}%
\pgfpathlineto{\pgfqpoint{3.354663in}{1.071177in}}%
\pgfpathlineto{\pgfqpoint{3.373466in}{1.079294in}}%
\pgfpathlineto{\pgfqpoint{3.392269in}{1.080982in}}%
\pgfpathlineto{\pgfqpoint{3.411072in}{1.076047in}}%
\pgfpathlineto{\pgfqpoint{3.429875in}{1.064854in}}%
\pgfpathlineto{\pgfqpoint{3.448677in}{1.047208in}}%
\pgfpathlineto{\pgfqpoint{3.542692in}{0.936979in}}%
\pgfpathlineto{\pgfqpoint{3.561494in}{0.920338in}}%
\pgfpathlineto{\pgfqpoint{3.580297in}{0.909336in}}%
\pgfpathlineto{\pgfqpoint{3.599100in}{0.903269in}}%
\pgfpathlineto{\pgfqpoint{3.617903in}{0.901539in}}%
\pgfpathlineto{\pgfqpoint{3.636706in}{0.904508in}}%
\pgfpathlineto{\pgfqpoint{3.655508in}{0.914249in}}%
\pgfpathlineto{\pgfqpoint{3.674311in}{0.931873in}}%
\pgfpathlineto{\pgfqpoint{3.730720in}{0.997156in}}%
\pgfpathlineto{\pgfqpoint{3.768325in}{1.038492in}}%
\pgfpathlineto{\pgfqpoint{3.787128in}{1.056095in}}%
\pgfpathlineto{\pgfqpoint{3.805931in}{1.069340in}}%
\pgfpathlineto{\pgfqpoint{3.824734in}{1.078376in}}%
\pgfpathlineto{\pgfqpoint{3.843536in}{1.081367in}}%
\pgfpathlineto{\pgfqpoint{3.862339in}{1.076774in}}%
\pgfpathlineto{\pgfqpoint{3.881142in}{1.065558in}}%
\pgfpathlineto{\pgfqpoint{3.899945in}{1.049900in}}%
\pgfpathlineto{\pgfqpoint{3.918748in}{1.032062in}}%
\pgfpathlineto{\pgfqpoint{3.937550in}{1.012388in}}%
\pgfpathlineto{\pgfqpoint{3.956353in}{0.990235in}}%
\pgfpathlineto{\pgfqpoint{3.993959in}{0.942896in}}%
\pgfpathlineto{\pgfqpoint{4.012762in}{0.923905in}}%
\pgfpathlineto{\pgfqpoint{4.031565in}{0.911301in}}%
\pgfpathlineto{\pgfqpoint{4.050367in}{0.905128in}}%
\pgfpathlineto{\pgfqpoint{4.069170in}{0.904230in}}%
\pgfpathlineto{\pgfqpoint{4.087973in}{0.907477in}}%
\pgfpathlineto{\pgfqpoint{4.106776in}{0.914121in}}%
\pgfpathlineto{\pgfqpoint{4.125579in}{0.925614in}}%
\pgfpathlineto{\pgfqpoint{4.144381in}{0.943131in}}%
\pgfpathlineto{\pgfqpoint{4.163184in}{0.965433in}}%
\pgfpathlineto{\pgfqpoint{4.200790in}{1.013798in}}%
\pgfpathlineto{\pgfqpoint{4.219593in}{1.035331in}}%
\pgfpathlineto{\pgfqpoint{4.238395in}{1.052570in}}%
\pgfpathlineto{\pgfqpoint{4.257198in}{1.064469in}}%
\pgfpathlineto{\pgfqpoint{4.276001in}{1.072138in}}%
\pgfpathlineto{\pgfqpoint{4.294804in}{1.076774in}}%
\pgfpathlineto{\pgfqpoint{4.313607in}{1.077158in}}%
\pgfpathlineto{\pgfqpoint{4.332409in}{1.071070in}}%
\pgfpathlineto{\pgfqpoint{4.351212in}{1.057654in}}%
\pgfpathlineto{\pgfqpoint{4.370015in}{1.038792in}}%
\pgfpathlineto{\pgfqpoint{4.426423in}{0.973295in}}%
\pgfpathlineto{\pgfqpoint{4.464029in}{0.930549in}}%
\pgfpathlineto{\pgfqpoint{4.482832in}{0.914805in}}%
\pgfpathlineto{\pgfqpoint{4.501635in}{0.905598in}}%
\pgfpathlineto{\pgfqpoint{4.520438in}{0.902543in}}%
\pgfpathlineto{\pgfqpoint{4.539240in}{0.904615in}}%
\pgfpathlineto{\pgfqpoint{4.558043in}{0.910874in}}%
\pgfpathlineto{\pgfqpoint{4.576846in}{0.921448in}}%
\pgfpathlineto{\pgfqpoint{4.595649in}{0.936808in}}%
\pgfpathlineto{\pgfqpoint{4.614452in}{0.957316in}}%
\pgfpathlineto{\pgfqpoint{4.670860in}{1.028431in}}%
\pgfpathlineto{\pgfqpoint{4.689663in}{1.046674in}}%
\pgfpathlineto{\pgfqpoint{4.708466in}{1.059897in}}%
\pgfpathlineto{\pgfqpoint{4.727268in}{1.069190in}}%
\pgfpathlineto{\pgfqpoint{4.746071in}{1.074872in}}%
\pgfpathlineto{\pgfqpoint{4.764874in}{1.076304in}}%
\pgfpathlineto{\pgfqpoint{4.783677in}{1.072587in}}%
\pgfpathlineto{\pgfqpoint{4.802480in}{1.062760in}}%
\pgfpathlineto{\pgfqpoint{4.821282in}{1.046290in}}%
\pgfpathlineto{\pgfqpoint{4.840085in}{1.024201in}}%
\pgfpathlineto{\pgfqpoint{4.877691in}{0.976435in}}%
\pgfpathlineto{\pgfqpoint{4.896494in}{0.955564in}}%
\pgfpathlineto{\pgfqpoint{4.915296in}{0.937107in}}%
\pgfpathlineto{\pgfqpoint{4.934099in}{0.921662in}}%
\pgfpathlineto{\pgfqpoint{4.952902in}{0.909550in}}%
\pgfpathlineto{\pgfqpoint{4.971705in}{0.901581in}}%
\pgfpathlineto{\pgfqpoint{4.990508in}{0.899018in}}%
\pgfpathlineto{\pgfqpoint{5.009311in}{0.903141in}}%
\pgfpathlineto{\pgfqpoint{5.028113in}{0.914420in}}%
\pgfpathlineto{\pgfqpoint{5.046916in}{0.931937in}}%
\pgfpathlineto{\pgfqpoint{5.103325in}{0.997541in}}%
\pgfpathlineto{\pgfqpoint{5.159733in}{1.055646in}}%
\pgfpathlineto{\pgfqpoint{5.178536in}{1.070045in}}%
\pgfpathlineto{\pgfqpoint{5.197339in}{1.077607in}}%
\pgfpathlineto{\pgfqpoint{5.216141in}{1.078205in}}%
\pgfpathlineto{\pgfqpoint{5.234944in}{1.073270in}}%
\pgfpathlineto{\pgfqpoint{5.253747in}{1.063871in}}%
\pgfpathlineto{\pgfqpoint{5.272550in}{1.049622in}}%
\pgfpathlineto{\pgfqpoint{5.291353in}{1.030866in}}%
\pgfpathlineto{\pgfqpoint{5.328958in}{0.986838in}}%
\pgfpathlineto{\pgfqpoint{5.366564in}{0.943473in}}%
\pgfpathlineto{\pgfqpoint{5.385367in}{0.924866in}}%
\pgfpathlineto{\pgfqpoint{5.404169in}{0.911173in}}%
\pgfpathlineto{\pgfqpoint{5.422972in}{0.903803in}}%
\pgfpathlineto{\pgfqpoint{5.441775in}{0.901688in}}%
\pgfpathlineto{\pgfqpoint{5.460578in}{0.903504in}}%
\pgfpathlineto{\pgfqpoint{5.479381in}{0.910639in}}%
\pgfpathlineto{\pgfqpoint{5.498184in}{0.924866in}}%
\pgfpathlineto{\pgfqpoint{5.516986in}{0.944883in}}%
\pgfpathlineto{\pgfqpoint{5.573395in}{1.012217in}}%
\pgfpathlineto{\pgfqpoint{5.592198in}{1.032212in}}%
\pgfpathlineto{\pgfqpoint{5.611000in}{1.049857in}}%
\pgfpathlineto{\pgfqpoint{5.629803in}{1.064597in}}%
\pgfpathlineto{\pgfqpoint{5.648606in}{1.074851in}}%
\pgfpathlineto{\pgfqpoint{5.667409in}{1.078953in}}%
\pgfpathlineto{\pgfqpoint{5.686212in}{1.076859in}}%
\pgfpathlineto{\pgfqpoint{5.705014in}{1.068613in}}%
\pgfpathlineto{\pgfqpoint{5.723817in}{1.054386in}}%
\pgfpathlineto{\pgfqpoint{5.742620in}{1.035844in}}%
\pgfpathlineto{\pgfqpoint{5.780226in}{0.994486in}}%
\pgfpathlineto{\pgfqpoint{5.817831in}{0.948472in}}%
\pgfpathlineto{\pgfqpoint{5.836634in}{0.929267in}}%
\pgfpathlineto{\pgfqpoint{5.855437in}{0.915745in}}%
\pgfpathlineto{\pgfqpoint{5.874240in}{0.907477in}}%
\pgfpathlineto{\pgfqpoint{5.893042in}{0.903632in}}%
\pgfpathlineto{\pgfqpoint{5.911845in}{0.904230in}}%
\pgfpathlineto{\pgfqpoint{5.930648in}{0.909635in}}%
\pgfpathlineto{\pgfqpoint{5.949451in}{0.920615in}}%
\pgfpathlineto{\pgfqpoint{5.968254in}{0.937705in}}%
\pgfpathlineto{\pgfqpoint{5.987056in}{0.960328in}}%
\pgfpathlineto{\pgfqpoint{6.024662in}{1.009952in}}%
\pgfpathlineto{\pgfqpoint{6.043465in}{1.029969in}}%
\pgfpathlineto{\pgfqpoint{6.062268in}{1.046076in}}%
\pgfpathlineto{\pgfqpoint{6.081071in}{1.059385in}}%
\pgfpathlineto{\pgfqpoint{6.099873in}{1.070344in}}%
\pgfpathlineto{\pgfqpoint{6.118676in}{1.077970in}}%
\pgfpathlineto{\pgfqpoint{6.137479in}{1.080063in}}%
\pgfpathlineto{\pgfqpoint{6.156282in}{1.074232in}}%
\pgfpathlineto{\pgfqpoint{6.175085in}{1.060602in}}%
\pgfpathlineto{\pgfqpoint{6.193887in}{1.041868in}}%
\pgfpathlineto{\pgfqpoint{6.231493in}{0.999314in}}%
\pgfpathlineto{\pgfqpoint{6.269099in}{0.956696in}}%
\pgfpathlineto{\pgfqpoint{6.287901in}{0.937342in}}%
\pgfpathlineto{\pgfqpoint{6.306704in}{0.920807in}}%
\pgfpathlineto{\pgfqpoint{6.325507in}{0.908738in}}%
\pgfpathlineto{\pgfqpoint{6.344310in}{0.902051in}}%
\pgfpathlineto{\pgfqpoint{6.363113in}{0.901325in}}%
\pgfpathlineto{\pgfqpoint{6.381915in}{0.906879in}}%
\pgfpathlineto{\pgfqpoint{6.400718in}{0.918586in}}%
\pgfpathlineto{\pgfqpoint{6.419521in}{0.935270in}}%
\pgfpathlineto{\pgfqpoint{6.455845in}{0.974195in}}%
\pgfpathlineto{\pgfqpoint{6.455845in}{0.974195in}}%
\pgfusepath{stroke}%
\end{pgfscope}%
\begin{pgfscope}%
\pgfpathrectangle{\pgfqpoint{0.895256in}{0.220000in}}{\pgfqpoint{5.550589in}{1.540000in}}%
\pgfusepath{clip}%
\pgfsetbuttcap%
\pgfsetroundjoin%
\pgfsetlinewidth{1.505625pt}%
\definecolor{currentstroke}{rgb}{0.501961,0.501961,0.501961}%
\pgfsetstrokecolor{currentstroke}%
\pgfsetdash{{5.550000pt}{2.400000pt}}{0.000000pt}%
\pgfusepath{stroke}%
\end{pgfscope}%
\begin{pgfscope}%
\pgfpathrectangle{\pgfqpoint{0.895256in}{0.220000in}}{\pgfqpoint{5.550589in}{1.540000in}}%
\pgfusepath{clip}%
\pgfsetbuttcap%
\pgfsetroundjoin%
\pgfsetlinewidth{1.505625pt}%
\definecolor{currentstroke}{rgb}{0.501961,0.501961,0.501961}%
\pgfsetstrokecolor{currentstroke}%
\pgfsetdash{{5.550000pt}{2.400000pt}}{0.000000pt}%
\pgfusepath{stroke}%
\end{pgfscope}%
\begin{pgfscope}%
\pgfpathrectangle{\pgfqpoint{0.895256in}{0.220000in}}{\pgfqpoint{5.550589in}{1.540000in}}%
\pgfusepath{clip}%
\pgfsetbuttcap%
\pgfsetroundjoin%
\pgfsetlinewidth{1.505625pt}%
\definecolor{currentstroke}{rgb}{0.501961,0.501961,0.501961}%
\pgfsetstrokecolor{currentstroke}%
\pgfsetdash{{5.550000pt}{2.400000pt}}{0.000000pt}%
\pgfusepath{stroke}%
\end{pgfscope}%
\begin{pgfscope}%
\pgfpathrectangle{\pgfqpoint{0.895256in}{0.220000in}}{\pgfqpoint{5.550589in}{1.540000in}}%
\pgfusepath{clip}%
\pgfsetbuttcap%
\pgfsetroundjoin%
\pgfsetlinewidth{1.505625pt}%
\definecolor{currentstroke}{rgb}{0.501961,0.501961,0.501961}%
\pgfsetstrokecolor{currentstroke}%
\pgfsetdash{{5.550000pt}{2.400000pt}}{0.000000pt}%
\pgfusepath{stroke}%
\end{pgfscope}%
\begin{pgfscope}%
\pgfpathrectangle{\pgfqpoint{0.895256in}{0.220000in}}{\pgfqpoint{5.550589in}{1.540000in}}%
\pgfusepath{clip}%
\pgfsetbuttcap%
\pgfsetroundjoin%
\pgfsetlinewidth{1.505625pt}%
\definecolor{currentstroke}{rgb}{0.501961,0.501961,0.501961}%
\pgfsetstrokecolor{currentstroke}%
\pgfsetdash{{5.550000pt}{2.400000pt}}{0.000000pt}%
\pgfusepath{stroke}%
\end{pgfscope}%
\begin{pgfscope}%
\pgfpathrectangle{\pgfqpoint{0.895256in}{0.220000in}}{\pgfqpoint{5.550589in}{1.540000in}}%
\pgfusepath{clip}%
\pgfsetbuttcap%
\pgfsetroundjoin%
\pgfsetlinewidth{1.505625pt}%
\definecolor{currentstroke}{rgb}{0.501961,0.501961,0.501961}%
\pgfsetstrokecolor{currentstroke}%
\pgfsetdash{{5.550000pt}{2.400000pt}}{0.000000pt}%
\pgfusepath{stroke}%
\end{pgfscope}%
\begin{pgfscope}%
\pgfpathrectangle{\pgfqpoint{0.895256in}{0.220000in}}{\pgfqpoint{5.550589in}{1.540000in}}%
\pgfusepath{clip}%
\pgfsetbuttcap%
\pgfsetroundjoin%
\pgfsetlinewidth{1.505625pt}%
\definecolor{currentstroke}{rgb}{0.501961,0.501961,0.501961}%
\pgfsetstrokecolor{currentstroke}%
\pgfsetdash{{5.550000pt}{2.400000pt}}{0.000000pt}%
\pgfusepath{stroke}%
\end{pgfscope}%
\begin{pgfscope}%
\pgfpathrectangle{\pgfqpoint{0.895256in}{0.220000in}}{\pgfqpoint{5.550589in}{1.540000in}}%
\pgfusepath{clip}%
\pgfsetbuttcap%
\pgfsetroundjoin%
\pgfsetlinewidth{1.505625pt}%
\definecolor{currentstroke}{rgb}{0.501961,0.501961,0.501961}%
\pgfsetstrokecolor{currentstroke}%
\pgfsetdash{{5.550000pt}{2.400000pt}}{0.000000pt}%
\pgfusepath{stroke}%
\end{pgfscope}%
\begin{pgfscope}%
\pgfpathrectangle{\pgfqpoint{0.895256in}{0.220000in}}{\pgfqpoint{5.550589in}{1.540000in}}%
\pgfusepath{clip}%
\pgfsetbuttcap%
\pgfsetroundjoin%
\pgfsetlinewidth{1.505625pt}%
\definecolor{currentstroke}{rgb}{0.501961,0.501961,0.501961}%
\pgfsetstrokecolor{currentstroke}%
\pgfsetdash{{5.550000pt}{2.400000pt}}{0.000000pt}%
\pgfusepath{stroke}%
\end{pgfscope}%
\begin{pgfscope}%
\pgfpathrectangle{\pgfqpoint{0.895256in}{0.220000in}}{\pgfqpoint{5.550589in}{1.540000in}}%
\pgfusepath{clip}%
\pgfsetbuttcap%
\pgfsetroundjoin%
\pgfsetlinewidth{1.505625pt}%
\definecolor{currentstroke}{rgb}{0.501961,0.501961,0.501961}%
\pgfsetstrokecolor{currentstroke}%
\pgfsetdash{{5.550000pt}{2.400000pt}}{0.000000pt}%
\pgfpathmoveto{\pgfqpoint{0.985510in}{0.220000in}}%
\pgfpathlineto{\pgfqpoint{0.985510in}{1.760000in}}%
\pgfusepath{stroke}%
\end{pgfscope}%
\begin{pgfscope}%
\pgfpathrectangle{\pgfqpoint{0.895256in}{0.220000in}}{\pgfqpoint{5.550589in}{1.540000in}}%
\pgfusepath{clip}%
\pgfsetbuttcap%
\pgfsetroundjoin%
\pgfsetlinewidth{1.505625pt}%
\definecolor{currentstroke}{rgb}{0.501961,0.501961,0.501961}%
\pgfsetstrokecolor{currentstroke}%
\pgfsetdash{{5.550000pt}{2.400000pt}}{0.000000pt}%
\pgfpathmoveto{\pgfqpoint{1.436777in}{0.220000in}}%
\pgfpathlineto{\pgfqpoint{1.436777in}{1.760000in}}%
\pgfusepath{stroke}%
\end{pgfscope}%
\begin{pgfscope}%
\pgfpathrectangle{\pgfqpoint{0.895256in}{0.220000in}}{\pgfqpoint{5.550589in}{1.540000in}}%
\pgfusepath{clip}%
\pgfsetbuttcap%
\pgfsetroundjoin%
\pgfsetlinewidth{1.505625pt}%
\definecolor{currentstroke}{rgb}{0.501961,0.501961,0.501961}%
\pgfsetstrokecolor{currentstroke}%
\pgfsetdash{{5.550000pt}{2.400000pt}}{0.000000pt}%
\pgfpathmoveto{\pgfqpoint{1.888044in}{0.220000in}}%
\pgfpathlineto{\pgfqpoint{1.888044in}{1.760000in}}%
\pgfusepath{stroke}%
\end{pgfscope}%
\begin{pgfscope}%
\pgfpathrectangle{\pgfqpoint{0.895256in}{0.220000in}}{\pgfqpoint{5.550589in}{1.540000in}}%
\pgfusepath{clip}%
\pgfsetbuttcap%
\pgfsetroundjoin%
\pgfsetlinewidth{1.505625pt}%
\definecolor{currentstroke}{rgb}{0.501961,0.501961,0.501961}%
\pgfsetstrokecolor{currentstroke}%
\pgfsetdash{{5.550000pt}{2.400000pt}}{0.000000pt}%
\pgfpathmoveto{\pgfqpoint{2.339312in}{0.220000in}}%
\pgfpathlineto{\pgfqpoint{2.339312in}{1.760000in}}%
\pgfusepath{stroke}%
\end{pgfscope}%
\begin{pgfscope}%
\pgfpathrectangle{\pgfqpoint{0.895256in}{0.220000in}}{\pgfqpoint{5.550589in}{1.540000in}}%
\pgfusepath{clip}%
\pgfsetbuttcap%
\pgfsetroundjoin%
\pgfsetlinewidth{1.505625pt}%
\definecolor{currentstroke}{rgb}{0.501961,0.501961,0.501961}%
\pgfsetstrokecolor{currentstroke}%
\pgfsetdash{{5.550000pt}{2.400000pt}}{0.000000pt}%
\pgfpathmoveto{\pgfqpoint{2.790579in}{0.220000in}}%
\pgfpathlineto{\pgfqpoint{2.790579in}{1.760000in}}%
\pgfusepath{stroke}%
\end{pgfscope}%
\begin{pgfscope}%
\pgfpathrectangle{\pgfqpoint{0.895256in}{0.220000in}}{\pgfqpoint{5.550589in}{1.540000in}}%
\pgfusepath{clip}%
\pgfsetbuttcap%
\pgfsetroundjoin%
\pgfsetlinewidth{1.505625pt}%
\definecolor{currentstroke}{rgb}{0.501961,0.501961,0.501961}%
\pgfsetstrokecolor{currentstroke}%
\pgfsetdash{{5.550000pt}{2.400000pt}}{0.000000pt}%
\pgfpathmoveto{\pgfqpoint{3.241847in}{0.220000in}}%
\pgfpathlineto{\pgfqpoint{3.241847in}{1.760000in}}%
\pgfusepath{stroke}%
\end{pgfscope}%
\begin{pgfscope}%
\pgfpathrectangle{\pgfqpoint{0.895256in}{0.220000in}}{\pgfqpoint{5.550589in}{1.540000in}}%
\pgfusepath{clip}%
\pgfsetbuttcap%
\pgfsetroundjoin%
\pgfsetlinewidth{1.505625pt}%
\definecolor{currentstroke}{rgb}{0.501961,0.501961,0.501961}%
\pgfsetstrokecolor{currentstroke}%
\pgfsetdash{{5.550000pt}{2.400000pt}}{0.000000pt}%
\pgfpathmoveto{\pgfqpoint{3.693114in}{0.220000in}}%
\pgfpathlineto{\pgfqpoint{3.693114in}{1.760000in}}%
\pgfusepath{stroke}%
\end{pgfscope}%
\begin{pgfscope}%
\pgfpathrectangle{\pgfqpoint{0.895256in}{0.220000in}}{\pgfqpoint{5.550589in}{1.540000in}}%
\pgfusepath{clip}%
\pgfsetbuttcap%
\pgfsetroundjoin%
\pgfsetlinewidth{1.505625pt}%
\definecolor{currentstroke}{rgb}{0.501961,0.501961,0.501961}%
\pgfsetstrokecolor{currentstroke}%
\pgfsetdash{{5.550000pt}{2.400000pt}}{0.000000pt}%
\pgfpathmoveto{\pgfqpoint{4.144381in}{0.220000in}}%
\pgfpathlineto{\pgfqpoint{4.144381in}{1.760000in}}%
\pgfusepath{stroke}%
\end{pgfscope}%
\begin{pgfscope}%
\pgfpathrectangle{\pgfqpoint{0.895256in}{0.220000in}}{\pgfqpoint{5.550589in}{1.540000in}}%
\pgfusepath{clip}%
\pgfsetbuttcap%
\pgfsetroundjoin%
\pgfsetlinewidth{1.505625pt}%
\definecolor{currentstroke}{rgb}{0.501961,0.501961,0.501961}%
\pgfsetstrokecolor{currentstroke}%
\pgfsetdash{{5.550000pt}{2.400000pt}}{0.000000pt}%
\pgfpathmoveto{\pgfqpoint{4.595649in}{0.220000in}}%
\pgfpathlineto{\pgfqpoint{4.595649in}{1.760000in}}%
\pgfusepath{stroke}%
\end{pgfscope}%
\begin{pgfscope}%
\pgfpathrectangle{\pgfqpoint{0.895256in}{0.220000in}}{\pgfqpoint{5.550589in}{1.540000in}}%
\pgfusepath{clip}%
\pgfsetbuttcap%
\pgfsetroundjoin%
\pgfsetlinewidth{1.505625pt}%
\definecolor{currentstroke}{rgb}{0.501961,0.501961,0.501961}%
\pgfsetstrokecolor{currentstroke}%
\pgfsetdash{{5.550000pt}{2.400000pt}}{0.000000pt}%
\pgfpathmoveto{\pgfqpoint{5.046916in}{0.220000in}}%
\pgfpathlineto{\pgfqpoint{5.046916in}{1.760000in}}%
\pgfusepath{stroke}%
\end{pgfscope}%
\begin{pgfscope}%
\pgfpathrectangle{\pgfqpoint{0.895256in}{0.220000in}}{\pgfqpoint{5.550589in}{1.540000in}}%
\pgfusepath{clip}%
\pgfsetbuttcap%
\pgfsetroundjoin%
\pgfsetlinewidth{1.505625pt}%
\definecolor{currentstroke}{rgb}{0.501961,0.501961,0.501961}%
\pgfsetstrokecolor{currentstroke}%
\pgfsetdash{{5.550000pt}{2.400000pt}}{0.000000pt}%
\pgfpathmoveto{\pgfqpoint{5.498184in}{0.220000in}}%
\pgfpathlineto{\pgfqpoint{5.498184in}{1.760000in}}%
\pgfusepath{stroke}%
\end{pgfscope}%
\begin{pgfscope}%
\pgfpathrectangle{\pgfqpoint{0.895256in}{0.220000in}}{\pgfqpoint{5.550589in}{1.540000in}}%
\pgfusepath{clip}%
\pgfsetbuttcap%
\pgfsetroundjoin%
\pgfsetlinewidth{1.505625pt}%
\definecolor{currentstroke}{rgb}{0.501961,0.501961,0.501961}%
\pgfsetstrokecolor{currentstroke}%
\pgfsetdash{{5.550000pt}{2.400000pt}}{0.000000pt}%
\pgfpathmoveto{\pgfqpoint{5.949451in}{0.220000in}}%
\pgfpathlineto{\pgfqpoint{5.949451in}{1.760000in}}%
\pgfusepath{stroke}%
\end{pgfscope}%
\begin{pgfscope}%
\pgfpathrectangle{\pgfqpoint{0.895256in}{0.220000in}}{\pgfqpoint{5.550589in}{1.540000in}}%
\pgfusepath{clip}%
\pgfsetbuttcap%
\pgfsetroundjoin%
\pgfsetlinewidth{1.505625pt}%
\definecolor{currentstroke}{rgb}{0.501961,0.501961,0.501961}%
\pgfsetstrokecolor{currentstroke}%
\pgfsetdash{{5.550000pt}{2.400000pt}}{0.000000pt}%
\pgfpathmoveto{\pgfqpoint{6.400718in}{0.220000in}}%
\pgfpathlineto{\pgfqpoint{6.400718in}{1.760000in}}%
\pgfusepath{stroke}%
\end{pgfscope}%
\begin{pgfscope}%
\pgfpathrectangle{\pgfqpoint{0.895256in}{0.220000in}}{\pgfqpoint{5.550589in}{1.540000in}}%
\pgfusepath{clip}%
\pgfsetbuttcap%
\pgfsetroundjoin%
\pgfsetlinewidth{1.505625pt}%
\definecolor{currentstroke}{rgb}{0.501961,0.501961,0.501961}%
\pgfsetstrokecolor{currentstroke}%
\pgfsetdash{{5.550000pt}{2.400000pt}}{0.000000pt}%
\pgfusepath{stroke}%
\end{pgfscope}%
\begin{pgfscope}%
\pgfpathrectangle{\pgfqpoint{0.895256in}{0.220000in}}{\pgfqpoint{5.550589in}{1.540000in}}%
\pgfusepath{clip}%
\pgfsetbuttcap%
\pgfsetroundjoin%
\pgfsetlinewidth{1.505625pt}%
\definecolor{currentstroke}{rgb}{0.501961,0.501961,0.501961}%
\pgfsetstrokecolor{currentstroke}%
\pgfsetdash{{5.550000pt}{2.400000pt}}{0.000000pt}%
\pgfusepath{stroke}%
\end{pgfscope}%
\begin{pgfscope}%
\pgfpathrectangle{\pgfqpoint{0.895256in}{0.220000in}}{\pgfqpoint{5.550589in}{1.540000in}}%
\pgfusepath{clip}%
\pgfsetbuttcap%
\pgfsetroundjoin%
\pgfsetlinewidth{1.505625pt}%
\definecolor{currentstroke}{rgb}{0.501961,0.501961,0.501961}%
\pgfsetstrokecolor{currentstroke}%
\pgfsetdash{{5.550000pt}{2.400000pt}}{0.000000pt}%
\pgfusepath{stroke}%
\end{pgfscope}%
\begin{pgfscope}%
\pgfpathrectangle{\pgfqpoint{0.895256in}{0.220000in}}{\pgfqpoint{5.550589in}{1.540000in}}%
\pgfusepath{clip}%
\pgfsetbuttcap%
\pgfsetroundjoin%
\pgfsetlinewidth{1.505625pt}%
\definecolor{currentstroke}{rgb}{0.501961,0.501961,0.501961}%
\pgfsetstrokecolor{currentstroke}%
\pgfsetdash{{5.550000pt}{2.400000pt}}{0.000000pt}%
\pgfusepath{stroke}%
\end{pgfscope}%
\begin{pgfscope}%
\pgfpathrectangle{\pgfqpoint{0.895256in}{0.220000in}}{\pgfqpoint{5.550589in}{1.540000in}}%
\pgfusepath{clip}%
\pgfsetbuttcap%
\pgfsetroundjoin%
\pgfsetlinewidth{1.505625pt}%
\definecolor{currentstroke}{rgb}{0.501961,0.501961,0.501961}%
\pgfsetstrokecolor{currentstroke}%
\pgfsetdash{{5.550000pt}{2.400000pt}}{0.000000pt}%
\pgfusepath{stroke}%
\end{pgfscope}%
\begin{pgfscope}%
\pgfpathrectangle{\pgfqpoint{0.895256in}{0.220000in}}{\pgfqpoint{5.550589in}{1.540000in}}%
\pgfusepath{clip}%
\pgfsetbuttcap%
\pgfsetroundjoin%
\pgfsetlinewidth{1.505625pt}%
\definecolor{currentstroke}{rgb}{0.501961,0.501961,0.501961}%
\pgfsetstrokecolor{currentstroke}%
\pgfsetdash{{5.550000pt}{2.400000pt}}{0.000000pt}%
\pgfusepath{stroke}%
\end{pgfscope}%
\begin{pgfscope}%
\pgfpathrectangle{\pgfqpoint{0.895256in}{0.220000in}}{\pgfqpoint{5.550589in}{1.540000in}}%
\pgfusepath{clip}%
\pgfsetbuttcap%
\pgfsetroundjoin%
\pgfsetlinewidth{1.505625pt}%
\definecolor{currentstroke}{rgb}{0.501961,0.501961,0.501961}%
\pgfsetstrokecolor{currentstroke}%
\pgfsetdash{{5.550000pt}{2.400000pt}}{0.000000pt}%
\pgfusepath{stroke}%
\end{pgfscope}%
\begin{pgfscope}%
\pgfpathrectangle{\pgfqpoint{0.895256in}{0.220000in}}{\pgfqpoint{5.550589in}{1.540000in}}%
\pgfusepath{clip}%
\pgfsetbuttcap%
\pgfsetroundjoin%
\pgfsetlinewidth{1.505625pt}%
\definecolor{currentstroke}{rgb}{0.501961,0.501961,0.501961}%
\pgfsetstrokecolor{currentstroke}%
\pgfsetdash{{5.550000pt}{2.400000pt}}{0.000000pt}%
\pgfusepath{stroke}%
\end{pgfscope}%
\begin{pgfscope}%
\pgfpathrectangle{\pgfqpoint{0.895256in}{0.220000in}}{\pgfqpoint{5.550589in}{1.540000in}}%
\pgfusepath{clip}%
\pgfsetbuttcap%
\pgfsetroundjoin%
\pgfsetlinewidth{1.505625pt}%
\definecolor{currentstroke}{rgb}{0.501961,0.501961,0.501961}%
\pgfsetstrokecolor{currentstroke}%
\pgfsetdash{{5.550000pt}{2.400000pt}}{0.000000pt}%
\pgfusepath{stroke}%
\end{pgfscope}%
\begin{pgfscope}%
\pgfpathrectangle{\pgfqpoint{0.895256in}{0.220000in}}{\pgfqpoint{5.550589in}{1.540000in}}%
\pgfusepath{clip}%
\pgfsetbuttcap%
\pgfsetroundjoin%
\pgfsetlinewidth{1.505625pt}%
\definecolor{currentstroke}{rgb}{0.501961,0.501961,0.501961}%
\pgfsetstrokecolor{currentstroke}%
\pgfsetdash{{5.550000pt}{2.400000pt}}{0.000000pt}%
\pgfusepath{stroke}%
\end{pgfscope}%
\begin{pgfscope}%
\pgfpathrectangle{\pgfqpoint{0.895256in}{0.220000in}}{\pgfqpoint{5.550589in}{1.540000in}}%
\pgfusepath{clip}%
\pgfsetbuttcap%
\pgfsetroundjoin%
\pgfsetlinewidth{1.505625pt}%
\definecolor{currentstroke}{rgb}{0.501961,0.501961,0.501961}%
\pgfsetstrokecolor{currentstroke}%
\pgfsetdash{{5.550000pt}{2.400000pt}}{0.000000pt}%
\pgfusepath{stroke}%
\end{pgfscope}%
\begin{pgfscope}%
\pgfpathrectangle{\pgfqpoint{0.895256in}{0.220000in}}{\pgfqpoint{5.550589in}{1.540000in}}%
\pgfusepath{clip}%
\pgfsetbuttcap%
\pgfsetroundjoin%
\pgfsetlinewidth{1.505625pt}%
\definecolor{currentstroke}{rgb}{0.501961,0.501961,0.501961}%
\pgfsetstrokecolor{currentstroke}%
\pgfsetdash{{5.550000pt}{2.400000pt}}{0.000000pt}%
\pgfusepath{stroke}%
\end{pgfscope}%
\begin{pgfscope}%
\pgfpathrectangle{\pgfqpoint{0.895256in}{0.220000in}}{\pgfqpoint{5.550589in}{1.540000in}}%
\pgfusepath{clip}%
\pgfsetbuttcap%
\pgfsetroundjoin%
\pgfsetlinewidth{1.505625pt}%
\definecolor{currentstroke}{rgb}{0.501961,0.501961,0.501961}%
\pgfsetstrokecolor{currentstroke}%
\pgfsetdash{{5.550000pt}{2.400000pt}}{0.000000pt}%
\pgfusepath{stroke}%
\end{pgfscope}%
\begin{pgfscope}%
\pgfpathrectangle{\pgfqpoint{0.895256in}{0.220000in}}{\pgfqpoint{5.550589in}{1.540000in}}%
\pgfusepath{clip}%
\pgfsetbuttcap%
\pgfsetroundjoin%
\pgfsetlinewidth{1.505625pt}%
\definecolor{currentstroke}{rgb}{0.501961,0.501961,0.501961}%
\pgfsetstrokecolor{currentstroke}%
\pgfsetdash{{5.550000pt}{2.400000pt}}{0.000000pt}%
\pgfusepath{stroke}%
\end{pgfscope}%
\begin{pgfscope}%
\pgfpathrectangle{\pgfqpoint{0.895256in}{0.220000in}}{\pgfqpoint{5.550589in}{1.540000in}}%
\pgfusepath{clip}%
\pgfsetbuttcap%
\pgfsetroundjoin%
\pgfsetlinewidth{1.505625pt}%
\definecolor{currentstroke}{rgb}{0.501961,0.501961,0.501961}%
\pgfsetstrokecolor{currentstroke}%
\pgfsetdash{{5.550000pt}{2.400000pt}}{0.000000pt}%
\pgfusepath{stroke}%
\end{pgfscope}%
\begin{pgfscope}%
\pgfpathrectangle{\pgfqpoint{0.895256in}{0.220000in}}{\pgfqpoint{5.550589in}{1.540000in}}%
\pgfusepath{clip}%
\pgfsetbuttcap%
\pgfsetroundjoin%
\pgfsetlinewidth{1.505625pt}%
\definecolor{currentstroke}{rgb}{0.501961,0.501961,0.501961}%
\pgfsetstrokecolor{currentstroke}%
\pgfsetdash{{5.550000pt}{2.400000pt}}{0.000000pt}%
\pgfusepath{stroke}%
\end{pgfscope}%
\begin{pgfscope}%
\pgfpathrectangle{\pgfqpoint{0.895256in}{0.220000in}}{\pgfqpoint{5.550589in}{1.540000in}}%
\pgfusepath{clip}%
\pgfsetbuttcap%
\pgfsetroundjoin%
\pgfsetlinewidth{1.505625pt}%
\definecolor{currentstroke}{rgb}{0.501961,0.501961,0.501961}%
\pgfsetstrokecolor{currentstroke}%
\pgfsetdash{{5.550000pt}{2.400000pt}}{0.000000pt}%
\pgfusepath{stroke}%
\end{pgfscope}%
\begin{pgfscope}%
\pgfpathrectangle{\pgfqpoint{0.895256in}{0.220000in}}{\pgfqpoint{5.550589in}{1.540000in}}%
\pgfusepath{clip}%
\pgfsetbuttcap%
\pgfsetroundjoin%
\pgfsetlinewidth{1.505625pt}%
\definecolor{currentstroke}{rgb}{0.501961,0.501961,0.501961}%
\pgfsetstrokecolor{currentstroke}%
\pgfsetdash{{5.550000pt}{2.400000pt}}{0.000000pt}%
\pgfusepath{stroke}%
\end{pgfscope}%
\begin{pgfscope}%
\pgfsetrectcap%
\pgfsetmiterjoin%
\pgfsetlinewidth{0.803000pt}%
\definecolor{currentstroke}{rgb}{0.000000,0.000000,0.000000}%
\pgfsetstrokecolor{currentstroke}%
\pgfsetdash{}{0pt}%
\pgfpathmoveto{\pgfqpoint{0.895256in}{0.220000in}}%
\pgfpathlineto{\pgfqpoint{0.895256in}{1.760000in}}%
\pgfusepath{stroke}%
\end{pgfscope}%
\begin{pgfscope}%
\pgfsetrectcap%
\pgfsetmiterjoin%
\pgfsetlinewidth{0.803000pt}%
\definecolor{currentstroke}{rgb}{0.000000,0.000000,0.000000}%
\pgfsetstrokecolor{currentstroke}%
\pgfsetdash{}{0pt}%
\pgfpathmoveto{\pgfqpoint{6.445845in}{0.220000in}}%
\pgfpathlineto{\pgfqpoint{6.445845in}{1.760000in}}%
\pgfusepath{stroke}%
\end{pgfscope}%
\begin{pgfscope}%
\pgfsetrectcap%
\pgfsetmiterjoin%
\pgfsetlinewidth{0.803000pt}%
\definecolor{currentstroke}{rgb}{0.000000,0.000000,0.000000}%
\pgfsetstrokecolor{currentstroke}%
\pgfsetdash{}{0pt}%
\pgfpathmoveto{\pgfqpoint{0.895256in}{0.220000in}}%
\pgfpathlineto{\pgfqpoint{6.445845in}{0.220000in}}%
\pgfusepath{stroke}%
\end{pgfscope}%
\begin{pgfscope}%
\pgfsetrectcap%
\pgfsetmiterjoin%
\pgfsetlinewidth{0.803000pt}%
\definecolor{currentstroke}{rgb}{0.000000,0.000000,0.000000}%
\pgfsetstrokecolor{currentstroke}%
\pgfsetdash{}{0pt}%
\pgfpathmoveto{\pgfqpoint{0.895256in}{1.760000in}}%
\pgfpathlineto{\pgfqpoint{6.445845in}{1.760000in}}%
\pgfusepath{stroke}%
\end{pgfscope}%
\begin{pgfscope}%
\pgfpathrectangle{\pgfqpoint{0.895256in}{0.220000in}}{\pgfqpoint{5.550589in}{1.540000in}}%
\pgfusepath{clip}%
\definecolor{textcolor}{rgb}{1.000000,0.000000,0.000000}%
\pgfsetstrokecolor{textcolor}%
\pgfsetfillcolor{textcolor}%
\pgftext[x=-2.568221in,y=1.165000in,left,base]{\color{textcolor}{\rmfamily\fontsize{8.000000}{9.600000}\selectfont\catcode`\^=\active\def^{\ifmmode\sp\else\^{}\fi}\catcode`\%=\active\def%{\%}0.31}}%
\end{pgfscope}%
\begin{pgfscope}%
\pgfpathrectangle{\pgfqpoint{0.895256in}{0.220000in}}{\pgfqpoint{5.550589in}{1.540000in}}%
\pgfusepath{clip}%
\definecolor{textcolor}{rgb}{1.000000,0.000000,0.000000}%
\pgfsetstrokecolor{textcolor}%
\pgfsetfillcolor{textcolor}%
\pgftext[x=-2.116953in,y=1.165000in,left,base]{\color{textcolor}{\rmfamily\fontsize{8.000000}{9.600000}\selectfont\catcode`\^=\active\def^{\ifmmode\sp\else\^{}\fi}\catcode`\%=\active\def%{\%}0.37}}%
\end{pgfscope}%
\begin{pgfscope}%
\pgfpathrectangle{\pgfqpoint{0.895256in}{0.220000in}}{\pgfqpoint{5.550589in}{1.540000in}}%
\pgfusepath{clip}%
\definecolor{textcolor}{rgb}{1.000000,0.000000,0.000000}%
\pgfsetstrokecolor{textcolor}%
\pgfsetfillcolor{textcolor}%
\pgftext[x=-1.665686in,y=1.165000in,left,base]{\color{textcolor}{\rmfamily\fontsize{8.000000}{9.600000}\selectfont\catcode`\^=\active\def^{\ifmmode\sp\else\^{}\fi}\catcode`\%=\active\def%{\%}0.38}}%
\end{pgfscope}%
\begin{pgfscope}%
\pgfpathrectangle{\pgfqpoint{0.895256in}{0.220000in}}{\pgfqpoint{5.550589in}{1.540000in}}%
\pgfusepath{clip}%
\definecolor{textcolor}{rgb}{1.000000,0.000000,0.000000}%
\pgfsetstrokecolor{textcolor}%
\pgfsetfillcolor{textcolor}%
\pgftext[x=-1.214419in,y=1.165000in,left,base]{\color{textcolor}{\rmfamily\fontsize{8.000000}{9.600000}\selectfont\catcode`\^=\active\def^{\ifmmode\sp\else\^{}\fi}\catcode`\%=\active\def%{\%}0.43}}%
\end{pgfscope}%
\begin{pgfscope}%
\pgfpathrectangle{\pgfqpoint{0.895256in}{0.220000in}}{\pgfqpoint{5.550589in}{1.540000in}}%
\pgfusepath{clip}%
\definecolor{textcolor}{rgb}{1.000000,0.000000,0.000000}%
\pgfsetstrokecolor{textcolor}%
\pgfsetfillcolor{textcolor}%
\pgftext[x=-0.763151in,y=1.165000in,left,base]{\color{textcolor}{\rmfamily\fontsize{8.000000}{9.600000}\selectfont\catcode`\^=\active\def^{\ifmmode\sp\else\^{}\fi}\catcode`\%=\active\def%{\%}0.48}}%
\end{pgfscope}%
\begin{pgfscope}%
\pgfpathrectangle{\pgfqpoint{0.895256in}{0.220000in}}{\pgfqpoint{5.550589in}{1.540000in}}%
\pgfusepath{clip}%
\definecolor{textcolor}{rgb}{1.000000,0.000000,0.000000}%
\pgfsetstrokecolor{textcolor}%
\pgfsetfillcolor{textcolor}%
\pgftext[x=-0.311884in,y=1.165000in,left,base]{\color{textcolor}{\rmfamily\fontsize{8.000000}{9.600000}\selectfont\catcode`\^=\active\def^{\ifmmode\sp\else\^{}\fi}\catcode`\%=\active\def%{\%}0.46}}%
\end{pgfscope}%
\begin{pgfscope}%
\pgfpathrectangle{\pgfqpoint{0.895256in}{0.220000in}}{\pgfqpoint{5.550589in}{1.540000in}}%
\pgfusepath{clip}%
\definecolor{textcolor}{rgb}{1.000000,0.000000,0.000000}%
\pgfsetstrokecolor{textcolor}%
\pgfsetfillcolor{textcolor}%
\pgftext[x=0.139383in,y=1.165000in,left,base]{\color{textcolor}{\rmfamily\fontsize{8.000000}{9.600000}\selectfont\catcode`\^=\active\def^{\ifmmode\sp\else\^{}\fi}\catcode`\%=\active\def%{\%}0.48}}%
\end{pgfscope}%
\begin{pgfscope}%
\pgfpathrectangle{\pgfqpoint{0.895256in}{0.220000in}}{\pgfqpoint{5.550589in}{1.540000in}}%
\pgfusepath{clip}%
\definecolor{textcolor}{rgb}{1.000000,0.000000,0.000000}%
\pgfsetstrokecolor{textcolor}%
\pgfsetfillcolor{textcolor}%
\pgftext[x=0.590651in,y=1.165000in,left,base]{\color{textcolor}{\rmfamily\fontsize{8.000000}{9.600000}\selectfont\catcode`\^=\active\def^{\ifmmode\sp\else\^{}\fi}\catcode`\%=\active\def%{\%}0.50}}%
\end{pgfscope}%
\begin{pgfscope}%
\pgfpathrectangle{\pgfqpoint{0.895256in}{0.220000in}}{\pgfqpoint{5.550589in}{1.540000in}}%
\pgfusepath{clip}%
\definecolor{textcolor}{rgb}{1.000000,0.000000,0.000000}%
\pgfsetstrokecolor{textcolor}%
\pgfsetfillcolor{textcolor}%
\pgftext[x=1.041918in,y=1.165000in,left,base]{\color{textcolor}{\rmfamily\fontsize{8.000000}{9.600000}\selectfont\catcode`\^=\active\def^{\ifmmode\sp\else\^{}\fi}\catcode`\%=\active\def%{\%}0.50}}%
\end{pgfscope}%
\begin{pgfscope}%
\pgfpathrectangle{\pgfqpoint{0.895256in}{0.220000in}}{\pgfqpoint{5.550589in}{1.540000in}}%
\pgfusepath{clip}%
\definecolor{textcolor}{rgb}{1.000000,0.000000,0.000000}%
\pgfsetstrokecolor{textcolor}%
\pgfsetfillcolor{textcolor}%
\pgftext[x=1.493186in,y=1.165000in,left,base]{\color{textcolor}{\rmfamily\fontsize{8.000000}{9.600000}\selectfont\catcode`\^=\active\def^{\ifmmode\sp\else\^{}\fi}\catcode`\%=\active\def%{\%}0.49}}%
\end{pgfscope}%
\begin{pgfscope}%
\pgfpathrectangle{\pgfqpoint{0.895256in}{0.220000in}}{\pgfqpoint{5.550589in}{1.540000in}}%
\pgfusepath{clip}%
\definecolor{textcolor}{rgb}{1.000000,0.000000,0.000000}%
\pgfsetstrokecolor{textcolor}%
\pgfsetfillcolor{textcolor}%
\pgftext[x=1.944453in,y=1.165000in,left,base]{\color{textcolor}{\rmfamily\fontsize{8.000000}{9.600000}\selectfont\catcode`\^=\active\def^{\ifmmode\sp\else\^{}\fi}\catcode`\%=\active\def%{\%}0.49}}%
\end{pgfscope}%
\begin{pgfscope}%
\pgfpathrectangle{\pgfqpoint{0.895256in}{0.220000in}}{\pgfqpoint{5.550589in}{1.540000in}}%
\pgfusepath{clip}%
\definecolor{textcolor}{rgb}{1.000000,0.000000,0.000000}%
\pgfsetstrokecolor{textcolor}%
\pgfsetfillcolor{textcolor}%
\pgftext[x=2.395720in,y=1.165000in,left,base]{\color{textcolor}{\rmfamily\fontsize{8.000000}{9.600000}\selectfont\catcode`\^=\active\def^{\ifmmode\sp\else\^{}\fi}\catcode`\%=\active\def%{\%}0.48}}%
\end{pgfscope}%
\begin{pgfscope}%
\pgfpathrectangle{\pgfqpoint{0.895256in}{0.220000in}}{\pgfqpoint{5.550589in}{1.540000in}}%
\pgfusepath{clip}%
\definecolor{textcolor}{rgb}{1.000000,0.000000,0.000000}%
\pgfsetstrokecolor{textcolor}%
\pgfsetfillcolor{textcolor}%
\pgftext[x=2.846988in,y=1.165000in,left,base]{\color{textcolor}{\rmfamily\fontsize{8.000000}{9.600000}\selectfont\catcode`\^=\active\def^{\ifmmode\sp\else\^{}\fi}\catcode`\%=\active\def%{\%}0.48}}%
\end{pgfscope}%
\begin{pgfscope}%
\pgfpathrectangle{\pgfqpoint{0.895256in}{0.220000in}}{\pgfqpoint{5.550589in}{1.540000in}}%
\pgfusepath{clip}%
\definecolor{textcolor}{rgb}{1.000000,0.000000,0.000000}%
\pgfsetstrokecolor{textcolor}%
\pgfsetfillcolor{textcolor}%
\pgftext[x=3.298255in,y=1.165000in,left,base]{\color{textcolor}{\rmfamily\fontsize{8.000000}{9.600000}\selectfont\catcode`\^=\active\def^{\ifmmode\sp\else\^{}\fi}\catcode`\%=\active\def%{\%}0.46}}%
\end{pgfscope}%
\begin{pgfscope}%
\pgfpathrectangle{\pgfqpoint{0.895256in}{0.220000in}}{\pgfqpoint{5.550589in}{1.540000in}}%
\pgfusepath{clip}%
\definecolor{textcolor}{rgb}{1.000000,0.000000,0.000000}%
\pgfsetstrokecolor{textcolor}%
\pgfsetfillcolor{textcolor}%
\pgftext[x=3.749522in,y=1.165000in,left,base]{\color{textcolor}{\rmfamily\fontsize{8.000000}{9.600000}\selectfont\catcode`\^=\active\def^{\ifmmode\sp\else\^{}\fi}\catcode`\%=\active\def%{\%}0.43}}%
\end{pgfscope}%
\begin{pgfscope}%
\pgfpathrectangle{\pgfqpoint{0.895256in}{0.220000in}}{\pgfqpoint{5.550589in}{1.540000in}}%
\pgfusepath{clip}%
\definecolor{textcolor}{rgb}{1.000000,0.000000,0.000000}%
\pgfsetstrokecolor{textcolor}%
\pgfsetfillcolor{textcolor}%
\pgftext[x=4.200790in,y=1.165000in,left,base]{\color{textcolor}{\rmfamily\fontsize{8.000000}{9.600000}\selectfont\catcode`\^=\active\def^{\ifmmode\sp\else\^{}\fi}\catcode`\%=\active\def%{\%}0.43}}%
\end{pgfscope}%
\begin{pgfscope}%
\pgfpathrectangle{\pgfqpoint{0.895256in}{0.220000in}}{\pgfqpoint{5.550589in}{1.540000in}}%
\pgfusepath{clip}%
\definecolor{textcolor}{rgb}{1.000000,0.000000,0.000000}%
\pgfsetstrokecolor{textcolor}%
\pgfsetfillcolor{textcolor}%
\pgftext[x=4.652057in,y=1.165000in,left,base]{\color{textcolor}{\rmfamily\fontsize{8.000000}{9.600000}\selectfont\catcode`\^=\active\def^{\ifmmode\sp\else\^{}\fi}\catcode`\%=\active\def%{\%}0.39}}%
\end{pgfscope}%
\begin{pgfscope}%
\pgfpathrectangle{\pgfqpoint{0.895256in}{0.220000in}}{\pgfqpoint{5.550589in}{1.540000in}}%
\pgfusepath{clip}%
\definecolor{textcolor}{rgb}{1.000000,0.000000,0.000000}%
\pgfsetstrokecolor{textcolor}%
\pgfsetfillcolor{textcolor}%
\pgftext[x=5.103325in,y=1.165000in,left,base]{\color{textcolor}{\rmfamily\fontsize{8.000000}{9.600000}\selectfont\catcode`\^=\active\def^{\ifmmode\sp\else\^{}\fi}\catcode`\%=\active\def%{\%}0.37}}%
\end{pgfscope}%
\begin{pgfscope}%
\pgfpathrectangle{\pgfqpoint{0.895256in}{0.220000in}}{\pgfqpoint{5.550589in}{1.540000in}}%
\pgfusepath{clip}%
\definecolor{textcolor}{rgb}{1.000000,0.000000,0.000000}%
\pgfsetstrokecolor{textcolor}%
\pgfsetfillcolor{textcolor}%
\pgftext[x=5.554592in,y=1.165000in,left,base]{\color{textcolor}{\rmfamily\fontsize{8.000000}{9.600000}\selectfont\catcode`\^=\active\def^{\ifmmode\sp\else\^{}\fi}\catcode`\%=\active\def%{\%}0.34}}%
\end{pgfscope}%
\begin{pgfscope}%
\pgfpathrectangle{\pgfqpoint{0.895256in}{0.220000in}}{\pgfqpoint{5.550589in}{1.540000in}}%
\pgfusepath{clip}%
\definecolor{textcolor}{rgb}{1.000000,0.000000,0.000000}%
\pgfsetstrokecolor{textcolor}%
\pgfsetfillcolor{textcolor}%
\pgftext[x=6.005859in,y=1.165000in,left,base]{\color{textcolor}{\rmfamily\fontsize{8.000000}{9.600000}\selectfont\catcode`\^=\active\def^{\ifmmode\sp\else\^{}\fi}\catcode`\%=\active\def%{\%}0.31}}%
\end{pgfscope}%
\begin{pgfscope}%
\pgfpathrectangle{\pgfqpoint{0.895256in}{0.220000in}}{\pgfqpoint{5.550589in}{1.540000in}}%
\pgfusepath{clip}%
\definecolor{textcolor}{rgb}{1.000000,0.000000,0.000000}%
\pgfsetstrokecolor{textcolor}%
\pgfsetfillcolor{textcolor}%
\pgftext[x=6.457127in,y=1.165000in,left,base]{\color{textcolor}{\rmfamily\fontsize{8.000000}{9.600000}\selectfont\catcode`\^=\active\def^{\ifmmode\sp\else\^{}\fi}\catcode`\%=\active\def%{\%}0.29}}%
\end{pgfscope}%
\begin{pgfscope}%
\pgfpathrectangle{\pgfqpoint{0.895256in}{0.220000in}}{\pgfqpoint{5.550589in}{1.540000in}}%
\pgfusepath{clip}%
\definecolor{textcolor}{rgb}{1.000000,0.000000,0.000000}%
\pgfsetstrokecolor{textcolor}%
\pgfsetfillcolor{textcolor}%
\pgftext[x=6.908394in,y=1.165000in,left,base]{\color{textcolor}{\rmfamily\fontsize{8.000000}{9.600000}\selectfont\catcode`\^=\active\def^{\ifmmode\sp\else\^{}\fi}\catcode`\%=\active\def%{\%}0.25}}%
\end{pgfscope}%
\begin{pgfscope}%
\pgfpathrectangle{\pgfqpoint{0.895256in}{0.220000in}}{\pgfqpoint{5.550589in}{1.540000in}}%
\pgfusepath{clip}%
\definecolor{textcolor}{rgb}{1.000000,0.000000,0.000000}%
\pgfsetstrokecolor{textcolor}%
\pgfsetfillcolor{textcolor}%
\pgftext[x=7.359661in,y=1.165000in,left,base]{\color{textcolor}{\rmfamily\fontsize{8.000000}{9.600000}\selectfont\catcode`\^=\active\def^{\ifmmode\sp\else\^{}\fi}\catcode`\%=\active\def%{\%}0.22}}%
\end{pgfscope}%
\begin{pgfscope}%
\pgfpathrectangle{\pgfqpoint{0.895256in}{0.220000in}}{\pgfqpoint{5.550589in}{1.540000in}}%
\pgfusepath{clip}%
\definecolor{textcolor}{rgb}{1.000000,0.000000,0.000000}%
\pgfsetstrokecolor{textcolor}%
\pgfsetfillcolor{textcolor}%
\pgftext[x=7.810929in,y=1.165000in,left,base]{\color{textcolor}{\rmfamily\fontsize{8.000000}{9.600000}\selectfont\catcode`\^=\active\def^{\ifmmode\sp\else\^{}\fi}\catcode`\%=\active\def%{\%}0.18}}%
\end{pgfscope}%
\begin{pgfscope}%
\pgfpathrectangle{\pgfqpoint{0.895256in}{0.220000in}}{\pgfqpoint{5.550589in}{1.540000in}}%
\pgfusepath{clip}%
\definecolor{textcolor}{rgb}{1.000000,0.000000,0.000000}%
\pgfsetstrokecolor{textcolor}%
\pgfsetfillcolor{textcolor}%
\pgftext[x=8.262196in,y=1.165000in,left,base]{\color{textcolor}{\rmfamily\fontsize{8.000000}{9.600000}\selectfont\catcode`\^=\active\def^{\ifmmode\sp\else\^{}\fi}\catcode`\%=\active\def%{\%}0.14}}%
\end{pgfscope}%
\begin{pgfscope}%
\pgfpathrectangle{\pgfqpoint{0.895256in}{0.220000in}}{\pgfqpoint{5.550589in}{1.540000in}}%
\pgfusepath{clip}%
\definecolor{textcolor}{rgb}{1.000000,0.000000,0.000000}%
\pgfsetstrokecolor{textcolor}%
\pgfsetfillcolor{textcolor}%
\pgftext[x=8.713464in,y=1.165000in,left,base]{\color{textcolor}{\rmfamily\fontsize{8.000000}{9.600000}\selectfont\catcode`\^=\active\def^{\ifmmode\sp\else\^{}\fi}\catcode`\%=\active\def%{\%}0.11}}%
\end{pgfscope}%
\begin{pgfscope}%
\pgfpathrectangle{\pgfqpoint{0.895256in}{0.220000in}}{\pgfqpoint{5.550589in}{1.540000in}}%
\pgfusepath{clip}%
\definecolor{textcolor}{rgb}{1.000000,0.000000,0.000000}%
\pgfsetstrokecolor{textcolor}%
\pgfsetfillcolor{textcolor}%
\pgftext[x=9.164731in,y=1.165000in,left,base]{\color{textcolor}{\rmfamily\fontsize{8.000000}{9.600000}\selectfont\catcode`\^=\active\def^{\ifmmode\sp\else\^{}\fi}\catcode`\%=\active\def%{\%}0.07}}%
\end{pgfscope}%
\begin{pgfscope}%
\pgfpathrectangle{\pgfqpoint{0.895256in}{0.220000in}}{\pgfqpoint{5.550589in}{1.540000in}}%
\pgfusepath{clip}%
\definecolor{textcolor}{rgb}{1.000000,0.000000,0.000000}%
\pgfsetstrokecolor{textcolor}%
\pgfsetfillcolor{textcolor}%
\pgftext[x=9.615998in,y=1.165000in,left,base]{\color{textcolor}{\rmfamily\fontsize{8.000000}{9.600000}\selectfont\catcode`\^=\active\def^{\ifmmode\sp\else\^{}\fi}\catcode`\%=\active\def%{\%}0.04}}%
\end{pgfscope}%
\begin{pgfscope}%
\pgfpathrectangle{\pgfqpoint{0.895256in}{0.220000in}}{\pgfqpoint{5.550589in}{1.540000in}}%
\pgfusepath{clip}%
\definecolor{textcolor}{rgb}{1.000000,0.000000,0.000000}%
\pgfsetstrokecolor{textcolor}%
\pgfsetfillcolor{textcolor}%
\pgftext[x=10.067266in,y=1.165000in,left,base]{\color{textcolor}{\rmfamily\fontsize{8.000000}{9.600000}\selectfont\catcode`\^=\active\def^{\ifmmode\sp\else\^{}\fi}\catcode`\%=\active\def%{\%}-0.00}}%
\end{pgfscope}%
\begin{pgfscope}%
\pgfpathrectangle{\pgfqpoint{0.895256in}{0.220000in}}{\pgfqpoint{5.550589in}{1.540000in}}%
\pgfusepath{clip}%
\definecolor{textcolor}{rgb}{1.000000,0.000000,0.000000}%
\pgfsetstrokecolor{textcolor}%
\pgfsetfillcolor{textcolor}%
\pgftext[x=10.518533in,y=1.165000in,left,base]{\color{textcolor}{\rmfamily\fontsize{8.000000}{9.600000}\selectfont\catcode`\^=\active\def^{\ifmmode\sp\else\^{}\fi}\catcode`\%=\active\def%{\%}-0.04}}%
\end{pgfscope}%
\begin{pgfscope}%
\pgfpathrectangle{\pgfqpoint{0.895256in}{0.220000in}}{\pgfqpoint{5.550589in}{1.540000in}}%
\pgfusepath{clip}%
\definecolor{textcolor}{rgb}{1.000000,0.000000,0.000000}%
\pgfsetstrokecolor{textcolor}%
\pgfsetfillcolor{textcolor}%
\pgftext[x=10.969800in,y=1.165000in,left,base]{\color{textcolor}{\rmfamily\fontsize{8.000000}{9.600000}\selectfont\catcode`\^=\active\def^{\ifmmode\sp\else\^{}\fi}\catcode`\%=\active\def%{\%}-0.08}}%
\end{pgfscope}%
\begin{pgfscope}%
\pgfpathrectangle{\pgfqpoint{0.895256in}{0.220000in}}{\pgfqpoint{5.550589in}{1.540000in}}%
\pgfusepath{clip}%
\definecolor{textcolor}{rgb}{1.000000,0.000000,0.000000}%
\pgfsetstrokecolor{textcolor}%
\pgfsetfillcolor{textcolor}%
\pgftext[x=11.421068in,y=1.165000in,left,base]{\color{textcolor}{\rmfamily\fontsize{8.000000}{9.600000}\selectfont\catcode`\^=\active\def^{\ifmmode\sp\else\^{}\fi}\catcode`\%=\active\def%{\%}-0.12}}%
\end{pgfscope}%
\begin{pgfscope}%
\pgfpathrectangle{\pgfqpoint{0.895256in}{0.220000in}}{\pgfqpoint{5.550589in}{1.540000in}}%
\pgfusepath{clip}%
\definecolor{textcolor}{rgb}{1.000000,0.000000,0.000000}%
\pgfsetstrokecolor{textcolor}%
\pgfsetfillcolor{textcolor}%
\pgftext[x=11.872335in,y=1.165000in,left,base]{\color{textcolor}{\rmfamily\fontsize{8.000000}{9.600000}\selectfont\catcode`\^=\active\def^{\ifmmode\sp\else\^{}\fi}\catcode`\%=\active\def%{\%}-0.15}}%
\end{pgfscope}%
\begin{pgfscope}%
\pgfpathrectangle{\pgfqpoint{0.895256in}{0.220000in}}{\pgfqpoint{5.550589in}{1.540000in}}%
\pgfusepath{clip}%
\definecolor{textcolor}{rgb}{1.000000,0.000000,0.000000}%
\pgfsetstrokecolor{textcolor}%
\pgfsetfillcolor{textcolor}%
\pgftext[x=12.323603in,y=1.165000in,left,base]{\color{textcolor}{\rmfamily\fontsize{8.000000}{9.600000}\selectfont\catcode`\^=\active\def^{\ifmmode\sp\else\^{}\fi}\catcode`\%=\active\def%{\%}-0.19}}%
\end{pgfscope}%
\begin{pgfscope}%
\pgfpathrectangle{\pgfqpoint{0.895256in}{0.220000in}}{\pgfqpoint{5.550589in}{1.540000in}}%
\pgfusepath{clip}%
\definecolor{textcolor}{rgb}{1.000000,0.000000,0.000000}%
\pgfsetstrokecolor{textcolor}%
\pgfsetfillcolor{textcolor}%
\pgftext[x=12.774870in,y=1.165000in,left,base]{\color{textcolor}{\rmfamily\fontsize{8.000000}{9.600000}\selectfont\catcode`\^=\active\def^{\ifmmode\sp\else\^{}\fi}\catcode`\%=\active\def%{\%}-0.22}}%
\end{pgfscope}%
\begin{pgfscope}%
\pgfpathrectangle{\pgfqpoint{0.895256in}{0.220000in}}{\pgfqpoint{5.550589in}{1.540000in}}%
\pgfusepath{clip}%
\definecolor{textcolor}{rgb}{1.000000,0.000000,0.000000}%
\pgfsetstrokecolor{textcolor}%
\pgfsetfillcolor{textcolor}%
\pgftext[x=13.226137in,y=1.165000in,left,base]{\color{textcolor}{\rmfamily\fontsize{8.000000}{9.600000}\selectfont\catcode`\^=\active\def^{\ifmmode\sp\else\^{}\fi}\catcode`\%=\active\def%{\%}-0.25}}%
\end{pgfscope}%
\begin{pgfscope}%
\pgfpathrectangle{\pgfqpoint{0.895256in}{0.220000in}}{\pgfqpoint{5.550589in}{1.540000in}}%
\pgfusepath{clip}%
\definecolor{textcolor}{rgb}{1.000000,0.000000,0.000000}%
\pgfsetstrokecolor{textcolor}%
\pgfsetfillcolor{textcolor}%
\pgftext[x=13.677405in,y=1.165000in,left,base]{\color{textcolor}{\rmfamily\fontsize{8.000000}{9.600000}\selectfont\catcode`\^=\active\def^{\ifmmode\sp\else\^{}\fi}\catcode`\%=\active\def%{\%}-0.28}}%
\end{pgfscope}%
\begin{pgfscope}%
\pgfpathrectangle{\pgfqpoint{0.895256in}{0.220000in}}{\pgfqpoint{5.550589in}{1.540000in}}%
\pgfusepath{clip}%
\definecolor{textcolor}{rgb}{1.000000,0.000000,0.000000}%
\pgfsetstrokecolor{textcolor}%
\pgfsetfillcolor{textcolor}%
\pgftext[x=14.128672in,y=1.165000in,left,base]{\color{textcolor}{\rmfamily\fontsize{8.000000}{9.600000}\selectfont\catcode`\^=\active\def^{\ifmmode\sp\else\^{}\fi}\catcode`\%=\active\def%{\%}-0.31}}%
\end{pgfscope}%
\begin{pgfscope}%
\pgfpathrectangle{\pgfqpoint{0.895256in}{0.220000in}}{\pgfqpoint{5.550589in}{1.540000in}}%
\pgfusepath{clip}%
\definecolor{textcolor}{rgb}{1.000000,0.000000,0.000000}%
\pgfsetstrokecolor{textcolor}%
\pgfsetfillcolor{textcolor}%
\pgftext[x=14.579939in,y=1.165000in,left,base]{\color{textcolor}{\rmfamily\fontsize{8.000000}{9.600000}\selectfont\catcode`\^=\active\def^{\ifmmode\sp\else\^{}\fi}\catcode`\%=\active\def%{\%}-0.34}}%
\end{pgfscope}%
\begin{pgfscope}%
\pgfpathrectangle{\pgfqpoint{0.895256in}{0.220000in}}{\pgfqpoint{5.550589in}{1.540000in}}%
\pgfusepath{clip}%
\definecolor{textcolor}{rgb}{1.000000,0.000000,0.000000}%
\pgfsetstrokecolor{textcolor}%
\pgfsetfillcolor{textcolor}%
\pgftext[x=15.031207in,y=1.165000in,left,base]{\color{textcolor}{\rmfamily\fontsize{8.000000}{9.600000}\selectfont\catcode`\^=\active\def^{\ifmmode\sp\else\^{}\fi}\catcode`\%=\active\def%{\%}-0.37}}%
\end{pgfscope}%
\begin{pgfscope}%
\definecolor{textcolor}{rgb}{0.000000,0.000000,0.000000}%
\pgfsetstrokecolor{textcolor}%
\pgfsetfillcolor{textcolor}%
\pgftext[x=3.670551in,y=1.843333in,,base]{\color{textcolor}{\rmfamily\fontsize{12.000000}{14.400000}\selectfont\catcode`\^=\active\def^{\ifmmode\sp\else\^{}\fi}\catcode`\%=\active\def%{\%}Clock Drift of Sine Wave (2000 Hz)}}%
\end{pgfscope}%
\begin{pgfscope}%
\pgfsetbuttcap%
\pgfsetmiterjoin%
\definecolor{currentfill}{rgb}{1.000000,1.000000,1.000000}%
\pgfsetfillcolor{currentfill}%
\pgfsetfillopacity{0.800000}%
\pgfsetlinewidth{1.003750pt}%
\definecolor{currentstroke}{rgb}{0.800000,0.800000,0.800000}%
\pgfsetstrokecolor{currentstroke}%
\pgfsetstrokeopacity{0.800000}%
\pgfsetdash{}{0pt}%
\pgfpathmoveto{\pgfqpoint{5.354408in}{1.261543in}}%
\pgfpathlineto{\pgfqpoint{6.348623in}{1.261543in}}%
\pgfpathquadraticcurveto{\pgfqpoint{6.376401in}{1.261543in}}{\pgfqpoint{6.376401in}{1.289321in}}%
\pgfpathlineto{\pgfqpoint{6.376401in}{1.662778in}}%
\pgfpathquadraticcurveto{\pgfqpoint{6.376401in}{1.690556in}}{\pgfqpoint{6.348623in}{1.690556in}}%
\pgfpathlineto{\pgfqpoint{5.354408in}{1.690556in}}%
\pgfpathquadraticcurveto{\pgfqpoint{5.326631in}{1.690556in}}{\pgfqpoint{5.326631in}{1.662778in}}%
\pgfpathlineto{\pgfqpoint{5.326631in}{1.289321in}}%
\pgfpathquadraticcurveto{\pgfqpoint{5.326631in}{1.261543in}}{\pgfqpoint{5.354408in}{1.261543in}}%
\pgfpathlineto{\pgfqpoint{5.354408in}{1.261543in}}%
\pgfpathclose%
\pgfusepath{stroke,fill}%
\end{pgfscope}%
\begin{pgfscope}%
\pgfsetrectcap%
\pgfsetroundjoin%
\pgfsetlinewidth{1.505625pt}%
\definecolor{currentstroke}{rgb}{0.121569,0.466667,0.705882}%
\pgfsetstrokecolor{currentstroke}%
\pgfsetdash{}{0pt}%
\pgfpathmoveto{\pgfqpoint{5.382186in}{1.586389in}}%
\pgfpathlineto{\pgfqpoint{5.521075in}{1.586389in}}%
\pgfpathlineto{\pgfqpoint{5.659964in}{1.586389in}}%
\pgfusepath{stroke}%
\end{pgfscope}%
\begin{pgfscope}%
\definecolor{textcolor}{rgb}{0.000000,0.000000,0.000000}%
\pgfsetstrokecolor{textcolor}%
\pgfsetfillcolor{textcolor}%
\pgftext[x=5.771075in,y=1.537778in,left,base]{\color{textcolor}{\rmfamily\fontsize{10.000000}{12.000000}\selectfont\catcode`\^=\active\def^{\ifmmode\sp\else\^{}\fi}\catcode`\%=\active\def%{\%}Playback}}%
\end{pgfscope}%
\begin{pgfscope}%
\pgfsetrectcap%
\pgfsetroundjoin%
\pgfsetlinewidth{1.505625pt}%
\definecolor{currentstroke}{rgb}{1.000000,0.498039,0.054902}%
\pgfsetstrokecolor{currentstroke}%
\pgfsetdash{}{0pt}%
\pgfpathmoveto{\pgfqpoint{5.382186in}{1.392716in}}%
\pgfpathlineto{\pgfqpoint{5.521075in}{1.392716in}}%
\pgfpathlineto{\pgfqpoint{5.659964in}{1.392716in}}%
\pgfusepath{stroke}%
\end{pgfscope}%
\begin{pgfscope}%
\definecolor{textcolor}{rgb}{0.000000,0.000000,0.000000}%
\pgfsetstrokecolor{textcolor}%
\pgfsetfillcolor{textcolor}%
\pgftext[x=5.771075in,y=1.344105in,left,base]{\color{textcolor}{\rmfamily\fontsize{10.000000}{12.000000}\selectfont\catcode`\^=\active\def^{\ifmmode\sp\else\^{}\fi}\catcode`\%=\active\def%{\%}Capture}}%
\end{pgfscope}%
\end{pgfpicture}%
\makeatother%
\endgroup%
}
    \caption{The plot to show the clock drift.}
    \smallskip
    \small
    The red text is the similarity score between the played sound wave and the recorded sound wave, which is calculated by dot product, max is \(0.5\).
    \label{fig:drift}
\end{figure}

As illustrated in Figure \ref{fig:drift}, at the beginning of the recording, the recorded sound wave is in sync with the played sound wave. However, as time progresses, the recorded sound wave gradually drifts out of sync with the played sound wave. \textbf{Around 30 cycles}, the recorded sound wave is almost \(\pi/2\) out of phase with the played sound wave.

\textbf{This issue has a significant impact on Phase Shift Keying (PSK) modulation, as the phase of the received signal is crucial for correct demodulation.}

\subsubsection{Speaker to Microphone Latency}

Latency is the time delay between the played sound wave and the recorded sound wave.
Latency is crucial in our communication system, as we implement a CSMA protocol, which requires to sense the medium before transmitting. If the latency is too large, the sender may not be able to sense the medium correctly.

\begin{figure}[H]
    \noindent\makebox[\textwidth]{%% Creator: Matplotlib, PGF backend
%%
%% To include the figure in your LaTeX document, write
%%   \input{<filename>.pgf}
%%
%% Make sure the required packages are loaded in your preamble
%%   \usepackage{pgf}
%%
%% Also ensure that all the required font packages are loaded; for instance,
%% the lmodern package is sometimes necessary when using math font.
%%   \usepackage{lmodern}
%%
%% Figures using additional raster images can only be included by \input if
%% they are in the same directory as the main LaTeX file. For loading figures
%% from other directories you can use the `import` package
%%   \usepackage{import}
%%
%% and then include the figures with
%%   \import{<path to file>}{<filename>.pgf}
%%
%% Matplotlib used the following preamble
%%   \def\mathdefault#1{#1}
%%   \everymath=\expandafter{\the\everymath\displaystyle}
%%   \IfFileExists{scrextend.sty}{
%%     \usepackage[fontsize=10.000000pt]{scrextend}
%%   }{
%%     \renewcommand{\normalsize}{\fontsize{10.000000}{12.000000}\selectfont}
%%     \normalsize
%%   }
%%   
%%   \makeatletter\@ifpackageloaded{underscore}{}{\usepackage[strings]{underscore}}\makeatother
%%
\begingroup%
\makeatletter%
\begin{pgfpicture}%
\pgfpathrectangle{\pgfpointorigin}{\pgfqpoint{7.162050in}{2.000000in}}%
\pgfusepath{use as bounding box, clip}%
\begin{pgfscope}%
\pgfsetbuttcap%
\pgfsetmiterjoin%
\definecolor{currentfill}{rgb}{1.000000,1.000000,1.000000}%
\pgfsetfillcolor{currentfill}%
\pgfsetlinewidth{0.000000pt}%
\definecolor{currentstroke}{rgb}{1.000000,1.000000,1.000000}%
\pgfsetstrokecolor{currentstroke}%
\pgfsetdash{}{0pt}%
\pgfpathmoveto{\pgfqpoint{0.000000in}{0.000000in}}%
\pgfpathlineto{\pgfqpoint{7.162050in}{0.000000in}}%
\pgfpathlineto{\pgfqpoint{7.162050in}{2.000000in}}%
\pgfpathlineto{\pgfqpoint{0.000000in}{2.000000in}}%
\pgfpathlineto{\pgfqpoint{0.000000in}{0.000000in}}%
\pgfpathclose%
\pgfusepath{fill}%
\end{pgfscope}%
\begin{pgfscope}%
\pgfsetbuttcap%
\pgfsetmiterjoin%
\definecolor{currentfill}{rgb}{1.000000,1.000000,1.000000}%
\pgfsetfillcolor{currentfill}%
\pgfsetlinewidth{0.000000pt}%
\definecolor{currentstroke}{rgb}{0.000000,0.000000,0.000000}%
\pgfsetstrokecolor{currentstroke}%
\pgfsetstrokeopacity{0.000000}%
\pgfsetdash{}{0pt}%
\pgfpathmoveto{\pgfqpoint{0.895256in}{0.220000in}}%
\pgfpathlineto{\pgfqpoint{6.445845in}{0.220000in}}%
\pgfpathlineto{\pgfqpoint{6.445845in}{1.760000in}}%
\pgfpathlineto{\pgfqpoint{0.895256in}{1.760000in}}%
\pgfpathlineto{\pgfqpoint{0.895256in}{0.220000in}}%
\pgfpathclose%
\pgfusepath{fill}%
\end{pgfscope}%
\begin{pgfscope}%
\pgfsetbuttcap%
\pgfsetroundjoin%
\definecolor{currentfill}{rgb}{0.000000,0.000000,0.000000}%
\pgfsetfillcolor{currentfill}%
\pgfsetlinewidth{0.803000pt}%
\definecolor{currentstroke}{rgb}{0.000000,0.000000,0.000000}%
\pgfsetstrokecolor{currentstroke}%
\pgfsetdash{}{0pt}%
\pgfsys@defobject{currentmarker}{\pgfqpoint{0.000000in}{-0.048611in}}{\pgfqpoint{0.000000in}{0.000000in}}{%
\pgfpathmoveto{\pgfqpoint{0.000000in}{0.000000in}}%
\pgfpathlineto{\pgfqpoint{0.000000in}{-0.048611in}}%
\pgfusepath{stroke,fill}%
}%
\begin{pgfscope}%
\pgfsys@transformshift{1.015921in}{0.220000in}%
\pgfsys@useobject{currentmarker}{}%
\end{pgfscope}%
\end{pgfscope}%
\begin{pgfscope}%
\definecolor{textcolor}{rgb}{0.000000,0.000000,0.000000}%
\pgfsetstrokecolor{textcolor}%
\pgfsetfillcolor{textcolor}%
\pgftext[x=1.015921in,y=0.122778in,,top]{\color{textcolor}{\rmfamily\fontsize{10.000000}{12.000000}\selectfont\catcode`\^=\active\def^{\ifmmode\sp\else\^{}\fi}\catcode`\%=\active\def%{\%}$\mathdefault{0.10}$}}%
\end{pgfscope}%
\begin{pgfscope}%
\pgfsetbuttcap%
\pgfsetroundjoin%
\definecolor{currentfill}{rgb}{0.000000,0.000000,0.000000}%
\pgfsetfillcolor{currentfill}%
\pgfsetlinewidth{0.803000pt}%
\definecolor{currentstroke}{rgb}{0.000000,0.000000,0.000000}%
\pgfsetstrokecolor{currentstroke}%
\pgfsetdash{}{0pt}%
\pgfsys@defobject{currentmarker}{\pgfqpoint{0.000000in}{-0.048611in}}{\pgfqpoint{0.000000in}{0.000000in}}{%
\pgfpathmoveto{\pgfqpoint{0.000000in}{0.000000in}}%
\pgfpathlineto{\pgfqpoint{0.000000in}{-0.048611in}}%
\pgfusepath{stroke,fill}%
}%
\begin{pgfscope}%
\pgfsys@transformshift{2.222571in}{0.220000in}%
\pgfsys@useobject{currentmarker}{}%
\end{pgfscope}%
\end{pgfscope}%
\begin{pgfscope}%
\definecolor{textcolor}{rgb}{0.000000,0.000000,0.000000}%
\pgfsetstrokecolor{textcolor}%
\pgfsetfillcolor{textcolor}%
\pgftext[x=2.222571in,y=0.122778in,,top]{\color{textcolor}{\rmfamily\fontsize{10.000000}{12.000000}\selectfont\catcode`\^=\active\def^{\ifmmode\sp\else\^{}\fi}\catcode`\%=\active\def%{\%}$\mathdefault{0.12}$}}%
\end{pgfscope}%
\begin{pgfscope}%
\pgfsetbuttcap%
\pgfsetroundjoin%
\definecolor{currentfill}{rgb}{0.000000,0.000000,0.000000}%
\pgfsetfillcolor{currentfill}%
\pgfsetlinewidth{0.803000pt}%
\definecolor{currentstroke}{rgb}{0.000000,0.000000,0.000000}%
\pgfsetstrokecolor{currentstroke}%
\pgfsetdash{}{0pt}%
\pgfsys@defobject{currentmarker}{\pgfqpoint{0.000000in}{-0.048611in}}{\pgfqpoint{0.000000in}{0.000000in}}{%
\pgfpathmoveto{\pgfqpoint{0.000000in}{0.000000in}}%
\pgfpathlineto{\pgfqpoint{0.000000in}{-0.048611in}}%
\pgfusepath{stroke,fill}%
}%
\begin{pgfscope}%
\pgfsys@transformshift{3.429221in}{0.220000in}%
\pgfsys@useobject{currentmarker}{}%
\end{pgfscope}%
\end{pgfscope}%
\begin{pgfscope}%
\definecolor{textcolor}{rgb}{0.000000,0.000000,0.000000}%
\pgfsetstrokecolor{textcolor}%
\pgfsetfillcolor{textcolor}%
\pgftext[x=3.429221in,y=0.122778in,,top]{\color{textcolor}{\rmfamily\fontsize{10.000000}{12.000000}\selectfont\catcode`\^=\active\def^{\ifmmode\sp\else\^{}\fi}\catcode`\%=\active\def%{\%}$\mathdefault{0.14}$}}%
\end{pgfscope}%
\begin{pgfscope}%
\pgfsetbuttcap%
\pgfsetroundjoin%
\definecolor{currentfill}{rgb}{0.000000,0.000000,0.000000}%
\pgfsetfillcolor{currentfill}%
\pgfsetlinewidth{0.803000pt}%
\definecolor{currentstroke}{rgb}{0.000000,0.000000,0.000000}%
\pgfsetstrokecolor{currentstroke}%
\pgfsetdash{}{0pt}%
\pgfsys@defobject{currentmarker}{\pgfqpoint{0.000000in}{-0.048611in}}{\pgfqpoint{0.000000in}{0.000000in}}{%
\pgfpathmoveto{\pgfqpoint{0.000000in}{0.000000in}}%
\pgfpathlineto{\pgfqpoint{0.000000in}{-0.048611in}}%
\pgfusepath{stroke,fill}%
}%
\begin{pgfscope}%
\pgfsys@transformshift{4.635870in}{0.220000in}%
\pgfsys@useobject{currentmarker}{}%
\end{pgfscope}%
\end{pgfscope}%
\begin{pgfscope}%
\definecolor{textcolor}{rgb}{0.000000,0.000000,0.000000}%
\pgfsetstrokecolor{textcolor}%
\pgfsetfillcolor{textcolor}%
\pgftext[x=4.635870in,y=0.122778in,,top]{\color{textcolor}{\rmfamily\fontsize{10.000000}{12.000000}\selectfont\catcode`\^=\active\def^{\ifmmode\sp\else\^{}\fi}\catcode`\%=\active\def%{\%}$\mathdefault{0.16}$}}%
\end{pgfscope}%
\begin{pgfscope}%
\pgfsetbuttcap%
\pgfsetroundjoin%
\definecolor{currentfill}{rgb}{0.000000,0.000000,0.000000}%
\pgfsetfillcolor{currentfill}%
\pgfsetlinewidth{0.803000pt}%
\definecolor{currentstroke}{rgb}{0.000000,0.000000,0.000000}%
\pgfsetstrokecolor{currentstroke}%
\pgfsetdash{}{0pt}%
\pgfsys@defobject{currentmarker}{\pgfqpoint{0.000000in}{-0.048611in}}{\pgfqpoint{0.000000in}{0.000000in}}{%
\pgfpathmoveto{\pgfqpoint{0.000000in}{0.000000in}}%
\pgfpathlineto{\pgfqpoint{0.000000in}{-0.048611in}}%
\pgfusepath{stroke,fill}%
}%
\begin{pgfscope}%
\pgfsys@transformshift{5.842520in}{0.220000in}%
\pgfsys@useobject{currentmarker}{}%
\end{pgfscope}%
\end{pgfscope}%
\begin{pgfscope}%
\definecolor{textcolor}{rgb}{0.000000,0.000000,0.000000}%
\pgfsetstrokecolor{textcolor}%
\pgfsetfillcolor{textcolor}%
\pgftext[x=5.842520in,y=0.122778in,,top]{\color{textcolor}{\rmfamily\fontsize{10.000000}{12.000000}\selectfont\catcode`\^=\active\def^{\ifmmode\sp\else\^{}\fi}\catcode`\%=\active\def%{\%}$\mathdefault{0.18}$}}%
\end{pgfscope}%
\begin{pgfscope}%
\definecolor{textcolor}{rgb}{0.000000,0.000000,0.000000}%
\pgfsetstrokecolor{textcolor}%
\pgfsetfillcolor{textcolor}%
\pgftext[x=3.670551in,y=-0.056234in,,top]{\color{textcolor}{\rmfamily\fontsize{10.000000}{12.000000}\selectfont\catcode`\^=\active\def^{\ifmmode\sp\else\^{}\fi}\catcode`\%=\active\def%{\%}Time (s)}}%
\end{pgfscope}%
\begin{pgfscope}%
\pgfsetbuttcap%
\pgfsetroundjoin%
\definecolor{currentfill}{rgb}{0.000000,0.000000,0.000000}%
\pgfsetfillcolor{currentfill}%
\pgfsetlinewidth{0.803000pt}%
\definecolor{currentstroke}{rgb}{0.000000,0.000000,0.000000}%
\pgfsetstrokecolor{currentstroke}%
\pgfsetdash{}{0pt}%
\pgfsys@defobject{currentmarker}{\pgfqpoint{-0.048611in}{0.000000in}}{\pgfqpoint{-0.000000in}{0.000000in}}{%
\pgfpathmoveto{\pgfqpoint{-0.000000in}{0.000000in}}%
\pgfpathlineto{\pgfqpoint{-0.048611in}{0.000000in}}%
\pgfusepath{stroke,fill}%
}%
\begin{pgfscope}%
\pgfsys@transformshift{0.895256in}{0.290000in}%
\pgfsys@useobject{currentmarker}{}%
\end{pgfscope}%
\end{pgfscope}%
\begin{pgfscope}%
\definecolor{textcolor}{rgb}{0.000000,0.000000,0.000000}%
\pgfsetstrokecolor{textcolor}%
\pgfsetfillcolor{textcolor}%
\pgftext[x=0.512539in, y=0.241775in, left, base]{\color{textcolor}{\rmfamily\fontsize{10.000000}{12.000000}\selectfont\catcode`\^=\active\def^{\ifmmode\sp\else\^{}\fi}\catcode`\%=\active\def%{\%}$\mathdefault{\ensuremath{-}1.0}$}}%
\end{pgfscope}%
\begin{pgfscope}%
\pgfsetbuttcap%
\pgfsetroundjoin%
\definecolor{currentfill}{rgb}{0.000000,0.000000,0.000000}%
\pgfsetfillcolor{currentfill}%
\pgfsetlinewidth{0.803000pt}%
\definecolor{currentstroke}{rgb}{0.000000,0.000000,0.000000}%
\pgfsetstrokecolor{currentstroke}%
\pgfsetdash{}{0pt}%
\pgfsys@defobject{currentmarker}{\pgfqpoint{-0.048611in}{0.000000in}}{\pgfqpoint{-0.000000in}{0.000000in}}{%
\pgfpathmoveto{\pgfqpoint{-0.000000in}{0.000000in}}%
\pgfpathlineto{\pgfqpoint{-0.048611in}{0.000000in}}%
\pgfusepath{stroke,fill}%
}%
\begin{pgfscope}%
\pgfsys@transformshift{0.895256in}{0.640000in}%
\pgfsys@useobject{currentmarker}{}%
\end{pgfscope}%
\end{pgfscope}%
\begin{pgfscope}%
\definecolor{textcolor}{rgb}{0.000000,0.000000,0.000000}%
\pgfsetstrokecolor{textcolor}%
\pgfsetfillcolor{textcolor}%
\pgftext[x=0.512539in, y=0.591775in, left, base]{\color{textcolor}{\rmfamily\fontsize{10.000000}{12.000000}\selectfont\catcode`\^=\active\def^{\ifmmode\sp\else\^{}\fi}\catcode`\%=\active\def%{\%}$\mathdefault{\ensuremath{-}0.5}$}}%
\end{pgfscope}%
\begin{pgfscope}%
\pgfsetbuttcap%
\pgfsetroundjoin%
\definecolor{currentfill}{rgb}{0.000000,0.000000,0.000000}%
\pgfsetfillcolor{currentfill}%
\pgfsetlinewidth{0.803000pt}%
\definecolor{currentstroke}{rgb}{0.000000,0.000000,0.000000}%
\pgfsetstrokecolor{currentstroke}%
\pgfsetdash{}{0pt}%
\pgfsys@defobject{currentmarker}{\pgfqpoint{-0.048611in}{0.000000in}}{\pgfqpoint{-0.000000in}{0.000000in}}{%
\pgfpathmoveto{\pgfqpoint{-0.000000in}{0.000000in}}%
\pgfpathlineto{\pgfqpoint{-0.048611in}{0.000000in}}%
\pgfusepath{stroke,fill}%
}%
\begin{pgfscope}%
\pgfsys@transformshift{0.895256in}{0.990000in}%
\pgfsys@useobject{currentmarker}{}%
\end{pgfscope}%
\end{pgfscope}%
\begin{pgfscope}%
\definecolor{textcolor}{rgb}{0.000000,0.000000,0.000000}%
\pgfsetstrokecolor{textcolor}%
\pgfsetfillcolor{textcolor}%
\pgftext[x=0.620564in, y=0.941775in, left, base]{\color{textcolor}{\rmfamily\fontsize{10.000000}{12.000000}\selectfont\catcode`\^=\active\def^{\ifmmode\sp\else\^{}\fi}\catcode`\%=\active\def%{\%}$\mathdefault{0.0}$}}%
\end{pgfscope}%
\begin{pgfscope}%
\pgfsetbuttcap%
\pgfsetroundjoin%
\definecolor{currentfill}{rgb}{0.000000,0.000000,0.000000}%
\pgfsetfillcolor{currentfill}%
\pgfsetlinewidth{0.803000pt}%
\definecolor{currentstroke}{rgb}{0.000000,0.000000,0.000000}%
\pgfsetstrokecolor{currentstroke}%
\pgfsetdash{}{0pt}%
\pgfsys@defobject{currentmarker}{\pgfqpoint{-0.048611in}{0.000000in}}{\pgfqpoint{-0.000000in}{0.000000in}}{%
\pgfpathmoveto{\pgfqpoint{-0.000000in}{0.000000in}}%
\pgfpathlineto{\pgfqpoint{-0.048611in}{0.000000in}}%
\pgfusepath{stroke,fill}%
}%
\begin{pgfscope}%
\pgfsys@transformshift{0.895256in}{1.340000in}%
\pgfsys@useobject{currentmarker}{}%
\end{pgfscope}%
\end{pgfscope}%
\begin{pgfscope}%
\definecolor{textcolor}{rgb}{0.000000,0.000000,0.000000}%
\pgfsetstrokecolor{textcolor}%
\pgfsetfillcolor{textcolor}%
\pgftext[x=0.620564in, y=1.291775in, left, base]{\color{textcolor}{\rmfamily\fontsize{10.000000}{12.000000}\selectfont\catcode`\^=\active\def^{\ifmmode\sp\else\^{}\fi}\catcode`\%=\active\def%{\%}$\mathdefault{0.5}$}}%
\end{pgfscope}%
\begin{pgfscope}%
\pgfsetbuttcap%
\pgfsetroundjoin%
\definecolor{currentfill}{rgb}{0.000000,0.000000,0.000000}%
\pgfsetfillcolor{currentfill}%
\pgfsetlinewidth{0.803000pt}%
\definecolor{currentstroke}{rgb}{0.000000,0.000000,0.000000}%
\pgfsetstrokecolor{currentstroke}%
\pgfsetdash{}{0pt}%
\pgfsys@defobject{currentmarker}{\pgfqpoint{-0.048611in}{0.000000in}}{\pgfqpoint{-0.000000in}{0.000000in}}{%
\pgfpathmoveto{\pgfqpoint{-0.000000in}{0.000000in}}%
\pgfpathlineto{\pgfqpoint{-0.048611in}{0.000000in}}%
\pgfusepath{stroke,fill}%
}%
\begin{pgfscope}%
\pgfsys@transformshift{0.895256in}{1.690000in}%
\pgfsys@useobject{currentmarker}{}%
\end{pgfscope}%
\end{pgfscope}%
\begin{pgfscope}%
\definecolor{textcolor}{rgb}{0.000000,0.000000,0.000000}%
\pgfsetstrokecolor{textcolor}%
\pgfsetfillcolor{textcolor}%
\pgftext[x=0.620564in, y=1.641775in, left, base]{\color{textcolor}{\rmfamily\fontsize{10.000000}{12.000000}\selectfont\catcode`\^=\active\def^{\ifmmode\sp\else\^{}\fi}\catcode`\%=\active\def%{\%}$\mathdefault{1.0}$}}%
\end{pgfscope}%
\begin{pgfscope}%
\definecolor{textcolor}{rgb}{0.000000,0.000000,0.000000}%
\pgfsetstrokecolor{textcolor}%
\pgfsetfillcolor{textcolor}%
\pgftext[x=0.456984in,y=0.990000in,,bottom,rotate=90.000000]{\color{textcolor}{\rmfamily\fontsize{10.000000}{12.000000}\selectfont\catcode`\^=\active\def^{\ifmmode\sp\else\^{}\fi}\catcode`\%=\active\def%{\%}Amplitude}}%
\end{pgfscope}%
\begin{pgfscope}%
\pgfpathrectangle{\pgfqpoint{0.895256in}{0.220000in}}{\pgfqpoint{5.550589in}{1.540000in}}%
\pgfusepath{clip}%
\pgfsetrectcap%
\pgfsetroundjoin%
\pgfsetlinewidth{1.505625pt}%
\definecolor{currentstroke}{rgb}{0.121569,0.466667,0.705882}%
\pgfsetstrokecolor{currentstroke}%
\pgfsetdash{}{0pt}%
\pgfpathmoveto{\pgfqpoint{0.893999in}{0.990000in}}%
\pgfpathlineto{\pgfqpoint{1.015921in}{0.990000in}}%
\pgfpathlineto{\pgfqpoint{1.029747in}{1.404399in}}%
\pgfpathlineto{\pgfqpoint{1.037289in}{1.571008in}}%
\pgfpathlineto{\pgfqpoint{1.042317in}{1.644796in}}%
\pgfpathlineto{\pgfqpoint{1.046087in}{1.677588in}}%
\pgfpathlineto{\pgfqpoint{1.048601in}{1.688120in}}%
\pgfpathlineto{\pgfqpoint{1.049858in}{1.689893in}}%
\pgfpathlineto{\pgfqpoint{1.051115in}{1.689380in}}%
\pgfpathlineto{\pgfqpoint{1.053629in}{1.681369in}}%
\pgfpathlineto{\pgfqpoint{1.056143in}{1.664194in}}%
\pgfpathlineto{\pgfqpoint{1.059914in}{1.621809in}}%
\pgfpathlineto{\pgfqpoint{1.064941in}{1.536294in}}%
\pgfpathlineto{\pgfqpoint{1.071226in}{1.389487in}}%
\pgfpathlineto{\pgfqpoint{1.081281in}{1.092243in}}%
\pgfpathlineto{\pgfqpoint{1.098878in}{0.560968in}}%
\pgfpathlineto{\pgfqpoint{1.106420in}{0.398972in}}%
\pgfpathlineto{\pgfqpoint{1.111448in}{0.328945in}}%
\pgfpathlineto{\pgfqpoint{1.115218in}{0.299207in}}%
\pgfpathlineto{\pgfqpoint{1.117732in}{0.290790in}}%
\pgfpathlineto{\pgfqpoint{1.118989in}{0.290043in}}%
\pgfpathlineto{\pgfqpoint{1.120246in}{0.291624in}}%
\pgfpathlineto{\pgfqpoint{1.122760in}{0.301728in}}%
\pgfpathlineto{\pgfqpoint{1.126531in}{0.333922in}}%
\pgfpathlineto{\pgfqpoint{1.131558in}{0.406962in}}%
\pgfpathlineto{\pgfqpoint{1.137843in}{0.540993in}}%
\pgfpathlineto{\pgfqpoint{1.146642in}{0.787693in}}%
\pgfpathlineto{\pgfqpoint{1.170523in}{1.492670in}}%
\pgfpathlineto{\pgfqpoint{1.176808in}{1.610338in}}%
\pgfpathlineto{\pgfqpoint{1.181836in}{1.667996in}}%
\pgfpathlineto{\pgfqpoint{1.185606in}{1.687821in}}%
\pgfpathlineto{\pgfqpoint{1.186863in}{1.689829in}}%
\pgfpathlineto{\pgfqpoint{1.188120in}{1.689509in}}%
\pgfpathlineto{\pgfqpoint{1.190634in}{1.681925in}}%
\pgfpathlineto{\pgfqpoint{1.193148in}{1.665176in}}%
\pgfpathlineto{\pgfqpoint{1.196919in}{1.623369in}}%
\pgfpathlineto{\pgfqpoint{1.201946in}{1.538579in}}%
\pgfpathlineto{\pgfqpoint{1.208231in}{1.392499in}}%
\pgfpathlineto{\pgfqpoint{1.218286in}{1.095875in}}%
\pgfpathlineto{\pgfqpoint{1.235883in}{0.563873in}}%
\pgfpathlineto{\pgfqpoint{1.243425in}{0.400959in}}%
\pgfpathlineto{\pgfqpoint{1.248453in}{0.330162in}}%
\pgfpathlineto{\pgfqpoint{1.252223in}{0.299806in}}%
\pgfpathlineto{\pgfqpoint{1.254737in}{0.290961in}}%
\pgfpathlineto{\pgfqpoint{1.255994in}{0.290021in}}%
\pgfpathlineto{\pgfqpoint{1.257251in}{0.291389in}}%
\pgfpathlineto{\pgfqpoint{1.259765in}{0.301066in}}%
\pgfpathlineto{\pgfqpoint{1.263536in}{0.332640in}}%
\pgfpathlineto{\pgfqpoint{1.268564in}{0.404954in}}%
\pgfpathlineto{\pgfqpoint{1.274848in}{0.538195in}}%
\pgfpathlineto{\pgfqpoint{1.283647in}{0.784189in}}%
\pgfpathlineto{\pgfqpoint{1.307528in}{1.490128in}}%
\pgfpathlineto{\pgfqpoint{1.313813in}{1.608628in}}%
\pgfpathlineto{\pgfqpoint{1.318841in}{1.667078in}}%
\pgfpathlineto{\pgfqpoint{1.322611in}{1.687543in}}%
\pgfpathlineto{\pgfqpoint{1.323868in}{1.689744in}}%
\pgfpathlineto{\pgfqpoint{1.325125in}{1.689637in}}%
\pgfpathlineto{\pgfqpoint{1.327639in}{1.682480in}}%
\pgfpathlineto{\pgfqpoint{1.330153in}{1.666138in}}%
\pgfpathlineto{\pgfqpoint{1.333924in}{1.624928in}}%
\pgfpathlineto{\pgfqpoint{1.338951in}{1.540844in}}%
\pgfpathlineto{\pgfqpoint{1.345236in}{1.395490in}}%
\pgfpathlineto{\pgfqpoint{1.355291in}{1.099485in}}%
\pgfpathlineto{\pgfqpoint{1.372888in}{0.566800in}}%
\pgfpathlineto{\pgfqpoint{1.380430in}{0.402946in}}%
\pgfpathlineto{\pgfqpoint{1.385458in}{0.331401in}}%
\pgfpathlineto{\pgfqpoint{1.389228in}{0.300425in}}%
\pgfpathlineto{\pgfqpoint{1.391742in}{0.291175in}}%
\pgfpathlineto{\pgfqpoint{1.392999in}{0.290000in}}%
\pgfpathlineto{\pgfqpoint{1.394256in}{0.291175in}}%
\pgfpathlineto{\pgfqpoint{1.396770in}{0.300425in}}%
\pgfpathlineto{\pgfqpoint{1.400541in}{0.331401in}}%
\pgfpathlineto{\pgfqpoint{1.405569in}{0.402946in}}%
\pgfpathlineto{\pgfqpoint{1.411853in}{0.535396in}}%
\pgfpathlineto{\pgfqpoint{1.420652in}{0.780686in}}%
\pgfpathlineto{\pgfqpoint{1.444533in}{1.487543in}}%
\pgfpathlineto{\pgfqpoint{1.450818in}{1.606898in}}%
\pgfpathlineto{\pgfqpoint{1.455846in}{1.666138in}}%
\pgfpathlineto{\pgfqpoint{1.459616in}{1.687223in}}%
\pgfpathlineto{\pgfqpoint{1.462130in}{1.689744in}}%
\pgfpathlineto{\pgfqpoint{1.463387in}{1.687543in}}%
\pgfpathlineto{\pgfqpoint{1.465901in}{1.676178in}}%
\pgfpathlineto{\pgfqpoint{1.469672in}{1.642168in}}%
\pgfpathlineto{\pgfqpoint{1.474700in}{1.566885in}}%
\pgfpathlineto{\pgfqpoint{1.480984in}{1.430504in}}%
\pgfpathlineto{\pgfqpoint{1.489783in}{1.181754in}}%
\pgfpathlineto{\pgfqpoint{1.512407in}{0.508159in}}%
\pgfpathlineto{\pgfqpoint{1.518692in}{0.383783in}}%
\pgfpathlineto{\pgfqpoint{1.523720in}{0.319887in}}%
\pgfpathlineto{\pgfqpoint{1.527490in}{0.295084in}}%
\pgfpathlineto{\pgfqpoint{1.530004in}{0.290021in}}%
\pgfpathlineto{\pgfqpoint{1.531261in}{0.290961in}}%
\pgfpathlineto{\pgfqpoint{1.533775in}{0.299806in}}%
\pgfpathlineto{\pgfqpoint{1.537546in}{0.330162in}}%
\pgfpathlineto{\pgfqpoint{1.542574in}{0.400959in}}%
\pgfpathlineto{\pgfqpoint{1.548858in}{0.532619in}}%
\pgfpathlineto{\pgfqpoint{1.557657in}{0.777182in}}%
\pgfpathlineto{\pgfqpoint{1.581538in}{1.484958in}}%
\pgfpathlineto{\pgfqpoint{1.587823in}{1.605168in}}%
\pgfpathlineto{\pgfqpoint{1.592851in}{1.665176in}}%
\pgfpathlineto{\pgfqpoint{1.596621in}{1.686881in}}%
\pgfpathlineto{\pgfqpoint{1.599135in}{1.689829in}}%
\pgfpathlineto{\pgfqpoint{1.600392in}{1.687821in}}%
\pgfpathlineto{\pgfqpoint{1.602906in}{1.676904in}}%
\pgfpathlineto{\pgfqpoint{1.606677in}{1.643493in}}%
\pgfpathlineto{\pgfqpoint{1.611705in}{1.568936in}}%
\pgfpathlineto{\pgfqpoint{1.617989in}{1.433345in}}%
\pgfpathlineto{\pgfqpoint{1.626788in}{1.185279in}}%
\pgfpathlineto{\pgfqpoint{1.649412in}{0.510829in}}%
\pgfpathlineto{\pgfqpoint{1.655697in}{0.385642in}}%
\pgfpathlineto{\pgfqpoint{1.660725in}{0.320955in}}%
\pgfpathlineto{\pgfqpoint{1.664495in}{0.295533in}}%
\pgfpathlineto{\pgfqpoint{1.667009in}{0.290043in}}%
\pgfpathlineto{\pgfqpoint{1.668266in}{0.290790in}}%
\pgfpathlineto{\pgfqpoint{1.670780in}{0.299207in}}%
\pgfpathlineto{\pgfqpoint{1.674551in}{0.328945in}}%
\pgfpathlineto{\pgfqpoint{1.679579in}{0.398972in}}%
\pgfpathlineto{\pgfqpoint{1.685863in}{0.529842in}}%
\pgfpathlineto{\pgfqpoint{1.694662in}{0.773700in}}%
\pgfpathlineto{\pgfqpoint{1.718543in}{1.482373in}}%
\pgfpathlineto{\pgfqpoint{1.724828in}{1.603395in}}%
\pgfpathlineto{\pgfqpoint{1.729856in}{1.664194in}}%
\pgfpathlineto{\pgfqpoint{1.733626in}{1.686518in}}%
\pgfpathlineto{\pgfqpoint{1.736140in}{1.689893in}}%
\pgfpathlineto{\pgfqpoint{1.737397in}{1.688120in}}%
\pgfpathlineto{\pgfqpoint{1.739911in}{1.677588in}}%
\pgfpathlineto{\pgfqpoint{1.743682in}{1.644796in}}%
\pgfpathlineto{\pgfqpoint{1.748710in}{1.571008in}}%
\pgfpathlineto{\pgfqpoint{1.754994in}{1.436187in}}%
\pgfpathlineto{\pgfqpoint{1.763793in}{1.188804in}}%
\pgfpathlineto{\pgfqpoint{1.786417in}{0.513499in}}%
\pgfpathlineto{\pgfqpoint{1.792702in}{0.387501in}}%
\pgfpathlineto{\pgfqpoint{1.797730in}{0.322044in}}%
\pgfpathlineto{\pgfqpoint{1.801500in}{0.296003in}}%
\pgfpathlineto{\pgfqpoint{1.804014in}{0.290107in}}%
\pgfpathlineto{\pgfqpoint{1.805271in}{0.290620in}}%
\pgfpathlineto{\pgfqpoint{1.807785in}{0.298631in}}%
\pgfpathlineto{\pgfqpoint{1.810299in}{0.315806in}}%
\pgfpathlineto{\pgfqpoint{1.814070in}{0.358191in}}%
\pgfpathlineto{\pgfqpoint{1.819097in}{0.443706in}}%
\pgfpathlineto{\pgfqpoint{1.825382in}{0.590513in}}%
\pgfpathlineto{\pgfqpoint{1.835437in}{0.887757in}}%
\pgfpathlineto{\pgfqpoint{1.853034in}{1.419032in}}%
\pgfpathlineto{\pgfqpoint{1.860576in}{1.581028in}}%
\pgfpathlineto{\pgfqpoint{1.865604in}{1.651055in}}%
\pgfpathlineto{\pgfqpoint{1.869375in}{1.680793in}}%
\pgfpathlineto{\pgfqpoint{1.871888in}{1.689210in}}%
\pgfpathlineto{\pgfqpoint{1.873145in}{1.689957in}}%
\pgfpathlineto{\pgfqpoint{1.874402in}{1.688376in}}%
\pgfpathlineto{\pgfqpoint{1.876916in}{1.678272in}}%
\pgfpathlineto{\pgfqpoint{1.880687in}{1.646078in}}%
\pgfpathlineto{\pgfqpoint{1.885715in}{1.573038in}}%
\pgfpathlineto{\pgfqpoint{1.891999in}{1.439007in}}%
\pgfpathlineto{\pgfqpoint{1.900798in}{1.192307in}}%
\pgfpathlineto{\pgfqpoint{1.924679in}{0.487330in}}%
\pgfpathlineto{\pgfqpoint{1.930964in}{0.369662in}}%
\pgfpathlineto{\pgfqpoint{1.935992in}{0.312004in}}%
\pgfpathlineto{\pgfqpoint{1.939762in}{0.292179in}}%
\pgfpathlineto{\pgfqpoint{1.941019in}{0.290171in}}%
\pgfpathlineto{\pgfqpoint{1.942276in}{0.290491in}}%
\pgfpathlineto{\pgfqpoint{1.944790in}{0.298075in}}%
\pgfpathlineto{\pgfqpoint{1.947304in}{0.314824in}}%
\pgfpathlineto{\pgfqpoint{1.951075in}{0.356631in}}%
\pgfpathlineto{\pgfqpoint{1.956102in}{0.441421in}}%
\pgfpathlineto{\pgfqpoint{1.962387in}{0.587501in}}%
\pgfpathlineto{\pgfqpoint{1.972443in}{0.884125in}}%
\pgfpathlineto{\pgfqpoint{1.990039in}{1.416127in}}%
\pgfpathlineto{\pgfqpoint{1.997581in}{1.579041in}}%
\pgfpathlineto{\pgfqpoint{2.002609in}{1.649838in}}%
\pgfpathlineto{\pgfqpoint{2.006380in}{1.680194in}}%
\pgfpathlineto{\pgfqpoint{2.008893in}{1.689039in}}%
\pgfpathlineto{\pgfqpoint{2.010150in}{1.689979in}}%
\pgfpathlineto{\pgfqpoint{2.011407in}{1.688611in}}%
\pgfpathlineto{\pgfqpoint{2.013921in}{1.678934in}}%
\pgfpathlineto{\pgfqpoint{2.017692in}{1.647360in}}%
\pgfpathlineto{\pgfqpoint{2.022720in}{1.575046in}}%
\pgfpathlineto{\pgfqpoint{2.029004in}{1.441805in}}%
\pgfpathlineto{\pgfqpoint{2.037803in}{1.195811in}}%
\pgfpathlineto{\pgfqpoint{2.061684in}{0.489872in}}%
\pgfpathlineto{\pgfqpoint{2.067969in}{0.371372in}}%
\pgfpathlineto{\pgfqpoint{2.072997in}{0.312922in}}%
\pgfpathlineto{\pgfqpoint{2.076767in}{0.292457in}}%
\pgfpathlineto{\pgfqpoint{2.078024in}{0.290256in}}%
\pgfpathlineto{\pgfqpoint{2.079281in}{0.290363in}}%
\pgfpathlineto{\pgfqpoint{2.081795in}{0.297520in}}%
\pgfpathlineto{\pgfqpoint{2.084309in}{0.313862in}}%
\pgfpathlineto{\pgfqpoint{2.088080in}{0.355072in}}%
\pgfpathlineto{\pgfqpoint{2.093107in}{0.439156in}}%
\pgfpathlineto{\pgfqpoint{2.099392in}{0.584510in}}%
\pgfpathlineto{\pgfqpoint{2.109448in}{0.880515in}}%
\pgfpathlineto{\pgfqpoint{2.127045in}{1.413200in}}%
\pgfpathlineto{\pgfqpoint{2.134586in}{1.577054in}}%
\pgfpathlineto{\pgfqpoint{2.139614in}{1.648599in}}%
\pgfpathlineto{\pgfqpoint{2.143385in}{1.679575in}}%
\pgfpathlineto{\pgfqpoint{2.145898in}{1.688825in}}%
\pgfpathlineto{\pgfqpoint{2.147155in}{1.690000in}}%
\pgfpathlineto{\pgfqpoint{2.148412in}{1.688825in}}%
\pgfpathlineto{\pgfqpoint{2.150926in}{1.679575in}}%
\pgfpathlineto{\pgfqpoint{2.154697in}{1.648599in}}%
\pgfpathlineto{\pgfqpoint{2.159725in}{1.577054in}}%
\pgfpathlineto{\pgfqpoint{2.166009in}{1.444604in}}%
\pgfpathlineto{\pgfqpoint{2.174808in}{1.199314in}}%
\pgfpathlineto{\pgfqpoint{2.198689in}{0.492457in}}%
\pgfpathlineto{\pgfqpoint{2.204974in}{0.373102in}}%
\pgfpathlineto{\pgfqpoint{2.210002in}{0.313862in}}%
\pgfpathlineto{\pgfqpoint{2.213772in}{0.292777in}}%
\pgfpathlineto{\pgfqpoint{2.216286in}{0.290256in}}%
\pgfpathlineto{\pgfqpoint{2.217543in}{0.292457in}}%
\pgfpathlineto{\pgfqpoint{2.220057in}{0.303822in}}%
\pgfpathlineto{\pgfqpoint{2.223828in}{0.337832in}}%
\pgfpathlineto{\pgfqpoint{2.228856in}{0.413115in}}%
\pgfpathlineto{\pgfqpoint{2.235140in}{0.549496in}}%
\pgfpathlineto{\pgfqpoint{2.243939in}{0.798246in}}%
\pgfpathlineto{\pgfqpoint{2.266563in}{1.471841in}}%
\pgfpathlineto{\pgfqpoint{2.272848in}{1.596217in}}%
\pgfpathlineto{\pgfqpoint{2.277876in}{1.660113in}}%
\pgfpathlineto{\pgfqpoint{2.281647in}{1.684916in}}%
\pgfpathlineto{\pgfqpoint{2.284160in}{1.689979in}}%
\pgfpathlineto{\pgfqpoint{2.285417in}{1.689039in}}%
\pgfpathlineto{\pgfqpoint{2.287931in}{1.680194in}}%
\pgfpathlineto{\pgfqpoint{2.291702in}{1.649838in}}%
\pgfpathlineto{\pgfqpoint{2.296730in}{1.579041in}}%
\pgfpathlineto{\pgfqpoint{2.303014in}{1.447381in}}%
\pgfpathlineto{\pgfqpoint{2.311813in}{1.202818in}}%
\pgfpathlineto{\pgfqpoint{2.335694in}{0.495042in}}%
\pgfpathlineto{\pgfqpoint{2.341979in}{0.374832in}}%
\pgfpathlineto{\pgfqpoint{2.347007in}{0.314824in}}%
\pgfpathlineto{\pgfqpoint{2.350777in}{0.293119in}}%
\pgfpathlineto{\pgfqpoint{2.353291in}{0.290171in}}%
\pgfpathlineto{\pgfqpoint{2.354548in}{0.292179in}}%
\pgfpathlineto{\pgfqpoint{2.357062in}{0.303096in}}%
\pgfpathlineto{\pgfqpoint{2.360833in}{0.336507in}}%
\pgfpathlineto{\pgfqpoint{2.365861in}{0.411064in}}%
\pgfpathlineto{\pgfqpoint{2.372145in}{0.546655in}}%
\pgfpathlineto{\pgfqpoint{2.380944in}{0.794721in}}%
\pgfpathlineto{\pgfqpoint{2.403568in}{1.469171in}}%
\pgfpathlineto{\pgfqpoint{2.409853in}{1.594358in}}%
\pgfpathlineto{\pgfqpoint{2.414881in}{1.659045in}}%
\pgfpathlineto{\pgfqpoint{2.418652in}{1.684467in}}%
\pgfpathlineto{\pgfqpoint{2.421165in}{1.689957in}}%
\pgfpathlineto{\pgfqpoint{2.422422in}{1.689210in}}%
\pgfpathlineto{\pgfqpoint{2.424936in}{1.680793in}}%
\pgfpathlineto{\pgfqpoint{2.428707in}{1.651055in}}%
\pgfpathlineto{\pgfqpoint{2.433735in}{1.581028in}}%
\pgfpathlineto{\pgfqpoint{2.440019in}{1.450158in}}%
\pgfpathlineto{\pgfqpoint{2.448818in}{1.206300in}}%
\pgfpathlineto{\pgfqpoint{2.472699in}{0.497627in}}%
\pgfpathlineto{\pgfqpoint{2.478984in}{0.376605in}}%
\pgfpathlineto{\pgfqpoint{2.484012in}{0.315806in}}%
\pgfpathlineto{\pgfqpoint{2.487783in}{0.293482in}}%
\pgfpathlineto{\pgfqpoint{2.490296in}{0.290107in}}%
\pgfpathlineto{\pgfqpoint{2.491553in}{0.291880in}}%
\pgfpathlineto{\pgfqpoint{2.494067in}{0.302412in}}%
\pgfpathlineto{\pgfqpoint{2.497838in}{0.335204in}}%
\pgfpathlineto{\pgfqpoint{2.502866in}{0.408992in}}%
\pgfpathlineto{\pgfqpoint{2.509150in}{0.543813in}}%
\pgfpathlineto{\pgfqpoint{2.517949in}{0.791196in}}%
\pgfpathlineto{\pgfqpoint{2.540573in}{1.466501in}}%
\pgfpathlineto{\pgfqpoint{2.546858in}{1.592499in}}%
\pgfpathlineto{\pgfqpoint{2.551886in}{1.657956in}}%
\pgfpathlineto{\pgfqpoint{2.555657in}{1.683997in}}%
\pgfpathlineto{\pgfqpoint{2.558170in}{1.689893in}}%
\pgfpathlineto{\pgfqpoint{2.559427in}{1.689380in}}%
\pgfpathlineto{\pgfqpoint{2.561941in}{1.681369in}}%
\pgfpathlineto{\pgfqpoint{2.564455in}{1.664194in}}%
\pgfpathlineto{\pgfqpoint{2.568226in}{1.621809in}}%
\pgfpathlineto{\pgfqpoint{2.573254in}{1.536294in}}%
\pgfpathlineto{\pgfqpoint{2.579538in}{1.389487in}}%
\pgfpathlineto{\pgfqpoint{2.589594in}{1.092243in}}%
\pgfpathlineto{\pgfqpoint{2.607191in}{0.560968in}}%
\pgfpathlineto{\pgfqpoint{2.614732in}{0.398972in}}%
\pgfpathlineto{\pgfqpoint{2.619760in}{0.328945in}}%
\pgfpathlineto{\pgfqpoint{2.623531in}{0.299207in}}%
\pgfpathlineto{\pgfqpoint{2.626044in}{0.290790in}}%
\pgfpathlineto{\pgfqpoint{2.627301in}{0.290043in}}%
\pgfpathlineto{\pgfqpoint{2.628558in}{0.291624in}}%
\pgfpathlineto{\pgfqpoint{2.631072in}{0.301728in}}%
\pgfpathlineto{\pgfqpoint{2.634843in}{0.333922in}}%
\pgfpathlineto{\pgfqpoint{2.639871in}{0.406962in}}%
\pgfpathlineto{\pgfqpoint{2.646155in}{0.540993in}}%
\pgfpathlineto{\pgfqpoint{2.654954in}{0.787693in}}%
\pgfpathlineto{\pgfqpoint{2.678835in}{1.492670in}}%
\pgfpathlineto{\pgfqpoint{2.685120in}{1.610338in}}%
\pgfpathlineto{\pgfqpoint{2.690148in}{1.667996in}}%
\pgfpathlineto{\pgfqpoint{2.693919in}{1.687821in}}%
\pgfpathlineto{\pgfqpoint{2.695175in}{1.689829in}}%
\pgfpathlineto{\pgfqpoint{2.696432in}{1.689509in}}%
\pgfpathlineto{\pgfqpoint{2.698946in}{1.681925in}}%
\pgfpathlineto{\pgfqpoint{2.701460in}{1.665176in}}%
\pgfpathlineto{\pgfqpoint{2.705231in}{1.623369in}}%
\pgfpathlineto{\pgfqpoint{2.710259in}{1.538579in}}%
\pgfpathlineto{\pgfqpoint{2.716543in}{1.392499in}}%
\pgfpathlineto{\pgfqpoint{2.726599in}{1.095875in}}%
\pgfpathlineto{\pgfqpoint{2.744196in}{0.563873in}}%
\pgfpathlineto{\pgfqpoint{2.751737in}{0.400959in}}%
\pgfpathlineto{\pgfqpoint{2.756765in}{0.330162in}}%
\pgfpathlineto{\pgfqpoint{2.760536in}{0.299806in}}%
\pgfpathlineto{\pgfqpoint{2.763049in}{0.290961in}}%
\pgfpathlineto{\pgfqpoint{2.764306in}{0.290021in}}%
\pgfpathlineto{\pgfqpoint{2.765563in}{0.291389in}}%
\pgfpathlineto{\pgfqpoint{2.768077in}{0.301066in}}%
\pgfpathlineto{\pgfqpoint{2.771848in}{0.332640in}}%
\pgfpathlineto{\pgfqpoint{2.776876in}{0.404954in}}%
\pgfpathlineto{\pgfqpoint{2.783160in}{0.538195in}}%
\pgfpathlineto{\pgfqpoint{2.791959in}{0.784189in}}%
\pgfpathlineto{\pgfqpoint{2.815840in}{1.490128in}}%
\pgfpathlineto{\pgfqpoint{2.822125in}{1.608628in}}%
\pgfpathlineto{\pgfqpoint{2.827153in}{1.667078in}}%
\pgfpathlineto{\pgfqpoint{2.830924in}{1.687543in}}%
\pgfpathlineto{\pgfqpoint{2.832180in}{1.689744in}}%
\pgfpathlineto{\pgfqpoint{2.833437in}{1.689637in}}%
\pgfpathlineto{\pgfqpoint{2.835951in}{1.682480in}}%
\pgfpathlineto{\pgfqpoint{2.838465in}{1.666138in}}%
\pgfpathlineto{\pgfqpoint{2.842236in}{1.624928in}}%
\pgfpathlineto{\pgfqpoint{2.847264in}{1.540844in}}%
\pgfpathlineto{\pgfqpoint{2.853548in}{1.395490in}}%
\pgfpathlineto{\pgfqpoint{2.863604in}{1.099485in}}%
\pgfpathlineto{\pgfqpoint{2.881201in}{0.566800in}}%
\pgfpathlineto{\pgfqpoint{2.888742in}{0.402946in}}%
\pgfpathlineto{\pgfqpoint{2.893770in}{0.331401in}}%
\pgfpathlineto{\pgfqpoint{2.897541in}{0.300425in}}%
\pgfpathlineto{\pgfqpoint{2.900054in}{0.291175in}}%
\pgfpathlineto{\pgfqpoint{2.901311in}{0.290000in}}%
\pgfpathlineto{\pgfqpoint{2.902568in}{0.291175in}}%
\pgfpathlineto{\pgfqpoint{2.905082in}{0.300425in}}%
\pgfpathlineto{\pgfqpoint{2.908853in}{0.331401in}}%
\pgfpathlineto{\pgfqpoint{2.913881in}{0.402946in}}%
\pgfpathlineto{\pgfqpoint{2.920165in}{0.535396in}}%
\pgfpathlineto{\pgfqpoint{2.928964in}{0.780686in}}%
\pgfpathlineto{\pgfqpoint{2.952845in}{1.487543in}}%
\pgfpathlineto{\pgfqpoint{2.959130in}{1.606898in}}%
\pgfpathlineto{\pgfqpoint{2.964158in}{1.666138in}}%
\pgfpathlineto{\pgfqpoint{2.967929in}{1.687223in}}%
\pgfpathlineto{\pgfqpoint{2.970442in}{1.689744in}}%
\pgfpathlineto{\pgfqpoint{2.971699in}{1.687543in}}%
\pgfpathlineto{\pgfqpoint{2.974213in}{1.676178in}}%
\pgfpathlineto{\pgfqpoint{2.977984in}{1.642168in}}%
\pgfpathlineto{\pgfqpoint{2.983012in}{1.566885in}}%
\pgfpathlineto{\pgfqpoint{2.989296in}{1.430504in}}%
\pgfpathlineto{\pgfqpoint{2.998095in}{1.181754in}}%
\pgfpathlineto{\pgfqpoint{3.020719in}{0.508159in}}%
\pgfpathlineto{\pgfqpoint{3.027004in}{0.383783in}}%
\pgfpathlineto{\pgfqpoint{3.032032in}{0.319887in}}%
\pgfpathlineto{\pgfqpoint{3.035803in}{0.295084in}}%
\pgfpathlineto{\pgfqpoint{3.038316in}{0.290021in}}%
\pgfpathlineto{\pgfqpoint{3.039573in}{0.290961in}}%
\pgfpathlineto{\pgfqpoint{3.042087in}{0.299806in}}%
\pgfpathlineto{\pgfqpoint{3.045858in}{0.330162in}}%
\pgfpathlineto{\pgfqpoint{3.050886in}{0.400959in}}%
\pgfpathlineto{\pgfqpoint{3.057170in}{0.532619in}}%
\pgfpathlineto{\pgfqpoint{3.065969in}{0.777182in}}%
\pgfpathlineto{\pgfqpoint{3.089850in}{1.484958in}}%
\pgfpathlineto{\pgfqpoint{3.096135in}{1.605168in}}%
\pgfpathlineto{\pgfqpoint{3.101163in}{1.665176in}}%
\pgfpathlineto{\pgfqpoint{3.104934in}{1.686881in}}%
\pgfpathlineto{\pgfqpoint{3.107447in}{1.689829in}}%
\pgfpathlineto{\pgfqpoint{3.108704in}{1.687821in}}%
\pgfpathlineto{\pgfqpoint{3.111218in}{1.676904in}}%
\pgfpathlineto{\pgfqpoint{3.114989in}{1.643493in}}%
\pgfpathlineto{\pgfqpoint{3.120017in}{1.568936in}}%
\pgfpathlineto{\pgfqpoint{3.126301in}{1.433345in}}%
\pgfpathlineto{\pgfqpoint{3.135100in}{1.185279in}}%
\pgfpathlineto{\pgfqpoint{3.157724in}{0.510829in}}%
\pgfpathlineto{\pgfqpoint{3.164009in}{0.385642in}}%
\pgfpathlineto{\pgfqpoint{3.169037in}{0.320955in}}%
\pgfpathlineto{\pgfqpoint{3.172808in}{0.295533in}}%
\pgfpathlineto{\pgfqpoint{3.175321in}{0.290043in}}%
\pgfpathlineto{\pgfqpoint{3.176578in}{0.290790in}}%
\pgfpathlineto{\pgfqpoint{3.179092in}{0.299207in}}%
\pgfpathlineto{\pgfqpoint{3.182863in}{0.328945in}}%
\pgfpathlineto{\pgfqpoint{3.187891in}{0.398972in}}%
\pgfpathlineto{\pgfqpoint{3.194175in}{0.529842in}}%
\pgfpathlineto{\pgfqpoint{3.202974in}{0.773700in}}%
\pgfpathlineto{\pgfqpoint{3.226855in}{1.482373in}}%
\pgfpathlineto{\pgfqpoint{3.233140in}{1.603395in}}%
\pgfpathlineto{\pgfqpoint{3.238168in}{1.664194in}}%
\pgfpathlineto{\pgfqpoint{3.241939in}{1.686518in}}%
\pgfpathlineto{\pgfqpoint{3.244452in}{1.689893in}}%
\pgfpathlineto{\pgfqpoint{3.245709in}{1.688120in}}%
\pgfpathlineto{\pgfqpoint{3.248223in}{1.677588in}}%
\pgfpathlineto{\pgfqpoint{3.251994in}{1.644796in}}%
\pgfpathlineto{\pgfqpoint{3.257022in}{1.571008in}}%
\pgfpathlineto{\pgfqpoint{3.263306in}{1.436187in}}%
\pgfpathlineto{\pgfqpoint{3.272105in}{1.188804in}}%
\pgfpathlineto{\pgfqpoint{3.294730in}{0.513499in}}%
\pgfpathlineto{\pgfqpoint{3.301014in}{0.387501in}}%
\pgfpathlineto{\pgfqpoint{3.306042in}{0.322044in}}%
\pgfpathlineto{\pgfqpoint{3.309813in}{0.296003in}}%
\pgfpathlineto{\pgfqpoint{3.312326in}{0.290107in}}%
\pgfpathlineto{\pgfqpoint{3.313583in}{0.290620in}}%
\pgfpathlineto{\pgfqpoint{3.316097in}{0.298631in}}%
\pgfpathlineto{\pgfqpoint{3.318611in}{0.315806in}}%
\pgfpathlineto{\pgfqpoint{3.322382in}{0.358191in}}%
\pgfpathlineto{\pgfqpoint{3.327410in}{0.443706in}}%
\pgfpathlineto{\pgfqpoint{3.333694in}{0.590513in}}%
\pgfpathlineto{\pgfqpoint{3.343750in}{0.887757in}}%
\pgfpathlineto{\pgfqpoint{3.361347in}{1.419032in}}%
\pgfpathlineto{\pgfqpoint{3.368888in}{1.581028in}}%
\pgfpathlineto{\pgfqpoint{3.373916in}{1.651055in}}%
\pgfpathlineto{\pgfqpoint{3.377687in}{1.680793in}}%
\pgfpathlineto{\pgfqpoint{3.380201in}{1.689210in}}%
\pgfpathlineto{\pgfqpoint{3.381457in}{1.689957in}}%
\pgfpathlineto{\pgfqpoint{3.382714in}{1.688376in}}%
\pgfpathlineto{\pgfqpoint{3.385228in}{1.678272in}}%
\pgfpathlineto{\pgfqpoint{3.388999in}{1.646078in}}%
\pgfpathlineto{\pgfqpoint{3.394027in}{1.573038in}}%
\pgfpathlineto{\pgfqpoint{3.400311in}{1.439007in}}%
\pgfpathlineto{\pgfqpoint{3.409110in}{1.192307in}}%
\pgfpathlineto{\pgfqpoint{3.432991in}{0.487330in}}%
\pgfpathlineto{\pgfqpoint{3.439276in}{0.369662in}}%
\pgfpathlineto{\pgfqpoint{3.444304in}{0.312004in}}%
\pgfpathlineto{\pgfqpoint{3.448075in}{0.292179in}}%
\pgfpathlineto{\pgfqpoint{3.449332in}{0.290171in}}%
\pgfpathlineto{\pgfqpoint{3.450588in}{0.290491in}}%
\pgfpathlineto{\pgfqpoint{3.453102in}{0.298075in}}%
\pgfpathlineto{\pgfqpoint{3.455616in}{0.314824in}}%
\pgfpathlineto{\pgfqpoint{3.459387in}{0.356631in}}%
\pgfpathlineto{\pgfqpoint{3.464415in}{0.441421in}}%
\pgfpathlineto{\pgfqpoint{3.470699in}{0.587501in}}%
\pgfpathlineto{\pgfqpoint{3.480755in}{0.884125in}}%
\pgfpathlineto{\pgfqpoint{3.498352in}{1.416127in}}%
\pgfpathlineto{\pgfqpoint{3.505893in}{1.579041in}}%
\pgfpathlineto{\pgfqpoint{3.510921in}{1.649838in}}%
\pgfpathlineto{\pgfqpoint{3.514692in}{1.680194in}}%
\pgfpathlineto{\pgfqpoint{3.517206in}{1.689039in}}%
\pgfpathlineto{\pgfqpoint{3.518462in}{1.689979in}}%
\pgfpathlineto{\pgfqpoint{3.519719in}{1.688611in}}%
\pgfpathlineto{\pgfqpoint{3.522233in}{1.678934in}}%
\pgfpathlineto{\pgfqpoint{3.526004in}{1.647360in}}%
\pgfpathlineto{\pgfqpoint{3.531032in}{1.575046in}}%
\pgfpathlineto{\pgfqpoint{3.537316in}{1.441805in}}%
\pgfpathlineto{\pgfqpoint{3.546115in}{1.195811in}}%
\pgfpathlineto{\pgfqpoint{3.569996in}{0.489872in}}%
\pgfpathlineto{\pgfqpoint{3.576281in}{0.371372in}}%
\pgfpathlineto{\pgfqpoint{3.581309in}{0.312922in}}%
\pgfpathlineto{\pgfqpoint{3.585080in}{0.292457in}}%
\pgfpathlineto{\pgfqpoint{3.586337in}{0.290256in}}%
\pgfpathlineto{\pgfqpoint{3.587593in}{0.290363in}}%
\pgfpathlineto{\pgfqpoint{3.590107in}{0.297520in}}%
\pgfpathlineto{\pgfqpoint{3.592621in}{0.313862in}}%
\pgfpathlineto{\pgfqpoint{3.596392in}{0.355072in}}%
\pgfpathlineto{\pgfqpoint{3.601420in}{0.439156in}}%
\pgfpathlineto{\pgfqpoint{3.607704in}{0.584510in}}%
\pgfpathlineto{\pgfqpoint{3.617760in}{0.880515in}}%
\pgfpathlineto{\pgfqpoint{3.635357in}{1.413200in}}%
\pgfpathlineto{\pgfqpoint{3.642898in}{1.577054in}}%
\pgfpathlineto{\pgfqpoint{3.647926in}{1.648599in}}%
\pgfpathlineto{\pgfqpoint{3.651697in}{1.679575in}}%
\pgfpathlineto{\pgfqpoint{3.654211in}{1.688825in}}%
\pgfpathlineto{\pgfqpoint{3.655468in}{1.690000in}}%
\pgfpathlineto{\pgfqpoint{3.656724in}{1.688825in}}%
\pgfpathlineto{\pgfqpoint{3.659238in}{1.679575in}}%
\pgfpathlineto{\pgfqpoint{3.663009in}{1.648599in}}%
\pgfpathlineto{\pgfqpoint{3.668037in}{1.577054in}}%
\pgfpathlineto{\pgfqpoint{3.674321in}{1.444604in}}%
\pgfpathlineto{\pgfqpoint{3.683120in}{1.199314in}}%
\pgfpathlineto{\pgfqpoint{3.707002in}{0.492457in}}%
\pgfpathlineto{\pgfqpoint{3.713286in}{0.373102in}}%
\pgfpathlineto{\pgfqpoint{3.718314in}{0.313862in}}%
\pgfpathlineto{\pgfqpoint{3.722085in}{0.292777in}}%
\pgfpathlineto{\pgfqpoint{3.724598in}{0.290256in}}%
\pgfpathlineto{\pgfqpoint{3.725855in}{0.292457in}}%
\pgfpathlineto{\pgfqpoint{3.728369in}{0.303822in}}%
\pgfpathlineto{\pgfqpoint{3.732140in}{0.337832in}}%
\pgfpathlineto{\pgfqpoint{3.737168in}{0.413115in}}%
\pgfpathlineto{\pgfqpoint{3.743452in}{0.549496in}}%
\pgfpathlineto{\pgfqpoint{3.752251in}{0.798246in}}%
\pgfpathlineto{\pgfqpoint{3.774876in}{1.471841in}}%
\pgfpathlineto{\pgfqpoint{3.781160in}{1.596217in}}%
\pgfpathlineto{\pgfqpoint{3.786188in}{1.660113in}}%
\pgfpathlineto{\pgfqpoint{3.789959in}{1.684916in}}%
\pgfpathlineto{\pgfqpoint{3.792473in}{1.689979in}}%
\pgfpathlineto{\pgfqpoint{3.793729in}{1.689039in}}%
\pgfpathlineto{\pgfqpoint{3.796243in}{1.680194in}}%
\pgfpathlineto{\pgfqpoint{3.800014in}{1.649838in}}%
\pgfpathlineto{\pgfqpoint{3.805042in}{1.579041in}}%
\pgfpathlineto{\pgfqpoint{3.811326in}{1.447381in}}%
\pgfpathlineto{\pgfqpoint{3.820125in}{1.202818in}}%
\pgfpathlineto{\pgfqpoint{3.844007in}{0.495042in}}%
\pgfpathlineto{\pgfqpoint{3.850291in}{0.374832in}}%
\pgfpathlineto{\pgfqpoint{3.855319in}{0.314824in}}%
\pgfpathlineto{\pgfqpoint{3.859090in}{0.293119in}}%
\pgfpathlineto{\pgfqpoint{3.861603in}{0.290171in}}%
\pgfpathlineto{\pgfqpoint{3.862860in}{0.292179in}}%
\pgfpathlineto{\pgfqpoint{3.865374in}{0.303096in}}%
\pgfpathlineto{\pgfqpoint{3.869145in}{0.336507in}}%
\pgfpathlineto{\pgfqpoint{3.874173in}{0.411064in}}%
\pgfpathlineto{\pgfqpoint{3.880457in}{0.546655in}}%
\pgfpathlineto{\pgfqpoint{3.889256in}{0.794721in}}%
\pgfpathlineto{\pgfqpoint{3.911881in}{1.469171in}}%
\pgfpathlineto{\pgfqpoint{3.918165in}{1.594358in}}%
\pgfpathlineto{\pgfqpoint{3.923193in}{1.659045in}}%
\pgfpathlineto{\pgfqpoint{3.926964in}{1.684467in}}%
\pgfpathlineto{\pgfqpoint{3.929478in}{1.689957in}}%
\pgfpathlineto{\pgfqpoint{3.930734in}{1.689210in}}%
\pgfpathlineto{\pgfqpoint{3.933248in}{1.680793in}}%
\pgfpathlineto{\pgfqpoint{3.937019in}{1.651055in}}%
\pgfpathlineto{\pgfqpoint{3.942047in}{1.581028in}}%
\pgfpathlineto{\pgfqpoint{3.948331in}{1.450158in}}%
\pgfpathlineto{\pgfqpoint{3.957130in}{1.206300in}}%
\pgfpathlineto{\pgfqpoint{3.981012in}{0.497627in}}%
\pgfpathlineto{\pgfqpoint{3.987296in}{0.376605in}}%
\pgfpathlineto{\pgfqpoint{3.992324in}{0.315806in}}%
\pgfpathlineto{\pgfqpoint{3.996095in}{0.293482in}}%
\pgfpathlineto{\pgfqpoint{3.998609in}{0.290107in}}%
\pgfpathlineto{\pgfqpoint{3.999865in}{0.291880in}}%
\pgfpathlineto{\pgfqpoint{4.002379in}{0.302412in}}%
\pgfpathlineto{\pgfqpoint{4.006150in}{0.335204in}}%
\pgfpathlineto{\pgfqpoint{4.011178in}{0.408992in}}%
\pgfpathlineto{\pgfqpoint{4.017462in}{0.543813in}}%
\pgfpathlineto{\pgfqpoint{4.026261in}{0.791196in}}%
\pgfpathlineto{\pgfqpoint{4.048886in}{1.466501in}}%
\pgfpathlineto{\pgfqpoint{4.055170in}{1.592499in}}%
\pgfpathlineto{\pgfqpoint{4.060198in}{1.657956in}}%
\pgfpathlineto{\pgfqpoint{4.063969in}{1.683997in}}%
\pgfpathlineto{\pgfqpoint{4.066483in}{1.689893in}}%
\pgfpathlineto{\pgfqpoint{4.067739in}{1.689380in}}%
\pgfpathlineto{\pgfqpoint{4.070253in}{1.681369in}}%
\pgfpathlineto{\pgfqpoint{4.072767in}{1.664194in}}%
\pgfpathlineto{\pgfqpoint{4.076538in}{1.621809in}}%
\pgfpathlineto{\pgfqpoint{4.081566in}{1.536294in}}%
\pgfpathlineto{\pgfqpoint{4.087850in}{1.389487in}}%
\pgfpathlineto{\pgfqpoint{4.097906in}{1.092243in}}%
\pgfpathlineto{\pgfqpoint{4.115503in}{0.560968in}}%
\pgfpathlineto{\pgfqpoint{4.123044in}{0.398972in}}%
\pgfpathlineto{\pgfqpoint{4.128072in}{0.328945in}}%
\pgfpathlineto{\pgfqpoint{4.131843in}{0.299207in}}%
\pgfpathlineto{\pgfqpoint{4.134357in}{0.290790in}}%
\pgfpathlineto{\pgfqpoint{4.135614in}{0.290043in}}%
\pgfpathlineto{\pgfqpoint{4.136870in}{0.291624in}}%
\pgfpathlineto{\pgfqpoint{4.139384in}{0.301728in}}%
\pgfpathlineto{\pgfqpoint{4.143155in}{0.333922in}}%
\pgfpathlineto{\pgfqpoint{4.148183in}{0.406962in}}%
\pgfpathlineto{\pgfqpoint{4.154467in}{0.540993in}}%
\pgfpathlineto{\pgfqpoint{4.163266in}{0.787693in}}%
\pgfpathlineto{\pgfqpoint{4.187148in}{1.492670in}}%
\pgfpathlineto{\pgfqpoint{4.193432in}{1.610338in}}%
\pgfpathlineto{\pgfqpoint{4.198460in}{1.667996in}}%
\pgfpathlineto{\pgfqpoint{4.202231in}{1.687821in}}%
\pgfpathlineto{\pgfqpoint{4.203488in}{1.689829in}}%
\pgfpathlineto{\pgfqpoint{4.204745in}{1.689509in}}%
\pgfpathlineto{\pgfqpoint{4.207258in}{1.681925in}}%
\pgfpathlineto{\pgfqpoint{4.209772in}{1.665176in}}%
\pgfpathlineto{\pgfqpoint{4.213543in}{1.623369in}}%
\pgfpathlineto{\pgfqpoint{4.218571in}{1.538579in}}%
\pgfpathlineto{\pgfqpoint{4.224855in}{1.392499in}}%
\pgfpathlineto{\pgfqpoint{4.234911in}{1.095875in}}%
\pgfpathlineto{\pgfqpoint{4.252508in}{0.563873in}}%
\pgfpathlineto{\pgfqpoint{4.260049in}{0.400959in}}%
\pgfpathlineto{\pgfqpoint{4.265077in}{0.330162in}}%
\pgfpathlineto{\pgfqpoint{4.268848in}{0.299806in}}%
\pgfpathlineto{\pgfqpoint{4.271362in}{0.290961in}}%
\pgfpathlineto{\pgfqpoint{4.272619in}{0.290021in}}%
\pgfpathlineto{\pgfqpoint{4.273875in}{0.291389in}}%
\pgfpathlineto{\pgfqpoint{4.276389in}{0.301066in}}%
\pgfpathlineto{\pgfqpoint{4.280160in}{0.332640in}}%
\pgfpathlineto{\pgfqpoint{4.285188in}{0.404954in}}%
\pgfpathlineto{\pgfqpoint{4.291472in}{0.538195in}}%
\pgfpathlineto{\pgfqpoint{4.300271in}{0.784189in}}%
\pgfpathlineto{\pgfqpoint{4.324153in}{1.490128in}}%
\pgfpathlineto{\pgfqpoint{4.330437in}{1.608628in}}%
\pgfpathlineto{\pgfqpoint{4.335465in}{1.667078in}}%
\pgfpathlineto{\pgfqpoint{4.339236in}{1.687543in}}%
\pgfpathlineto{\pgfqpoint{4.340493in}{1.689744in}}%
\pgfpathlineto{\pgfqpoint{4.341750in}{1.689637in}}%
\pgfpathlineto{\pgfqpoint{4.344263in}{1.682480in}}%
\pgfpathlineto{\pgfqpoint{4.346777in}{1.666138in}}%
\pgfpathlineto{\pgfqpoint{4.350548in}{1.624928in}}%
\pgfpathlineto{\pgfqpoint{4.355576in}{1.540844in}}%
\pgfpathlineto{\pgfqpoint{4.361860in}{1.395490in}}%
\pgfpathlineto{\pgfqpoint{4.371916in}{1.099485in}}%
\pgfpathlineto{\pgfqpoint{4.389513in}{0.566800in}}%
\pgfpathlineto{\pgfqpoint{4.397054in}{0.402946in}}%
\pgfpathlineto{\pgfqpoint{4.402082in}{0.331401in}}%
\pgfpathlineto{\pgfqpoint{4.405853in}{0.300425in}}%
\pgfpathlineto{\pgfqpoint{4.408367in}{0.291175in}}%
\pgfpathlineto{\pgfqpoint{4.409624in}{0.290000in}}%
\pgfpathlineto{\pgfqpoint{4.410881in}{0.291175in}}%
\pgfpathlineto{\pgfqpoint{4.413394in}{0.300425in}}%
\pgfpathlineto{\pgfqpoint{4.417165in}{0.331401in}}%
\pgfpathlineto{\pgfqpoint{4.422193in}{0.402946in}}%
\pgfpathlineto{\pgfqpoint{4.428477in}{0.535396in}}%
\pgfpathlineto{\pgfqpoint{4.437276in}{0.780686in}}%
\pgfpathlineto{\pgfqpoint{4.461158in}{1.487543in}}%
\pgfpathlineto{\pgfqpoint{4.467442in}{1.606898in}}%
\pgfpathlineto{\pgfqpoint{4.472470in}{1.666138in}}%
\pgfpathlineto{\pgfqpoint{4.476241in}{1.687223in}}%
\pgfpathlineto{\pgfqpoint{4.478755in}{1.689744in}}%
\pgfpathlineto{\pgfqpoint{4.480011in}{1.687543in}}%
\pgfpathlineto{\pgfqpoint{4.482525in}{1.676178in}}%
\pgfpathlineto{\pgfqpoint{4.486296in}{1.642168in}}%
\pgfpathlineto{\pgfqpoint{4.491324in}{1.566885in}}%
\pgfpathlineto{\pgfqpoint{4.497608in}{1.430504in}}%
\pgfpathlineto{\pgfqpoint{4.506407in}{1.181754in}}%
\pgfpathlineto{\pgfqpoint{4.529032in}{0.508159in}}%
\pgfpathlineto{\pgfqpoint{4.535316in}{0.383783in}}%
\pgfpathlineto{\pgfqpoint{4.540344in}{0.319887in}}%
\pgfpathlineto{\pgfqpoint{4.544115in}{0.295084in}}%
\pgfpathlineto{\pgfqpoint{4.546629in}{0.290021in}}%
\pgfpathlineto{\pgfqpoint{4.547886in}{0.290961in}}%
\pgfpathlineto{\pgfqpoint{4.550399in}{0.299806in}}%
\pgfpathlineto{\pgfqpoint{4.554170in}{0.330162in}}%
\pgfpathlineto{\pgfqpoint{4.559198in}{0.400959in}}%
\pgfpathlineto{\pgfqpoint{4.565483in}{0.532619in}}%
\pgfpathlineto{\pgfqpoint{4.574281in}{0.777182in}}%
\pgfpathlineto{\pgfqpoint{4.598163in}{1.484958in}}%
\pgfpathlineto{\pgfqpoint{4.604447in}{1.605168in}}%
\pgfpathlineto{\pgfqpoint{4.609475in}{1.665176in}}%
\pgfpathlineto{\pgfqpoint{4.613246in}{1.686881in}}%
\pgfpathlineto{\pgfqpoint{4.615760in}{1.689829in}}%
\pgfpathlineto{\pgfqpoint{4.617017in}{1.687821in}}%
\pgfpathlineto{\pgfqpoint{4.619530in}{1.676904in}}%
\pgfpathlineto{\pgfqpoint{4.623301in}{1.643493in}}%
\pgfpathlineto{\pgfqpoint{4.628329in}{1.568936in}}%
\pgfpathlineto{\pgfqpoint{4.634613in}{1.433345in}}%
\pgfpathlineto{\pgfqpoint{4.643412in}{1.185279in}}%
\pgfpathlineto{\pgfqpoint{4.666037in}{0.510829in}}%
\pgfpathlineto{\pgfqpoint{4.672321in}{0.385642in}}%
\pgfpathlineto{\pgfqpoint{4.677349in}{0.320955in}}%
\pgfpathlineto{\pgfqpoint{4.681120in}{0.295533in}}%
\pgfpathlineto{\pgfqpoint{4.683634in}{0.290043in}}%
\pgfpathlineto{\pgfqpoint{4.684891in}{0.290790in}}%
\pgfpathlineto{\pgfqpoint{4.687404in}{0.299207in}}%
\pgfpathlineto{\pgfqpoint{4.691175in}{0.328945in}}%
\pgfpathlineto{\pgfqpoint{4.696203in}{0.398972in}}%
\pgfpathlineto{\pgfqpoint{4.702488in}{0.529842in}}%
\pgfpathlineto{\pgfqpoint{4.711286in}{0.773700in}}%
\pgfpathlineto{\pgfqpoint{4.735168in}{1.482373in}}%
\pgfpathlineto{\pgfqpoint{4.741452in}{1.603395in}}%
\pgfpathlineto{\pgfqpoint{4.746480in}{1.664194in}}%
\pgfpathlineto{\pgfqpoint{4.750251in}{1.686518in}}%
\pgfpathlineto{\pgfqpoint{4.752765in}{1.689893in}}%
\pgfpathlineto{\pgfqpoint{4.754022in}{1.688120in}}%
\pgfpathlineto{\pgfqpoint{4.756535in}{1.677588in}}%
\pgfpathlineto{\pgfqpoint{4.760306in}{1.644796in}}%
\pgfpathlineto{\pgfqpoint{4.765334in}{1.571008in}}%
\pgfpathlineto{\pgfqpoint{4.771619in}{1.436187in}}%
\pgfpathlineto{\pgfqpoint{4.780417in}{1.188804in}}%
\pgfpathlineto{\pgfqpoint{4.803042in}{0.513499in}}%
\pgfpathlineto{\pgfqpoint{4.809326in}{0.387501in}}%
\pgfpathlineto{\pgfqpoint{4.814354in}{0.322044in}}%
\pgfpathlineto{\pgfqpoint{4.818125in}{0.296003in}}%
\pgfpathlineto{\pgfqpoint{4.820639in}{0.290107in}}%
\pgfpathlineto{\pgfqpoint{4.821896in}{0.290620in}}%
\pgfpathlineto{\pgfqpoint{4.824409in}{0.298631in}}%
\pgfpathlineto{\pgfqpoint{4.826923in}{0.315806in}}%
\pgfpathlineto{\pgfqpoint{4.830694in}{0.358191in}}%
\pgfpathlineto{\pgfqpoint{4.835722in}{0.443706in}}%
\pgfpathlineto{\pgfqpoint{4.842006in}{0.590513in}}%
\pgfpathlineto{\pgfqpoint{4.852062in}{0.887757in}}%
\pgfpathlineto{\pgfqpoint{4.869659in}{1.419032in}}%
\pgfpathlineto{\pgfqpoint{4.877200in}{1.581028in}}%
\pgfpathlineto{\pgfqpoint{4.882228in}{1.651055in}}%
\pgfpathlineto{\pgfqpoint{4.885999in}{1.680793in}}%
\pgfpathlineto{\pgfqpoint{4.888513in}{1.689210in}}%
\pgfpathlineto{\pgfqpoint{4.889770in}{1.689957in}}%
\pgfpathlineto{\pgfqpoint{4.891027in}{1.688376in}}%
\pgfpathlineto{\pgfqpoint{4.893540in}{1.678272in}}%
\pgfpathlineto{\pgfqpoint{4.897311in}{1.646078in}}%
\pgfpathlineto{\pgfqpoint{4.902339in}{1.573038in}}%
\pgfpathlineto{\pgfqpoint{4.908624in}{1.439007in}}%
\pgfpathlineto{\pgfqpoint{4.917422in}{1.192307in}}%
\pgfpathlineto{\pgfqpoint{4.941304in}{0.487330in}}%
\pgfpathlineto{\pgfqpoint{4.947588in}{0.369662in}}%
\pgfpathlineto{\pgfqpoint{4.952616in}{0.312004in}}%
\pgfpathlineto{\pgfqpoint{4.956387in}{0.292179in}}%
\pgfpathlineto{\pgfqpoint{4.957644in}{0.290171in}}%
\pgfpathlineto{\pgfqpoint{4.958901in}{0.290491in}}%
\pgfpathlineto{\pgfqpoint{4.961414in}{0.298075in}}%
\pgfpathlineto{\pgfqpoint{4.963928in}{0.314824in}}%
\pgfpathlineto{\pgfqpoint{4.967699in}{0.356631in}}%
\pgfpathlineto{\pgfqpoint{4.972727in}{0.441421in}}%
\pgfpathlineto{\pgfqpoint{4.979011in}{0.587501in}}%
\pgfpathlineto{\pgfqpoint{4.989067in}{0.884125in}}%
\pgfpathlineto{\pgfqpoint{5.006664in}{1.416127in}}%
\pgfpathlineto{\pgfqpoint{5.014205in}{1.579041in}}%
\pgfpathlineto{\pgfqpoint{5.019233in}{1.649838in}}%
\pgfpathlineto{\pgfqpoint{5.023004in}{1.680194in}}%
\pgfpathlineto{\pgfqpoint{5.025518in}{1.689039in}}%
\pgfpathlineto{\pgfqpoint{5.026775in}{1.689979in}}%
\pgfpathlineto{\pgfqpoint{5.028032in}{1.688611in}}%
\pgfpathlineto{\pgfqpoint{5.030545in}{1.678934in}}%
\pgfpathlineto{\pgfqpoint{5.034316in}{1.647360in}}%
\pgfpathlineto{\pgfqpoint{5.039344in}{1.575046in}}%
\pgfpathlineto{\pgfqpoint{5.045629in}{1.441805in}}%
\pgfpathlineto{\pgfqpoint{5.054427in}{1.195811in}}%
\pgfpathlineto{\pgfqpoint{5.078309in}{0.489872in}}%
\pgfpathlineto{\pgfqpoint{5.084593in}{0.371372in}}%
\pgfpathlineto{\pgfqpoint{5.089621in}{0.312922in}}%
\pgfpathlineto{\pgfqpoint{5.093392in}{0.292457in}}%
\pgfpathlineto{\pgfqpoint{5.094649in}{0.290256in}}%
\pgfpathlineto{\pgfqpoint{5.095906in}{0.290363in}}%
\pgfpathlineto{\pgfqpoint{5.098419in}{0.297520in}}%
\pgfpathlineto{\pgfqpoint{5.100933in}{0.313862in}}%
\pgfpathlineto{\pgfqpoint{5.104704in}{0.355072in}}%
\pgfpathlineto{\pgfqpoint{5.109732in}{0.439156in}}%
\pgfpathlineto{\pgfqpoint{5.116016in}{0.584510in}}%
\pgfpathlineto{\pgfqpoint{5.126072in}{0.880515in}}%
\pgfpathlineto{\pgfqpoint{5.143669in}{1.413200in}}%
\pgfpathlineto{\pgfqpoint{5.151210in}{1.577054in}}%
\pgfpathlineto{\pgfqpoint{5.156238in}{1.648599in}}%
\pgfpathlineto{\pgfqpoint{5.160009in}{1.679575in}}%
\pgfpathlineto{\pgfqpoint{5.162523in}{1.688825in}}%
\pgfpathlineto{\pgfqpoint{5.163780in}{1.690000in}}%
\pgfpathlineto{\pgfqpoint{5.165037in}{1.688825in}}%
\pgfpathlineto{\pgfqpoint{5.167550in}{1.679575in}}%
\pgfpathlineto{\pgfqpoint{5.171321in}{1.648599in}}%
\pgfpathlineto{\pgfqpoint{5.176349in}{1.577054in}}%
\pgfpathlineto{\pgfqpoint{5.182634in}{1.444604in}}%
\pgfpathlineto{\pgfqpoint{5.191432in}{1.199314in}}%
\pgfpathlineto{\pgfqpoint{5.215314in}{0.492457in}}%
\pgfpathlineto{\pgfqpoint{5.221598in}{0.373102in}}%
\pgfpathlineto{\pgfqpoint{5.226626in}{0.313862in}}%
\pgfpathlineto{\pgfqpoint{5.230397in}{0.292777in}}%
\pgfpathlineto{\pgfqpoint{5.232911in}{0.290256in}}%
\pgfpathlineto{\pgfqpoint{5.234168in}{0.292457in}}%
\pgfpathlineto{\pgfqpoint{5.236681in}{0.303822in}}%
\pgfpathlineto{\pgfqpoint{5.240452in}{0.337832in}}%
\pgfpathlineto{\pgfqpoint{5.245480in}{0.413115in}}%
\pgfpathlineto{\pgfqpoint{5.251765in}{0.549496in}}%
\pgfpathlineto{\pgfqpoint{5.260563in}{0.798246in}}%
\pgfpathlineto{\pgfqpoint{5.283188in}{1.471841in}}%
\pgfpathlineto{\pgfqpoint{5.289472in}{1.596217in}}%
\pgfpathlineto{\pgfqpoint{5.294500in}{1.660113in}}%
\pgfpathlineto{\pgfqpoint{5.298271in}{1.684916in}}%
\pgfpathlineto{\pgfqpoint{5.300785in}{1.689979in}}%
\pgfpathlineto{\pgfqpoint{5.302042in}{1.689039in}}%
\pgfpathlineto{\pgfqpoint{5.304555in}{1.680194in}}%
\pgfpathlineto{\pgfqpoint{5.308326in}{1.649838in}}%
\pgfpathlineto{\pgfqpoint{5.313354in}{1.579041in}}%
\pgfpathlineto{\pgfqpoint{5.319639in}{1.447381in}}%
\pgfpathlineto{\pgfqpoint{5.328437in}{1.202818in}}%
\pgfpathlineto{\pgfqpoint{5.352319in}{0.495042in}}%
\pgfpathlineto{\pgfqpoint{5.358603in}{0.374832in}}%
\pgfpathlineto{\pgfqpoint{5.363631in}{0.314824in}}%
\pgfpathlineto{\pgfqpoint{5.367402in}{0.293119in}}%
\pgfpathlineto{\pgfqpoint{5.369916in}{0.290171in}}%
\pgfpathlineto{\pgfqpoint{5.371173in}{0.292179in}}%
\pgfpathlineto{\pgfqpoint{5.373686in}{0.303096in}}%
\pgfpathlineto{\pgfqpoint{5.377457in}{0.336507in}}%
\pgfpathlineto{\pgfqpoint{5.382485in}{0.411064in}}%
\pgfpathlineto{\pgfqpoint{5.388770in}{0.546655in}}%
\pgfpathlineto{\pgfqpoint{5.397568in}{0.794721in}}%
\pgfpathlineto{\pgfqpoint{5.420193in}{1.469171in}}%
\pgfpathlineto{\pgfqpoint{5.426477in}{1.594358in}}%
\pgfpathlineto{\pgfqpoint{5.431505in}{1.659045in}}%
\pgfpathlineto{\pgfqpoint{5.435276in}{1.684467in}}%
\pgfpathlineto{\pgfqpoint{5.437790in}{1.689957in}}%
\pgfpathlineto{\pgfqpoint{5.439047in}{1.689210in}}%
\pgfpathlineto{\pgfqpoint{5.441560in}{1.680793in}}%
\pgfpathlineto{\pgfqpoint{5.445331in}{1.651055in}}%
\pgfpathlineto{\pgfqpoint{5.450359in}{1.581028in}}%
\pgfpathlineto{\pgfqpoint{5.456644in}{1.450158in}}%
\pgfpathlineto{\pgfqpoint{5.465442in}{1.206300in}}%
\pgfpathlineto{\pgfqpoint{5.489324in}{0.497627in}}%
\pgfpathlineto{\pgfqpoint{5.495608in}{0.376605in}}%
\pgfpathlineto{\pgfqpoint{5.500636in}{0.315806in}}%
\pgfpathlineto{\pgfqpoint{5.504407in}{0.293482in}}%
\pgfpathlineto{\pgfqpoint{5.506921in}{0.290107in}}%
\pgfpathlineto{\pgfqpoint{5.508178in}{0.291880in}}%
\pgfpathlineto{\pgfqpoint{5.510691in}{0.302412in}}%
\pgfpathlineto{\pgfqpoint{5.514462in}{0.335204in}}%
\pgfpathlineto{\pgfqpoint{5.519490in}{0.408992in}}%
\pgfpathlineto{\pgfqpoint{5.525775in}{0.543813in}}%
\pgfpathlineto{\pgfqpoint{5.534573in}{0.791196in}}%
\pgfpathlineto{\pgfqpoint{5.557198in}{1.466501in}}%
\pgfpathlineto{\pgfqpoint{5.563482in}{1.592499in}}%
\pgfpathlineto{\pgfqpoint{5.568510in}{1.657956in}}%
\pgfpathlineto{\pgfqpoint{5.572281in}{1.683997in}}%
\pgfpathlineto{\pgfqpoint{5.574795in}{1.689893in}}%
\pgfpathlineto{\pgfqpoint{5.576052in}{1.689380in}}%
\pgfpathlineto{\pgfqpoint{5.578566in}{1.681369in}}%
\pgfpathlineto{\pgfqpoint{5.581079in}{1.664194in}}%
\pgfpathlineto{\pgfqpoint{5.584850in}{1.621809in}}%
\pgfpathlineto{\pgfqpoint{5.589878in}{1.536294in}}%
\pgfpathlineto{\pgfqpoint{5.596162in}{1.389487in}}%
\pgfpathlineto{\pgfqpoint{5.606218in}{1.092243in}}%
\pgfpathlineto{\pgfqpoint{5.623815in}{0.560968in}}%
\pgfpathlineto{\pgfqpoint{5.631356in}{0.398972in}}%
\pgfpathlineto{\pgfqpoint{5.636384in}{0.328945in}}%
\pgfpathlineto{\pgfqpoint{5.640155in}{0.299207in}}%
\pgfpathlineto{\pgfqpoint{5.642669in}{0.290790in}}%
\pgfpathlineto{\pgfqpoint{5.643926in}{0.290043in}}%
\pgfpathlineto{\pgfqpoint{5.645183in}{0.291624in}}%
\pgfpathlineto{\pgfqpoint{5.647696in}{0.301728in}}%
\pgfpathlineto{\pgfqpoint{5.651467in}{0.333922in}}%
\pgfpathlineto{\pgfqpoint{5.656495in}{0.406962in}}%
\pgfpathlineto{\pgfqpoint{5.662780in}{0.540993in}}%
\pgfpathlineto{\pgfqpoint{5.671578in}{0.787693in}}%
\pgfpathlineto{\pgfqpoint{5.695460in}{1.492670in}}%
\pgfpathlineto{\pgfqpoint{5.701744in}{1.610338in}}%
\pgfpathlineto{\pgfqpoint{5.706772in}{1.667996in}}%
\pgfpathlineto{\pgfqpoint{5.710543in}{1.687821in}}%
\pgfpathlineto{\pgfqpoint{5.711800in}{1.689829in}}%
\pgfpathlineto{\pgfqpoint{5.713057in}{1.689509in}}%
\pgfpathlineto{\pgfqpoint{5.715571in}{1.681925in}}%
\pgfpathlineto{\pgfqpoint{5.718084in}{1.665176in}}%
\pgfpathlineto{\pgfqpoint{5.721855in}{1.623369in}}%
\pgfpathlineto{\pgfqpoint{5.726883in}{1.538579in}}%
\pgfpathlineto{\pgfqpoint{5.733168in}{1.392499in}}%
\pgfpathlineto{\pgfqpoint{5.743223in}{1.095875in}}%
\pgfpathlineto{\pgfqpoint{5.760820in}{0.563873in}}%
\pgfpathlineto{\pgfqpoint{5.768361in}{0.400959in}}%
\pgfpathlineto{\pgfqpoint{5.773389in}{0.330162in}}%
\pgfpathlineto{\pgfqpoint{5.777160in}{0.299806in}}%
\pgfpathlineto{\pgfqpoint{5.779674in}{0.290961in}}%
\pgfpathlineto{\pgfqpoint{5.780931in}{0.290021in}}%
\pgfpathlineto{\pgfqpoint{5.782188in}{0.291389in}}%
\pgfpathlineto{\pgfqpoint{5.784702in}{0.301066in}}%
\pgfpathlineto{\pgfqpoint{5.788472in}{0.332640in}}%
\pgfpathlineto{\pgfqpoint{5.793500in}{0.404954in}}%
\pgfpathlineto{\pgfqpoint{5.799785in}{0.538195in}}%
\pgfpathlineto{\pgfqpoint{5.808583in}{0.784189in}}%
\pgfpathlineto{\pgfqpoint{5.832465in}{1.490128in}}%
\pgfpathlineto{\pgfqpoint{5.838749in}{1.608628in}}%
\pgfpathlineto{\pgfqpoint{5.843777in}{1.667078in}}%
\pgfpathlineto{\pgfqpoint{5.847548in}{1.687543in}}%
\pgfpathlineto{\pgfqpoint{5.848805in}{1.689744in}}%
\pgfpathlineto{\pgfqpoint{5.850062in}{1.689637in}}%
\pgfpathlineto{\pgfqpoint{5.852576in}{1.682480in}}%
\pgfpathlineto{\pgfqpoint{5.855089in}{1.666138in}}%
\pgfpathlineto{\pgfqpoint{5.858860in}{1.624928in}}%
\pgfpathlineto{\pgfqpoint{5.863888in}{1.540844in}}%
\pgfpathlineto{\pgfqpoint{5.870173in}{1.395490in}}%
\pgfpathlineto{\pgfqpoint{5.880228in}{1.099485in}}%
\pgfpathlineto{\pgfqpoint{5.897825in}{0.566800in}}%
\pgfpathlineto{\pgfqpoint{5.905366in}{0.402946in}}%
\pgfpathlineto{\pgfqpoint{5.910394in}{0.331401in}}%
\pgfpathlineto{\pgfqpoint{5.914165in}{0.300425in}}%
\pgfpathlineto{\pgfqpoint{5.916679in}{0.291175in}}%
\pgfpathlineto{\pgfqpoint{5.917936in}{0.290000in}}%
\pgfpathlineto{\pgfqpoint{5.919193in}{0.291175in}}%
\pgfpathlineto{\pgfqpoint{5.921707in}{0.300425in}}%
\pgfpathlineto{\pgfqpoint{5.925477in}{0.331401in}}%
\pgfpathlineto{\pgfqpoint{5.930505in}{0.402946in}}%
\pgfpathlineto{\pgfqpoint{5.936790in}{0.535396in}}%
\pgfpathlineto{\pgfqpoint{5.945588in}{0.780686in}}%
\pgfpathlineto{\pgfqpoint{5.969470in}{1.487543in}}%
\pgfpathlineto{\pgfqpoint{5.975754in}{1.606898in}}%
\pgfpathlineto{\pgfqpoint{5.980782in}{1.666138in}}%
\pgfpathlineto{\pgfqpoint{5.984553in}{1.687223in}}%
\pgfpathlineto{\pgfqpoint{5.987067in}{1.689744in}}%
\pgfpathlineto{\pgfqpoint{5.988324in}{1.687543in}}%
\pgfpathlineto{\pgfqpoint{5.990837in}{1.676178in}}%
\pgfpathlineto{\pgfqpoint{5.994608in}{1.642168in}}%
\pgfpathlineto{\pgfqpoint{5.999636in}{1.566885in}}%
\pgfpathlineto{\pgfqpoint{6.005921in}{1.430504in}}%
\pgfpathlineto{\pgfqpoint{6.014719in}{1.181754in}}%
\pgfpathlineto{\pgfqpoint{6.037344in}{0.508159in}}%
\pgfpathlineto{\pgfqpoint{6.043628in}{0.383783in}}%
\pgfpathlineto{\pgfqpoint{6.048656in}{0.319887in}}%
\pgfpathlineto{\pgfqpoint{6.052427in}{0.295084in}}%
\pgfpathlineto{\pgfqpoint{6.054941in}{0.290021in}}%
\pgfpathlineto{\pgfqpoint{6.056198in}{0.290961in}}%
\pgfpathlineto{\pgfqpoint{6.058712in}{0.299806in}}%
\pgfpathlineto{\pgfqpoint{6.062482in}{0.330162in}}%
\pgfpathlineto{\pgfqpoint{6.067510in}{0.400959in}}%
\pgfpathlineto{\pgfqpoint{6.073795in}{0.532619in}}%
\pgfpathlineto{\pgfqpoint{6.082593in}{0.777182in}}%
\pgfpathlineto{\pgfqpoint{6.106475in}{1.484958in}}%
\pgfpathlineto{\pgfqpoint{6.112759in}{1.605168in}}%
\pgfpathlineto{\pgfqpoint{6.117787in}{1.665176in}}%
\pgfpathlineto{\pgfqpoint{6.121558in}{1.686881in}}%
\pgfpathlineto{\pgfqpoint{6.124072in}{1.689829in}}%
\pgfpathlineto{\pgfqpoint{6.125329in}{1.687821in}}%
\pgfpathlineto{\pgfqpoint{6.127843in}{1.676904in}}%
\pgfpathlineto{\pgfqpoint{6.131613in}{1.643493in}}%
\pgfpathlineto{\pgfqpoint{6.136641in}{1.568936in}}%
\pgfpathlineto{\pgfqpoint{6.142926in}{1.433345in}}%
\pgfpathlineto{\pgfqpoint{6.151724in}{1.185279in}}%
\pgfpathlineto{\pgfqpoint{6.174349in}{0.510829in}}%
\pgfpathlineto{\pgfqpoint{6.180633in}{0.385642in}}%
\pgfpathlineto{\pgfqpoint{6.185661in}{0.320955in}}%
\pgfpathlineto{\pgfqpoint{6.189432in}{0.295533in}}%
\pgfpathlineto{\pgfqpoint{6.191946in}{0.290043in}}%
\pgfpathlineto{\pgfqpoint{6.193203in}{0.290790in}}%
\pgfpathlineto{\pgfqpoint{6.195717in}{0.299207in}}%
\pgfpathlineto{\pgfqpoint{6.199487in}{0.328945in}}%
\pgfpathlineto{\pgfqpoint{6.204515in}{0.398972in}}%
\pgfpathlineto{\pgfqpoint{6.210800in}{0.529842in}}%
\pgfpathlineto{\pgfqpoint{6.219598in}{0.773700in}}%
\pgfpathlineto{\pgfqpoint{6.243480in}{1.482373in}}%
\pgfpathlineto{\pgfqpoint{6.249764in}{1.603395in}}%
\pgfpathlineto{\pgfqpoint{6.254792in}{1.664194in}}%
\pgfpathlineto{\pgfqpoint{6.258563in}{1.686518in}}%
\pgfpathlineto{\pgfqpoint{6.261077in}{1.689893in}}%
\pgfpathlineto{\pgfqpoint{6.262334in}{1.688120in}}%
\pgfpathlineto{\pgfqpoint{6.264848in}{1.677588in}}%
\pgfpathlineto{\pgfqpoint{6.268618in}{1.644796in}}%
\pgfpathlineto{\pgfqpoint{6.273646in}{1.571008in}}%
\pgfpathlineto{\pgfqpoint{6.279931in}{1.436187in}}%
\pgfpathlineto{\pgfqpoint{6.288729in}{1.188804in}}%
\pgfpathlineto{\pgfqpoint{6.311354in}{0.513499in}}%
\pgfpathlineto{\pgfqpoint{6.317638in}{0.387501in}}%
\pgfpathlineto{\pgfqpoint{6.322666in}{0.322044in}}%
\pgfpathlineto{\pgfqpoint{6.326437in}{0.296003in}}%
\pgfpathlineto{\pgfqpoint{6.328951in}{0.290107in}}%
\pgfpathlineto{\pgfqpoint{6.330208in}{0.290620in}}%
\pgfpathlineto{\pgfqpoint{6.332722in}{0.298631in}}%
\pgfpathlineto{\pgfqpoint{6.335235in}{0.315806in}}%
\pgfpathlineto{\pgfqpoint{6.339006in}{0.358191in}}%
\pgfpathlineto{\pgfqpoint{6.344034in}{0.443706in}}%
\pgfpathlineto{\pgfqpoint{6.350319in}{0.590513in}}%
\pgfpathlineto{\pgfqpoint{6.360374in}{0.887757in}}%
\pgfpathlineto{\pgfqpoint{6.377971in}{1.419032in}}%
\pgfpathlineto{\pgfqpoint{6.385513in}{1.581028in}}%
\pgfpathlineto{\pgfqpoint{6.390540in}{1.651055in}}%
\pgfpathlineto{\pgfqpoint{6.394311in}{1.680793in}}%
\pgfpathlineto{\pgfqpoint{6.396825in}{1.689210in}}%
\pgfpathlineto{\pgfqpoint{6.398082in}{1.689957in}}%
\pgfpathlineto{\pgfqpoint{6.399339in}{1.688376in}}%
\pgfpathlineto{\pgfqpoint{6.401853in}{1.678272in}}%
\pgfpathlineto{\pgfqpoint{6.405623in}{1.646078in}}%
\pgfpathlineto{\pgfqpoint{6.410651in}{1.573038in}}%
\pgfpathlineto{\pgfqpoint{6.416936in}{1.439007in}}%
\pgfpathlineto{\pgfqpoint{6.425734in}{1.192307in}}%
\pgfpathlineto{\pgfqpoint{6.447102in}{0.546655in}}%
\pgfpathlineto{\pgfqpoint{6.447102in}{0.546655in}}%
\pgfusepath{stroke}%
\end{pgfscope}%
\begin{pgfscope}%
\pgfpathrectangle{\pgfqpoint{0.895256in}{0.220000in}}{\pgfqpoint{5.550589in}{1.540000in}}%
\pgfusepath{clip}%
\pgfsetrectcap%
\pgfsetroundjoin%
\pgfsetlinewidth{1.505625pt}%
\definecolor{currentstroke}{rgb}{1.000000,0.498039,0.054902}%
\pgfsetstrokecolor{currentstroke}%
\pgfsetdash{}{0pt}%
\pgfpathmoveto{\pgfqpoint{0.893999in}{0.989808in}}%
\pgfpathlineto{\pgfqpoint{0.904055in}{0.989893in}}%
\pgfpathlineto{\pgfqpoint{0.914110in}{0.990107in}}%
\pgfpathlineto{\pgfqpoint{0.919138in}{0.990021in}}%
\pgfpathlineto{\pgfqpoint{0.926679in}{0.989957in}}%
\pgfpathlineto{\pgfqpoint{0.939249in}{0.989979in}}%
\pgfpathlineto{\pgfqpoint{0.999581in}{0.989893in}}%
\pgfpathlineto{\pgfqpoint{1.020949in}{0.989850in}}%
\pgfpathlineto{\pgfqpoint{1.056143in}{0.989979in}}%
\pgfpathlineto{\pgfqpoint{1.122760in}{0.989957in}}%
\pgfpathlineto{\pgfqpoint{1.129045in}{0.990064in}}%
\pgfpathlineto{\pgfqpoint{1.157954in}{0.990192in}}%
\pgfpathlineto{\pgfqpoint{1.222057in}{0.990021in}}%
\pgfpathlineto{\pgfqpoint{1.225828in}{0.989680in}}%
\pgfpathlineto{\pgfqpoint{1.611705in}{0.990449in}}%
\pgfpathlineto{\pgfqpoint{1.614218in}{0.990726in}}%
\pgfpathlineto{\pgfqpoint{1.620503in}{0.989957in}}%
\pgfpathlineto{\pgfqpoint{1.623017in}{0.990470in}}%
\pgfpathlineto{\pgfqpoint{1.625531in}{0.989744in}}%
\pgfpathlineto{\pgfqpoint{1.629302in}{0.990620in}}%
\pgfpathlineto{\pgfqpoint{1.634329in}{0.990043in}}%
\pgfpathlineto{\pgfqpoint{1.636843in}{0.990620in}}%
\pgfpathlineto{\pgfqpoint{1.645642in}{0.990150in}}%
\pgfpathlineto{\pgfqpoint{1.658211in}{0.990171in}}%
\pgfpathlineto{\pgfqpoint{1.663239in}{0.989979in}}%
\pgfpathlineto{\pgfqpoint{1.668266in}{0.989979in}}%
\pgfpathlineto{\pgfqpoint{1.697176in}{0.990021in}}%
\pgfpathlineto{\pgfqpoint{1.707231in}{0.990214in}}%
\pgfpathlineto{\pgfqpoint{2.176065in}{0.989295in}}%
\pgfpathlineto{\pgfqpoint{2.178579in}{0.988996in}}%
\pgfpathlineto{\pgfqpoint{2.186120in}{0.989808in}}%
\pgfpathlineto{\pgfqpoint{2.187377in}{0.989167in}}%
\pgfpathlineto{\pgfqpoint{2.189891in}{0.989551in}}%
\pgfpathlineto{\pgfqpoint{2.193662in}{0.990064in}}%
\pgfpathlineto{\pgfqpoint{2.196175in}{0.990940in}}%
\pgfpathlineto{\pgfqpoint{2.201203in}{0.990043in}}%
\pgfpathlineto{\pgfqpoint{2.208745in}{0.990192in}}%
\pgfpathlineto{\pgfqpoint{2.223828in}{0.990021in}}%
\pgfpathlineto{\pgfqpoint{2.285417in}{0.989808in}}%
\pgfpathlineto{\pgfqpoint{2.290445in}{0.990021in}}%
\pgfpathlineto{\pgfqpoint{2.301757in}{0.990085in}}%
\pgfpathlineto{\pgfqpoint{2.308042in}{0.990021in}}%
\pgfpathlineto{\pgfqpoint{2.338208in}{0.989979in}}%
\pgfpathlineto{\pgfqpoint{2.348264in}{0.989893in}}%
\pgfpathlineto{\pgfqpoint{2.407339in}{0.989893in}}%
\pgfpathlineto{\pgfqpoint{2.421165in}{0.990043in}}%
\pgfpathlineto{\pgfqpoint{2.445047in}{0.989893in}}%
\pgfpathlineto{\pgfqpoint{2.451332in}{0.989979in}}%
\pgfpathlineto{\pgfqpoint{2.608447in}{0.989850in}}%
\pgfpathlineto{\pgfqpoint{2.610961in}{0.988697in}}%
\pgfpathlineto{\pgfqpoint{2.614732in}{0.989850in}}%
\pgfpathlineto{\pgfqpoint{2.617246in}{0.989808in}}%
\pgfpathlineto{\pgfqpoint{2.619760in}{0.990107in}}%
\pgfpathlineto{\pgfqpoint{2.626044in}{0.989786in}}%
\pgfpathlineto{\pgfqpoint{2.628558in}{0.990235in}}%
\pgfpathlineto{\pgfqpoint{2.631072in}{0.990064in}}%
\pgfpathlineto{\pgfqpoint{2.633586in}{0.990534in}}%
\pgfpathlineto{\pgfqpoint{2.639871in}{0.989936in}}%
\pgfpathlineto{\pgfqpoint{2.643641in}{0.990214in}}%
\pgfpathlineto{\pgfqpoint{2.656211in}{0.990385in}}%
\pgfpathlineto{\pgfqpoint{2.668780in}{0.990214in}}%
\pgfpathlineto{\pgfqpoint{2.673808in}{0.989808in}}%
\pgfpathlineto{\pgfqpoint{2.678835in}{0.989936in}}%
\pgfpathlineto{\pgfqpoint{2.686377in}{0.989979in}}%
\pgfpathlineto{\pgfqpoint{2.698946in}{0.990192in}}%
\pgfpathlineto{\pgfqpoint{2.705231in}{0.990043in}}%
\pgfpathlineto{\pgfqpoint{2.769334in}{0.990021in}}%
\pgfpathlineto{\pgfqpoint{2.802014in}{0.990171in}}%
\pgfpathlineto{\pgfqpoint{2.808299in}{0.990043in}}%
\pgfpathlineto{\pgfqpoint{2.840979in}{0.990043in}}%
\pgfpathlineto{\pgfqpoint{2.847264in}{0.989936in}}%
\pgfpathlineto{\pgfqpoint{2.862347in}{0.989979in}}%
\pgfpathlineto{\pgfqpoint{2.871145in}{0.990021in}}%
\pgfpathlineto{\pgfqpoint{2.891256in}{0.989957in}}%
\pgfpathlineto{\pgfqpoint{2.898798in}{0.990150in}}%
\pgfpathlineto{\pgfqpoint{2.910110in}{0.990021in}}%
\pgfpathlineto{\pgfqpoint{3.622787in}{0.990021in}}%
\pgfpathlineto{\pgfqpoint{3.625301in}{0.988590in}}%
\pgfpathlineto{\pgfqpoint{3.631586in}{0.990620in}}%
\pgfpathlineto{\pgfqpoint{3.636614in}{0.989231in}}%
\pgfpathlineto{\pgfqpoint{3.640384in}{0.990299in}}%
\pgfpathlineto{\pgfqpoint{3.642898in}{0.989850in}}%
\pgfpathlineto{\pgfqpoint{3.647926in}{0.990150in}}%
\pgfpathlineto{\pgfqpoint{3.652954in}{0.990449in}}%
\pgfpathlineto{\pgfqpoint{3.657981in}{0.990641in}}%
\pgfpathlineto{\pgfqpoint{3.674321in}{0.990513in}}%
\pgfpathlineto{\pgfqpoint{3.679349in}{0.990620in}}%
\pgfpathlineto{\pgfqpoint{3.689405in}{0.990192in}}%
\pgfpathlineto{\pgfqpoint{3.699460in}{0.990470in}}%
\pgfpathlineto{\pgfqpoint{3.705745in}{0.990320in}}%
\pgfpathlineto{\pgfqpoint{3.710772in}{0.990043in}}%
\pgfpathlineto{\pgfqpoint{3.714543in}{0.989786in}}%
\pgfpathlineto{\pgfqpoint{3.947075in}{0.990513in}}%
\pgfpathlineto{\pgfqpoint{3.949588in}{0.990876in}}%
\pgfpathlineto{\pgfqpoint{3.958387in}{0.990064in}}%
\pgfpathlineto{\pgfqpoint{3.967185in}{0.989936in}}%
\pgfpathlineto{\pgfqpoint{4.036316in}{0.990171in}}%
\pgfpathlineto{\pgfqpoint{4.046372in}{0.989786in}}%
\pgfpathlineto{\pgfqpoint{4.071510in}{0.990043in}}%
\pgfpathlineto{\pgfqpoint{4.095392in}{0.989893in}}%
\pgfpathlineto{\pgfqpoint{4.100420in}{0.990107in}}%
\pgfpathlineto{\pgfqpoint{4.104190in}{0.990150in}}%
\pgfpathlineto{\pgfqpoint{4.111732in}{0.990214in}}%
\pgfpathlineto{\pgfqpoint{4.134357in}{0.989936in}}%
\pgfpathlineto{\pgfqpoint{4.151954in}{0.990021in}}%
\pgfpathlineto{\pgfqpoint{4.160752in}{0.989979in}}%
\pgfpathlineto{\pgfqpoint{4.177092in}{0.989979in}}%
\pgfpathlineto{\pgfqpoint{4.180863in}{0.990128in}}%
\pgfpathlineto{\pgfqpoint{4.195946in}{0.990085in}}%
\pgfpathlineto{\pgfqpoint{4.218571in}{0.990021in}}%
\pgfpathlineto{\pgfqpoint{4.241195in}{0.990150in}}%
\pgfpathlineto{\pgfqpoint{4.247480in}{0.989957in}}%
\pgfpathlineto{\pgfqpoint{4.614503in}{0.989295in}}%
\pgfpathlineto{\pgfqpoint{4.618273in}{0.997327in}}%
\pgfpathlineto{\pgfqpoint{4.619530in}{0.996708in}}%
\pgfpathlineto{\pgfqpoint{4.622044in}{0.989380in}}%
\pgfpathlineto{\pgfqpoint{4.629586in}{0.989380in}}%
\pgfpathlineto{\pgfqpoint{4.634613in}{0.989957in}}%
\pgfpathlineto{\pgfqpoint{4.637127in}{0.988633in}}%
\pgfpathlineto{\pgfqpoint{4.650954in}{0.990406in}}%
\pgfpathlineto{\pgfqpoint{4.653467in}{0.990021in}}%
\pgfpathlineto{\pgfqpoint{4.655981in}{0.990342in}}%
\pgfpathlineto{\pgfqpoint{4.664780in}{0.989786in}}%
\pgfpathlineto{\pgfqpoint{4.679863in}{0.989893in}}%
\pgfpathlineto{\pgfqpoint{4.684891in}{0.989466in}}%
\pgfpathlineto{\pgfqpoint{4.689918in}{0.989765in}}%
\pgfpathlineto{\pgfqpoint{4.721341in}{0.989722in}}%
\pgfpathlineto{\pgfqpoint{4.730140in}{0.989231in}}%
\pgfpathlineto{\pgfqpoint{4.732654in}{0.989252in}}%
\pgfpathlineto{\pgfqpoint{4.737681in}{0.990278in}}%
\pgfpathlineto{\pgfqpoint{4.742709in}{0.989423in}}%
\pgfpathlineto{\pgfqpoint{4.746480in}{0.990021in}}%
\pgfpathlineto{\pgfqpoint{4.748994in}{0.991004in}}%
\pgfpathlineto{\pgfqpoint{4.752765in}{0.990107in}}%
\pgfpathlineto{\pgfqpoint{4.761563in}{0.989829in}}%
\pgfpathlineto{\pgfqpoint{4.770362in}{0.989979in}}%
\pgfpathlineto{\pgfqpoint{4.781674in}{0.990043in}}%
\pgfpathlineto{\pgfqpoint{4.800528in}{0.990043in}}%
\pgfpathlineto{\pgfqpoint{4.814354in}{0.989872in}}%
\pgfpathlineto{\pgfqpoint{4.821896in}{0.990107in}}%
\pgfpathlineto{\pgfqpoint{4.828180in}{0.989957in}}%
\pgfpathlineto{\pgfqpoint{4.838236in}{0.990085in}}%
\pgfpathlineto{\pgfqpoint{4.849548in}{0.990000in}}%
\pgfpathlineto{\pgfqpoint{4.902339in}{0.990043in}}%
\pgfpathlineto{\pgfqpoint{4.908624in}{0.990192in}}%
\pgfpathlineto{\pgfqpoint{4.911137in}{0.989979in}}%
\pgfpathlineto{\pgfqpoint{4.916165in}{0.989957in}}%
\pgfpathlineto{\pgfqpoint{4.924964in}{0.989936in}}%
\pgfpathlineto{\pgfqpoint{4.928734in}{0.989786in}}%
\pgfpathlineto{\pgfqpoint{4.943817in}{0.990000in}}%
\pgfpathlineto{\pgfqpoint{4.955130in}{0.990128in}}%
\pgfpathlineto{\pgfqpoint{4.994095in}{0.989850in}}%
\pgfpathlineto{\pgfqpoint{5.001636in}{0.990150in}}%
\pgfpathlineto{\pgfqpoint{5.017976in}{0.989957in}}%
\pgfpathlineto{\pgfqpoint{5.050656in}{0.989722in}}%
\pgfpathlineto{\pgfqpoint{5.056941in}{0.990363in}}%
\pgfpathlineto{\pgfqpoint{5.066996in}{0.990064in}}%
\pgfpathlineto{\pgfqpoint{5.129843in}{0.990128in}}%
\pgfpathlineto{\pgfqpoint{5.139898in}{0.989872in}}%
\pgfpathlineto{\pgfqpoint{5.146183in}{0.990150in}}%
\pgfpathlineto{\pgfqpoint{5.628843in}{0.989509in}}%
\pgfpathlineto{\pgfqpoint{5.631356in}{0.989124in}}%
\pgfpathlineto{\pgfqpoint{5.640155in}{0.990534in}}%
\pgfpathlineto{\pgfqpoint{5.642669in}{0.988996in}}%
\pgfpathlineto{\pgfqpoint{5.653981in}{0.990000in}}%
\pgfpathlineto{\pgfqpoint{5.674092in}{0.989915in}}%
\pgfpathlineto{\pgfqpoint{5.682890in}{0.989850in}}%
\pgfpathlineto{\pgfqpoint{5.708029in}{0.990085in}}%
\pgfpathlineto{\pgfqpoint{5.710543in}{0.990150in}}%
\pgfpathlineto{\pgfqpoint{5.711800in}{0.988825in}}%
\pgfpathlineto{\pgfqpoint{5.715571in}{0.989637in}}%
\pgfpathlineto{\pgfqpoint{5.718084in}{0.989829in}}%
\pgfpathlineto{\pgfqpoint{5.720598in}{0.991132in}}%
\pgfpathlineto{\pgfqpoint{5.725626in}{0.990000in}}%
\pgfpathlineto{\pgfqpoint{5.736938in}{0.990064in}}%
\pgfpathlineto{\pgfqpoint{5.773389in}{0.989765in}}%
\pgfpathlineto{\pgfqpoint{5.783445in}{0.990064in}}%
\pgfpathlineto{\pgfqpoint{5.806069in}{0.990363in}}%
\pgfpathlineto{\pgfqpoint{5.814868in}{0.989915in}}%
\pgfpathlineto{\pgfqpoint{5.833722in}{0.989957in}}%
\pgfpathlineto{\pgfqpoint{5.838749in}{0.990790in}}%
\pgfpathlineto{\pgfqpoint{5.840006in}{0.989316in}}%
\pgfpathlineto{\pgfqpoint{5.841263in}{0.995832in}}%
\pgfpathlineto{\pgfqpoint{5.847548in}{1.057590in}}%
\pgfpathlineto{\pgfqpoint{5.856346in}{1.125864in}}%
\pgfpathlineto{\pgfqpoint{5.862631in}{1.157694in}}%
\pgfpathlineto{\pgfqpoint{5.867659in}{1.171238in}}%
\pgfpathlineto{\pgfqpoint{5.870173in}{1.173780in}}%
\pgfpathlineto{\pgfqpoint{5.872686in}{1.173353in}}%
\pgfpathlineto{\pgfqpoint{5.875200in}{1.169956in}}%
\pgfpathlineto{\pgfqpoint{5.878971in}{1.159382in}}%
\pgfpathlineto{\pgfqpoint{5.883999in}{1.136631in}}%
\pgfpathlineto{\pgfqpoint{5.890283in}{1.094654in}}%
\pgfpathlineto{\pgfqpoint{5.900339in}{1.006898in}}%
\pgfpathlineto{\pgfqpoint{5.919193in}{0.832496in}}%
\pgfpathlineto{\pgfqpoint{5.926734in}{0.782957in}}%
\pgfpathlineto{\pgfqpoint{5.931762in}{0.762086in}}%
\pgfpathlineto{\pgfqpoint{5.935533in}{0.753113in}}%
\pgfpathlineto{\pgfqpoint{5.938047in}{0.750550in}}%
\pgfpathlineto{\pgfqpoint{5.940560in}{0.750934in}}%
\pgfpathlineto{\pgfqpoint{5.944331in}{0.757429in}}%
\pgfpathlineto{\pgfqpoint{5.948102in}{0.769861in}}%
\pgfpathlineto{\pgfqpoint{5.953130in}{0.795667in}}%
\pgfpathlineto{\pgfqpoint{5.960671in}{0.851636in}}%
\pgfpathlineto{\pgfqpoint{5.969470in}{0.934415in}}%
\pgfpathlineto{\pgfqpoint{5.988324in}{1.114713in}}%
\pgfpathlineto{\pgfqpoint{5.995865in}{1.165940in}}%
\pgfpathlineto{\pgfqpoint{6.000893in}{1.187687in}}%
\pgfpathlineto{\pgfqpoint{6.004664in}{1.197257in}}%
\pgfpathlineto{\pgfqpoint{6.007178in}{1.199885in}}%
\pgfpathlineto{\pgfqpoint{6.009691in}{1.199949in}}%
\pgfpathlineto{\pgfqpoint{6.012205in}{1.196659in}}%
\pgfpathlineto{\pgfqpoint{6.014719in}{1.190720in}}%
\pgfpathlineto{\pgfqpoint{6.019747in}{1.170789in}}%
\pgfpathlineto{\pgfqpoint{6.026031in}{1.132123in}}%
\pgfpathlineto{\pgfqpoint{6.034830in}{1.058317in}}%
\pgfpathlineto{\pgfqpoint{6.059968in}{0.826728in}}%
\pgfpathlineto{\pgfqpoint{6.066253in}{0.790177in}}%
\pgfpathlineto{\pgfqpoint{6.071281in}{0.772233in}}%
\pgfpathlineto{\pgfqpoint{6.075052in}{0.766038in}}%
\pgfpathlineto{\pgfqpoint{6.077565in}{0.764863in}}%
\pgfpathlineto{\pgfqpoint{6.080079in}{0.767533in}}%
\pgfpathlineto{\pgfqpoint{6.082593in}{0.772553in}}%
\pgfpathlineto{\pgfqpoint{6.086364in}{0.785306in}}%
\pgfpathlineto{\pgfqpoint{6.091392in}{0.811689in}}%
\pgfpathlineto{\pgfqpoint{6.097676in}{0.857020in}}%
\pgfpathlineto{\pgfqpoint{6.106475in}{0.937342in}}%
\pgfpathlineto{\pgfqpoint{6.126586in}{1.127061in}}%
\pgfpathlineto{\pgfqpoint{6.134127in}{1.176258in}}%
\pgfpathlineto{\pgfqpoint{6.139155in}{1.196766in}}%
\pgfpathlineto{\pgfqpoint{6.142926in}{1.205225in}}%
\pgfpathlineto{\pgfqpoint{6.145439in}{1.207340in}}%
\pgfpathlineto{\pgfqpoint{6.147953in}{1.206336in}}%
\pgfpathlineto{\pgfqpoint{6.150467in}{1.202662in}}%
\pgfpathlineto{\pgfqpoint{6.154238in}{1.191596in}}%
\pgfpathlineto{\pgfqpoint{6.159266in}{1.167692in}}%
\pgfpathlineto{\pgfqpoint{6.165550in}{1.124988in}}%
\pgfpathlineto{\pgfqpoint{6.175606in}{1.035181in}}%
\pgfpathlineto{\pgfqpoint{6.195717in}{0.846274in}}%
\pgfpathlineto{\pgfqpoint{6.203258in}{0.797889in}}%
\pgfpathlineto{\pgfqpoint{6.208286in}{0.778043in}}%
\pgfpathlineto{\pgfqpoint{6.212057in}{0.770161in}}%
\pgfpathlineto{\pgfqpoint{6.214570in}{0.768516in}}%
\pgfpathlineto{\pgfqpoint{6.217084in}{0.769477in}}%
\pgfpathlineto{\pgfqpoint{6.219598in}{0.773813in}}%
\pgfpathlineto{\pgfqpoint{6.223369in}{0.785007in}}%
\pgfpathlineto{\pgfqpoint{6.228397in}{0.809702in}}%
\pgfpathlineto{\pgfqpoint{6.234681in}{0.853260in}}%
\pgfpathlineto{\pgfqpoint{6.243480in}{0.932194in}}%
\pgfpathlineto{\pgfqpoint{6.264848in}{1.133362in}}%
\pgfpathlineto{\pgfqpoint{6.272389in}{1.181043in}}%
\pgfpathlineto{\pgfqpoint{6.277417in}{1.200590in}}%
\pgfpathlineto{\pgfqpoint{6.281188in}{1.208109in}}%
\pgfpathlineto{\pgfqpoint{6.283701in}{1.209370in}}%
\pgfpathlineto{\pgfqpoint{6.286215in}{1.208066in}}%
\pgfpathlineto{\pgfqpoint{6.288729in}{1.203794in}}%
\pgfpathlineto{\pgfqpoint{6.292500in}{1.192066in}}%
\pgfpathlineto{\pgfqpoint{6.297528in}{1.167158in}}%
\pgfpathlineto{\pgfqpoint{6.305069in}{1.112961in}}%
\pgfpathlineto{\pgfqpoint{6.316382in}{1.007111in}}%
\pgfpathlineto{\pgfqpoint{6.331465in}{0.864091in}}%
\pgfpathlineto{\pgfqpoint{6.339006in}{0.810428in}}%
\pgfpathlineto{\pgfqpoint{6.344034in}{0.785926in}}%
\pgfpathlineto{\pgfqpoint{6.349062in}{0.772211in}}%
\pgfpathlineto{\pgfqpoint{6.351575in}{0.769776in}}%
\pgfpathlineto{\pgfqpoint{6.354089in}{0.769904in}}%
\pgfpathlineto{\pgfqpoint{6.356603in}{0.773130in}}%
\pgfpathlineto{\pgfqpoint{6.360374in}{0.783170in}}%
\pgfpathlineto{\pgfqpoint{6.365402in}{0.805921in}}%
\pgfpathlineto{\pgfqpoint{6.371686in}{0.847983in}}%
\pgfpathlineto{\pgfqpoint{6.380485in}{0.925358in}}%
\pgfpathlineto{\pgfqpoint{6.405623in}{1.155195in}}%
\pgfpathlineto{\pgfqpoint{6.411908in}{1.189481in}}%
\pgfpathlineto{\pgfqpoint{6.416936in}{1.205781in}}%
\pgfpathlineto{\pgfqpoint{6.420706in}{1.209882in}}%
\pgfpathlineto{\pgfqpoint{6.423220in}{1.209391in}}%
\pgfpathlineto{\pgfqpoint{6.425734in}{1.205909in}}%
\pgfpathlineto{\pgfqpoint{6.429505in}{1.195527in}}%
\pgfpathlineto{\pgfqpoint{6.434533in}{1.172092in}}%
\pgfpathlineto{\pgfqpoint{6.440817in}{1.129731in}}%
\pgfpathlineto{\pgfqpoint{6.447102in}{1.076282in}}%
\pgfpathlineto{\pgfqpoint{6.447102in}{1.076282in}}%
\pgfusepath{stroke}%
\end{pgfscope}%
\begin{pgfscope}%
\pgfpathrectangle{\pgfqpoint{0.895256in}{0.220000in}}{\pgfqpoint{5.550589in}{1.540000in}}%
\pgfusepath{clip}%
\pgfsetbuttcap%
\pgfsetroundjoin%
\pgfsetlinewidth{1.505625pt}%
\definecolor{currentstroke}{rgb}{0.501961,0.501961,0.501961}%
\pgfsetstrokecolor{currentstroke}%
\pgfsetdash{{5.550000pt}{2.400000pt}}{0.000000pt}%
\pgfpathmoveto{\pgfqpoint{5.842520in}{0.220000in}}%
\pgfpathlineto{\pgfqpoint{5.842520in}{1.760000in}}%
\pgfusepath{stroke}%
\end{pgfscope}%
\begin{pgfscope}%
\pgfpathrectangle{\pgfqpoint{0.895256in}{0.220000in}}{\pgfqpoint{5.550589in}{1.540000in}}%
\pgfusepath{clip}%
\pgfsetbuttcap%
\pgfsetroundjoin%
\pgfsetlinewidth{1.505625pt}%
\definecolor{currentstroke}{rgb}{0.501961,0.501961,0.501961}%
\pgfsetstrokecolor{currentstroke}%
\pgfsetdash{{5.550000pt}{2.400000pt}}{0.000000pt}%
\pgfpathmoveto{\pgfqpoint{1.015921in}{0.220000in}}%
\pgfpathlineto{\pgfqpoint{1.015921in}{1.760000in}}%
\pgfusepath{stroke}%
\end{pgfscope}%
\begin{pgfscope}%
\pgfsetrectcap%
\pgfsetmiterjoin%
\pgfsetlinewidth{0.803000pt}%
\definecolor{currentstroke}{rgb}{0.000000,0.000000,0.000000}%
\pgfsetstrokecolor{currentstroke}%
\pgfsetdash{}{0pt}%
\pgfpathmoveto{\pgfqpoint{0.895256in}{0.220000in}}%
\pgfpathlineto{\pgfqpoint{0.895256in}{1.760000in}}%
\pgfusepath{stroke}%
\end{pgfscope}%
\begin{pgfscope}%
\pgfsetrectcap%
\pgfsetmiterjoin%
\pgfsetlinewidth{0.803000pt}%
\definecolor{currentstroke}{rgb}{0.000000,0.000000,0.000000}%
\pgfsetstrokecolor{currentstroke}%
\pgfsetdash{}{0pt}%
\pgfpathmoveto{\pgfqpoint{6.445845in}{0.220000in}}%
\pgfpathlineto{\pgfqpoint{6.445845in}{1.760000in}}%
\pgfusepath{stroke}%
\end{pgfscope}%
\begin{pgfscope}%
\pgfsetrectcap%
\pgfsetmiterjoin%
\pgfsetlinewidth{0.803000pt}%
\definecolor{currentstroke}{rgb}{0.000000,0.000000,0.000000}%
\pgfsetstrokecolor{currentstroke}%
\pgfsetdash{}{0pt}%
\pgfpathmoveto{\pgfqpoint{0.895256in}{0.220000in}}%
\pgfpathlineto{\pgfqpoint{6.445845in}{0.220000in}}%
\pgfusepath{stroke}%
\end{pgfscope}%
\begin{pgfscope}%
\pgfsetrectcap%
\pgfsetmiterjoin%
\pgfsetlinewidth{0.803000pt}%
\definecolor{currentstroke}{rgb}{0.000000,0.000000,0.000000}%
\pgfsetstrokecolor{currentstroke}%
\pgfsetdash{}{0pt}%
\pgfpathmoveto{\pgfqpoint{0.895256in}{1.760000in}}%
\pgfpathlineto{\pgfqpoint{6.445845in}{1.760000in}}%
\pgfusepath{stroke}%
\end{pgfscope}%
\begin{pgfscope}%
\definecolor{textcolor}{rgb}{0.000000,0.000000,0.000000}%
\pgfsetstrokecolor{textcolor}%
\pgfsetfillcolor{textcolor}%
\pgftext[x=3.670551in,y=1.843333in,,base]{\color{textcolor}{\rmfamily\fontsize{12.000000}{14.400000}\selectfont\catcode`\^=\active\def^{\ifmmode\sp\else\^{}\fi}\catcode`\%=\active\def%{\%}Latency}}%
\end{pgfscope}%
\begin{pgfscope}%
\pgfsetbuttcap%
\pgfsetmiterjoin%
\definecolor{currentfill}{rgb}{1.000000,1.000000,1.000000}%
\pgfsetfillcolor{currentfill}%
\pgfsetfillopacity{0.800000}%
\pgfsetlinewidth{1.003750pt}%
\definecolor{currentstroke}{rgb}{0.800000,0.800000,0.800000}%
\pgfsetstrokecolor{currentstroke}%
\pgfsetstrokeopacity{0.800000}%
\pgfsetdash{}{0pt}%
\pgfpathmoveto{\pgfqpoint{5.354408in}{1.261543in}}%
\pgfpathlineto{\pgfqpoint{6.348623in}{1.261543in}}%
\pgfpathquadraticcurveto{\pgfqpoint{6.376401in}{1.261543in}}{\pgfqpoint{6.376401in}{1.289321in}}%
\pgfpathlineto{\pgfqpoint{6.376401in}{1.662778in}}%
\pgfpathquadraticcurveto{\pgfqpoint{6.376401in}{1.690556in}}{\pgfqpoint{6.348623in}{1.690556in}}%
\pgfpathlineto{\pgfqpoint{5.354408in}{1.690556in}}%
\pgfpathquadraticcurveto{\pgfqpoint{5.326631in}{1.690556in}}{\pgfqpoint{5.326631in}{1.662778in}}%
\pgfpathlineto{\pgfqpoint{5.326631in}{1.289321in}}%
\pgfpathquadraticcurveto{\pgfqpoint{5.326631in}{1.261543in}}{\pgfqpoint{5.354408in}{1.261543in}}%
\pgfpathlineto{\pgfqpoint{5.354408in}{1.261543in}}%
\pgfpathclose%
\pgfusepath{stroke,fill}%
\end{pgfscope}%
\begin{pgfscope}%
\pgfsetrectcap%
\pgfsetroundjoin%
\pgfsetlinewidth{1.505625pt}%
\definecolor{currentstroke}{rgb}{0.121569,0.466667,0.705882}%
\pgfsetstrokecolor{currentstroke}%
\pgfsetdash{}{0pt}%
\pgfpathmoveto{\pgfqpoint{5.382186in}{1.586389in}}%
\pgfpathlineto{\pgfqpoint{5.521075in}{1.586389in}}%
\pgfpathlineto{\pgfqpoint{5.659964in}{1.586389in}}%
\pgfusepath{stroke}%
\end{pgfscope}%
\begin{pgfscope}%
\definecolor{textcolor}{rgb}{0.000000,0.000000,0.000000}%
\pgfsetstrokecolor{textcolor}%
\pgfsetfillcolor{textcolor}%
\pgftext[x=5.771075in,y=1.537778in,left,base]{\color{textcolor}{\rmfamily\fontsize{10.000000}{12.000000}\selectfont\catcode`\^=\active\def^{\ifmmode\sp\else\^{}\fi}\catcode`\%=\active\def%{\%}Playback}}%
\end{pgfscope}%
\begin{pgfscope}%
\pgfsetrectcap%
\pgfsetroundjoin%
\pgfsetlinewidth{1.505625pt}%
\definecolor{currentstroke}{rgb}{1.000000,0.498039,0.054902}%
\pgfsetstrokecolor{currentstroke}%
\pgfsetdash{}{0pt}%
\pgfpathmoveto{\pgfqpoint{5.382186in}{1.392716in}}%
\pgfpathlineto{\pgfqpoint{5.521075in}{1.392716in}}%
\pgfpathlineto{\pgfqpoint{5.659964in}{1.392716in}}%
\pgfusepath{stroke}%
\end{pgfscope}%
\begin{pgfscope}%
\definecolor{textcolor}{rgb}{0.000000,0.000000,0.000000}%
\pgfsetstrokecolor{textcolor}%
\pgfsetfillcolor{textcolor}%
\pgftext[x=5.771075in,y=1.344105in,left,base]{\color{textcolor}{\rmfamily\fontsize{10.000000}{12.000000}\selectfont\catcode`\^=\active\def^{\ifmmode\sp\else\^{}\fi}\catcode`\%=\active\def%{\%}Capture}}%
\end{pgfscope}%
\end{pgfpicture}%
\makeatother%
\endgroup%
}
    \caption{The plot to show the latency on the provided sound card.}
    \label{fig:lat}
\end{figure}

We measure the latency of the provided sound card by playing a sound wave and recording it at the same time. As illustrated in Figure \ref{fig:lat}, the latency of the provided sound card is around \(0.08\) seconds.
The experiment environment is 1) Linux, 2) miniaudio framework, 3) use a wire to connect the output and input of the sound card.

\textbf{The latency of 0.08 seconds is unacceptable for our communication system. We find that the latency is much smaller when we use Windows.}

\subsection{Clock Synchronization}

Clock synchronization is the process of synchronizing the clocks of the recording device and the playback device.
This is crucial for our communication system, as the timing of the received signal is crucial for correct demodulation and decoding, no matter the modulation scheme.

We use chirp to synchronize the clocks.
A chirp signal also denotes the beginning of a frame.

Chirp signal is described by the following equation, where \(f_0\) and \(f_1\) are the start and end frequencies, respectively, and \(T\) is the duration of the chirp signal:

\begin{gather*}
    f(t) = ct+f_0\\
    c=\frac{f_1-f_0}{T}\\
    x(t)=\sin\left(2\pi \left(\frac{c}{2}t^2 + f_0 t\right)\right)
\end{gather*}

\begin{figure}[H]
    \noindent\makebox[\textwidth]{%% Creator: Matplotlib, PGF backend
%%
%% To include the figure in your LaTeX document, write
%%   \input{<filename>.pgf}
%%
%% Make sure the required packages are loaded in your preamble
%%   \usepackage{pgf}
%%
%% Also ensure that all the required font packages are loaded; for instance,
%% the lmodern package is sometimes necessary when using math font.
%%   \usepackage{lmodern}
%%
%% Figures using additional raster images can only be included by \input if
%% they are in the same directory as the main LaTeX file. For loading figures
%% from other directories you can use the `import` package
%%   \usepackage{import}
%%
%% and then include the figures with
%%   \import{<path to file>}{<filename>.pgf}
%%
%% Matplotlib used the following preamble
%%   \def\mathdefault#1{#1}
%%   \everymath=\expandafter{\the\everymath\displaystyle}
%%   \IfFileExists{scrextend.sty}{
%%     \usepackage[fontsize=10.000000pt]{scrextend}
%%   }{
%%     \renewcommand{\normalsize}{\fontsize{10.000000}{12.000000}\selectfont}
%%     \normalsize
%%   }
%%   
%%   \makeatletter\@ifpackageloaded{underscore}{}{\usepackage[strings]{underscore}}\makeatother
%%
\begingroup%
\makeatletter%
\begin{pgfpicture}%
\pgfpathrectangle{\pgfpointorigin}{\pgfqpoint{7.162050in}{2.000000in}}%
\pgfusepath{use as bounding box, clip}%
\begin{pgfscope}%
\pgfsetbuttcap%
\pgfsetmiterjoin%
\definecolor{currentfill}{rgb}{1.000000,1.000000,1.000000}%
\pgfsetfillcolor{currentfill}%
\pgfsetlinewidth{0.000000pt}%
\definecolor{currentstroke}{rgb}{1.000000,1.000000,1.000000}%
\pgfsetstrokecolor{currentstroke}%
\pgfsetdash{}{0pt}%
\pgfpathmoveto{\pgfqpoint{0.000000in}{0.000000in}}%
\pgfpathlineto{\pgfqpoint{7.162050in}{0.000000in}}%
\pgfpathlineto{\pgfqpoint{7.162050in}{2.000000in}}%
\pgfpathlineto{\pgfqpoint{0.000000in}{2.000000in}}%
\pgfpathlineto{\pgfqpoint{0.000000in}{0.000000in}}%
\pgfpathclose%
\pgfusepath{fill}%
\end{pgfscope}%
\begin{pgfscope}%
\pgfsetbuttcap%
\pgfsetmiterjoin%
\definecolor{currentfill}{rgb}{1.000000,1.000000,1.000000}%
\pgfsetfillcolor{currentfill}%
\pgfsetlinewidth{0.000000pt}%
\definecolor{currentstroke}{rgb}{0.000000,0.000000,0.000000}%
\pgfsetstrokecolor{currentstroke}%
\pgfsetstrokeopacity{0.000000}%
\pgfsetdash{}{0pt}%
\pgfpathmoveto{\pgfqpoint{0.895256in}{0.220000in}}%
\pgfpathlineto{\pgfqpoint{6.445845in}{0.220000in}}%
\pgfpathlineto{\pgfqpoint{6.445845in}{1.760000in}}%
\pgfpathlineto{\pgfqpoint{0.895256in}{1.760000in}}%
\pgfpathlineto{\pgfqpoint{0.895256in}{0.220000in}}%
\pgfpathclose%
\pgfusepath{fill}%
\end{pgfscope}%
\begin{pgfscope}%
\pgfsetbuttcap%
\pgfsetroundjoin%
\definecolor{currentfill}{rgb}{0.000000,0.000000,0.000000}%
\pgfsetfillcolor{currentfill}%
\pgfsetlinewidth{0.803000pt}%
\definecolor{currentstroke}{rgb}{0.000000,0.000000,0.000000}%
\pgfsetstrokecolor{currentstroke}%
\pgfsetdash{}{0pt}%
\pgfsys@defobject{currentmarker}{\pgfqpoint{0.000000in}{-0.048611in}}{\pgfqpoint{0.000000in}{0.000000in}}{%
\pgfpathmoveto{\pgfqpoint{0.000000in}{0.000000in}}%
\pgfpathlineto{\pgfqpoint{0.000000in}{-0.048611in}}%
\pgfusepath{stroke,fill}%
}%
\begin{pgfscope}%
\pgfsys@transformshift{1.147556in}{0.220000in}%
\pgfsys@useobject{currentmarker}{}%
\end{pgfscope}%
\end{pgfscope}%
\begin{pgfscope}%
\definecolor{textcolor}{rgb}{0.000000,0.000000,0.000000}%
\pgfsetstrokecolor{textcolor}%
\pgfsetfillcolor{textcolor}%
\pgftext[x=1.147556in,y=0.122778in,,top]{\color{textcolor}{\rmfamily\fontsize{10.000000}{12.000000}\selectfont\catcode`\^=\active\def^{\ifmmode\sp\else\^{}\fi}\catcode`\%=\active\def%{\%}$\mathdefault{0.000}$}}%
\end{pgfscope}%
\begin{pgfscope}%
\pgfsetbuttcap%
\pgfsetroundjoin%
\definecolor{currentfill}{rgb}{0.000000,0.000000,0.000000}%
\pgfsetfillcolor{currentfill}%
\pgfsetlinewidth{0.803000pt}%
\definecolor{currentstroke}{rgb}{0.000000,0.000000,0.000000}%
\pgfsetstrokecolor{currentstroke}%
\pgfsetdash{}{0pt}%
\pgfsys@defobject{currentmarker}{\pgfqpoint{0.000000in}{-0.048611in}}{\pgfqpoint{0.000000in}{0.000000in}}{%
\pgfpathmoveto{\pgfqpoint{0.000000in}{0.000000in}}%
\pgfpathlineto{\pgfqpoint{0.000000in}{-0.048611in}}%
\pgfusepath{stroke,fill}%
}%
\begin{pgfscope}%
\pgfsys@transformshift{1.779951in}{0.220000in}%
\pgfsys@useobject{currentmarker}{}%
\end{pgfscope}%
\end{pgfscope}%
\begin{pgfscope}%
\definecolor{textcolor}{rgb}{0.000000,0.000000,0.000000}%
\pgfsetstrokecolor{textcolor}%
\pgfsetfillcolor{textcolor}%
\pgftext[x=1.779951in,y=0.122778in,,top]{\color{textcolor}{\rmfamily\fontsize{10.000000}{12.000000}\selectfont\catcode`\^=\active\def^{\ifmmode\sp\else\^{}\fi}\catcode`\%=\active\def%{\%}$\mathdefault{0.001}$}}%
\end{pgfscope}%
\begin{pgfscope}%
\pgfsetbuttcap%
\pgfsetroundjoin%
\definecolor{currentfill}{rgb}{0.000000,0.000000,0.000000}%
\pgfsetfillcolor{currentfill}%
\pgfsetlinewidth{0.803000pt}%
\definecolor{currentstroke}{rgb}{0.000000,0.000000,0.000000}%
\pgfsetstrokecolor{currentstroke}%
\pgfsetdash{}{0pt}%
\pgfsys@defobject{currentmarker}{\pgfqpoint{0.000000in}{-0.048611in}}{\pgfqpoint{0.000000in}{0.000000in}}{%
\pgfpathmoveto{\pgfqpoint{0.000000in}{0.000000in}}%
\pgfpathlineto{\pgfqpoint{0.000000in}{-0.048611in}}%
\pgfusepath{stroke,fill}%
}%
\begin{pgfscope}%
\pgfsys@transformshift{2.412347in}{0.220000in}%
\pgfsys@useobject{currentmarker}{}%
\end{pgfscope}%
\end{pgfscope}%
\begin{pgfscope}%
\definecolor{textcolor}{rgb}{0.000000,0.000000,0.000000}%
\pgfsetstrokecolor{textcolor}%
\pgfsetfillcolor{textcolor}%
\pgftext[x=2.412347in,y=0.122778in,,top]{\color{textcolor}{\rmfamily\fontsize{10.000000}{12.000000}\selectfont\catcode`\^=\active\def^{\ifmmode\sp\else\^{}\fi}\catcode`\%=\active\def%{\%}$\mathdefault{0.002}$}}%
\end{pgfscope}%
\begin{pgfscope}%
\pgfsetbuttcap%
\pgfsetroundjoin%
\definecolor{currentfill}{rgb}{0.000000,0.000000,0.000000}%
\pgfsetfillcolor{currentfill}%
\pgfsetlinewidth{0.803000pt}%
\definecolor{currentstroke}{rgb}{0.000000,0.000000,0.000000}%
\pgfsetstrokecolor{currentstroke}%
\pgfsetdash{}{0pt}%
\pgfsys@defobject{currentmarker}{\pgfqpoint{0.000000in}{-0.048611in}}{\pgfqpoint{0.000000in}{0.000000in}}{%
\pgfpathmoveto{\pgfqpoint{0.000000in}{0.000000in}}%
\pgfpathlineto{\pgfqpoint{0.000000in}{-0.048611in}}%
\pgfusepath{stroke,fill}%
}%
\begin{pgfscope}%
\pgfsys@transformshift{3.044742in}{0.220000in}%
\pgfsys@useobject{currentmarker}{}%
\end{pgfscope}%
\end{pgfscope}%
\begin{pgfscope}%
\definecolor{textcolor}{rgb}{0.000000,0.000000,0.000000}%
\pgfsetstrokecolor{textcolor}%
\pgfsetfillcolor{textcolor}%
\pgftext[x=3.044742in,y=0.122778in,,top]{\color{textcolor}{\rmfamily\fontsize{10.000000}{12.000000}\selectfont\catcode`\^=\active\def^{\ifmmode\sp\else\^{}\fi}\catcode`\%=\active\def%{\%}$\mathdefault{0.003}$}}%
\end{pgfscope}%
\begin{pgfscope}%
\pgfsetbuttcap%
\pgfsetroundjoin%
\definecolor{currentfill}{rgb}{0.000000,0.000000,0.000000}%
\pgfsetfillcolor{currentfill}%
\pgfsetlinewidth{0.803000pt}%
\definecolor{currentstroke}{rgb}{0.000000,0.000000,0.000000}%
\pgfsetstrokecolor{currentstroke}%
\pgfsetdash{}{0pt}%
\pgfsys@defobject{currentmarker}{\pgfqpoint{0.000000in}{-0.048611in}}{\pgfqpoint{0.000000in}{0.000000in}}{%
\pgfpathmoveto{\pgfqpoint{0.000000in}{0.000000in}}%
\pgfpathlineto{\pgfqpoint{0.000000in}{-0.048611in}}%
\pgfusepath{stroke,fill}%
}%
\begin{pgfscope}%
\pgfsys@transformshift{3.677138in}{0.220000in}%
\pgfsys@useobject{currentmarker}{}%
\end{pgfscope}%
\end{pgfscope}%
\begin{pgfscope}%
\definecolor{textcolor}{rgb}{0.000000,0.000000,0.000000}%
\pgfsetstrokecolor{textcolor}%
\pgfsetfillcolor{textcolor}%
\pgftext[x=3.677138in,y=0.122778in,,top]{\color{textcolor}{\rmfamily\fontsize{10.000000}{12.000000}\selectfont\catcode`\^=\active\def^{\ifmmode\sp\else\^{}\fi}\catcode`\%=\active\def%{\%}$\mathdefault{0.004}$}}%
\end{pgfscope}%
\begin{pgfscope}%
\pgfsetbuttcap%
\pgfsetroundjoin%
\definecolor{currentfill}{rgb}{0.000000,0.000000,0.000000}%
\pgfsetfillcolor{currentfill}%
\pgfsetlinewidth{0.803000pt}%
\definecolor{currentstroke}{rgb}{0.000000,0.000000,0.000000}%
\pgfsetstrokecolor{currentstroke}%
\pgfsetdash{}{0pt}%
\pgfsys@defobject{currentmarker}{\pgfqpoint{0.000000in}{-0.048611in}}{\pgfqpoint{0.000000in}{0.000000in}}{%
\pgfpathmoveto{\pgfqpoint{0.000000in}{0.000000in}}%
\pgfpathlineto{\pgfqpoint{0.000000in}{-0.048611in}}%
\pgfusepath{stroke,fill}%
}%
\begin{pgfscope}%
\pgfsys@transformshift{4.309534in}{0.220000in}%
\pgfsys@useobject{currentmarker}{}%
\end{pgfscope}%
\end{pgfscope}%
\begin{pgfscope}%
\definecolor{textcolor}{rgb}{0.000000,0.000000,0.000000}%
\pgfsetstrokecolor{textcolor}%
\pgfsetfillcolor{textcolor}%
\pgftext[x=4.309534in,y=0.122778in,,top]{\color{textcolor}{\rmfamily\fontsize{10.000000}{12.000000}\selectfont\catcode`\^=\active\def^{\ifmmode\sp\else\^{}\fi}\catcode`\%=\active\def%{\%}$\mathdefault{0.005}$}}%
\end{pgfscope}%
\begin{pgfscope}%
\pgfsetbuttcap%
\pgfsetroundjoin%
\definecolor{currentfill}{rgb}{0.000000,0.000000,0.000000}%
\pgfsetfillcolor{currentfill}%
\pgfsetlinewidth{0.803000pt}%
\definecolor{currentstroke}{rgb}{0.000000,0.000000,0.000000}%
\pgfsetstrokecolor{currentstroke}%
\pgfsetdash{}{0pt}%
\pgfsys@defobject{currentmarker}{\pgfqpoint{0.000000in}{-0.048611in}}{\pgfqpoint{0.000000in}{0.000000in}}{%
\pgfpathmoveto{\pgfqpoint{0.000000in}{0.000000in}}%
\pgfpathlineto{\pgfqpoint{0.000000in}{-0.048611in}}%
\pgfusepath{stroke,fill}%
}%
\begin{pgfscope}%
\pgfsys@transformshift{4.941929in}{0.220000in}%
\pgfsys@useobject{currentmarker}{}%
\end{pgfscope}%
\end{pgfscope}%
\begin{pgfscope}%
\definecolor{textcolor}{rgb}{0.000000,0.000000,0.000000}%
\pgfsetstrokecolor{textcolor}%
\pgfsetfillcolor{textcolor}%
\pgftext[x=4.941929in,y=0.122778in,,top]{\color{textcolor}{\rmfamily\fontsize{10.000000}{12.000000}\selectfont\catcode`\^=\active\def^{\ifmmode\sp\else\^{}\fi}\catcode`\%=\active\def%{\%}$\mathdefault{0.006}$}}%
\end{pgfscope}%
\begin{pgfscope}%
\pgfsetbuttcap%
\pgfsetroundjoin%
\definecolor{currentfill}{rgb}{0.000000,0.000000,0.000000}%
\pgfsetfillcolor{currentfill}%
\pgfsetlinewidth{0.803000pt}%
\definecolor{currentstroke}{rgb}{0.000000,0.000000,0.000000}%
\pgfsetstrokecolor{currentstroke}%
\pgfsetdash{}{0pt}%
\pgfsys@defobject{currentmarker}{\pgfqpoint{0.000000in}{-0.048611in}}{\pgfqpoint{0.000000in}{0.000000in}}{%
\pgfpathmoveto{\pgfqpoint{0.000000in}{0.000000in}}%
\pgfpathlineto{\pgfqpoint{0.000000in}{-0.048611in}}%
\pgfusepath{stroke,fill}%
}%
\begin{pgfscope}%
\pgfsys@transformshift{5.574325in}{0.220000in}%
\pgfsys@useobject{currentmarker}{}%
\end{pgfscope}%
\end{pgfscope}%
\begin{pgfscope}%
\definecolor{textcolor}{rgb}{0.000000,0.000000,0.000000}%
\pgfsetstrokecolor{textcolor}%
\pgfsetfillcolor{textcolor}%
\pgftext[x=5.574325in,y=0.122778in,,top]{\color{textcolor}{\rmfamily\fontsize{10.000000}{12.000000}\selectfont\catcode`\^=\active\def^{\ifmmode\sp\else\^{}\fi}\catcode`\%=\active\def%{\%}$\mathdefault{0.007}$}}%
\end{pgfscope}%
\begin{pgfscope}%
\pgfsetbuttcap%
\pgfsetroundjoin%
\definecolor{currentfill}{rgb}{0.000000,0.000000,0.000000}%
\pgfsetfillcolor{currentfill}%
\pgfsetlinewidth{0.803000pt}%
\definecolor{currentstroke}{rgb}{0.000000,0.000000,0.000000}%
\pgfsetstrokecolor{currentstroke}%
\pgfsetdash{}{0pt}%
\pgfsys@defobject{currentmarker}{\pgfqpoint{0.000000in}{-0.048611in}}{\pgfqpoint{0.000000in}{0.000000in}}{%
\pgfpathmoveto{\pgfqpoint{0.000000in}{0.000000in}}%
\pgfpathlineto{\pgfqpoint{0.000000in}{-0.048611in}}%
\pgfusepath{stroke,fill}%
}%
\begin{pgfscope}%
\pgfsys@transformshift{6.206720in}{0.220000in}%
\pgfsys@useobject{currentmarker}{}%
\end{pgfscope}%
\end{pgfscope}%
\begin{pgfscope}%
\definecolor{textcolor}{rgb}{0.000000,0.000000,0.000000}%
\pgfsetstrokecolor{textcolor}%
\pgfsetfillcolor{textcolor}%
\pgftext[x=6.206720in,y=0.122778in,,top]{\color{textcolor}{\rmfamily\fontsize{10.000000}{12.000000}\selectfont\catcode`\^=\active\def^{\ifmmode\sp\else\^{}\fi}\catcode`\%=\active\def%{\%}$\mathdefault{0.008}$}}%
\end{pgfscope}%
\begin{pgfscope}%
\definecolor{textcolor}{rgb}{0.000000,0.000000,0.000000}%
\pgfsetstrokecolor{textcolor}%
\pgfsetfillcolor{textcolor}%
\pgftext[x=3.670551in,y=-0.056234in,,top]{\color{textcolor}{\rmfamily\fontsize{10.000000}{12.000000}\selectfont\catcode`\^=\active\def^{\ifmmode\sp\else\^{}\fi}\catcode`\%=\active\def%{\%}Time (s)}}%
\end{pgfscope}%
\begin{pgfscope}%
\pgfsetbuttcap%
\pgfsetroundjoin%
\definecolor{currentfill}{rgb}{0.000000,0.000000,0.000000}%
\pgfsetfillcolor{currentfill}%
\pgfsetlinewidth{0.803000pt}%
\definecolor{currentstroke}{rgb}{0.000000,0.000000,0.000000}%
\pgfsetstrokecolor{currentstroke}%
\pgfsetdash{}{0pt}%
\pgfsys@defobject{currentmarker}{\pgfqpoint{-0.048611in}{0.000000in}}{\pgfqpoint{-0.000000in}{0.000000in}}{%
\pgfpathmoveto{\pgfqpoint{-0.000000in}{0.000000in}}%
\pgfpathlineto{\pgfqpoint{-0.048611in}{0.000000in}}%
\pgfusepath{stroke,fill}%
}%
\begin{pgfscope}%
\pgfsys@transformshift{0.895256in}{0.289965in}%
\pgfsys@useobject{currentmarker}{}%
\end{pgfscope}%
\end{pgfscope}%
\begin{pgfscope}%
\definecolor{textcolor}{rgb}{0.000000,0.000000,0.000000}%
\pgfsetstrokecolor{textcolor}%
\pgfsetfillcolor{textcolor}%
\pgftext[x=0.512539in, y=0.241740in, left, base]{\color{textcolor}{\rmfamily\fontsize{10.000000}{12.000000}\selectfont\catcode`\^=\active\def^{\ifmmode\sp\else\^{}\fi}\catcode`\%=\active\def%{\%}$\mathdefault{\ensuremath{-}1.0}$}}%
\end{pgfscope}%
\begin{pgfscope}%
\pgfsetbuttcap%
\pgfsetroundjoin%
\definecolor{currentfill}{rgb}{0.000000,0.000000,0.000000}%
\pgfsetfillcolor{currentfill}%
\pgfsetlinewidth{0.803000pt}%
\definecolor{currentstroke}{rgb}{0.000000,0.000000,0.000000}%
\pgfsetstrokecolor{currentstroke}%
\pgfsetdash{}{0pt}%
\pgfsys@defobject{currentmarker}{\pgfqpoint{-0.048611in}{0.000000in}}{\pgfqpoint{-0.000000in}{0.000000in}}{%
\pgfpathmoveto{\pgfqpoint{-0.000000in}{0.000000in}}%
\pgfpathlineto{\pgfqpoint{-0.048611in}{0.000000in}}%
\pgfusepath{stroke,fill}%
}%
\begin{pgfscope}%
\pgfsys@transformshift{0.895256in}{0.639983in}%
\pgfsys@useobject{currentmarker}{}%
\end{pgfscope}%
\end{pgfscope}%
\begin{pgfscope}%
\definecolor{textcolor}{rgb}{0.000000,0.000000,0.000000}%
\pgfsetstrokecolor{textcolor}%
\pgfsetfillcolor{textcolor}%
\pgftext[x=0.512539in, y=0.591757in, left, base]{\color{textcolor}{\rmfamily\fontsize{10.000000}{12.000000}\selectfont\catcode`\^=\active\def^{\ifmmode\sp\else\^{}\fi}\catcode`\%=\active\def%{\%}$\mathdefault{\ensuremath{-}0.5}$}}%
\end{pgfscope}%
\begin{pgfscope}%
\pgfsetbuttcap%
\pgfsetroundjoin%
\definecolor{currentfill}{rgb}{0.000000,0.000000,0.000000}%
\pgfsetfillcolor{currentfill}%
\pgfsetlinewidth{0.803000pt}%
\definecolor{currentstroke}{rgb}{0.000000,0.000000,0.000000}%
\pgfsetstrokecolor{currentstroke}%
\pgfsetdash{}{0pt}%
\pgfsys@defobject{currentmarker}{\pgfqpoint{-0.048611in}{0.000000in}}{\pgfqpoint{-0.000000in}{0.000000in}}{%
\pgfpathmoveto{\pgfqpoint{-0.000000in}{0.000000in}}%
\pgfpathlineto{\pgfqpoint{-0.048611in}{0.000000in}}%
\pgfusepath{stroke,fill}%
}%
\begin{pgfscope}%
\pgfsys@transformshift{0.895256in}{0.990000in}%
\pgfsys@useobject{currentmarker}{}%
\end{pgfscope}%
\end{pgfscope}%
\begin{pgfscope}%
\definecolor{textcolor}{rgb}{0.000000,0.000000,0.000000}%
\pgfsetstrokecolor{textcolor}%
\pgfsetfillcolor{textcolor}%
\pgftext[x=0.620564in, y=0.941775in, left, base]{\color{textcolor}{\rmfamily\fontsize{10.000000}{12.000000}\selectfont\catcode`\^=\active\def^{\ifmmode\sp\else\^{}\fi}\catcode`\%=\active\def%{\%}$\mathdefault{0.0}$}}%
\end{pgfscope}%
\begin{pgfscope}%
\pgfsetbuttcap%
\pgfsetroundjoin%
\definecolor{currentfill}{rgb}{0.000000,0.000000,0.000000}%
\pgfsetfillcolor{currentfill}%
\pgfsetlinewidth{0.803000pt}%
\definecolor{currentstroke}{rgb}{0.000000,0.000000,0.000000}%
\pgfsetstrokecolor{currentstroke}%
\pgfsetdash{}{0pt}%
\pgfsys@defobject{currentmarker}{\pgfqpoint{-0.048611in}{0.000000in}}{\pgfqpoint{-0.000000in}{0.000000in}}{%
\pgfpathmoveto{\pgfqpoint{-0.000000in}{0.000000in}}%
\pgfpathlineto{\pgfqpoint{-0.048611in}{0.000000in}}%
\pgfusepath{stroke,fill}%
}%
\begin{pgfscope}%
\pgfsys@transformshift{0.895256in}{1.340017in}%
\pgfsys@useobject{currentmarker}{}%
\end{pgfscope}%
\end{pgfscope}%
\begin{pgfscope}%
\definecolor{textcolor}{rgb}{0.000000,0.000000,0.000000}%
\pgfsetstrokecolor{textcolor}%
\pgfsetfillcolor{textcolor}%
\pgftext[x=0.620564in, y=1.291792in, left, base]{\color{textcolor}{\rmfamily\fontsize{10.000000}{12.000000}\selectfont\catcode`\^=\active\def^{\ifmmode\sp\else\^{}\fi}\catcode`\%=\active\def%{\%}$\mathdefault{0.5}$}}%
\end{pgfscope}%
\begin{pgfscope}%
\pgfsetbuttcap%
\pgfsetroundjoin%
\definecolor{currentfill}{rgb}{0.000000,0.000000,0.000000}%
\pgfsetfillcolor{currentfill}%
\pgfsetlinewidth{0.803000pt}%
\definecolor{currentstroke}{rgb}{0.000000,0.000000,0.000000}%
\pgfsetstrokecolor{currentstroke}%
\pgfsetdash{}{0pt}%
\pgfsys@defobject{currentmarker}{\pgfqpoint{-0.048611in}{0.000000in}}{\pgfqpoint{-0.000000in}{0.000000in}}{%
\pgfpathmoveto{\pgfqpoint{-0.000000in}{0.000000in}}%
\pgfpathlineto{\pgfqpoint{-0.048611in}{0.000000in}}%
\pgfusepath{stroke,fill}%
}%
\begin{pgfscope}%
\pgfsys@transformshift{0.895256in}{1.690035in}%
\pgfsys@useobject{currentmarker}{}%
\end{pgfscope}%
\end{pgfscope}%
\begin{pgfscope}%
\definecolor{textcolor}{rgb}{0.000000,0.000000,0.000000}%
\pgfsetstrokecolor{textcolor}%
\pgfsetfillcolor{textcolor}%
\pgftext[x=0.620564in, y=1.641810in, left, base]{\color{textcolor}{\rmfamily\fontsize{10.000000}{12.000000}\selectfont\catcode`\^=\active\def^{\ifmmode\sp\else\^{}\fi}\catcode`\%=\active\def%{\%}$\mathdefault{1.0}$}}%
\end{pgfscope}%
\begin{pgfscope}%
\definecolor{textcolor}{rgb}{0.000000,0.000000,0.000000}%
\pgfsetstrokecolor{textcolor}%
\pgfsetfillcolor{textcolor}%
\pgftext[x=0.456984in,y=0.990000in,,bottom,rotate=90.000000]{\color{textcolor}{\rmfamily\fontsize{10.000000}{12.000000}\selectfont\catcode`\^=\active\def^{\ifmmode\sp\else\^{}\fi}\catcode`\%=\active\def%{\%}Amplitude}}%
\end{pgfscope}%
\begin{pgfscope}%
\pgfpathrectangle{\pgfqpoint{0.895256in}{0.220000in}}{\pgfqpoint{5.550589in}{1.540000in}}%
\pgfusepath{clip}%
\pgfsetrectcap%
\pgfsetroundjoin%
\pgfsetlinewidth{1.505625pt}%
\definecolor{currentstroke}{rgb}{0.121569,0.466667,0.705882}%
\pgfsetstrokecolor{currentstroke}%
\pgfsetdash{}{0pt}%
\pgfpathmoveto{\pgfqpoint{1.147556in}{0.990000in}}%
\pgfpathlineto{\pgfqpoint{1.173906in}{1.216148in}}%
\pgfpathlineto{\pgfqpoint{1.200255in}{1.419931in}}%
\pgfpathlineto{\pgfqpoint{1.213430in}{1.505999in}}%
\pgfpathlineto{\pgfqpoint{1.226605in}{1.577955in}}%
\pgfpathlineto{\pgfqpoint{1.239780in}{1.633529in}}%
\pgfpathlineto{\pgfqpoint{1.252955in}{1.670871in}}%
\pgfpathlineto{\pgfqpoint{1.266130in}{1.688617in}}%
\pgfpathlineto{\pgfqpoint{1.279305in}{1.685952in}}%
\pgfpathlineto{\pgfqpoint{1.292480in}{1.662650in}}%
\pgfpathlineto{\pgfqpoint{1.305655in}{1.619103in}}%
\pgfpathlineto{\pgfqpoint{1.318830in}{1.556333in}}%
\pgfpathlineto{\pgfqpoint{1.332004in}{1.475976in}}%
\pgfpathlineto{\pgfqpoint{1.345179in}{1.380257in}}%
\pgfpathlineto{\pgfqpoint{1.358354in}{1.271937in}}%
\pgfpathlineto{\pgfqpoint{1.384704in}{1.030783in}}%
\pgfpathlineto{\pgfqpoint{1.424229in}{0.665226in}}%
\pgfpathlineto{\pgfqpoint{1.437404in}{0.558376in}}%
\pgfpathlineto{\pgfqpoint{1.450579in}{0.465381in}}%
\pgfpathlineto{\pgfqpoint{1.463754in}{0.389560in}}%
\pgfpathlineto{\pgfqpoint{1.476928in}{0.333726in}}%
\pgfpathlineto{\pgfqpoint{1.490103in}{0.300068in}}%
\pgfpathlineto{\pgfqpoint{1.503278in}{0.290056in}}%
\pgfpathlineto{\pgfqpoint{1.516453in}{0.304363in}}%
\pgfpathlineto{\pgfqpoint{1.529628in}{0.342819in}}%
\pgfpathlineto{\pgfqpoint{1.542803in}{0.404387in}}%
\pgfpathlineto{\pgfqpoint{1.555978in}{0.487175in}}%
\pgfpathlineto{\pgfqpoint{1.569153in}{0.588477in}}%
\pgfpathlineto{\pgfqpoint{1.582328in}{0.704845in}}%
\pgfpathlineto{\pgfqpoint{1.608678in}{0.965963in}}%
\pgfpathlineto{\pgfqpoint{1.635027in}{1.232842in}}%
\pgfpathlineto{\pgfqpoint{1.648202in}{1.355808in}}%
\pgfpathlineto{\pgfqpoint{1.661377in}{1.465263in}}%
\pgfpathlineto{\pgfqpoint{1.674552in}{1.556789in}}%
\pgfpathlineto{\pgfqpoint{1.687727in}{1.626573in}}%
\pgfpathlineto{\pgfqpoint{1.700902in}{1.671585in}}%
\pgfpathlineto{\pgfqpoint{1.714077in}{1.689714in}}%
\pgfpathlineto{\pgfqpoint{1.727252in}{1.679891in}}%
\pgfpathlineto{\pgfqpoint{1.740427in}{1.642156in}}%
\pgfpathlineto{\pgfqpoint{1.753602in}{1.577696in}}%
\pgfpathlineto{\pgfqpoint{1.766776in}{1.488823in}}%
\pgfpathlineto{\pgfqpoint{1.779951in}{1.378918in}}%
\pgfpathlineto{\pgfqpoint{1.793126in}{1.252315in}}%
\pgfpathlineto{\pgfqpoint{1.859001in}{0.564414in}}%
\pgfpathlineto{\pgfqpoint{1.872176in}{0.457868in}}%
\pgfpathlineto{\pgfqpoint{1.885351in}{0.374433in}}%
\pgfpathlineto{\pgfqpoint{1.898525in}{0.318116in}}%
\pgfpathlineto{\pgfqpoint{1.911700in}{0.291775in}}%
\pgfpathlineto{\pgfqpoint{1.924875in}{0.296967in}}%
\pgfpathlineto{\pgfqpoint{1.938050in}{0.333830in}}%
\pgfpathlineto{\pgfqpoint{1.951225in}{0.401043in}}%
\pgfpathlineto{\pgfqpoint{1.964400in}{0.495845in}}%
\pgfpathlineto{\pgfqpoint{1.977575in}{0.614123in}}%
\pgfpathlineto{\pgfqpoint{1.990750in}{0.750574in}}%
\pgfpathlineto{\pgfqpoint{2.043449in}{1.344142in}}%
\pgfpathlineto{\pgfqpoint{2.056624in}{1.468408in}}%
\pgfpathlineto{\pgfqpoint{2.069799in}{1.569527in}}%
\pgfpathlineto{\pgfqpoint{2.082974in}{1.642200in}}%
\pgfpathlineto{\pgfqpoint{2.096149in}{1.682458in}}%
\pgfpathlineto{\pgfqpoint{2.109324in}{1.687898in}}%
\pgfpathlineto{\pgfqpoint{2.122499in}{1.657850in}}%
\pgfpathlineto{\pgfqpoint{2.135674in}{1.593455in}}%
\pgfpathlineto{\pgfqpoint{2.148849in}{1.497656in}}%
\pgfpathlineto{\pgfqpoint{2.162024in}{1.375091in}}%
\pgfpathlineto{\pgfqpoint{2.175199in}{1.231891in}}%
\pgfpathlineto{\pgfqpoint{2.227898in}{0.609655in}}%
\pgfpathlineto{\pgfqpoint{2.241073in}{0.483784in}}%
\pgfpathlineto{\pgfqpoint{2.254248in}{0.385189in}}%
\pgfpathlineto{\pgfqpoint{2.267423in}{0.319608in}}%
\pgfpathlineto{\pgfqpoint{2.280598in}{0.291046in}}%
\pgfpathlineto{\pgfqpoint{2.293773in}{0.301499in}}%
\pgfpathlineto{\pgfqpoint{2.306948in}{0.350797in}}%
\pgfpathlineto{\pgfqpoint{2.320123in}{0.436556in}}%
\pgfpathlineto{\pgfqpoint{2.333297in}{0.554255in}}%
\pgfpathlineto{\pgfqpoint{2.346472in}{0.697455in}}%
\pgfpathlineto{\pgfqpoint{2.372822in}{1.027030in}}%
\pgfpathlineto{\pgfqpoint{2.399172in}{1.350146in}}%
\pgfpathlineto{\pgfqpoint{2.412347in}{1.484999in}}%
\pgfpathlineto{\pgfqpoint{2.425522in}{1.590593in}}%
\pgfpathlineto{\pgfqpoint{2.438697in}{1.660237in}}%
\pgfpathlineto{\pgfqpoint{2.451872in}{1.689318in}}%
\pgfpathlineto{\pgfqpoint{2.465047in}{1.675637in}}%
\pgfpathlineto{\pgfqpoint{2.478221in}{1.619600in}}%
\pgfpathlineto{\pgfqpoint{2.491396in}{1.524258in}}%
\pgfpathlineto{\pgfqpoint{2.504571in}{1.395180in}}%
\pgfpathlineto{\pgfqpoint{2.517746in}{1.240157in}}%
\pgfpathlineto{\pgfqpoint{2.570446in}{0.566121in}}%
\pgfpathlineto{\pgfqpoint{2.583621in}{0.438828in}}%
\pgfpathlineto{\pgfqpoint{2.596796in}{0.347248in}}%
\pgfpathlineto{\pgfqpoint{2.609971in}{0.297780in}}%
\pgfpathlineto{\pgfqpoint{2.623145in}{0.294126in}}%
\pgfpathlineto{\pgfqpoint{2.636320in}{0.336980in}}%
\pgfpathlineto{\pgfqpoint{2.649495in}{0.423913in}}%
\pgfpathlineto{\pgfqpoint{2.662670in}{0.549463in}}%
\pgfpathlineto{\pgfqpoint{2.675845in}{0.705445in}}%
\pgfpathlineto{\pgfqpoint{2.728545in}{1.407010in}}%
\pgfpathlineto{\pgfqpoint{2.741720in}{1.540288in}}%
\pgfpathlineto{\pgfqpoint{2.754895in}{1.634999in}}%
\pgfpathlineto{\pgfqpoint{2.768069in}{1.684023in}}%
\pgfpathlineto{\pgfqpoint{2.781244in}{1.683407in}}%
\pgfpathlineto{\pgfqpoint{2.794419in}{1.632728in}}%
\pgfpathlineto{\pgfqpoint{2.807594in}{1.535199in}}%
\pgfpathlineto{\pgfqpoint{2.820769in}{1.397512in}}%
\pgfpathlineto{\pgfqpoint{2.833944in}{1.229426in}}%
\pgfpathlineto{\pgfqpoint{2.873469in}{0.671160in}}%
\pgfpathlineto{\pgfqpoint{2.886644in}{0.513556in}}%
\pgfpathlineto{\pgfqpoint{2.899819in}{0.391625in}}%
\pgfpathlineto{\pgfqpoint{2.912993in}{0.315006in}}%
\pgfpathlineto{\pgfqpoint{2.926168in}{0.290000in}}%
\pgfpathlineto{\pgfqpoint{2.939343in}{0.319010in}}%
\pgfpathlineto{\pgfqpoint{2.952518in}{0.400271in}}%
\pgfpathlineto{\pgfqpoint{2.965693in}{0.527897in}}%
\pgfpathlineto{\pgfqpoint{2.978868in}{0.692261in}}%
\pgfpathlineto{\pgfqpoint{3.031568in}{1.438739in}}%
\pgfpathlineto{\pgfqpoint{3.044742in}{1.572058in}}%
\pgfpathlineto{\pgfqpoint{3.057917in}{1.658295in}}%
\pgfpathlineto{\pgfqpoint{3.071092in}{1.689960in}}%
\pgfpathlineto{\pgfqpoint{3.084267in}{1.663956in}}%
\pgfpathlineto{\pgfqpoint{3.097442in}{1.581937in}}%
\pgfpathlineto{\pgfqpoint{3.110617in}{1.450263in}}%
\pgfpathlineto{\pgfqpoint{3.123792in}{1.279561in}}%
\pgfpathlineto{\pgfqpoint{3.163317in}{0.684402in}}%
\pgfpathlineto{\pgfqpoint{3.176492in}{0.514781in}}%
\pgfpathlineto{\pgfqpoint{3.189666in}{0.385760in}}%
\pgfpathlineto{\pgfqpoint{3.202841in}{0.308908in}}%
\pgfpathlineto{\pgfqpoint{3.216016in}{0.291391in}}%
\pgfpathlineto{\pgfqpoint{3.229191in}{0.335255in}}%
\pgfpathlineto{\pgfqpoint{3.242366in}{0.437141in}}%
\pgfpathlineto{\pgfqpoint{3.255541in}{0.588477in}}%
\pgfpathlineto{\pgfqpoint{3.268716in}{0.776141in}}%
\pgfpathlineto{\pgfqpoint{3.308241in}{1.382878in}}%
\pgfpathlineto{\pgfqpoint{3.321416in}{1.538217in}}%
\pgfpathlineto{\pgfqpoint{3.334590in}{1.643556in}}%
\pgfpathlineto{\pgfqpoint{3.347765in}{1.688736in}}%
\pgfpathlineto{\pgfqpoint{3.360940in}{1.669056in}}%
\pgfpathlineto{\pgfqpoint{3.374115in}{1.585822in}}%
\pgfpathlineto{\pgfqpoint{3.387290in}{1.446340in}}%
\pgfpathlineto{\pgfqpoint{3.400465in}{1.263339in}}%
\pgfpathlineto{\pgfqpoint{3.439990in}{0.635652in}}%
\pgfpathlineto{\pgfqpoint{3.453165in}{0.467281in}}%
\pgfpathlineto{\pgfqpoint{3.466340in}{0.349250in}}%
\pgfpathlineto{\pgfqpoint{3.479514in}{0.293496in}}%
\pgfpathlineto{\pgfqpoint{3.492689in}{0.305995in}}%
\pgfpathlineto{\pgfqpoint{3.505864in}{0.386062in}}%
\pgfpathlineto{\pgfqpoint{3.519039in}{0.526286in}}%
\pgfpathlineto{\pgfqpoint{3.532214in}{0.713150in}}%
\pgfpathlineto{\pgfqpoint{3.571739in}{1.356418in}}%
\pgfpathlineto{\pgfqpoint{3.584914in}{1.525836in}}%
\pgfpathlineto{\pgfqpoint{3.598089in}{1.640781in}}%
\pgfpathlineto{\pgfqpoint{3.611264in}{1.688984in}}%
\pgfpathlineto{\pgfqpoint{3.624438in}{1.664931in}}%
\pgfpathlineto{\pgfqpoint{3.637613in}{1.570562in}}%
\pgfpathlineto{\pgfqpoint{3.650788in}{1.415207in}}%
\pgfpathlineto{\pgfqpoint{3.663963in}{1.214736in}}%
\pgfpathlineto{\pgfqpoint{3.677138in}{0.765264in}}%
\pgfpathlineto{\pgfqpoint{3.690313in}{0.564793in}}%
\pgfpathlineto{\pgfqpoint{3.703488in}{0.409438in}}%
\pgfpathlineto{\pgfqpoint{3.716663in}{0.315069in}}%
\pgfpathlineto{\pgfqpoint{3.729838in}{0.291016in}}%
\pgfpathlineto{\pgfqpoint{3.743013in}{0.339219in}}%
\pgfpathlineto{\pgfqpoint{3.756188in}{0.454164in}}%
\pgfpathlineto{\pgfqpoint{3.769362in}{0.623582in}}%
\pgfpathlineto{\pgfqpoint{3.795712in}{1.051725in}}%
\pgfpathlineto{\pgfqpoint{3.808887in}{1.266850in}}%
\pgfpathlineto{\pgfqpoint{3.822062in}{1.453714in}}%
\pgfpathlineto{\pgfqpoint{3.835237in}{1.593938in}}%
\pgfpathlineto{\pgfqpoint{3.848412in}{1.674005in}}%
\pgfpathlineto{\pgfqpoint{3.861587in}{1.686504in}}%
\pgfpathlineto{\pgfqpoint{3.874762in}{1.630750in}}%
\pgfpathlineto{\pgfqpoint{3.887937in}{1.512719in}}%
\pgfpathlineto{\pgfqpoint{3.901112in}{1.344348in}}%
\pgfpathlineto{\pgfqpoint{3.927461in}{0.926136in}}%
\pgfpathlineto{\pgfqpoint{3.940636in}{0.716661in}}%
\pgfpathlineto{\pgfqpoint{3.953811in}{0.533660in}}%
\pgfpathlineto{\pgfqpoint{3.966986in}{0.394178in}}%
\pgfpathlineto{\pgfqpoint{3.980161in}{0.310944in}}%
\pgfpathlineto{\pgfqpoint{3.993336in}{0.291264in}}%
\pgfpathlineto{\pgfqpoint{4.006511in}{0.336444in}}%
\pgfpathlineto{\pgfqpoint{4.019686in}{0.441783in}}%
\pgfpathlineto{\pgfqpoint{4.032861in}{0.597122in}}%
\pgfpathlineto{\pgfqpoint{4.046036in}{0.787876in}}%
\pgfpathlineto{\pgfqpoint{4.072385in}{1.203859in}}%
\pgfpathlineto{\pgfqpoint{4.085560in}{1.391523in}}%
\pgfpathlineto{\pgfqpoint{4.098735in}{1.542859in}}%
\pgfpathlineto{\pgfqpoint{4.111910in}{1.644745in}}%
\pgfpathlineto{\pgfqpoint{4.125085in}{1.688609in}}%
\pgfpathlineto{\pgfqpoint{4.138260in}{1.671092in}}%
\pgfpathlineto{\pgfqpoint{4.151435in}{1.594240in}}%
\pgfpathlineto{\pgfqpoint{4.164610in}{1.465219in}}%
\pgfpathlineto{\pgfqpoint{4.177785in}{1.295598in}}%
\pgfpathlineto{\pgfqpoint{4.230484in}{0.529737in}}%
\pgfpathlineto{\pgfqpoint{4.243659in}{0.398063in}}%
\pgfpathlineto{\pgfqpoint{4.256834in}{0.316044in}}%
\pgfpathlineto{\pgfqpoint{4.270009in}{0.290040in}}%
\pgfpathlineto{\pgfqpoint{4.283184in}{0.321705in}}%
\pgfpathlineto{\pgfqpoint{4.296359in}{0.407942in}}%
\pgfpathlineto{\pgfqpoint{4.309534in}{0.541261in}}%
\pgfpathlineto{\pgfqpoint{4.322709in}{0.710468in}}%
\pgfpathlineto{\pgfqpoint{4.362233in}{1.287739in}}%
\pgfpathlineto{\pgfqpoint{4.375408in}{1.452103in}}%
\pgfpathlineto{\pgfqpoint{4.388583in}{1.579729in}}%
\pgfpathlineto{\pgfqpoint{4.401758in}{1.660990in}}%
\pgfpathlineto{\pgfqpoint{4.414933in}{1.690000in}}%
\pgfpathlineto{\pgfqpoint{4.428108in}{1.664994in}}%
\pgfpathlineto{\pgfqpoint{4.441283in}{1.588375in}}%
\pgfpathlineto{\pgfqpoint{4.454458in}{1.466444in}}%
\pgfpathlineto{\pgfqpoint{4.467633in}{1.308840in}}%
\pgfpathlineto{\pgfqpoint{4.520332in}{0.582488in}}%
\pgfpathlineto{\pgfqpoint{4.533507in}{0.444801in}}%
\pgfpathlineto{\pgfqpoint{4.546682in}{0.347272in}}%
\pgfpathlineto{\pgfqpoint{4.559857in}{0.296593in}}%
\pgfpathlineto{\pgfqpoint{4.573032in}{0.295977in}}%
\pgfpathlineto{\pgfqpoint{4.586207in}{0.345001in}}%
\pgfpathlineto{\pgfqpoint{4.599382in}{0.439712in}}%
\pgfpathlineto{\pgfqpoint{4.612557in}{0.572990in}}%
\pgfpathlineto{\pgfqpoint{4.625731in}{0.735114in}}%
\pgfpathlineto{\pgfqpoint{4.665256in}{1.274555in}}%
\pgfpathlineto{\pgfqpoint{4.678431in}{1.430537in}}%
\pgfpathlineto{\pgfqpoint{4.691606in}{1.556087in}}%
\pgfpathlineto{\pgfqpoint{4.704781in}{1.643020in}}%
\pgfpathlineto{\pgfqpoint{4.717956in}{1.685874in}}%
\pgfpathlineto{\pgfqpoint{4.731131in}{1.682220in}}%
\pgfpathlineto{\pgfqpoint{4.744306in}{1.632752in}}%
\pgfpathlineto{\pgfqpoint{4.757481in}{1.541172in}}%
\pgfpathlineto{\pgfqpoint{4.770655in}{1.413879in}}%
\pgfpathlineto{\pgfqpoint{4.783830in}{1.259489in}}%
\pgfpathlineto{\pgfqpoint{4.836530in}{0.584820in}}%
\pgfpathlineto{\pgfqpoint{4.849705in}{0.455742in}}%
\pgfpathlineto{\pgfqpoint{4.862880in}{0.360400in}}%
\pgfpathlineto{\pgfqpoint{4.876055in}{0.304363in}}%
\pgfpathlineto{\pgfqpoint{4.889230in}{0.290682in}}%
\pgfpathlineto{\pgfqpoint{4.902405in}{0.319763in}}%
\pgfpathlineto{\pgfqpoint{4.915579in}{0.389407in}}%
\pgfpathlineto{\pgfqpoint{4.928754in}{0.495001in}}%
\pgfpathlineto{\pgfqpoint{4.941929in}{0.629854in}}%
\pgfpathlineto{\pgfqpoint{4.968279in}{0.952970in}}%
\pgfpathlineto{\pgfqpoint{4.994629in}{1.282545in}}%
\pgfpathlineto{\pgfqpoint{5.007804in}{1.425745in}}%
\pgfpathlineto{\pgfqpoint{5.020979in}{1.543444in}}%
\pgfpathlineto{\pgfqpoint{5.034154in}{1.629203in}}%
\pgfpathlineto{\pgfqpoint{5.047329in}{1.678501in}}%
\pgfpathlineto{\pgfqpoint{5.060503in}{1.688954in}}%
\pgfpathlineto{\pgfqpoint{5.073678in}{1.660392in}}%
\pgfpathlineto{\pgfqpoint{5.086853in}{1.594811in}}%
\pgfpathlineto{\pgfqpoint{5.100028in}{1.496216in}}%
\pgfpathlineto{\pgfqpoint{5.113203in}{1.370345in}}%
\pgfpathlineto{\pgfqpoint{5.139553in}{1.066211in}}%
\pgfpathlineto{\pgfqpoint{5.165903in}{0.748109in}}%
\pgfpathlineto{\pgfqpoint{5.179078in}{0.604909in}}%
\pgfpathlineto{\pgfqpoint{5.192253in}{0.482344in}}%
\pgfpathlineto{\pgfqpoint{5.205427in}{0.386545in}}%
\pgfpathlineto{\pgfqpoint{5.218602in}{0.322150in}}%
\pgfpathlineto{\pgfqpoint{5.231777in}{0.292102in}}%
\pgfpathlineto{\pgfqpoint{5.244952in}{0.297542in}}%
\pgfpathlineto{\pgfqpoint{5.258127in}{0.337800in}}%
\pgfpathlineto{\pgfqpoint{5.271302in}{0.410473in}}%
\pgfpathlineto{\pgfqpoint{5.284477in}{0.511592in}}%
\pgfpathlineto{\pgfqpoint{5.297652in}{0.635858in}}%
\pgfpathlineto{\pgfqpoint{5.324002in}{0.927799in}}%
\pgfpathlineto{\pgfqpoint{5.350351in}{1.229426in}}%
\pgfpathlineto{\pgfqpoint{5.363526in}{1.365877in}}%
\pgfpathlineto{\pgfqpoint{5.376701in}{1.484155in}}%
\pgfpathlineto{\pgfqpoint{5.389876in}{1.578957in}}%
\pgfpathlineto{\pgfqpoint{5.403051in}{1.646170in}}%
\pgfpathlineto{\pgfqpoint{5.416226in}{1.683033in}}%
\pgfpathlineto{\pgfqpoint{5.429401in}{1.688225in}}%
\pgfpathlineto{\pgfqpoint{5.442576in}{1.661884in}}%
\pgfpathlineto{\pgfqpoint{5.455751in}{1.605567in}}%
\pgfpathlineto{\pgfqpoint{5.468926in}{1.522132in}}%
\pgfpathlineto{\pgfqpoint{5.482100in}{1.415586in}}%
\pgfpathlineto{\pgfqpoint{5.495275in}{1.290866in}}%
\pgfpathlineto{\pgfqpoint{5.521625in}{1.009863in}}%
\pgfpathlineto{\pgfqpoint{5.547975in}{0.727685in}}%
\pgfpathlineto{\pgfqpoint{5.561150in}{0.601082in}}%
\pgfpathlineto{\pgfqpoint{5.574325in}{0.491177in}}%
\pgfpathlineto{\pgfqpoint{5.587500in}{0.402304in}}%
\pgfpathlineto{\pgfqpoint{5.600675in}{0.337844in}}%
\pgfpathlineto{\pgfqpoint{5.613850in}{0.300109in}}%
\pgfpathlineto{\pgfqpoint{5.627024in}{0.290286in}}%
\pgfpathlineto{\pgfqpoint{5.640199in}{0.308415in}}%
\pgfpathlineto{\pgfqpoint{5.653374in}{0.353427in}}%
\pgfpathlineto{\pgfqpoint{5.666549in}{0.423211in}}%
\pgfpathlineto{\pgfqpoint{5.679724in}{0.514737in}}%
\pgfpathlineto{\pgfqpoint{5.692899in}{0.624192in}}%
\pgfpathlineto{\pgfqpoint{5.719249in}{0.878794in}}%
\pgfpathlineto{\pgfqpoint{5.758774in}{1.275155in}}%
\pgfpathlineto{\pgfqpoint{5.771948in}{1.391523in}}%
\pgfpathlineto{\pgfqpoint{5.785123in}{1.492825in}}%
\pgfpathlineto{\pgfqpoint{5.798298in}{1.575613in}}%
\pgfpathlineto{\pgfqpoint{5.811473in}{1.637181in}}%
\pgfpathlineto{\pgfqpoint{5.824648in}{1.675637in}}%
\pgfpathlineto{\pgfqpoint{5.837823in}{1.689944in}}%
\pgfpathlineto{\pgfqpoint{5.850998in}{1.679932in}}%
\pgfpathlineto{\pgfqpoint{5.864173in}{1.646274in}}%
\pgfpathlineto{\pgfqpoint{5.877348in}{1.590440in}}%
\pgfpathlineto{\pgfqpoint{5.890523in}{1.514619in}}%
\pgfpathlineto{\pgfqpoint{5.903698in}{1.421624in}}%
\pgfpathlineto{\pgfqpoint{5.916872in}{1.314774in}}%
\pgfpathlineto{\pgfqpoint{5.943222in}{1.074567in}}%
\pgfpathlineto{\pgfqpoint{5.982747in}{0.708063in}}%
\pgfpathlineto{\pgfqpoint{5.995922in}{0.599743in}}%
\pgfpathlineto{\pgfqpoint{6.009097in}{0.504024in}}%
\pgfpathlineto{\pgfqpoint{6.022272in}{0.423667in}}%
\pgfpathlineto{\pgfqpoint{6.035447in}{0.360897in}}%
\pgfpathlineto{\pgfqpoint{6.048622in}{0.317350in}}%
\pgfpathlineto{\pgfqpoint{6.061796in}{0.294048in}}%
\pgfpathlineto{\pgfqpoint{6.074971in}{0.291383in}}%
\pgfpathlineto{\pgfqpoint{6.088146in}{0.309129in}}%
\pgfpathlineto{\pgfqpoint{6.101321in}{0.346471in}}%
\pgfpathlineto{\pgfqpoint{6.114496in}{0.402045in}}%
\pgfpathlineto{\pgfqpoint{6.127671in}{0.474001in}}%
\pgfpathlineto{\pgfqpoint{6.140846in}{0.560069in}}%
\pgfpathlineto{\pgfqpoint{6.154021in}{0.657641in}}%
\pgfpathlineto{\pgfqpoint{6.180371in}{0.875673in}}%
\pgfpathlineto{\pgfqpoint{6.193546in}{0.990000in}}%
\pgfpathlineto{\pgfqpoint{6.193546in}{0.990000in}}%
\pgfusepath{stroke}%
\end{pgfscope}%
\begin{pgfscope}%
\pgfsetrectcap%
\pgfsetmiterjoin%
\pgfsetlinewidth{0.803000pt}%
\definecolor{currentstroke}{rgb}{0.000000,0.000000,0.000000}%
\pgfsetstrokecolor{currentstroke}%
\pgfsetdash{}{0pt}%
\pgfpathmoveto{\pgfqpoint{0.895256in}{0.220000in}}%
\pgfpathlineto{\pgfqpoint{0.895256in}{1.760000in}}%
\pgfusepath{stroke}%
\end{pgfscope}%
\begin{pgfscope}%
\pgfsetrectcap%
\pgfsetmiterjoin%
\pgfsetlinewidth{0.803000pt}%
\definecolor{currentstroke}{rgb}{0.000000,0.000000,0.000000}%
\pgfsetstrokecolor{currentstroke}%
\pgfsetdash{}{0pt}%
\pgfpathmoveto{\pgfqpoint{6.445845in}{0.220000in}}%
\pgfpathlineto{\pgfqpoint{6.445845in}{1.760000in}}%
\pgfusepath{stroke}%
\end{pgfscope}%
\begin{pgfscope}%
\pgfsetrectcap%
\pgfsetmiterjoin%
\pgfsetlinewidth{0.803000pt}%
\definecolor{currentstroke}{rgb}{0.000000,0.000000,0.000000}%
\pgfsetstrokecolor{currentstroke}%
\pgfsetdash{}{0pt}%
\pgfpathmoveto{\pgfqpoint{0.895256in}{0.220000in}}%
\pgfpathlineto{\pgfqpoint{6.445845in}{0.220000in}}%
\pgfusepath{stroke}%
\end{pgfscope}%
\begin{pgfscope}%
\pgfsetrectcap%
\pgfsetmiterjoin%
\pgfsetlinewidth{0.803000pt}%
\definecolor{currentstroke}{rgb}{0.000000,0.000000,0.000000}%
\pgfsetstrokecolor{currentstroke}%
\pgfsetdash{}{0pt}%
\pgfpathmoveto{\pgfqpoint{0.895256in}{1.760000in}}%
\pgfpathlineto{\pgfqpoint{6.445845in}{1.760000in}}%
\pgfusepath{stroke}%
\end{pgfscope}%
\begin{pgfscope}%
\definecolor{textcolor}{rgb}{0.000000,0.000000,0.000000}%
\pgfsetstrokecolor{textcolor}%
\pgfsetfillcolor{textcolor}%
\pgftext[x=3.670551in,y=1.843333in,,base]{\color{textcolor}{\rmfamily\fontsize{12.000000}{14.400000}\selectfont\catcode`\^=\active\def^{\ifmmode\sp\else\^{}\fi}\catcode`\%=\active\def%{\%}Chirp Signal}}%
\end{pgfscope}%
\end{pgfpicture}%
\makeatother%
\endgroup%
}
    \caption{The plot of a chirp signal.}
    \label{fig:chirp}
\end{figure}

The Figure \ref{fig:chirp} shows the plot of a chirp signal.
The actual chrip signal we use is \(x(t)\) concatenated with \(-x(-t)\), which has a better performance in practice.

To find the chirp signal, we do a sliding dot product between the received signal and the chirp signal, and if the similarity score is larger than a threshold, we find the chirp signal.

\textbf{Our clock synchronization algorithm is extremely accurate, we can archive sample level synchronization, which means the error is less than 1 sample (1/48000 seconds).}

\subsection{Modulation Schemes}

We have experimented with several modulation schemes in our project, including Amplitude Shift Keying (ASK), Frequency Shift Keying (FSK), and Phase Shift Keying (PSK).

Thanks to Python, we can easily visualize the signals, and iterate fast.

\subsubsection{ASK}

ASK is a modulation scheme that encodes symbols in the amplitude of the wave.
Since we are reminded that transmitting sound waves through the air is not reliable, we didn't use ASK in our project 1. \textbf{We use FSK in Project 1, and use ASK in Project 2.}

However, for cable connection, ASK is a good choice.
We find that we can tranmit a bit every 2 samples, where \(\text{one} = [0,1], \text{zero}=[0,-1]\).

\textbf{By ASK, we archive a 24000 bps transmission rate.} (There are some overheads in the frame structure, so the actual transmission rate is less than 24000 bps.)

\begin{figure}[H]
    \noindent\makebox[\textwidth]{%% Creator: Matplotlib, PGF backend
%%
%% To include the figure in your LaTeX document, write
%%   \input{<filename>.pgf}
%%
%% Make sure the required packages are loaded in your preamble
%%   \usepackage{pgf}
%%
%% Also ensure that all the required font packages are loaded; for instance,
%% the lmodern package is sometimes necessary when using math font.
%%   \usepackage{lmodern}
%%
%% Figures using additional raster images can only be included by \input if
%% they are in the same directory as the main LaTeX file. For loading figures
%% from other directories you can use the `import` package
%%   \usepackage{import}
%%
%% and then include the figures with
%%   \import{<path to file>}{<filename>.pgf}
%%
%% Matplotlib used the following preamble
%%   \def\mathdefault#1{#1}
%%   \everymath=\expandafter{\the\everymath\displaystyle}
%%   \IfFileExists{scrextend.sty}{
%%     \usepackage[fontsize=10.000000pt]{scrextend}
%%   }{
%%     \renewcommand{\normalsize}{\fontsize{10.000000}{12.000000}\selectfont}
%%     \normalsize
%%   }
%%   
%%   \makeatletter\@ifpackageloaded{underscore}{}{\usepackage[strings]{underscore}}\makeatother
%%
\begingroup%
\makeatletter%
\begin{pgfpicture}%
\pgfpathrectangle{\pgfpointorigin}{\pgfqpoint{7.162050in}{2.000000in}}%
\pgfusepath{use as bounding box, clip}%
\begin{pgfscope}%
\pgfsetbuttcap%
\pgfsetmiterjoin%
\definecolor{currentfill}{rgb}{1.000000,1.000000,1.000000}%
\pgfsetfillcolor{currentfill}%
\pgfsetlinewidth{0.000000pt}%
\definecolor{currentstroke}{rgb}{1.000000,1.000000,1.000000}%
\pgfsetstrokecolor{currentstroke}%
\pgfsetdash{}{0pt}%
\pgfpathmoveto{\pgfqpoint{0.000000in}{0.000000in}}%
\pgfpathlineto{\pgfqpoint{7.162050in}{0.000000in}}%
\pgfpathlineto{\pgfqpoint{7.162050in}{2.000000in}}%
\pgfpathlineto{\pgfqpoint{0.000000in}{2.000000in}}%
\pgfpathlineto{\pgfqpoint{0.000000in}{0.000000in}}%
\pgfpathclose%
\pgfusepath{fill}%
\end{pgfscope}%
\begin{pgfscope}%
\pgfsetbuttcap%
\pgfsetmiterjoin%
\definecolor{currentfill}{rgb}{1.000000,1.000000,1.000000}%
\pgfsetfillcolor{currentfill}%
\pgfsetlinewidth{0.000000pt}%
\definecolor{currentstroke}{rgb}{0.000000,0.000000,0.000000}%
\pgfsetstrokecolor{currentstroke}%
\pgfsetstrokeopacity{0.000000}%
\pgfsetdash{}{0pt}%
\pgfpathmoveto{\pgfqpoint{0.895256in}{0.220000in}}%
\pgfpathlineto{\pgfqpoint{6.445845in}{0.220000in}}%
\pgfpathlineto{\pgfqpoint{6.445845in}{1.760000in}}%
\pgfpathlineto{\pgfqpoint{0.895256in}{1.760000in}}%
\pgfpathlineto{\pgfqpoint{0.895256in}{0.220000in}}%
\pgfpathclose%
\pgfusepath{fill}%
\end{pgfscope}%
\begin{pgfscope}%
\pgfsetbuttcap%
\pgfsetroundjoin%
\definecolor{currentfill}{rgb}{0.000000,0.000000,0.000000}%
\pgfsetfillcolor{currentfill}%
\pgfsetlinewidth{0.803000pt}%
\definecolor{currentstroke}{rgb}{0.000000,0.000000,0.000000}%
\pgfsetstrokecolor{currentstroke}%
\pgfsetdash{}{0pt}%
\pgfsys@defobject{currentmarker}{\pgfqpoint{0.000000in}{-0.048611in}}{\pgfqpoint{0.000000in}{0.000000in}}{%
\pgfpathmoveto{\pgfqpoint{0.000000in}{0.000000in}}%
\pgfpathlineto{\pgfqpoint{0.000000in}{-0.048611in}}%
\pgfusepath{stroke,fill}%
}%
\begin{pgfscope}%
\pgfsys@transformshift{1.147556in}{0.220000in}%
\pgfsys@useobject{currentmarker}{}%
\end{pgfscope}%
\end{pgfscope}%
\begin{pgfscope}%
\definecolor{textcolor}{rgb}{0.000000,0.000000,0.000000}%
\pgfsetstrokecolor{textcolor}%
\pgfsetfillcolor{textcolor}%
\pgftext[x=1.147556in,y=0.122778in,,top]{\color{textcolor}{\rmfamily\fontsize{10.000000}{12.000000}\selectfont\catcode`\^=\active\def^{\ifmmode\sp\else\^{}\fi}\catcode`\%=\active\def%{\%}$\mathdefault{0}$}}%
\end{pgfscope}%
\begin{pgfscope}%
\pgfsetbuttcap%
\pgfsetroundjoin%
\definecolor{currentfill}{rgb}{0.000000,0.000000,0.000000}%
\pgfsetfillcolor{currentfill}%
\pgfsetlinewidth{0.803000pt}%
\definecolor{currentstroke}{rgb}{0.000000,0.000000,0.000000}%
\pgfsetstrokecolor{currentstroke}%
\pgfsetdash{}{0pt}%
\pgfsys@defobject{currentmarker}{\pgfqpoint{0.000000in}{-0.048611in}}{\pgfqpoint{0.000000in}{0.000000in}}{%
\pgfpathmoveto{\pgfqpoint{0.000000in}{0.000000in}}%
\pgfpathlineto{\pgfqpoint{0.000000in}{-0.048611in}}%
\pgfusepath{stroke,fill}%
}%
\begin{pgfscope}%
\pgfsys@transformshift{2.209869in}{0.220000in}%
\pgfsys@useobject{currentmarker}{}%
\end{pgfscope}%
\end{pgfscope}%
\begin{pgfscope}%
\definecolor{textcolor}{rgb}{0.000000,0.000000,0.000000}%
\pgfsetstrokecolor{textcolor}%
\pgfsetfillcolor{textcolor}%
\pgftext[x=2.209869in,y=0.122778in,,top]{\color{textcolor}{\rmfamily\fontsize{10.000000}{12.000000}\selectfont\catcode`\^=\active\def^{\ifmmode\sp\else\^{}\fi}\catcode`\%=\active\def%{\%}$\mathdefault{2}$}}%
\end{pgfscope}%
\begin{pgfscope}%
\pgfsetbuttcap%
\pgfsetroundjoin%
\definecolor{currentfill}{rgb}{0.000000,0.000000,0.000000}%
\pgfsetfillcolor{currentfill}%
\pgfsetlinewidth{0.803000pt}%
\definecolor{currentstroke}{rgb}{0.000000,0.000000,0.000000}%
\pgfsetstrokecolor{currentstroke}%
\pgfsetdash{}{0pt}%
\pgfsys@defobject{currentmarker}{\pgfqpoint{0.000000in}{-0.048611in}}{\pgfqpoint{0.000000in}{0.000000in}}{%
\pgfpathmoveto{\pgfqpoint{0.000000in}{0.000000in}}%
\pgfpathlineto{\pgfqpoint{0.000000in}{-0.048611in}}%
\pgfusepath{stroke,fill}%
}%
\begin{pgfscope}%
\pgfsys@transformshift{3.272183in}{0.220000in}%
\pgfsys@useobject{currentmarker}{}%
\end{pgfscope}%
\end{pgfscope}%
\begin{pgfscope}%
\definecolor{textcolor}{rgb}{0.000000,0.000000,0.000000}%
\pgfsetstrokecolor{textcolor}%
\pgfsetfillcolor{textcolor}%
\pgftext[x=3.272183in,y=0.122778in,,top]{\color{textcolor}{\rmfamily\fontsize{10.000000}{12.000000}\selectfont\catcode`\^=\active\def^{\ifmmode\sp\else\^{}\fi}\catcode`\%=\active\def%{\%}$\mathdefault{4}$}}%
\end{pgfscope}%
\begin{pgfscope}%
\pgfsetbuttcap%
\pgfsetroundjoin%
\definecolor{currentfill}{rgb}{0.000000,0.000000,0.000000}%
\pgfsetfillcolor{currentfill}%
\pgfsetlinewidth{0.803000pt}%
\definecolor{currentstroke}{rgb}{0.000000,0.000000,0.000000}%
\pgfsetstrokecolor{currentstroke}%
\pgfsetdash{}{0pt}%
\pgfsys@defobject{currentmarker}{\pgfqpoint{0.000000in}{-0.048611in}}{\pgfqpoint{0.000000in}{0.000000in}}{%
\pgfpathmoveto{\pgfqpoint{0.000000in}{0.000000in}}%
\pgfpathlineto{\pgfqpoint{0.000000in}{-0.048611in}}%
\pgfusepath{stroke,fill}%
}%
\begin{pgfscope}%
\pgfsys@transformshift{4.334497in}{0.220000in}%
\pgfsys@useobject{currentmarker}{}%
\end{pgfscope}%
\end{pgfscope}%
\begin{pgfscope}%
\definecolor{textcolor}{rgb}{0.000000,0.000000,0.000000}%
\pgfsetstrokecolor{textcolor}%
\pgfsetfillcolor{textcolor}%
\pgftext[x=4.334497in,y=0.122778in,,top]{\color{textcolor}{\rmfamily\fontsize{10.000000}{12.000000}\selectfont\catcode`\^=\active\def^{\ifmmode\sp\else\^{}\fi}\catcode`\%=\active\def%{\%}$\mathdefault{6}$}}%
\end{pgfscope}%
\begin{pgfscope}%
\pgfsetbuttcap%
\pgfsetroundjoin%
\definecolor{currentfill}{rgb}{0.000000,0.000000,0.000000}%
\pgfsetfillcolor{currentfill}%
\pgfsetlinewidth{0.803000pt}%
\definecolor{currentstroke}{rgb}{0.000000,0.000000,0.000000}%
\pgfsetstrokecolor{currentstroke}%
\pgfsetdash{}{0pt}%
\pgfsys@defobject{currentmarker}{\pgfqpoint{0.000000in}{-0.048611in}}{\pgfqpoint{0.000000in}{0.000000in}}{%
\pgfpathmoveto{\pgfqpoint{0.000000in}{0.000000in}}%
\pgfpathlineto{\pgfqpoint{0.000000in}{-0.048611in}}%
\pgfusepath{stroke,fill}%
}%
\begin{pgfscope}%
\pgfsys@transformshift{5.396810in}{0.220000in}%
\pgfsys@useobject{currentmarker}{}%
\end{pgfscope}%
\end{pgfscope}%
\begin{pgfscope}%
\definecolor{textcolor}{rgb}{0.000000,0.000000,0.000000}%
\pgfsetstrokecolor{textcolor}%
\pgfsetfillcolor{textcolor}%
\pgftext[x=5.396810in,y=0.122778in,,top]{\color{textcolor}{\rmfamily\fontsize{10.000000}{12.000000}\selectfont\catcode`\^=\active\def^{\ifmmode\sp\else\^{}\fi}\catcode`\%=\active\def%{\%}$\mathdefault{8}$}}%
\end{pgfscope}%
\begin{pgfscope}%
\definecolor{textcolor}{rgb}{0.000000,0.000000,0.000000}%
\pgfsetstrokecolor{textcolor}%
\pgfsetfillcolor{textcolor}%
\pgftext[x=3.670551in,y=-0.056234in,,top]{\color{textcolor}{\rmfamily\fontsize{10.000000}{12.000000}\selectfont\catcode`\^=\active\def^{\ifmmode\sp\else\^{}\fi}\catcode`\%=\active\def%{\%}Samples}}%
\end{pgfscope}%
\begin{pgfscope}%
\pgfsetbuttcap%
\pgfsetroundjoin%
\definecolor{currentfill}{rgb}{0.000000,0.000000,0.000000}%
\pgfsetfillcolor{currentfill}%
\pgfsetlinewidth{0.803000pt}%
\definecolor{currentstroke}{rgb}{0.000000,0.000000,0.000000}%
\pgfsetstrokecolor{currentstroke}%
\pgfsetdash{}{0pt}%
\pgfsys@defobject{currentmarker}{\pgfqpoint{-0.048611in}{0.000000in}}{\pgfqpoint{-0.000000in}{0.000000in}}{%
\pgfpathmoveto{\pgfqpoint{-0.000000in}{0.000000in}}%
\pgfpathlineto{\pgfqpoint{-0.048611in}{0.000000in}}%
\pgfusepath{stroke,fill}%
}%
\begin{pgfscope}%
\pgfsys@transformshift{0.895256in}{0.290000in}%
\pgfsys@useobject{currentmarker}{}%
\end{pgfscope}%
\end{pgfscope}%
\begin{pgfscope}%
\definecolor{textcolor}{rgb}{0.000000,0.000000,0.000000}%
\pgfsetstrokecolor{textcolor}%
\pgfsetfillcolor{textcolor}%
\pgftext[x=0.512539in, y=0.241775in, left, base]{\color{textcolor}{\rmfamily\fontsize{10.000000}{12.000000}\selectfont\catcode`\^=\active\def^{\ifmmode\sp\else\^{}\fi}\catcode`\%=\active\def%{\%}$\mathdefault{\ensuremath{-}1.0}$}}%
\end{pgfscope}%
\begin{pgfscope}%
\pgfsetbuttcap%
\pgfsetroundjoin%
\definecolor{currentfill}{rgb}{0.000000,0.000000,0.000000}%
\pgfsetfillcolor{currentfill}%
\pgfsetlinewidth{0.803000pt}%
\definecolor{currentstroke}{rgb}{0.000000,0.000000,0.000000}%
\pgfsetstrokecolor{currentstroke}%
\pgfsetdash{}{0pt}%
\pgfsys@defobject{currentmarker}{\pgfqpoint{-0.048611in}{0.000000in}}{\pgfqpoint{-0.000000in}{0.000000in}}{%
\pgfpathmoveto{\pgfqpoint{-0.000000in}{0.000000in}}%
\pgfpathlineto{\pgfqpoint{-0.048611in}{0.000000in}}%
\pgfusepath{stroke,fill}%
}%
\begin{pgfscope}%
\pgfsys@transformshift{0.895256in}{0.640000in}%
\pgfsys@useobject{currentmarker}{}%
\end{pgfscope}%
\end{pgfscope}%
\begin{pgfscope}%
\definecolor{textcolor}{rgb}{0.000000,0.000000,0.000000}%
\pgfsetstrokecolor{textcolor}%
\pgfsetfillcolor{textcolor}%
\pgftext[x=0.512539in, y=0.591775in, left, base]{\color{textcolor}{\rmfamily\fontsize{10.000000}{12.000000}\selectfont\catcode`\^=\active\def^{\ifmmode\sp\else\^{}\fi}\catcode`\%=\active\def%{\%}$\mathdefault{\ensuremath{-}0.5}$}}%
\end{pgfscope}%
\begin{pgfscope}%
\pgfsetbuttcap%
\pgfsetroundjoin%
\definecolor{currentfill}{rgb}{0.000000,0.000000,0.000000}%
\pgfsetfillcolor{currentfill}%
\pgfsetlinewidth{0.803000pt}%
\definecolor{currentstroke}{rgb}{0.000000,0.000000,0.000000}%
\pgfsetstrokecolor{currentstroke}%
\pgfsetdash{}{0pt}%
\pgfsys@defobject{currentmarker}{\pgfqpoint{-0.048611in}{0.000000in}}{\pgfqpoint{-0.000000in}{0.000000in}}{%
\pgfpathmoveto{\pgfqpoint{-0.000000in}{0.000000in}}%
\pgfpathlineto{\pgfqpoint{-0.048611in}{0.000000in}}%
\pgfusepath{stroke,fill}%
}%
\begin{pgfscope}%
\pgfsys@transformshift{0.895256in}{0.990000in}%
\pgfsys@useobject{currentmarker}{}%
\end{pgfscope}%
\end{pgfscope}%
\begin{pgfscope}%
\definecolor{textcolor}{rgb}{0.000000,0.000000,0.000000}%
\pgfsetstrokecolor{textcolor}%
\pgfsetfillcolor{textcolor}%
\pgftext[x=0.620564in, y=0.941775in, left, base]{\color{textcolor}{\rmfamily\fontsize{10.000000}{12.000000}\selectfont\catcode`\^=\active\def^{\ifmmode\sp\else\^{}\fi}\catcode`\%=\active\def%{\%}$\mathdefault{0.0}$}}%
\end{pgfscope}%
\begin{pgfscope}%
\pgfsetbuttcap%
\pgfsetroundjoin%
\definecolor{currentfill}{rgb}{0.000000,0.000000,0.000000}%
\pgfsetfillcolor{currentfill}%
\pgfsetlinewidth{0.803000pt}%
\definecolor{currentstroke}{rgb}{0.000000,0.000000,0.000000}%
\pgfsetstrokecolor{currentstroke}%
\pgfsetdash{}{0pt}%
\pgfsys@defobject{currentmarker}{\pgfqpoint{-0.048611in}{0.000000in}}{\pgfqpoint{-0.000000in}{0.000000in}}{%
\pgfpathmoveto{\pgfqpoint{-0.000000in}{0.000000in}}%
\pgfpathlineto{\pgfqpoint{-0.048611in}{0.000000in}}%
\pgfusepath{stroke,fill}%
}%
\begin{pgfscope}%
\pgfsys@transformshift{0.895256in}{1.340000in}%
\pgfsys@useobject{currentmarker}{}%
\end{pgfscope}%
\end{pgfscope}%
\begin{pgfscope}%
\definecolor{textcolor}{rgb}{0.000000,0.000000,0.000000}%
\pgfsetstrokecolor{textcolor}%
\pgfsetfillcolor{textcolor}%
\pgftext[x=0.620564in, y=1.291775in, left, base]{\color{textcolor}{\rmfamily\fontsize{10.000000}{12.000000}\selectfont\catcode`\^=\active\def^{\ifmmode\sp\else\^{}\fi}\catcode`\%=\active\def%{\%}$\mathdefault{0.5}$}}%
\end{pgfscope}%
\begin{pgfscope}%
\pgfsetbuttcap%
\pgfsetroundjoin%
\definecolor{currentfill}{rgb}{0.000000,0.000000,0.000000}%
\pgfsetfillcolor{currentfill}%
\pgfsetlinewidth{0.803000pt}%
\definecolor{currentstroke}{rgb}{0.000000,0.000000,0.000000}%
\pgfsetstrokecolor{currentstroke}%
\pgfsetdash{}{0pt}%
\pgfsys@defobject{currentmarker}{\pgfqpoint{-0.048611in}{0.000000in}}{\pgfqpoint{-0.000000in}{0.000000in}}{%
\pgfpathmoveto{\pgfqpoint{-0.000000in}{0.000000in}}%
\pgfpathlineto{\pgfqpoint{-0.048611in}{0.000000in}}%
\pgfusepath{stroke,fill}%
}%
\begin{pgfscope}%
\pgfsys@transformshift{0.895256in}{1.690000in}%
\pgfsys@useobject{currentmarker}{}%
\end{pgfscope}%
\end{pgfscope}%
\begin{pgfscope}%
\definecolor{textcolor}{rgb}{0.000000,0.000000,0.000000}%
\pgfsetstrokecolor{textcolor}%
\pgfsetfillcolor{textcolor}%
\pgftext[x=0.620564in, y=1.641775in, left, base]{\color{textcolor}{\rmfamily\fontsize{10.000000}{12.000000}\selectfont\catcode`\^=\active\def^{\ifmmode\sp\else\^{}\fi}\catcode`\%=\active\def%{\%}$\mathdefault{1.0}$}}%
\end{pgfscope}%
\begin{pgfscope}%
\definecolor{textcolor}{rgb}{0.000000,0.000000,0.000000}%
\pgfsetstrokecolor{textcolor}%
\pgfsetfillcolor{textcolor}%
\pgftext[x=0.456984in,y=0.990000in,,bottom,rotate=90.000000]{\color{textcolor}{\rmfamily\fontsize{10.000000}{12.000000}\selectfont\catcode`\^=\active\def^{\ifmmode\sp\else\^{}\fi}\catcode`\%=\active\def%{\%}Amplitude}}%
\end{pgfscope}%
\begin{pgfscope}%
\pgfpathrectangle{\pgfqpoint{0.895256in}{0.220000in}}{\pgfqpoint{5.550589in}{1.540000in}}%
\pgfusepath{clip}%
\pgfsetrectcap%
\pgfsetroundjoin%
\pgfsetlinewidth{1.505625pt}%
\definecolor{currentstroke}{rgb}{0.121569,0.466667,0.705882}%
\pgfsetstrokecolor{currentstroke}%
\pgfsetdash{}{0pt}%
\pgfpathmoveto{\pgfqpoint{1.147556in}{0.990000in}}%
\pgfpathlineto{\pgfqpoint{1.678713in}{1.690000in}}%
\pgfusepath{stroke}%
\end{pgfscope}%
\begin{pgfscope}%
\pgfpathrectangle{\pgfqpoint{0.895256in}{0.220000in}}{\pgfqpoint{5.550589in}{1.540000in}}%
\pgfusepath{clip}%
\pgfsetbuttcap%
\pgfsetroundjoin%
\definecolor{currentfill}{rgb}{0.121569,0.466667,0.705882}%
\pgfsetfillcolor{currentfill}%
\pgfsetlinewidth{1.003750pt}%
\definecolor{currentstroke}{rgb}{0.121569,0.466667,0.705882}%
\pgfsetstrokecolor{currentstroke}%
\pgfsetdash{}{0pt}%
\pgfsys@defobject{currentmarker}{\pgfqpoint{-0.041667in}{-0.041667in}}{\pgfqpoint{0.041667in}{0.041667in}}{%
\pgfpathmoveto{\pgfqpoint{0.000000in}{-0.041667in}}%
\pgfpathcurveto{\pgfqpoint{0.011050in}{-0.041667in}}{\pgfqpoint{0.021649in}{-0.037276in}}{\pgfqpoint{0.029463in}{-0.029463in}}%
\pgfpathcurveto{\pgfqpoint{0.037276in}{-0.021649in}}{\pgfqpoint{0.041667in}{-0.011050in}}{\pgfqpoint{0.041667in}{0.000000in}}%
\pgfpathcurveto{\pgfqpoint{0.041667in}{0.011050in}}{\pgfqpoint{0.037276in}{0.021649in}}{\pgfqpoint{0.029463in}{0.029463in}}%
\pgfpathcurveto{\pgfqpoint{0.021649in}{0.037276in}}{\pgfqpoint{0.011050in}{0.041667in}}{\pgfqpoint{0.000000in}{0.041667in}}%
\pgfpathcurveto{\pgfqpoint{-0.011050in}{0.041667in}}{\pgfqpoint{-0.021649in}{0.037276in}}{\pgfqpoint{-0.029463in}{0.029463in}}%
\pgfpathcurveto{\pgfqpoint{-0.037276in}{0.021649in}}{\pgfqpoint{-0.041667in}{0.011050in}}{\pgfqpoint{-0.041667in}{0.000000in}}%
\pgfpathcurveto{\pgfqpoint{-0.041667in}{-0.011050in}}{\pgfqpoint{-0.037276in}{-0.021649in}}{\pgfqpoint{-0.029463in}{-0.029463in}}%
\pgfpathcurveto{\pgfqpoint{-0.021649in}{-0.037276in}}{\pgfqpoint{-0.011050in}{-0.041667in}}{\pgfqpoint{0.000000in}{-0.041667in}}%
\pgfpathlineto{\pgfqpoint{0.000000in}{-0.041667in}}%
\pgfpathclose%
\pgfusepath{stroke,fill}%
}%
\begin{pgfscope}%
\pgfsys@transformshift{1.147556in}{0.990000in}%
\pgfsys@useobject{currentmarker}{}%
\end{pgfscope}%
\begin{pgfscope}%
\pgfsys@transformshift{1.678713in}{1.690000in}%
\pgfsys@useobject{currentmarker}{}%
\end{pgfscope}%
\end{pgfscope}%
\begin{pgfscope}%
\pgfpathrectangle{\pgfqpoint{0.895256in}{0.220000in}}{\pgfqpoint{5.550589in}{1.540000in}}%
\pgfusepath{clip}%
\pgfsetrectcap%
\pgfsetroundjoin%
\pgfsetlinewidth{1.505625pt}%
\definecolor{currentstroke}{rgb}{1.000000,0.498039,0.054902}%
\pgfsetstrokecolor{currentstroke}%
\pgfsetdash{}{0pt}%
\pgfpathmoveto{\pgfqpoint{1.147556in}{0.971287in}}%
\pgfpathlineto{\pgfqpoint{1.678713in}{1.182346in}}%
\pgfusepath{stroke}%
\end{pgfscope}%
\begin{pgfscope}%
\pgfpathrectangle{\pgfqpoint{0.895256in}{0.220000in}}{\pgfqpoint{5.550589in}{1.540000in}}%
\pgfusepath{clip}%
\pgfsetbuttcap%
\pgfsetroundjoin%
\definecolor{currentfill}{rgb}{1.000000,0.498039,0.054902}%
\pgfsetfillcolor{currentfill}%
\pgfsetlinewidth{1.003750pt}%
\definecolor{currentstroke}{rgb}{1.000000,0.498039,0.054902}%
\pgfsetstrokecolor{currentstroke}%
\pgfsetdash{}{0pt}%
\pgfsys@defobject{currentmarker}{\pgfqpoint{-0.041667in}{-0.041667in}}{\pgfqpoint{0.041667in}{0.041667in}}{%
\pgfpathmoveto{\pgfqpoint{0.000000in}{-0.041667in}}%
\pgfpathcurveto{\pgfqpoint{0.011050in}{-0.041667in}}{\pgfqpoint{0.021649in}{-0.037276in}}{\pgfqpoint{0.029463in}{-0.029463in}}%
\pgfpathcurveto{\pgfqpoint{0.037276in}{-0.021649in}}{\pgfqpoint{0.041667in}{-0.011050in}}{\pgfqpoint{0.041667in}{0.000000in}}%
\pgfpathcurveto{\pgfqpoint{0.041667in}{0.011050in}}{\pgfqpoint{0.037276in}{0.021649in}}{\pgfqpoint{0.029463in}{0.029463in}}%
\pgfpathcurveto{\pgfqpoint{0.021649in}{0.037276in}}{\pgfqpoint{0.011050in}{0.041667in}}{\pgfqpoint{0.000000in}{0.041667in}}%
\pgfpathcurveto{\pgfqpoint{-0.011050in}{0.041667in}}{\pgfqpoint{-0.021649in}{0.037276in}}{\pgfqpoint{-0.029463in}{0.029463in}}%
\pgfpathcurveto{\pgfqpoint{-0.037276in}{0.021649in}}{\pgfqpoint{-0.041667in}{0.011050in}}{\pgfqpoint{-0.041667in}{0.000000in}}%
\pgfpathcurveto{\pgfqpoint{-0.041667in}{-0.011050in}}{\pgfqpoint{-0.037276in}{-0.021649in}}{\pgfqpoint{-0.029463in}{-0.029463in}}%
\pgfpathcurveto{\pgfqpoint{-0.021649in}{-0.037276in}}{\pgfqpoint{-0.011050in}{-0.041667in}}{\pgfqpoint{0.000000in}{-0.041667in}}%
\pgfpathlineto{\pgfqpoint{0.000000in}{-0.041667in}}%
\pgfpathclose%
\pgfusepath{stroke,fill}%
}%
\begin{pgfscope}%
\pgfsys@transformshift{1.147556in}{0.971287in}%
\pgfsys@useobject{currentmarker}{}%
\end{pgfscope}%
\begin{pgfscope}%
\pgfsys@transformshift{1.678713in}{1.182346in}%
\pgfsys@useobject{currentmarker}{}%
\end{pgfscope}%
\end{pgfscope}%
\begin{pgfscope}%
\pgfpathrectangle{\pgfqpoint{0.895256in}{0.220000in}}{\pgfqpoint{5.550589in}{1.540000in}}%
\pgfusepath{clip}%
\pgfsetbuttcap%
\pgfsetroundjoin%
\pgfsetlinewidth{1.505625pt}%
\definecolor{currentstroke}{rgb}{0.501961,0.501961,0.501961}%
\pgfsetstrokecolor{currentstroke}%
\pgfsetdash{{5.550000pt}{2.400000pt}}{0.000000pt}%
\pgfpathmoveto{\pgfqpoint{1.944291in}{0.220000in}}%
\pgfpathlineto{\pgfqpoint{1.944291in}{1.760000in}}%
\pgfusepath{stroke}%
\end{pgfscope}%
\begin{pgfscope}%
\pgfpathrectangle{\pgfqpoint{0.895256in}{0.220000in}}{\pgfqpoint{5.550589in}{1.540000in}}%
\pgfusepath{clip}%
\pgfsetrectcap%
\pgfsetroundjoin%
\pgfsetlinewidth{1.505625pt}%
\definecolor{currentstroke}{rgb}{0.121569,0.466667,0.705882}%
\pgfsetstrokecolor{currentstroke}%
\pgfsetdash{}{0pt}%
\pgfpathmoveto{\pgfqpoint{2.209869in}{0.990000in}}%
\pgfpathlineto{\pgfqpoint{2.741026in}{0.290000in}}%
\pgfusepath{stroke}%
\end{pgfscope}%
\begin{pgfscope}%
\pgfpathrectangle{\pgfqpoint{0.895256in}{0.220000in}}{\pgfqpoint{5.550589in}{1.540000in}}%
\pgfusepath{clip}%
\pgfsetbuttcap%
\pgfsetroundjoin%
\definecolor{currentfill}{rgb}{0.121569,0.466667,0.705882}%
\pgfsetfillcolor{currentfill}%
\pgfsetlinewidth{1.003750pt}%
\definecolor{currentstroke}{rgb}{0.121569,0.466667,0.705882}%
\pgfsetstrokecolor{currentstroke}%
\pgfsetdash{}{0pt}%
\pgfsys@defobject{currentmarker}{\pgfqpoint{-0.041667in}{-0.041667in}}{\pgfqpoint{0.041667in}{0.041667in}}{%
\pgfpathmoveto{\pgfqpoint{0.000000in}{-0.041667in}}%
\pgfpathcurveto{\pgfqpoint{0.011050in}{-0.041667in}}{\pgfqpoint{0.021649in}{-0.037276in}}{\pgfqpoint{0.029463in}{-0.029463in}}%
\pgfpathcurveto{\pgfqpoint{0.037276in}{-0.021649in}}{\pgfqpoint{0.041667in}{-0.011050in}}{\pgfqpoint{0.041667in}{0.000000in}}%
\pgfpathcurveto{\pgfqpoint{0.041667in}{0.011050in}}{\pgfqpoint{0.037276in}{0.021649in}}{\pgfqpoint{0.029463in}{0.029463in}}%
\pgfpathcurveto{\pgfqpoint{0.021649in}{0.037276in}}{\pgfqpoint{0.011050in}{0.041667in}}{\pgfqpoint{0.000000in}{0.041667in}}%
\pgfpathcurveto{\pgfqpoint{-0.011050in}{0.041667in}}{\pgfqpoint{-0.021649in}{0.037276in}}{\pgfqpoint{-0.029463in}{0.029463in}}%
\pgfpathcurveto{\pgfqpoint{-0.037276in}{0.021649in}}{\pgfqpoint{-0.041667in}{0.011050in}}{\pgfqpoint{-0.041667in}{0.000000in}}%
\pgfpathcurveto{\pgfqpoint{-0.041667in}{-0.011050in}}{\pgfqpoint{-0.037276in}{-0.021649in}}{\pgfqpoint{-0.029463in}{-0.029463in}}%
\pgfpathcurveto{\pgfqpoint{-0.021649in}{-0.037276in}}{\pgfqpoint{-0.011050in}{-0.041667in}}{\pgfqpoint{0.000000in}{-0.041667in}}%
\pgfpathlineto{\pgfqpoint{0.000000in}{-0.041667in}}%
\pgfpathclose%
\pgfusepath{stroke,fill}%
}%
\begin{pgfscope}%
\pgfsys@transformshift{2.209869in}{0.990000in}%
\pgfsys@useobject{currentmarker}{}%
\end{pgfscope}%
\begin{pgfscope}%
\pgfsys@transformshift{2.741026in}{0.290000in}%
\pgfsys@useobject{currentmarker}{}%
\end{pgfscope}%
\end{pgfscope}%
\begin{pgfscope}%
\pgfpathrectangle{\pgfqpoint{0.895256in}{0.220000in}}{\pgfqpoint{5.550589in}{1.540000in}}%
\pgfusepath{clip}%
\pgfsetrectcap%
\pgfsetroundjoin%
\pgfsetlinewidth{1.505625pt}%
\definecolor{currentstroke}{rgb}{1.000000,0.498039,0.054902}%
\pgfsetstrokecolor{currentstroke}%
\pgfsetdash{}{0pt}%
\pgfpathmoveto{\pgfqpoint{2.209869in}{1.076774in}}%
\pgfpathlineto{\pgfqpoint{2.741026in}{0.784815in}}%
\pgfusepath{stroke}%
\end{pgfscope}%
\begin{pgfscope}%
\pgfpathrectangle{\pgfqpoint{0.895256in}{0.220000in}}{\pgfqpoint{5.550589in}{1.540000in}}%
\pgfusepath{clip}%
\pgfsetbuttcap%
\pgfsetroundjoin%
\definecolor{currentfill}{rgb}{1.000000,0.498039,0.054902}%
\pgfsetfillcolor{currentfill}%
\pgfsetlinewidth{1.003750pt}%
\definecolor{currentstroke}{rgb}{1.000000,0.498039,0.054902}%
\pgfsetstrokecolor{currentstroke}%
\pgfsetdash{}{0pt}%
\pgfsys@defobject{currentmarker}{\pgfqpoint{-0.041667in}{-0.041667in}}{\pgfqpoint{0.041667in}{0.041667in}}{%
\pgfpathmoveto{\pgfqpoint{0.000000in}{-0.041667in}}%
\pgfpathcurveto{\pgfqpoint{0.011050in}{-0.041667in}}{\pgfqpoint{0.021649in}{-0.037276in}}{\pgfqpoint{0.029463in}{-0.029463in}}%
\pgfpathcurveto{\pgfqpoint{0.037276in}{-0.021649in}}{\pgfqpoint{0.041667in}{-0.011050in}}{\pgfqpoint{0.041667in}{0.000000in}}%
\pgfpathcurveto{\pgfqpoint{0.041667in}{0.011050in}}{\pgfqpoint{0.037276in}{0.021649in}}{\pgfqpoint{0.029463in}{0.029463in}}%
\pgfpathcurveto{\pgfqpoint{0.021649in}{0.037276in}}{\pgfqpoint{0.011050in}{0.041667in}}{\pgfqpoint{0.000000in}{0.041667in}}%
\pgfpathcurveto{\pgfqpoint{-0.011050in}{0.041667in}}{\pgfqpoint{-0.021649in}{0.037276in}}{\pgfqpoint{-0.029463in}{0.029463in}}%
\pgfpathcurveto{\pgfqpoint{-0.037276in}{0.021649in}}{\pgfqpoint{-0.041667in}{0.011050in}}{\pgfqpoint{-0.041667in}{0.000000in}}%
\pgfpathcurveto{\pgfqpoint{-0.041667in}{-0.011050in}}{\pgfqpoint{-0.037276in}{-0.021649in}}{\pgfqpoint{-0.029463in}{-0.029463in}}%
\pgfpathcurveto{\pgfqpoint{-0.021649in}{-0.037276in}}{\pgfqpoint{-0.011050in}{-0.041667in}}{\pgfqpoint{0.000000in}{-0.041667in}}%
\pgfpathlineto{\pgfqpoint{0.000000in}{-0.041667in}}%
\pgfpathclose%
\pgfusepath{stroke,fill}%
}%
\begin{pgfscope}%
\pgfsys@transformshift{2.209869in}{1.076774in}%
\pgfsys@useobject{currentmarker}{}%
\end{pgfscope}%
\begin{pgfscope}%
\pgfsys@transformshift{2.741026in}{0.784815in}%
\pgfsys@useobject{currentmarker}{}%
\end{pgfscope}%
\end{pgfscope}%
\begin{pgfscope}%
\pgfpathrectangle{\pgfqpoint{0.895256in}{0.220000in}}{\pgfqpoint{5.550589in}{1.540000in}}%
\pgfusepath{clip}%
\pgfsetbuttcap%
\pgfsetroundjoin%
\pgfsetlinewidth{1.505625pt}%
\definecolor{currentstroke}{rgb}{0.501961,0.501961,0.501961}%
\pgfsetstrokecolor{currentstroke}%
\pgfsetdash{{5.550000pt}{2.400000pt}}{0.000000pt}%
\pgfpathmoveto{\pgfqpoint{3.006605in}{0.220000in}}%
\pgfpathlineto{\pgfqpoint{3.006605in}{1.760000in}}%
\pgfusepath{stroke}%
\end{pgfscope}%
\begin{pgfscope}%
\pgfpathrectangle{\pgfqpoint{0.895256in}{0.220000in}}{\pgfqpoint{5.550589in}{1.540000in}}%
\pgfusepath{clip}%
\pgfsetrectcap%
\pgfsetroundjoin%
\pgfsetlinewidth{1.505625pt}%
\definecolor{currentstroke}{rgb}{0.121569,0.466667,0.705882}%
\pgfsetstrokecolor{currentstroke}%
\pgfsetdash{}{0pt}%
\pgfpathmoveto{\pgfqpoint{3.272183in}{0.990000in}}%
\pgfpathlineto{\pgfqpoint{3.803340in}{0.290000in}}%
\pgfusepath{stroke}%
\end{pgfscope}%
\begin{pgfscope}%
\pgfpathrectangle{\pgfqpoint{0.895256in}{0.220000in}}{\pgfqpoint{5.550589in}{1.540000in}}%
\pgfusepath{clip}%
\pgfsetbuttcap%
\pgfsetroundjoin%
\definecolor{currentfill}{rgb}{0.121569,0.466667,0.705882}%
\pgfsetfillcolor{currentfill}%
\pgfsetlinewidth{1.003750pt}%
\definecolor{currentstroke}{rgb}{0.121569,0.466667,0.705882}%
\pgfsetstrokecolor{currentstroke}%
\pgfsetdash{}{0pt}%
\pgfsys@defobject{currentmarker}{\pgfqpoint{-0.041667in}{-0.041667in}}{\pgfqpoint{0.041667in}{0.041667in}}{%
\pgfpathmoveto{\pgfqpoint{0.000000in}{-0.041667in}}%
\pgfpathcurveto{\pgfqpoint{0.011050in}{-0.041667in}}{\pgfqpoint{0.021649in}{-0.037276in}}{\pgfqpoint{0.029463in}{-0.029463in}}%
\pgfpathcurveto{\pgfqpoint{0.037276in}{-0.021649in}}{\pgfqpoint{0.041667in}{-0.011050in}}{\pgfqpoint{0.041667in}{0.000000in}}%
\pgfpathcurveto{\pgfqpoint{0.041667in}{0.011050in}}{\pgfqpoint{0.037276in}{0.021649in}}{\pgfqpoint{0.029463in}{0.029463in}}%
\pgfpathcurveto{\pgfqpoint{0.021649in}{0.037276in}}{\pgfqpoint{0.011050in}{0.041667in}}{\pgfqpoint{0.000000in}{0.041667in}}%
\pgfpathcurveto{\pgfqpoint{-0.011050in}{0.041667in}}{\pgfqpoint{-0.021649in}{0.037276in}}{\pgfqpoint{-0.029463in}{0.029463in}}%
\pgfpathcurveto{\pgfqpoint{-0.037276in}{0.021649in}}{\pgfqpoint{-0.041667in}{0.011050in}}{\pgfqpoint{-0.041667in}{0.000000in}}%
\pgfpathcurveto{\pgfqpoint{-0.041667in}{-0.011050in}}{\pgfqpoint{-0.037276in}{-0.021649in}}{\pgfqpoint{-0.029463in}{-0.029463in}}%
\pgfpathcurveto{\pgfqpoint{-0.021649in}{-0.037276in}}{\pgfqpoint{-0.011050in}{-0.041667in}}{\pgfqpoint{0.000000in}{-0.041667in}}%
\pgfpathlineto{\pgfqpoint{0.000000in}{-0.041667in}}%
\pgfpathclose%
\pgfusepath{stroke,fill}%
}%
\begin{pgfscope}%
\pgfsys@transformshift{3.272183in}{0.990000in}%
\pgfsys@useobject{currentmarker}{}%
\end{pgfscope}%
\begin{pgfscope}%
\pgfsys@transformshift{3.803340in}{0.290000in}%
\pgfsys@useobject{currentmarker}{}%
\end{pgfscope}%
\end{pgfscope}%
\begin{pgfscope}%
\pgfpathrectangle{\pgfqpoint{0.895256in}{0.220000in}}{\pgfqpoint{5.550589in}{1.540000in}}%
\pgfusepath{clip}%
\pgfsetrectcap%
\pgfsetroundjoin%
\pgfsetlinewidth{1.505625pt}%
\definecolor{currentstroke}{rgb}{1.000000,0.498039,0.054902}%
\pgfsetstrokecolor{currentstroke}%
\pgfsetdash{}{0pt}%
\pgfpathmoveto{\pgfqpoint{3.272183in}{0.932386in}}%
\pgfpathlineto{\pgfqpoint{3.803340in}{0.831428in}}%
\pgfusepath{stroke}%
\end{pgfscope}%
\begin{pgfscope}%
\pgfpathrectangle{\pgfqpoint{0.895256in}{0.220000in}}{\pgfqpoint{5.550589in}{1.540000in}}%
\pgfusepath{clip}%
\pgfsetbuttcap%
\pgfsetroundjoin%
\definecolor{currentfill}{rgb}{1.000000,0.498039,0.054902}%
\pgfsetfillcolor{currentfill}%
\pgfsetlinewidth{1.003750pt}%
\definecolor{currentstroke}{rgb}{1.000000,0.498039,0.054902}%
\pgfsetstrokecolor{currentstroke}%
\pgfsetdash{}{0pt}%
\pgfsys@defobject{currentmarker}{\pgfqpoint{-0.041667in}{-0.041667in}}{\pgfqpoint{0.041667in}{0.041667in}}{%
\pgfpathmoveto{\pgfqpoint{0.000000in}{-0.041667in}}%
\pgfpathcurveto{\pgfqpoint{0.011050in}{-0.041667in}}{\pgfqpoint{0.021649in}{-0.037276in}}{\pgfqpoint{0.029463in}{-0.029463in}}%
\pgfpathcurveto{\pgfqpoint{0.037276in}{-0.021649in}}{\pgfqpoint{0.041667in}{-0.011050in}}{\pgfqpoint{0.041667in}{0.000000in}}%
\pgfpathcurveto{\pgfqpoint{0.041667in}{0.011050in}}{\pgfqpoint{0.037276in}{0.021649in}}{\pgfqpoint{0.029463in}{0.029463in}}%
\pgfpathcurveto{\pgfqpoint{0.021649in}{0.037276in}}{\pgfqpoint{0.011050in}{0.041667in}}{\pgfqpoint{0.000000in}{0.041667in}}%
\pgfpathcurveto{\pgfqpoint{-0.011050in}{0.041667in}}{\pgfqpoint{-0.021649in}{0.037276in}}{\pgfqpoint{-0.029463in}{0.029463in}}%
\pgfpathcurveto{\pgfqpoint{-0.037276in}{0.021649in}}{\pgfqpoint{-0.041667in}{0.011050in}}{\pgfqpoint{-0.041667in}{0.000000in}}%
\pgfpathcurveto{\pgfqpoint{-0.041667in}{-0.011050in}}{\pgfqpoint{-0.037276in}{-0.021649in}}{\pgfqpoint{-0.029463in}{-0.029463in}}%
\pgfpathcurveto{\pgfqpoint{-0.021649in}{-0.037276in}}{\pgfqpoint{-0.011050in}{-0.041667in}}{\pgfqpoint{0.000000in}{-0.041667in}}%
\pgfpathlineto{\pgfqpoint{0.000000in}{-0.041667in}}%
\pgfpathclose%
\pgfusepath{stroke,fill}%
}%
\begin{pgfscope}%
\pgfsys@transformshift{3.272183in}{0.932386in}%
\pgfsys@useobject{currentmarker}{}%
\end{pgfscope}%
\begin{pgfscope}%
\pgfsys@transformshift{3.803340in}{0.831428in}%
\pgfsys@useobject{currentmarker}{}%
\end{pgfscope}%
\end{pgfscope}%
\begin{pgfscope}%
\pgfpathrectangle{\pgfqpoint{0.895256in}{0.220000in}}{\pgfqpoint{5.550589in}{1.540000in}}%
\pgfusepath{clip}%
\pgfsetbuttcap%
\pgfsetroundjoin%
\pgfsetlinewidth{1.505625pt}%
\definecolor{currentstroke}{rgb}{0.501961,0.501961,0.501961}%
\pgfsetstrokecolor{currentstroke}%
\pgfsetdash{{5.550000pt}{2.400000pt}}{0.000000pt}%
\pgfpathmoveto{\pgfqpoint{4.068918in}{0.220000in}}%
\pgfpathlineto{\pgfqpoint{4.068918in}{1.760000in}}%
\pgfusepath{stroke}%
\end{pgfscope}%
\begin{pgfscope}%
\pgfpathrectangle{\pgfqpoint{0.895256in}{0.220000in}}{\pgfqpoint{5.550589in}{1.540000in}}%
\pgfusepath{clip}%
\pgfsetrectcap%
\pgfsetroundjoin%
\pgfsetlinewidth{1.505625pt}%
\definecolor{currentstroke}{rgb}{0.121569,0.466667,0.705882}%
\pgfsetstrokecolor{currentstroke}%
\pgfsetdash{}{0pt}%
\pgfpathmoveto{\pgfqpoint{4.334497in}{0.990000in}}%
\pgfpathlineto{\pgfqpoint{4.865653in}{1.690000in}}%
\pgfusepath{stroke}%
\end{pgfscope}%
\begin{pgfscope}%
\pgfpathrectangle{\pgfqpoint{0.895256in}{0.220000in}}{\pgfqpoint{5.550589in}{1.540000in}}%
\pgfusepath{clip}%
\pgfsetbuttcap%
\pgfsetroundjoin%
\definecolor{currentfill}{rgb}{0.121569,0.466667,0.705882}%
\pgfsetfillcolor{currentfill}%
\pgfsetlinewidth{1.003750pt}%
\definecolor{currentstroke}{rgb}{0.121569,0.466667,0.705882}%
\pgfsetstrokecolor{currentstroke}%
\pgfsetdash{}{0pt}%
\pgfsys@defobject{currentmarker}{\pgfqpoint{-0.041667in}{-0.041667in}}{\pgfqpoint{0.041667in}{0.041667in}}{%
\pgfpathmoveto{\pgfqpoint{0.000000in}{-0.041667in}}%
\pgfpathcurveto{\pgfqpoint{0.011050in}{-0.041667in}}{\pgfqpoint{0.021649in}{-0.037276in}}{\pgfqpoint{0.029463in}{-0.029463in}}%
\pgfpathcurveto{\pgfqpoint{0.037276in}{-0.021649in}}{\pgfqpoint{0.041667in}{-0.011050in}}{\pgfqpoint{0.041667in}{0.000000in}}%
\pgfpathcurveto{\pgfqpoint{0.041667in}{0.011050in}}{\pgfqpoint{0.037276in}{0.021649in}}{\pgfqpoint{0.029463in}{0.029463in}}%
\pgfpathcurveto{\pgfqpoint{0.021649in}{0.037276in}}{\pgfqpoint{0.011050in}{0.041667in}}{\pgfqpoint{0.000000in}{0.041667in}}%
\pgfpathcurveto{\pgfqpoint{-0.011050in}{0.041667in}}{\pgfqpoint{-0.021649in}{0.037276in}}{\pgfqpoint{-0.029463in}{0.029463in}}%
\pgfpathcurveto{\pgfqpoint{-0.037276in}{0.021649in}}{\pgfqpoint{-0.041667in}{0.011050in}}{\pgfqpoint{-0.041667in}{0.000000in}}%
\pgfpathcurveto{\pgfqpoint{-0.041667in}{-0.011050in}}{\pgfqpoint{-0.037276in}{-0.021649in}}{\pgfqpoint{-0.029463in}{-0.029463in}}%
\pgfpathcurveto{\pgfqpoint{-0.021649in}{-0.037276in}}{\pgfqpoint{-0.011050in}{-0.041667in}}{\pgfqpoint{0.000000in}{-0.041667in}}%
\pgfpathlineto{\pgfqpoint{0.000000in}{-0.041667in}}%
\pgfpathclose%
\pgfusepath{stroke,fill}%
}%
\begin{pgfscope}%
\pgfsys@transformshift{4.334497in}{0.990000in}%
\pgfsys@useobject{currentmarker}{}%
\end{pgfscope}%
\begin{pgfscope}%
\pgfsys@transformshift{4.865653in}{1.690000in}%
\pgfsys@useobject{currentmarker}{}%
\end{pgfscope}%
\end{pgfscope}%
\begin{pgfscope}%
\pgfpathrectangle{\pgfqpoint{0.895256in}{0.220000in}}{\pgfqpoint{5.550589in}{1.540000in}}%
\pgfusepath{clip}%
\pgfsetrectcap%
\pgfsetroundjoin%
\pgfsetlinewidth{1.505625pt}%
\definecolor{currentstroke}{rgb}{1.000000,0.498039,0.054902}%
\pgfsetstrokecolor{currentstroke}%
\pgfsetdash{}{0pt}%
\pgfpathmoveto{\pgfqpoint{4.334497in}{0.901411in}}%
\pgfpathlineto{\pgfqpoint{4.865653in}{1.190699in}}%
\pgfusepath{stroke}%
\end{pgfscope}%
\begin{pgfscope}%
\pgfpathrectangle{\pgfqpoint{0.895256in}{0.220000in}}{\pgfqpoint{5.550589in}{1.540000in}}%
\pgfusepath{clip}%
\pgfsetbuttcap%
\pgfsetroundjoin%
\definecolor{currentfill}{rgb}{1.000000,0.498039,0.054902}%
\pgfsetfillcolor{currentfill}%
\pgfsetlinewidth{1.003750pt}%
\definecolor{currentstroke}{rgb}{1.000000,0.498039,0.054902}%
\pgfsetstrokecolor{currentstroke}%
\pgfsetdash{}{0pt}%
\pgfsys@defobject{currentmarker}{\pgfqpoint{-0.041667in}{-0.041667in}}{\pgfqpoint{0.041667in}{0.041667in}}{%
\pgfpathmoveto{\pgfqpoint{0.000000in}{-0.041667in}}%
\pgfpathcurveto{\pgfqpoint{0.011050in}{-0.041667in}}{\pgfqpoint{0.021649in}{-0.037276in}}{\pgfqpoint{0.029463in}{-0.029463in}}%
\pgfpathcurveto{\pgfqpoint{0.037276in}{-0.021649in}}{\pgfqpoint{0.041667in}{-0.011050in}}{\pgfqpoint{0.041667in}{0.000000in}}%
\pgfpathcurveto{\pgfqpoint{0.041667in}{0.011050in}}{\pgfqpoint{0.037276in}{0.021649in}}{\pgfqpoint{0.029463in}{0.029463in}}%
\pgfpathcurveto{\pgfqpoint{0.021649in}{0.037276in}}{\pgfqpoint{0.011050in}{0.041667in}}{\pgfqpoint{0.000000in}{0.041667in}}%
\pgfpathcurveto{\pgfqpoint{-0.011050in}{0.041667in}}{\pgfqpoint{-0.021649in}{0.037276in}}{\pgfqpoint{-0.029463in}{0.029463in}}%
\pgfpathcurveto{\pgfqpoint{-0.037276in}{0.021649in}}{\pgfqpoint{-0.041667in}{0.011050in}}{\pgfqpoint{-0.041667in}{0.000000in}}%
\pgfpathcurveto{\pgfqpoint{-0.041667in}{-0.011050in}}{\pgfqpoint{-0.037276in}{-0.021649in}}{\pgfqpoint{-0.029463in}{-0.029463in}}%
\pgfpathcurveto{\pgfqpoint{-0.021649in}{-0.037276in}}{\pgfqpoint{-0.011050in}{-0.041667in}}{\pgfqpoint{0.000000in}{-0.041667in}}%
\pgfpathlineto{\pgfqpoint{0.000000in}{-0.041667in}}%
\pgfpathclose%
\pgfusepath{stroke,fill}%
}%
\begin{pgfscope}%
\pgfsys@transformshift{4.334497in}{0.901411in}%
\pgfsys@useobject{currentmarker}{}%
\end{pgfscope}%
\begin{pgfscope}%
\pgfsys@transformshift{4.865653in}{1.190699in}%
\pgfsys@useobject{currentmarker}{}%
\end{pgfscope}%
\end{pgfscope}%
\begin{pgfscope}%
\pgfpathrectangle{\pgfqpoint{0.895256in}{0.220000in}}{\pgfqpoint{5.550589in}{1.540000in}}%
\pgfusepath{clip}%
\pgfsetbuttcap%
\pgfsetroundjoin%
\pgfsetlinewidth{1.505625pt}%
\definecolor{currentstroke}{rgb}{0.501961,0.501961,0.501961}%
\pgfsetstrokecolor{currentstroke}%
\pgfsetdash{{5.550000pt}{2.400000pt}}{0.000000pt}%
\pgfpathmoveto{\pgfqpoint{5.131232in}{0.220000in}}%
\pgfpathlineto{\pgfqpoint{5.131232in}{1.760000in}}%
\pgfusepath{stroke}%
\end{pgfscope}%
\begin{pgfscope}%
\pgfpathrectangle{\pgfqpoint{0.895256in}{0.220000in}}{\pgfqpoint{5.550589in}{1.540000in}}%
\pgfusepath{clip}%
\pgfsetrectcap%
\pgfsetroundjoin%
\pgfsetlinewidth{1.505625pt}%
\definecolor{currentstroke}{rgb}{0.121569,0.466667,0.705882}%
\pgfsetstrokecolor{currentstroke}%
\pgfsetdash{}{0pt}%
\pgfpathmoveto{\pgfqpoint{5.396810in}{0.990000in}}%
\pgfpathlineto{\pgfqpoint{5.927967in}{1.690000in}}%
\pgfusepath{stroke}%
\end{pgfscope}%
\begin{pgfscope}%
\pgfpathrectangle{\pgfqpoint{0.895256in}{0.220000in}}{\pgfqpoint{5.550589in}{1.540000in}}%
\pgfusepath{clip}%
\pgfsetbuttcap%
\pgfsetroundjoin%
\definecolor{currentfill}{rgb}{0.121569,0.466667,0.705882}%
\pgfsetfillcolor{currentfill}%
\pgfsetlinewidth{1.003750pt}%
\definecolor{currentstroke}{rgb}{0.121569,0.466667,0.705882}%
\pgfsetstrokecolor{currentstroke}%
\pgfsetdash{}{0pt}%
\pgfsys@defobject{currentmarker}{\pgfqpoint{-0.041667in}{-0.041667in}}{\pgfqpoint{0.041667in}{0.041667in}}{%
\pgfpathmoveto{\pgfqpoint{0.000000in}{-0.041667in}}%
\pgfpathcurveto{\pgfqpoint{0.011050in}{-0.041667in}}{\pgfqpoint{0.021649in}{-0.037276in}}{\pgfqpoint{0.029463in}{-0.029463in}}%
\pgfpathcurveto{\pgfqpoint{0.037276in}{-0.021649in}}{\pgfqpoint{0.041667in}{-0.011050in}}{\pgfqpoint{0.041667in}{0.000000in}}%
\pgfpathcurveto{\pgfqpoint{0.041667in}{0.011050in}}{\pgfqpoint{0.037276in}{0.021649in}}{\pgfqpoint{0.029463in}{0.029463in}}%
\pgfpathcurveto{\pgfqpoint{0.021649in}{0.037276in}}{\pgfqpoint{0.011050in}{0.041667in}}{\pgfqpoint{0.000000in}{0.041667in}}%
\pgfpathcurveto{\pgfqpoint{-0.011050in}{0.041667in}}{\pgfqpoint{-0.021649in}{0.037276in}}{\pgfqpoint{-0.029463in}{0.029463in}}%
\pgfpathcurveto{\pgfqpoint{-0.037276in}{0.021649in}}{\pgfqpoint{-0.041667in}{0.011050in}}{\pgfqpoint{-0.041667in}{0.000000in}}%
\pgfpathcurveto{\pgfqpoint{-0.041667in}{-0.011050in}}{\pgfqpoint{-0.037276in}{-0.021649in}}{\pgfqpoint{-0.029463in}{-0.029463in}}%
\pgfpathcurveto{\pgfqpoint{-0.021649in}{-0.037276in}}{\pgfqpoint{-0.011050in}{-0.041667in}}{\pgfqpoint{0.000000in}{-0.041667in}}%
\pgfpathlineto{\pgfqpoint{0.000000in}{-0.041667in}}%
\pgfpathclose%
\pgfusepath{stroke,fill}%
}%
\begin{pgfscope}%
\pgfsys@transformshift{5.396810in}{0.990000in}%
\pgfsys@useobject{currentmarker}{}%
\end{pgfscope}%
\begin{pgfscope}%
\pgfsys@transformshift{5.927967in}{1.690000in}%
\pgfsys@useobject{currentmarker}{}%
\end{pgfscope}%
\end{pgfscope}%
\begin{pgfscope}%
\pgfpathrectangle{\pgfqpoint{0.895256in}{0.220000in}}{\pgfqpoint{5.550589in}{1.540000in}}%
\pgfusepath{clip}%
\pgfsetrectcap%
\pgfsetroundjoin%
\pgfsetlinewidth{1.505625pt}%
\definecolor{currentstroke}{rgb}{1.000000,0.498039,0.054902}%
\pgfsetstrokecolor{currentstroke}%
\pgfsetdash{}{0pt}%
\pgfpathmoveto{\pgfqpoint{5.396810in}{1.066840in}}%
\pgfpathlineto{\pgfqpoint{5.927967in}{1.127680in}}%
\pgfusepath{stroke}%
\end{pgfscope}%
\begin{pgfscope}%
\pgfpathrectangle{\pgfqpoint{0.895256in}{0.220000in}}{\pgfqpoint{5.550589in}{1.540000in}}%
\pgfusepath{clip}%
\pgfsetbuttcap%
\pgfsetroundjoin%
\definecolor{currentfill}{rgb}{1.000000,0.498039,0.054902}%
\pgfsetfillcolor{currentfill}%
\pgfsetlinewidth{1.003750pt}%
\definecolor{currentstroke}{rgb}{1.000000,0.498039,0.054902}%
\pgfsetstrokecolor{currentstroke}%
\pgfsetdash{}{0pt}%
\pgfsys@defobject{currentmarker}{\pgfqpoint{-0.041667in}{-0.041667in}}{\pgfqpoint{0.041667in}{0.041667in}}{%
\pgfpathmoveto{\pgfqpoint{0.000000in}{-0.041667in}}%
\pgfpathcurveto{\pgfqpoint{0.011050in}{-0.041667in}}{\pgfqpoint{0.021649in}{-0.037276in}}{\pgfqpoint{0.029463in}{-0.029463in}}%
\pgfpathcurveto{\pgfqpoint{0.037276in}{-0.021649in}}{\pgfqpoint{0.041667in}{-0.011050in}}{\pgfqpoint{0.041667in}{0.000000in}}%
\pgfpathcurveto{\pgfqpoint{0.041667in}{0.011050in}}{\pgfqpoint{0.037276in}{0.021649in}}{\pgfqpoint{0.029463in}{0.029463in}}%
\pgfpathcurveto{\pgfqpoint{0.021649in}{0.037276in}}{\pgfqpoint{0.011050in}{0.041667in}}{\pgfqpoint{0.000000in}{0.041667in}}%
\pgfpathcurveto{\pgfqpoint{-0.011050in}{0.041667in}}{\pgfqpoint{-0.021649in}{0.037276in}}{\pgfqpoint{-0.029463in}{0.029463in}}%
\pgfpathcurveto{\pgfqpoint{-0.037276in}{0.021649in}}{\pgfqpoint{-0.041667in}{0.011050in}}{\pgfqpoint{-0.041667in}{0.000000in}}%
\pgfpathcurveto{\pgfqpoint{-0.041667in}{-0.011050in}}{\pgfqpoint{-0.037276in}{-0.021649in}}{\pgfqpoint{-0.029463in}{-0.029463in}}%
\pgfpathcurveto{\pgfqpoint{-0.021649in}{-0.037276in}}{\pgfqpoint{-0.011050in}{-0.041667in}}{\pgfqpoint{0.000000in}{-0.041667in}}%
\pgfpathlineto{\pgfqpoint{0.000000in}{-0.041667in}}%
\pgfpathclose%
\pgfusepath{stroke,fill}%
}%
\begin{pgfscope}%
\pgfsys@transformshift{5.396810in}{1.066840in}%
\pgfsys@useobject{currentmarker}{}%
\end{pgfscope}%
\begin{pgfscope}%
\pgfsys@transformshift{5.927967in}{1.127680in}%
\pgfsys@useobject{currentmarker}{}%
\end{pgfscope}%
\end{pgfscope}%
\begin{pgfscope}%
\pgfpathrectangle{\pgfqpoint{0.895256in}{0.220000in}}{\pgfqpoint{5.550589in}{1.540000in}}%
\pgfusepath{clip}%
\pgfsetbuttcap%
\pgfsetroundjoin%
\pgfsetlinewidth{1.505625pt}%
\definecolor{currentstroke}{rgb}{0.501961,0.501961,0.501961}%
\pgfsetstrokecolor{currentstroke}%
\pgfsetdash{{5.550000pt}{2.400000pt}}{0.000000pt}%
\pgfpathmoveto{\pgfqpoint{6.193546in}{0.220000in}}%
\pgfpathlineto{\pgfqpoint{6.193546in}{1.760000in}}%
\pgfusepath{stroke}%
\end{pgfscope}%
\begin{pgfscope}%
\pgfsetrectcap%
\pgfsetmiterjoin%
\pgfsetlinewidth{0.803000pt}%
\definecolor{currentstroke}{rgb}{0.000000,0.000000,0.000000}%
\pgfsetstrokecolor{currentstroke}%
\pgfsetdash{}{0pt}%
\pgfpathmoveto{\pgfqpoint{0.895256in}{0.220000in}}%
\pgfpathlineto{\pgfqpoint{0.895256in}{1.760000in}}%
\pgfusepath{stroke}%
\end{pgfscope}%
\begin{pgfscope}%
\pgfsetrectcap%
\pgfsetmiterjoin%
\pgfsetlinewidth{0.803000pt}%
\definecolor{currentstroke}{rgb}{0.000000,0.000000,0.000000}%
\pgfsetstrokecolor{currentstroke}%
\pgfsetdash{}{0pt}%
\pgfpathmoveto{\pgfqpoint{6.445845in}{0.220000in}}%
\pgfpathlineto{\pgfqpoint{6.445845in}{1.760000in}}%
\pgfusepath{stroke}%
\end{pgfscope}%
\begin{pgfscope}%
\pgfsetrectcap%
\pgfsetmiterjoin%
\pgfsetlinewidth{0.803000pt}%
\definecolor{currentstroke}{rgb}{0.000000,0.000000,0.000000}%
\pgfsetstrokecolor{currentstroke}%
\pgfsetdash{}{0pt}%
\pgfpathmoveto{\pgfqpoint{0.895256in}{0.220000in}}%
\pgfpathlineto{\pgfqpoint{6.445845in}{0.220000in}}%
\pgfusepath{stroke}%
\end{pgfscope}%
\begin{pgfscope}%
\pgfsetrectcap%
\pgfsetmiterjoin%
\pgfsetlinewidth{0.803000pt}%
\definecolor{currentstroke}{rgb}{0.000000,0.000000,0.000000}%
\pgfsetstrokecolor{currentstroke}%
\pgfsetdash{}{0pt}%
\pgfpathmoveto{\pgfqpoint{0.895256in}{1.760000in}}%
\pgfpathlineto{\pgfqpoint{6.445845in}{1.760000in}}%
\pgfusepath{stroke}%
\end{pgfscope}%
\begin{pgfscope}%
\definecolor{textcolor}{rgb}{0.000000,0.000000,0.000000}%
\pgfsetstrokecolor{textcolor}%
\pgfsetfillcolor{textcolor}%
\pgftext[x=3.670551in,y=1.843333in,,base]{\color{textcolor}{\rmfamily\fontsize{12.000000}{14.400000}\selectfont\catcode`\^=\active\def^{\ifmmode\sp\else\^{}\fi}\catcode`\%=\active\def%{\%}ASK on Cable}}%
\end{pgfscope}%
\begin{pgfscope}%
\pgfsetbuttcap%
\pgfsetmiterjoin%
\definecolor{currentfill}{rgb}{1.000000,1.000000,1.000000}%
\pgfsetfillcolor{currentfill}%
\pgfsetfillopacity{0.800000}%
\pgfsetlinewidth{1.003750pt}%
\definecolor{currentstroke}{rgb}{0.800000,0.800000,0.800000}%
\pgfsetstrokecolor{currentstroke}%
\pgfsetstrokeopacity{0.800000}%
\pgfsetdash{}{0pt}%
\pgfpathmoveto{\pgfqpoint{0.992478in}{0.289444in}}%
\pgfpathlineto{\pgfqpoint{1.986693in}{0.289444in}}%
\pgfpathquadraticcurveto{\pgfqpoint{2.014471in}{0.289444in}}{\pgfqpoint{2.014471in}{0.317222in}}%
\pgfpathlineto{\pgfqpoint{2.014471in}{0.690679in}}%
\pgfpathquadraticcurveto{\pgfqpoint{2.014471in}{0.718457in}}{\pgfqpoint{1.986693in}{0.718457in}}%
\pgfpathlineto{\pgfqpoint{0.992478in}{0.718457in}}%
\pgfpathquadraticcurveto{\pgfqpoint{0.964701in}{0.718457in}}{\pgfqpoint{0.964701in}{0.690679in}}%
\pgfpathlineto{\pgfqpoint{0.964701in}{0.317222in}}%
\pgfpathquadraticcurveto{\pgfqpoint{0.964701in}{0.289444in}}{\pgfqpoint{0.992478in}{0.289444in}}%
\pgfpathlineto{\pgfqpoint{0.992478in}{0.289444in}}%
\pgfpathclose%
\pgfusepath{stroke,fill}%
\end{pgfscope}%
\begin{pgfscope}%
\pgfsetrectcap%
\pgfsetroundjoin%
\pgfsetlinewidth{1.505625pt}%
\definecolor{currentstroke}{rgb}{0.121569,0.466667,0.705882}%
\pgfsetstrokecolor{currentstroke}%
\pgfsetdash{}{0pt}%
\pgfpathmoveto{\pgfqpoint{1.020256in}{0.614290in}}%
\pgfpathlineto{\pgfqpoint{1.159145in}{0.614290in}}%
\pgfpathlineto{\pgfqpoint{1.298034in}{0.614290in}}%
\pgfusepath{stroke}%
\end{pgfscope}%
\begin{pgfscope}%
\pgfsetbuttcap%
\pgfsetroundjoin%
\definecolor{currentfill}{rgb}{0.121569,0.466667,0.705882}%
\pgfsetfillcolor{currentfill}%
\pgfsetlinewidth{1.003750pt}%
\definecolor{currentstroke}{rgb}{0.121569,0.466667,0.705882}%
\pgfsetstrokecolor{currentstroke}%
\pgfsetdash{}{0pt}%
\pgfsys@defobject{currentmarker}{\pgfqpoint{-0.041667in}{-0.041667in}}{\pgfqpoint{0.041667in}{0.041667in}}{%
\pgfpathmoveto{\pgfqpoint{0.000000in}{-0.041667in}}%
\pgfpathcurveto{\pgfqpoint{0.011050in}{-0.041667in}}{\pgfqpoint{0.021649in}{-0.037276in}}{\pgfqpoint{0.029463in}{-0.029463in}}%
\pgfpathcurveto{\pgfqpoint{0.037276in}{-0.021649in}}{\pgfqpoint{0.041667in}{-0.011050in}}{\pgfqpoint{0.041667in}{0.000000in}}%
\pgfpathcurveto{\pgfqpoint{0.041667in}{0.011050in}}{\pgfqpoint{0.037276in}{0.021649in}}{\pgfqpoint{0.029463in}{0.029463in}}%
\pgfpathcurveto{\pgfqpoint{0.021649in}{0.037276in}}{\pgfqpoint{0.011050in}{0.041667in}}{\pgfqpoint{0.000000in}{0.041667in}}%
\pgfpathcurveto{\pgfqpoint{-0.011050in}{0.041667in}}{\pgfqpoint{-0.021649in}{0.037276in}}{\pgfqpoint{-0.029463in}{0.029463in}}%
\pgfpathcurveto{\pgfqpoint{-0.037276in}{0.021649in}}{\pgfqpoint{-0.041667in}{0.011050in}}{\pgfqpoint{-0.041667in}{0.000000in}}%
\pgfpathcurveto{\pgfqpoint{-0.041667in}{-0.011050in}}{\pgfqpoint{-0.037276in}{-0.021649in}}{\pgfqpoint{-0.029463in}{-0.029463in}}%
\pgfpathcurveto{\pgfqpoint{-0.021649in}{-0.037276in}}{\pgfqpoint{-0.011050in}{-0.041667in}}{\pgfqpoint{0.000000in}{-0.041667in}}%
\pgfpathlineto{\pgfqpoint{0.000000in}{-0.041667in}}%
\pgfpathclose%
\pgfusepath{stroke,fill}%
}%
\begin{pgfscope}%
\pgfsys@transformshift{1.159145in}{0.614290in}%
\pgfsys@useobject{currentmarker}{}%
\end{pgfscope}%
\end{pgfscope}%
\begin{pgfscope}%
\definecolor{textcolor}{rgb}{0.000000,0.000000,0.000000}%
\pgfsetstrokecolor{textcolor}%
\pgfsetfillcolor{textcolor}%
\pgftext[x=1.409145in,y=0.565679in,left,base]{\color{textcolor}{\rmfamily\fontsize{10.000000}{12.000000}\selectfont\catcode`\^=\active\def^{\ifmmode\sp\else\^{}\fi}\catcode`\%=\active\def%{\%}Playback}}%
\end{pgfscope}%
\begin{pgfscope}%
\pgfsetrectcap%
\pgfsetroundjoin%
\pgfsetlinewidth{1.505625pt}%
\definecolor{currentstroke}{rgb}{1.000000,0.498039,0.054902}%
\pgfsetstrokecolor{currentstroke}%
\pgfsetdash{}{0pt}%
\pgfpathmoveto{\pgfqpoint{1.020256in}{0.420617in}}%
\pgfpathlineto{\pgfqpoint{1.159145in}{0.420617in}}%
\pgfpathlineto{\pgfqpoint{1.298034in}{0.420617in}}%
\pgfusepath{stroke}%
\end{pgfscope}%
\begin{pgfscope}%
\pgfsetbuttcap%
\pgfsetroundjoin%
\definecolor{currentfill}{rgb}{1.000000,0.498039,0.054902}%
\pgfsetfillcolor{currentfill}%
\pgfsetlinewidth{1.003750pt}%
\definecolor{currentstroke}{rgb}{1.000000,0.498039,0.054902}%
\pgfsetstrokecolor{currentstroke}%
\pgfsetdash{}{0pt}%
\pgfsys@defobject{currentmarker}{\pgfqpoint{-0.041667in}{-0.041667in}}{\pgfqpoint{0.041667in}{0.041667in}}{%
\pgfpathmoveto{\pgfqpoint{0.000000in}{-0.041667in}}%
\pgfpathcurveto{\pgfqpoint{0.011050in}{-0.041667in}}{\pgfqpoint{0.021649in}{-0.037276in}}{\pgfqpoint{0.029463in}{-0.029463in}}%
\pgfpathcurveto{\pgfqpoint{0.037276in}{-0.021649in}}{\pgfqpoint{0.041667in}{-0.011050in}}{\pgfqpoint{0.041667in}{0.000000in}}%
\pgfpathcurveto{\pgfqpoint{0.041667in}{0.011050in}}{\pgfqpoint{0.037276in}{0.021649in}}{\pgfqpoint{0.029463in}{0.029463in}}%
\pgfpathcurveto{\pgfqpoint{0.021649in}{0.037276in}}{\pgfqpoint{0.011050in}{0.041667in}}{\pgfqpoint{0.000000in}{0.041667in}}%
\pgfpathcurveto{\pgfqpoint{-0.011050in}{0.041667in}}{\pgfqpoint{-0.021649in}{0.037276in}}{\pgfqpoint{-0.029463in}{0.029463in}}%
\pgfpathcurveto{\pgfqpoint{-0.037276in}{0.021649in}}{\pgfqpoint{-0.041667in}{0.011050in}}{\pgfqpoint{-0.041667in}{0.000000in}}%
\pgfpathcurveto{\pgfqpoint{-0.041667in}{-0.011050in}}{\pgfqpoint{-0.037276in}{-0.021649in}}{\pgfqpoint{-0.029463in}{-0.029463in}}%
\pgfpathcurveto{\pgfqpoint{-0.021649in}{-0.037276in}}{\pgfqpoint{-0.011050in}{-0.041667in}}{\pgfqpoint{0.000000in}{-0.041667in}}%
\pgfpathlineto{\pgfqpoint{0.000000in}{-0.041667in}}%
\pgfpathclose%
\pgfusepath{stroke,fill}%
}%
\begin{pgfscope}%
\pgfsys@transformshift{1.159145in}{0.420617in}%
\pgfsys@useobject{currentmarker}{}%
\end{pgfscope}%
\end{pgfscope}%
\begin{pgfscope}%
\definecolor{textcolor}{rgb}{0.000000,0.000000,0.000000}%
\pgfsetstrokecolor{textcolor}%
\pgfsetfillcolor{textcolor}%
\pgftext[x=1.409145in,y=0.372006in,left,base]{\color{textcolor}{\rmfamily\fontsize{10.000000}{12.000000}\selectfont\catcode`\^=\active\def^{\ifmmode\sp\else\^{}\fi}\catcode`\%=\active\def%{\%}Capture}}%
\end{pgfscope}%
\end{pgfpicture}%
\makeatother%
\endgroup%
}
    \caption{The plot of ASK modulation.}
    \label{fig:ask}
\end{figure}

The Figure \ref{fig:ask} shows the plot of ASK modulation.
The bit sequence is 1, 0, 0, 1, 1.

To modulate the bit sequence, we simply concate the signal of each bit.

To demodulate the signal, we do a dot product between the received signal and the signal of each bit, and choose the bit with the largest similarity score.

\subsubsection{FSK}

FSK is a modulation scheme that encodes symbols in the frequency of the wave.

We find that FSK is a good choice for our project 1. Because the frequency of the sound wave is more reliable than the amplitude of the sound wave when transmitting through the air.

Although the issue of clock drift is still a problem, we can still achieve a good performance with FSK. Because it has a higher tolerance to clock drift than PSK.

\begin{figure}[H]
    \noindent\makebox[\textwidth]{%% Creator: Matplotlib, PGF backend
%%
%% To include the figure in your LaTeX document, write
%%   \input{<filename>.pgf}
%%
%% Make sure the required packages are loaded in your preamble
%%   \usepackage{pgf}
%%
%% Also ensure that all the required font packages are loaded; for instance,
%% the lmodern package is sometimes necessary when using math font.
%%   \usepackage{lmodern}
%%
%% Figures using additional raster images can only be included by \input if
%% they are in the same directory as the main LaTeX file. For loading figures
%% from other directories you can use the `import` package
%%   \usepackage{import}
%%
%% and then include the figures with
%%   \import{<path to file>}{<filename>.pgf}
%%
%% Matplotlib used the following preamble
%%   \def\mathdefault#1{#1}
%%   \everymath=\expandafter{\the\everymath\displaystyle}
%%   \IfFileExists{scrextend.sty}{
%%     \usepackage[fontsize=10.000000pt]{scrextend}
%%   }{
%%     \renewcommand{\normalsize}{\fontsize{10.000000}{12.000000}\selectfont}
%%     \normalsize
%%   }
%%   
%%   \makeatletter\@ifpackageloaded{underscore}{}{\usepackage[strings]{underscore}}\makeatother
%%
\begingroup%
\makeatletter%
\begin{pgfpicture}%
\pgfpathrectangle{\pgfpointorigin}{\pgfqpoint{7.162050in}{2.000000in}}%
\pgfusepath{use as bounding box, clip}%
\begin{pgfscope}%
\pgfsetbuttcap%
\pgfsetmiterjoin%
\definecolor{currentfill}{rgb}{1.000000,1.000000,1.000000}%
\pgfsetfillcolor{currentfill}%
\pgfsetlinewidth{0.000000pt}%
\definecolor{currentstroke}{rgb}{1.000000,1.000000,1.000000}%
\pgfsetstrokecolor{currentstroke}%
\pgfsetdash{}{0pt}%
\pgfpathmoveto{\pgfqpoint{0.000000in}{0.000000in}}%
\pgfpathlineto{\pgfqpoint{7.162050in}{0.000000in}}%
\pgfpathlineto{\pgfqpoint{7.162050in}{2.000000in}}%
\pgfpathlineto{\pgfqpoint{0.000000in}{2.000000in}}%
\pgfpathlineto{\pgfqpoint{0.000000in}{0.000000in}}%
\pgfpathclose%
\pgfusepath{fill}%
\end{pgfscope}%
\begin{pgfscope}%
\pgfsetbuttcap%
\pgfsetmiterjoin%
\definecolor{currentfill}{rgb}{1.000000,1.000000,1.000000}%
\pgfsetfillcolor{currentfill}%
\pgfsetlinewidth{0.000000pt}%
\definecolor{currentstroke}{rgb}{0.000000,0.000000,0.000000}%
\pgfsetstrokecolor{currentstroke}%
\pgfsetstrokeopacity{0.000000}%
\pgfsetdash{}{0pt}%
\pgfpathmoveto{\pgfqpoint{0.895256in}{0.220000in}}%
\pgfpathlineto{\pgfqpoint{6.445845in}{0.220000in}}%
\pgfpathlineto{\pgfqpoint{6.445845in}{1.760000in}}%
\pgfpathlineto{\pgfqpoint{0.895256in}{1.760000in}}%
\pgfpathlineto{\pgfqpoint{0.895256in}{0.220000in}}%
\pgfpathclose%
\pgfusepath{fill}%
\end{pgfscope}%
\begin{pgfscope}%
\pgfsetbuttcap%
\pgfsetroundjoin%
\definecolor{currentfill}{rgb}{0.000000,0.000000,0.000000}%
\pgfsetfillcolor{currentfill}%
\pgfsetlinewidth{0.803000pt}%
\definecolor{currentstroke}{rgb}{0.000000,0.000000,0.000000}%
\pgfsetstrokecolor{currentstroke}%
\pgfsetdash{}{0pt}%
\pgfsys@defobject{currentmarker}{\pgfqpoint{0.000000in}{-0.048611in}}{\pgfqpoint{0.000000in}{0.000000in}}{%
\pgfpathmoveto{\pgfqpoint{0.000000in}{0.000000in}}%
\pgfpathlineto{\pgfqpoint{0.000000in}{-0.048611in}}%
\pgfusepath{stroke,fill}%
}%
\begin{pgfscope}%
\pgfsys@transformshift{1.147556in}{0.220000in}%
\pgfsys@useobject{currentmarker}{}%
\end{pgfscope}%
\end{pgfscope}%
\begin{pgfscope}%
\definecolor{textcolor}{rgb}{0.000000,0.000000,0.000000}%
\pgfsetstrokecolor{textcolor}%
\pgfsetfillcolor{textcolor}%
\pgftext[x=1.147556in,y=0.122778in,,top]{\color{textcolor}{\rmfamily\fontsize{10.000000}{12.000000}\selectfont\catcode`\^=\active\def^{\ifmmode\sp\else\^{}\fi}\catcode`\%=\active\def%{\%}$\mathdefault{0.000}$}}%
\end{pgfscope}%
\begin{pgfscope}%
\pgfsetbuttcap%
\pgfsetroundjoin%
\definecolor{currentfill}{rgb}{0.000000,0.000000,0.000000}%
\pgfsetfillcolor{currentfill}%
\pgfsetlinewidth{0.803000pt}%
\definecolor{currentstroke}{rgb}{0.000000,0.000000,0.000000}%
\pgfsetstrokecolor{currentstroke}%
\pgfsetdash{}{0pt}%
\pgfsys@defobject{currentmarker}{\pgfqpoint{0.000000in}{-0.048611in}}{\pgfqpoint{0.000000in}{0.000000in}}{%
\pgfpathmoveto{\pgfqpoint{0.000000in}{0.000000in}}%
\pgfpathlineto{\pgfqpoint{0.000000in}{-0.048611in}}%
\pgfusepath{stroke,fill}%
}%
\begin{pgfscope}%
\pgfsys@transformshift{2.160976in}{0.220000in}%
\pgfsys@useobject{currentmarker}{}%
\end{pgfscope}%
\end{pgfscope}%
\begin{pgfscope}%
\definecolor{textcolor}{rgb}{0.000000,0.000000,0.000000}%
\pgfsetstrokecolor{textcolor}%
\pgfsetfillcolor{textcolor}%
\pgftext[x=2.160976in,y=0.122778in,,top]{\color{textcolor}{\rmfamily\fontsize{10.000000}{12.000000}\selectfont\catcode`\^=\active\def^{\ifmmode\sp\else\^{}\fi}\catcode`\%=\active\def%{\%}$\mathdefault{0.001}$}}%
\end{pgfscope}%
\begin{pgfscope}%
\pgfsetbuttcap%
\pgfsetroundjoin%
\definecolor{currentfill}{rgb}{0.000000,0.000000,0.000000}%
\pgfsetfillcolor{currentfill}%
\pgfsetlinewidth{0.803000pt}%
\definecolor{currentstroke}{rgb}{0.000000,0.000000,0.000000}%
\pgfsetstrokecolor{currentstroke}%
\pgfsetdash{}{0pt}%
\pgfsys@defobject{currentmarker}{\pgfqpoint{0.000000in}{-0.048611in}}{\pgfqpoint{0.000000in}{0.000000in}}{%
\pgfpathmoveto{\pgfqpoint{0.000000in}{0.000000in}}%
\pgfpathlineto{\pgfqpoint{0.000000in}{-0.048611in}}%
\pgfusepath{stroke,fill}%
}%
\begin{pgfscope}%
\pgfsys@transformshift{3.174397in}{0.220000in}%
\pgfsys@useobject{currentmarker}{}%
\end{pgfscope}%
\end{pgfscope}%
\begin{pgfscope}%
\definecolor{textcolor}{rgb}{0.000000,0.000000,0.000000}%
\pgfsetstrokecolor{textcolor}%
\pgfsetfillcolor{textcolor}%
\pgftext[x=3.174397in,y=0.122778in,,top]{\color{textcolor}{\rmfamily\fontsize{10.000000}{12.000000}\selectfont\catcode`\^=\active\def^{\ifmmode\sp\else\^{}\fi}\catcode`\%=\active\def%{\%}$\mathdefault{0.002}$}}%
\end{pgfscope}%
\begin{pgfscope}%
\pgfsetbuttcap%
\pgfsetroundjoin%
\definecolor{currentfill}{rgb}{0.000000,0.000000,0.000000}%
\pgfsetfillcolor{currentfill}%
\pgfsetlinewidth{0.803000pt}%
\definecolor{currentstroke}{rgb}{0.000000,0.000000,0.000000}%
\pgfsetstrokecolor{currentstroke}%
\pgfsetdash{}{0pt}%
\pgfsys@defobject{currentmarker}{\pgfqpoint{0.000000in}{-0.048611in}}{\pgfqpoint{0.000000in}{0.000000in}}{%
\pgfpathmoveto{\pgfqpoint{0.000000in}{0.000000in}}%
\pgfpathlineto{\pgfqpoint{0.000000in}{-0.048611in}}%
\pgfusepath{stroke,fill}%
}%
\begin{pgfscope}%
\pgfsys@transformshift{4.187817in}{0.220000in}%
\pgfsys@useobject{currentmarker}{}%
\end{pgfscope}%
\end{pgfscope}%
\begin{pgfscope}%
\definecolor{textcolor}{rgb}{0.000000,0.000000,0.000000}%
\pgfsetstrokecolor{textcolor}%
\pgfsetfillcolor{textcolor}%
\pgftext[x=4.187817in,y=0.122778in,,top]{\color{textcolor}{\rmfamily\fontsize{10.000000}{12.000000}\selectfont\catcode`\^=\active\def^{\ifmmode\sp\else\^{}\fi}\catcode`\%=\active\def%{\%}$\mathdefault{0.003}$}}%
\end{pgfscope}%
\begin{pgfscope}%
\pgfsetbuttcap%
\pgfsetroundjoin%
\definecolor{currentfill}{rgb}{0.000000,0.000000,0.000000}%
\pgfsetfillcolor{currentfill}%
\pgfsetlinewidth{0.803000pt}%
\definecolor{currentstroke}{rgb}{0.000000,0.000000,0.000000}%
\pgfsetstrokecolor{currentstroke}%
\pgfsetdash{}{0pt}%
\pgfsys@defobject{currentmarker}{\pgfqpoint{0.000000in}{-0.048611in}}{\pgfqpoint{0.000000in}{0.000000in}}{%
\pgfpathmoveto{\pgfqpoint{0.000000in}{0.000000in}}%
\pgfpathlineto{\pgfqpoint{0.000000in}{-0.048611in}}%
\pgfusepath{stroke,fill}%
}%
\begin{pgfscope}%
\pgfsys@transformshift{5.201238in}{0.220000in}%
\pgfsys@useobject{currentmarker}{}%
\end{pgfscope}%
\end{pgfscope}%
\begin{pgfscope}%
\definecolor{textcolor}{rgb}{0.000000,0.000000,0.000000}%
\pgfsetstrokecolor{textcolor}%
\pgfsetfillcolor{textcolor}%
\pgftext[x=5.201238in,y=0.122778in,,top]{\color{textcolor}{\rmfamily\fontsize{10.000000}{12.000000}\selectfont\catcode`\^=\active\def^{\ifmmode\sp\else\^{}\fi}\catcode`\%=\active\def%{\%}$\mathdefault{0.004}$}}%
\end{pgfscope}%
\begin{pgfscope}%
\pgfsetbuttcap%
\pgfsetroundjoin%
\definecolor{currentfill}{rgb}{0.000000,0.000000,0.000000}%
\pgfsetfillcolor{currentfill}%
\pgfsetlinewidth{0.803000pt}%
\definecolor{currentstroke}{rgb}{0.000000,0.000000,0.000000}%
\pgfsetstrokecolor{currentstroke}%
\pgfsetdash{}{0pt}%
\pgfsys@defobject{currentmarker}{\pgfqpoint{0.000000in}{-0.048611in}}{\pgfqpoint{0.000000in}{0.000000in}}{%
\pgfpathmoveto{\pgfqpoint{0.000000in}{0.000000in}}%
\pgfpathlineto{\pgfqpoint{0.000000in}{-0.048611in}}%
\pgfusepath{stroke,fill}%
}%
\begin{pgfscope}%
\pgfsys@transformshift{6.214658in}{0.220000in}%
\pgfsys@useobject{currentmarker}{}%
\end{pgfscope}%
\end{pgfscope}%
\begin{pgfscope}%
\definecolor{textcolor}{rgb}{0.000000,0.000000,0.000000}%
\pgfsetstrokecolor{textcolor}%
\pgfsetfillcolor{textcolor}%
\pgftext[x=6.214658in,y=0.122778in,,top]{\color{textcolor}{\rmfamily\fontsize{10.000000}{12.000000}\selectfont\catcode`\^=\active\def^{\ifmmode\sp\else\^{}\fi}\catcode`\%=\active\def%{\%}$\mathdefault{0.005}$}}%
\end{pgfscope}%
\begin{pgfscope}%
\definecolor{textcolor}{rgb}{0.000000,0.000000,0.000000}%
\pgfsetstrokecolor{textcolor}%
\pgfsetfillcolor{textcolor}%
\pgftext[x=3.670551in,y=-0.056234in,,top]{\color{textcolor}{\rmfamily\fontsize{10.000000}{12.000000}\selectfont\catcode`\^=\active\def^{\ifmmode\sp\else\^{}\fi}\catcode`\%=\active\def%{\%}Time (s)}}%
\end{pgfscope}%
\begin{pgfscope}%
\pgfsetbuttcap%
\pgfsetroundjoin%
\definecolor{currentfill}{rgb}{0.000000,0.000000,0.000000}%
\pgfsetfillcolor{currentfill}%
\pgfsetlinewidth{0.803000pt}%
\definecolor{currentstroke}{rgb}{0.000000,0.000000,0.000000}%
\pgfsetstrokecolor{currentstroke}%
\pgfsetdash{}{0pt}%
\pgfsys@defobject{currentmarker}{\pgfqpoint{-0.048611in}{0.000000in}}{\pgfqpoint{-0.000000in}{0.000000in}}{%
\pgfpathmoveto{\pgfqpoint{-0.000000in}{0.000000in}}%
\pgfpathlineto{\pgfqpoint{-0.048611in}{0.000000in}}%
\pgfusepath{stroke,fill}%
}%
\begin{pgfscope}%
\pgfsys@transformshift{0.895256in}{0.290000in}%
\pgfsys@useobject{currentmarker}{}%
\end{pgfscope}%
\end{pgfscope}%
\begin{pgfscope}%
\definecolor{textcolor}{rgb}{0.000000,0.000000,0.000000}%
\pgfsetstrokecolor{textcolor}%
\pgfsetfillcolor{textcolor}%
\pgftext[x=0.512539in, y=0.241775in, left, base]{\color{textcolor}{\rmfamily\fontsize{10.000000}{12.000000}\selectfont\catcode`\^=\active\def^{\ifmmode\sp\else\^{}\fi}\catcode`\%=\active\def%{\%}$\mathdefault{\ensuremath{-}1.0}$}}%
\end{pgfscope}%
\begin{pgfscope}%
\pgfsetbuttcap%
\pgfsetroundjoin%
\definecolor{currentfill}{rgb}{0.000000,0.000000,0.000000}%
\pgfsetfillcolor{currentfill}%
\pgfsetlinewidth{0.803000pt}%
\definecolor{currentstroke}{rgb}{0.000000,0.000000,0.000000}%
\pgfsetstrokecolor{currentstroke}%
\pgfsetdash{}{0pt}%
\pgfsys@defobject{currentmarker}{\pgfqpoint{-0.048611in}{0.000000in}}{\pgfqpoint{-0.000000in}{0.000000in}}{%
\pgfpathmoveto{\pgfqpoint{-0.000000in}{0.000000in}}%
\pgfpathlineto{\pgfqpoint{-0.048611in}{0.000000in}}%
\pgfusepath{stroke,fill}%
}%
\begin{pgfscope}%
\pgfsys@transformshift{0.895256in}{0.640000in}%
\pgfsys@useobject{currentmarker}{}%
\end{pgfscope}%
\end{pgfscope}%
\begin{pgfscope}%
\definecolor{textcolor}{rgb}{0.000000,0.000000,0.000000}%
\pgfsetstrokecolor{textcolor}%
\pgfsetfillcolor{textcolor}%
\pgftext[x=0.512539in, y=0.591775in, left, base]{\color{textcolor}{\rmfamily\fontsize{10.000000}{12.000000}\selectfont\catcode`\^=\active\def^{\ifmmode\sp\else\^{}\fi}\catcode`\%=\active\def%{\%}$\mathdefault{\ensuremath{-}0.5}$}}%
\end{pgfscope}%
\begin{pgfscope}%
\pgfsetbuttcap%
\pgfsetroundjoin%
\definecolor{currentfill}{rgb}{0.000000,0.000000,0.000000}%
\pgfsetfillcolor{currentfill}%
\pgfsetlinewidth{0.803000pt}%
\definecolor{currentstroke}{rgb}{0.000000,0.000000,0.000000}%
\pgfsetstrokecolor{currentstroke}%
\pgfsetdash{}{0pt}%
\pgfsys@defobject{currentmarker}{\pgfqpoint{-0.048611in}{0.000000in}}{\pgfqpoint{-0.000000in}{0.000000in}}{%
\pgfpathmoveto{\pgfqpoint{-0.000000in}{0.000000in}}%
\pgfpathlineto{\pgfqpoint{-0.048611in}{0.000000in}}%
\pgfusepath{stroke,fill}%
}%
\begin{pgfscope}%
\pgfsys@transformshift{0.895256in}{0.990000in}%
\pgfsys@useobject{currentmarker}{}%
\end{pgfscope}%
\end{pgfscope}%
\begin{pgfscope}%
\definecolor{textcolor}{rgb}{0.000000,0.000000,0.000000}%
\pgfsetstrokecolor{textcolor}%
\pgfsetfillcolor{textcolor}%
\pgftext[x=0.620564in, y=0.941775in, left, base]{\color{textcolor}{\rmfamily\fontsize{10.000000}{12.000000}\selectfont\catcode`\^=\active\def^{\ifmmode\sp\else\^{}\fi}\catcode`\%=\active\def%{\%}$\mathdefault{0.0}$}}%
\end{pgfscope}%
\begin{pgfscope}%
\pgfsetbuttcap%
\pgfsetroundjoin%
\definecolor{currentfill}{rgb}{0.000000,0.000000,0.000000}%
\pgfsetfillcolor{currentfill}%
\pgfsetlinewidth{0.803000pt}%
\definecolor{currentstroke}{rgb}{0.000000,0.000000,0.000000}%
\pgfsetstrokecolor{currentstroke}%
\pgfsetdash{}{0pt}%
\pgfsys@defobject{currentmarker}{\pgfqpoint{-0.048611in}{0.000000in}}{\pgfqpoint{-0.000000in}{0.000000in}}{%
\pgfpathmoveto{\pgfqpoint{-0.000000in}{0.000000in}}%
\pgfpathlineto{\pgfqpoint{-0.048611in}{0.000000in}}%
\pgfusepath{stroke,fill}%
}%
\begin{pgfscope}%
\pgfsys@transformshift{0.895256in}{1.340000in}%
\pgfsys@useobject{currentmarker}{}%
\end{pgfscope}%
\end{pgfscope}%
\begin{pgfscope}%
\definecolor{textcolor}{rgb}{0.000000,0.000000,0.000000}%
\pgfsetstrokecolor{textcolor}%
\pgfsetfillcolor{textcolor}%
\pgftext[x=0.620564in, y=1.291775in, left, base]{\color{textcolor}{\rmfamily\fontsize{10.000000}{12.000000}\selectfont\catcode`\^=\active\def^{\ifmmode\sp\else\^{}\fi}\catcode`\%=\active\def%{\%}$\mathdefault{0.5}$}}%
\end{pgfscope}%
\begin{pgfscope}%
\pgfsetbuttcap%
\pgfsetroundjoin%
\definecolor{currentfill}{rgb}{0.000000,0.000000,0.000000}%
\pgfsetfillcolor{currentfill}%
\pgfsetlinewidth{0.803000pt}%
\definecolor{currentstroke}{rgb}{0.000000,0.000000,0.000000}%
\pgfsetstrokecolor{currentstroke}%
\pgfsetdash{}{0pt}%
\pgfsys@defobject{currentmarker}{\pgfqpoint{-0.048611in}{0.000000in}}{\pgfqpoint{-0.000000in}{0.000000in}}{%
\pgfpathmoveto{\pgfqpoint{-0.000000in}{0.000000in}}%
\pgfpathlineto{\pgfqpoint{-0.048611in}{0.000000in}}%
\pgfusepath{stroke,fill}%
}%
\begin{pgfscope}%
\pgfsys@transformshift{0.895256in}{1.690000in}%
\pgfsys@useobject{currentmarker}{}%
\end{pgfscope}%
\end{pgfscope}%
\begin{pgfscope}%
\definecolor{textcolor}{rgb}{0.000000,0.000000,0.000000}%
\pgfsetstrokecolor{textcolor}%
\pgfsetfillcolor{textcolor}%
\pgftext[x=0.620564in, y=1.641775in, left, base]{\color{textcolor}{\rmfamily\fontsize{10.000000}{12.000000}\selectfont\catcode`\^=\active\def^{\ifmmode\sp\else\^{}\fi}\catcode`\%=\active\def%{\%}$\mathdefault{1.0}$}}%
\end{pgfscope}%
\begin{pgfscope}%
\definecolor{textcolor}{rgb}{0.000000,0.000000,0.000000}%
\pgfsetstrokecolor{textcolor}%
\pgfsetfillcolor{textcolor}%
\pgftext[x=0.456984in,y=0.990000in,,bottom,rotate=90.000000]{\color{textcolor}{\rmfamily\fontsize{10.000000}{12.000000}\selectfont\catcode`\^=\active\def^{\ifmmode\sp\else\^{}\fi}\catcode`\%=\active\def%{\%}Amplitude}}%
\end{pgfscope}%
\begin{pgfscope}%
\pgfpathrectangle{\pgfqpoint{0.895256in}{0.220000in}}{\pgfqpoint{5.550589in}{1.540000in}}%
\pgfusepath{clip}%
\pgfsetrectcap%
\pgfsetroundjoin%
\pgfsetlinewidth{1.505625pt}%
\definecolor{currentstroke}{rgb}{0.121569,0.466667,0.705882}%
\pgfsetstrokecolor{currentstroke}%
\pgfsetdash{}{0pt}%
\pgfpathmoveto{\pgfqpoint{1.147556in}{0.990000in}}%
\pgfpathlineto{\pgfqpoint{1.189782in}{1.171173in}}%
\pgfpathlineto{\pgfqpoint{1.232007in}{1.340000in}}%
\pgfpathlineto{\pgfqpoint{1.253120in}{1.416133in}}%
\pgfpathlineto{\pgfqpoint{1.274233in}{1.484975in}}%
\pgfpathlineto{\pgfqpoint{1.295346in}{1.545347in}}%
\pgfpathlineto{\pgfqpoint{1.316459in}{1.596218in}}%
\pgfpathlineto{\pgfqpoint{1.337572in}{1.636716in}}%
\pgfpathlineto{\pgfqpoint{1.358685in}{1.666148in}}%
\pgfpathlineto{\pgfqpoint{1.379798in}{1.684011in}}%
\pgfpathlineto{\pgfqpoint{1.400911in}{1.690000in}}%
\pgfpathlineto{\pgfqpoint{1.422024in}{1.684011in}}%
\pgfpathlineto{\pgfqpoint{1.443137in}{1.666148in}}%
\pgfpathlineto{\pgfqpoint{1.464250in}{1.636716in}}%
\pgfpathlineto{\pgfqpoint{1.485363in}{1.596218in}}%
\pgfpathlineto{\pgfqpoint{1.506476in}{1.545347in}}%
\pgfpathlineto{\pgfqpoint{1.527588in}{1.484975in}}%
\pgfpathlineto{\pgfqpoint{1.548701in}{1.416133in}}%
\pgfpathlineto{\pgfqpoint{1.569814in}{1.340000in}}%
\pgfpathlineto{\pgfqpoint{1.590927in}{1.257878in}}%
\pgfpathlineto{\pgfqpoint{1.633153in}{1.081368in}}%
\pgfpathlineto{\pgfqpoint{1.696492in}{0.808827in}}%
\pgfpathlineto{\pgfqpoint{1.738718in}{0.640000in}}%
\pgfpathlineto{\pgfqpoint{1.759831in}{0.563867in}}%
\pgfpathlineto{\pgfqpoint{1.780944in}{0.495025in}}%
\pgfpathlineto{\pgfqpoint{1.802057in}{0.434653in}}%
\pgfpathlineto{\pgfqpoint{1.823169in}{0.383782in}}%
\pgfpathlineto{\pgfqpoint{1.844282in}{0.343284in}}%
\pgfpathlineto{\pgfqpoint{1.865395in}{0.313852in}}%
\pgfpathlineto{\pgfqpoint{1.886508in}{0.295989in}}%
\pgfpathlineto{\pgfqpoint{1.907621in}{0.290000in}}%
\pgfpathlineto{\pgfqpoint{1.928734in}{0.295989in}}%
\pgfpathlineto{\pgfqpoint{1.949847in}{0.313852in}}%
\pgfpathlineto{\pgfqpoint{1.970960in}{0.343284in}}%
\pgfpathlineto{\pgfqpoint{1.992073in}{0.383782in}}%
\pgfpathlineto{\pgfqpoint{2.013186in}{0.434653in}}%
\pgfpathlineto{\pgfqpoint{2.034299in}{0.495025in}}%
\pgfpathlineto{\pgfqpoint{2.055412in}{0.563867in}}%
\pgfpathlineto{\pgfqpoint{2.076525in}{0.640000in}}%
\pgfpathlineto{\pgfqpoint{2.097637in}{0.722122in}}%
\pgfpathlineto{\pgfqpoint{2.139863in}{0.898632in}}%
\pgfpathlineto{\pgfqpoint{2.160976in}{0.990000in}}%
\pgfpathlineto{\pgfqpoint{2.182089in}{1.171173in}}%
\pgfpathlineto{\pgfqpoint{2.203202in}{1.340000in}}%
\pgfpathlineto{\pgfqpoint{2.224315in}{1.484975in}}%
\pgfpathlineto{\pgfqpoint{2.245428in}{1.596218in}}%
\pgfpathlineto{\pgfqpoint{2.266541in}{1.666148in}}%
\pgfpathlineto{\pgfqpoint{2.287654in}{1.690000in}}%
\pgfpathlineto{\pgfqpoint{2.308767in}{1.666148in}}%
\pgfpathlineto{\pgfqpoint{2.329880in}{1.596218in}}%
\pgfpathlineto{\pgfqpoint{2.350993in}{1.484975in}}%
\pgfpathlineto{\pgfqpoint{2.372106in}{1.340000in}}%
\pgfpathlineto{\pgfqpoint{2.393218in}{1.171173in}}%
\pgfpathlineto{\pgfqpoint{2.435444in}{0.808827in}}%
\pgfpathlineto{\pgfqpoint{2.456557in}{0.640000in}}%
\pgfpathlineto{\pgfqpoint{2.477670in}{0.495025in}}%
\pgfpathlineto{\pgfqpoint{2.498783in}{0.383782in}}%
\pgfpathlineto{\pgfqpoint{2.519896in}{0.313852in}}%
\pgfpathlineto{\pgfqpoint{2.541009in}{0.290000in}}%
\pgfpathlineto{\pgfqpoint{2.562122in}{0.313852in}}%
\pgfpathlineto{\pgfqpoint{2.583235in}{0.383782in}}%
\pgfpathlineto{\pgfqpoint{2.604348in}{0.495025in}}%
\pgfpathlineto{\pgfqpoint{2.625461in}{0.640000in}}%
\pgfpathlineto{\pgfqpoint{2.646574in}{0.808827in}}%
\pgfpathlineto{\pgfqpoint{2.688799in}{1.171173in}}%
\pgfpathlineto{\pgfqpoint{2.709912in}{1.340000in}}%
\pgfpathlineto{\pgfqpoint{2.731025in}{1.484975in}}%
\pgfpathlineto{\pgfqpoint{2.752138in}{1.596218in}}%
\pgfpathlineto{\pgfqpoint{2.773251in}{1.666148in}}%
\pgfpathlineto{\pgfqpoint{2.794364in}{1.690000in}}%
\pgfpathlineto{\pgfqpoint{2.815477in}{1.666148in}}%
\pgfpathlineto{\pgfqpoint{2.836590in}{1.596218in}}%
\pgfpathlineto{\pgfqpoint{2.857703in}{1.484975in}}%
\pgfpathlineto{\pgfqpoint{2.878816in}{1.340000in}}%
\pgfpathlineto{\pgfqpoint{2.899929in}{1.171173in}}%
\pgfpathlineto{\pgfqpoint{2.942155in}{0.808827in}}%
\pgfpathlineto{\pgfqpoint{2.963268in}{0.640000in}}%
\pgfpathlineto{\pgfqpoint{2.984380in}{0.495025in}}%
\pgfpathlineto{\pgfqpoint{3.005493in}{0.383782in}}%
\pgfpathlineto{\pgfqpoint{3.026606in}{0.313852in}}%
\pgfpathlineto{\pgfqpoint{3.047719in}{0.290000in}}%
\pgfpathlineto{\pgfqpoint{3.068832in}{0.313852in}}%
\pgfpathlineto{\pgfqpoint{3.089945in}{0.383782in}}%
\pgfpathlineto{\pgfqpoint{3.111058in}{0.495025in}}%
\pgfpathlineto{\pgfqpoint{3.132171in}{0.640000in}}%
\pgfpathlineto{\pgfqpoint{3.153284in}{0.808827in}}%
\pgfpathlineto{\pgfqpoint{3.195510in}{1.171173in}}%
\pgfpathlineto{\pgfqpoint{3.216623in}{1.340000in}}%
\pgfpathlineto{\pgfqpoint{3.237736in}{1.484975in}}%
\pgfpathlineto{\pgfqpoint{3.258849in}{1.596218in}}%
\pgfpathlineto{\pgfqpoint{3.279961in}{1.666148in}}%
\pgfpathlineto{\pgfqpoint{3.301074in}{1.690000in}}%
\pgfpathlineto{\pgfqpoint{3.322187in}{1.666148in}}%
\pgfpathlineto{\pgfqpoint{3.343300in}{1.596218in}}%
\pgfpathlineto{\pgfqpoint{3.364413in}{1.484975in}}%
\pgfpathlineto{\pgfqpoint{3.385526in}{1.340000in}}%
\pgfpathlineto{\pgfqpoint{3.406639in}{1.171173in}}%
\pgfpathlineto{\pgfqpoint{3.448865in}{0.808827in}}%
\pgfpathlineto{\pgfqpoint{3.469978in}{0.640000in}}%
\pgfpathlineto{\pgfqpoint{3.491091in}{0.495025in}}%
\pgfpathlineto{\pgfqpoint{3.512204in}{0.383782in}}%
\pgfpathlineto{\pgfqpoint{3.533317in}{0.313852in}}%
\pgfpathlineto{\pgfqpoint{3.554430in}{0.290000in}}%
\pgfpathlineto{\pgfqpoint{3.575542in}{0.313852in}}%
\pgfpathlineto{\pgfqpoint{3.596655in}{0.383782in}}%
\pgfpathlineto{\pgfqpoint{3.617768in}{0.495025in}}%
\pgfpathlineto{\pgfqpoint{3.638881in}{0.640000in}}%
\pgfpathlineto{\pgfqpoint{3.659994in}{0.808827in}}%
\pgfpathlineto{\pgfqpoint{3.702220in}{1.171173in}}%
\pgfpathlineto{\pgfqpoint{3.723333in}{1.340000in}}%
\pgfpathlineto{\pgfqpoint{3.744446in}{1.484975in}}%
\pgfpathlineto{\pgfqpoint{3.765559in}{1.596218in}}%
\pgfpathlineto{\pgfqpoint{3.786672in}{1.666148in}}%
\pgfpathlineto{\pgfqpoint{3.807785in}{1.690000in}}%
\pgfpathlineto{\pgfqpoint{3.828898in}{1.666148in}}%
\pgfpathlineto{\pgfqpoint{3.850011in}{1.596218in}}%
\pgfpathlineto{\pgfqpoint{3.871123in}{1.484975in}}%
\pgfpathlineto{\pgfqpoint{3.892236in}{1.340000in}}%
\pgfpathlineto{\pgfqpoint{3.913349in}{1.171173in}}%
\pgfpathlineto{\pgfqpoint{3.955575in}{0.808827in}}%
\pgfpathlineto{\pgfqpoint{3.976688in}{0.640000in}}%
\pgfpathlineto{\pgfqpoint{3.997801in}{0.495025in}}%
\pgfpathlineto{\pgfqpoint{4.018914in}{0.383782in}}%
\pgfpathlineto{\pgfqpoint{4.040027in}{0.313852in}}%
\pgfpathlineto{\pgfqpoint{4.061140in}{0.290000in}}%
\pgfpathlineto{\pgfqpoint{4.082253in}{0.313852in}}%
\pgfpathlineto{\pgfqpoint{4.103366in}{0.383782in}}%
\pgfpathlineto{\pgfqpoint{4.124479in}{0.495025in}}%
\pgfpathlineto{\pgfqpoint{4.145592in}{0.640000in}}%
\pgfpathlineto{\pgfqpoint{4.166704in}{0.808827in}}%
\pgfpathlineto{\pgfqpoint{4.187817in}{0.990000in}}%
\pgfpathlineto{\pgfqpoint{4.230043in}{1.171173in}}%
\pgfpathlineto{\pgfqpoint{4.272269in}{1.340000in}}%
\pgfpathlineto{\pgfqpoint{4.293382in}{1.416133in}}%
\pgfpathlineto{\pgfqpoint{4.314495in}{1.484975in}}%
\pgfpathlineto{\pgfqpoint{4.335608in}{1.545347in}}%
\pgfpathlineto{\pgfqpoint{4.356721in}{1.596218in}}%
\pgfpathlineto{\pgfqpoint{4.377834in}{1.636716in}}%
\pgfpathlineto{\pgfqpoint{4.398947in}{1.666148in}}%
\pgfpathlineto{\pgfqpoint{4.420060in}{1.684011in}}%
\pgfpathlineto{\pgfqpoint{4.441172in}{1.690000in}}%
\pgfpathlineto{\pgfqpoint{4.462285in}{1.684011in}}%
\pgfpathlineto{\pgfqpoint{4.483398in}{1.666148in}}%
\pgfpathlineto{\pgfqpoint{4.504511in}{1.636716in}}%
\pgfpathlineto{\pgfqpoint{4.525624in}{1.596218in}}%
\pgfpathlineto{\pgfqpoint{4.546737in}{1.545347in}}%
\pgfpathlineto{\pgfqpoint{4.567850in}{1.484975in}}%
\pgfpathlineto{\pgfqpoint{4.588963in}{1.416133in}}%
\pgfpathlineto{\pgfqpoint{4.610076in}{1.340000in}}%
\pgfpathlineto{\pgfqpoint{4.631189in}{1.257878in}}%
\pgfpathlineto{\pgfqpoint{4.673415in}{1.081368in}}%
\pgfpathlineto{\pgfqpoint{4.736753in}{0.808827in}}%
\pgfpathlineto{\pgfqpoint{4.778979in}{0.640000in}}%
\pgfpathlineto{\pgfqpoint{4.800092in}{0.563867in}}%
\pgfpathlineto{\pgfqpoint{4.821205in}{0.495025in}}%
\pgfpathlineto{\pgfqpoint{4.842318in}{0.434653in}}%
\pgfpathlineto{\pgfqpoint{4.863431in}{0.383782in}}%
\pgfpathlineto{\pgfqpoint{4.884544in}{0.343284in}}%
\pgfpathlineto{\pgfqpoint{4.905657in}{0.313852in}}%
\pgfpathlineto{\pgfqpoint{4.926770in}{0.295989in}}%
\pgfpathlineto{\pgfqpoint{4.947883in}{0.290000in}}%
\pgfpathlineto{\pgfqpoint{4.968996in}{0.295989in}}%
\pgfpathlineto{\pgfqpoint{4.990109in}{0.313852in}}%
\pgfpathlineto{\pgfqpoint{5.011222in}{0.343284in}}%
\pgfpathlineto{\pgfqpoint{5.032334in}{0.383782in}}%
\pgfpathlineto{\pgfqpoint{5.053447in}{0.434653in}}%
\pgfpathlineto{\pgfqpoint{5.074560in}{0.495025in}}%
\pgfpathlineto{\pgfqpoint{5.095673in}{0.563867in}}%
\pgfpathlineto{\pgfqpoint{5.116786in}{0.640000in}}%
\pgfpathlineto{\pgfqpoint{5.137899in}{0.722122in}}%
\pgfpathlineto{\pgfqpoint{5.180125in}{0.898632in}}%
\pgfpathlineto{\pgfqpoint{5.243464in}{1.171173in}}%
\pgfpathlineto{\pgfqpoint{5.285690in}{1.340000in}}%
\pgfpathlineto{\pgfqpoint{5.306803in}{1.416133in}}%
\pgfpathlineto{\pgfqpoint{5.327915in}{1.484975in}}%
\pgfpathlineto{\pgfqpoint{5.349028in}{1.545347in}}%
\pgfpathlineto{\pgfqpoint{5.370141in}{1.596218in}}%
\pgfpathlineto{\pgfqpoint{5.391254in}{1.636716in}}%
\pgfpathlineto{\pgfqpoint{5.412367in}{1.666148in}}%
\pgfpathlineto{\pgfqpoint{5.433480in}{1.684011in}}%
\pgfpathlineto{\pgfqpoint{5.454593in}{1.690000in}}%
\pgfpathlineto{\pgfqpoint{5.475706in}{1.684011in}}%
\pgfpathlineto{\pgfqpoint{5.496819in}{1.666148in}}%
\pgfpathlineto{\pgfqpoint{5.517932in}{1.636716in}}%
\pgfpathlineto{\pgfqpoint{5.539045in}{1.596218in}}%
\pgfpathlineto{\pgfqpoint{5.560158in}{1.545347in}}%
\pgfpathlineto{\pgfqpoint{5.581271in}{1.484975in}}%
\pgfpathlineto{\pgfqpoint{5.602384in}{1.416133in}}%
\pgfpathlineto{\pgfqpoint{5.623496in}{1.340000in}}%
\pgfpathlineto{\pgfqpoint{5.644609in}{1.257878in}}%
\pgfpathlineto{\pgfqpoint{5.686835in}{1.081368in}}%
\pgfpathlineto{\pgfqpoint{5.750174in}{0.808827in}}%
\pgfpathlineto{\pgfqpoint{5.792400in}{0.640000in}}%
\pgfpathlineto{\pgfqpoint{5.813513in}{0.563867in}}%
\pgfpathlineto{\pgfqpoint{5.834626in}{0.495025in}}%
\pgfpathlineto{\pgfqpoint{5.855739in}{0.434653in}}%
\pgfpathlineto{\pgfqpoint{5.876852in}{0.383782in}}%
\pgfpathlineto{\pgfqpoint{5.897965in}{0.343284in}}%
\pgfpathlineto{\pgfqpoint{5.919077in}{0.313852in}}%
\pgfpathlineto{\pgfqpoint{5.940190in}{0.295989in}}%
\pgfpathlineto{\pgfqpoint{5.961303in}{0.290000in}}%
\pgfpathlineto{\pgfqpoint{5.982416in}{0.295989in}}%
\pgfpathlineto{\pgfqpoint{6.003529in}{0.313852in}}%
\pgfpathlineto{\pgfqpoint{6.024642in}{0.343284in}}%
\pgfpathlineto{\pgfqpoint{6.045755in}{0.383782in}}%
\pgfpathlineto{\pgfqpoint{6.066868in}{0.434653in}}%
\pgfpathlineto{\pgfqpoint{6.087981in}{0.495025in}}%
\pgfpathlineto{\pgfqpoint{6.109094in}{0.563867in}}%
\pgfpathlineto{\pgfqpoint{6.130207in}{0.640000in}}%
\pgfpathlineto{\pgfqpoint{6.151320in}{0.722122in}}%
\pgfpathlineto{\pgfqpoint{6.193546in}{0.898632in}}%
\pgfpathlineto{\pgfqpoint{6.193546in}{0.898632in}}%
\pgfusepath{stroke}%
\end{pgfscope}%
\begin{pgfscope}%
\pgfpathrectangle{\pgfqpoint{0.895256in}{0.220000in}}{\pgfqpoint{5.550589in}{1.540000in}}%
\pgfusepath{clip}%
\pgfsetrectcap%
\pgfsetroundjoin%
\pgfsetlinewidth{1.505625pt}%
\definecolor{currentstroke}{rgb}{1.000000,0.498039,0.054902}%
\pgfsetstrokecolor{currentstroke}%
\pgfsetdash{}{0pt}%
\pgfpathmoveto{\pgfqpoint{1.147556in}{0.992478in}}%
\pgfpathlineto{\pgfqpoint{1.168669in}{1.003757in}}%
\pgfpathlineto{\pgfqpoint{1.189782in}{1.009995in}}%
\pgfpathlineto{\pgfqpoint{1.232007in}{1.010593in}}%
\pgfpathlineto{\pgfqpoint{1.253120in}{1.016895in}}%
\pgfpathlineto{\pgfqpoint{1.274233in}{1.029862in}}%
\pgfpathlineto{\pgfqpoint{1.295346in}{1.039966in}}%
\pgfpathlineto{\pgfqpoint{1.316459in}{1.043683in}}%
\pgfpathlineto{\pgfqpoint{1.337572in}{1.043769in}}%
\pgfpathlineto{\pgfqpoint{1.358685in}{1.037659in}}%
\pgfpathlineto{\pgfqpoint{1.379798in}{1.023688in}}%
\pgfpathlineto{\pgfqpoint{1.400911in}{1.004847in}}%
\pgfpathlineto{\pgfqpoint{1.422024in}{0.988270in}}%
\pgfpathlineto{\pgfqpoint{1.443137in}{0.981541in}}%
\pgfpathlineto{\pgfqpoint{1.485363in}{0.984339in}}%
\pgfpathlineto{\pgfqpoint{1.506476in}{0.979554in}}%
\pgfpathlineto{\pgfqpoint{1.527588in}{0.968488in}}%
\pgfpathlineto{\pgfqpoint{1.569814in}{0.939991in}}%
\pgfpathlineto{\pgfqpoint{1.590927in}{0.935953in}}%
\pgfpathlineto{\pgfqpoint{1.633153in}{0.948279in}}%
\pgfpathlineto{\pgfqpoint{1.654266in}{0.950522in}}%
\pgfpathlineto{\pgfqpoint{1.675379in}{0.950757in}}%
\pgfpathlineto{\pgfqpoint{1.696492in}{0.945908in}}%
\pgfpathlineto{\pgfqpoint{1.738718in}{0.923221in}}%
\pgfpathlineto{\pgfqpoint{1.759831in}{0.921448in}}%
\pgfpathlineto{\pgfqpoint{1.780944in}{0.929865in}}%
\pgfpathlineto{\pgfqpoint{1.802057in}{0.941807in}}%
\pgfpathlineto{\pgfqpoint{1.823169in}{0.950544in}}%
\pgfpathlineto{\pgfqpoint{1.844282in}{0.952637in}}%
\pgfpathlineto{\pgfqpoint{1.865395in}{0.950031in}}%
\pgfpathlineto{\pgfqpoint{1.886508in}{0.951954in}}%
\pgfpathlineto{\pgfqpoint{1.928734in}{0.971693in}}%
\pgfpathlineto{\pgfqpoint{1.970960in}{0.978849in}}%
\pgfpathlineto{\pgfqpoint{1.992073in}{0.987180in}}%
\pgfpathlineto{\pgfqpoint{2.013186in}{1.000104in}}%
\pgfpathlineto{\pgfqpoint{2.034299in}{1.011042in}}%
\pgfpathlineto{\pgfqpoint{2.055412in}{1.014417in}}%
\pgfpathlineto{\pgfqpoint{2.076525in}{1.013178in}}%
\pgfpathlineto{\pgfqpoint{2.097637in}{1.016724in}}%
\pgfpathlineto{\pgfqpoint{2.139863in}{1.040137in}}%
\pgfpathlineto{\pgfqpoint{2.160976in}{1.041440in}}%
\pgfpathlineto{\pgfqpoint{2.182089in}{1.032938in}}%
\pgfpathlineto{\pgfqpoint{2.203202in}{1.028730in}}%
\pgfpathlineto{\pgfqpoint{2.224315in}{1.042338in}}%
\pgfpathlineto{\pgfqpoint{2.245428in}{1.067609in}}%
\pgfpathlineto{\pgfqpoint{2.266541in}{1.081559in}}%
\pgfpathlineto{\pgfqpoint{2.287654in}{1.072031in}}%
\pgfpathlineto{\pgfqpoint{2.329880in}{1.026188in}}%
\pgfpathlineto{\pgfqpoint{2.350993in}{1.014289in}}%
\pgfpathlineto{\pgfqpoint{2.372106in}{1.004932in}}%
\pgfpathlineto{\pgfqpoint{2.393218in}{0.980430in}}%
\pgfpathlineto{\pgfqpoint{2.414331in}{0.941593in}}%
\pgfpathlineto{\pgfqpoint{2.435444in}{0.910212in}}%
\pgfpathlineto{\pgfqpoint{2.456557in}{0.898804in}}%
\pgfpathlineto{\pgfqpoint{2.477670in}{0.897651in}}%
\pgfpathlineto{\pgfqpoint{2.519896in}{0.891498in}}%
\pgfpathlineto{\pgfqpoint{2.541009in}{0.899531in}}%
\pgfpathlineto{\pgfqpoint{2.562122in}{0.920402in}}%
\pgfpathlineto{\pgfqpoint{2.583235in}{0.944584in}}%
\pgfpathlineto{\pgfqpoint{2.604348in}{0.966651in}}%
\pgfpathlineto{\pgfqpoint{2.625461in}{0.992136in}}%
\pgfpathlineto{\pgfqpoint{2.667687in}{1.049921in}}%
\pgfpathlineto{\pgfqpoint{2.688799in}{1.075599in}}%
\pgfpathlineto{\pgfqpoint{2.709912in}{1.092176in}}%
\pgfpathlineto{\pgfqpoint{2.731025in}{1.095508in}}%
\pgfpathlineto{\pgfqpoint{2.752138in}{1.091620in}}%
\pgfpathlineto{\pgfqpoint{2.794364in}{1.086942in}}%
\pgfpathlineto{\pgfqpoint{2.815477in}{1.070835in}}%
\pgfpathlineto{\pgfqpoint{2.857703in}{1.011448in}}%
\pgfpathlineto{\pgfqpoint{2.921042in}{0.937427in}}%
\pgfpathlineto{\pgfqpoint{2.942155in}{0.912647in}}%
\pgfpathlineto{\pgfqpoint{2.963268in}{0.894105in}}%
\pgfpathlineto{\pgfqpoint{2.984380in}{0.886072in}}%
\pgfpathlineto{\pgfqpoint{3.005493in}{0.888251in}}%
\pgfpathlineto{\pgfqpoint{3.026606in}{0.896989in}}%
\pgfpathlineto{\pgfqpoint{3.047719in}{0.910062in}}%
\pgfpathlineto{\pgfqpoint{3.068832in}{0.927494in}}%
\pgfpathlineto{\pgfqpoint{3.111058in}{0.968403in}}%
\pgfpathlineto{\pgfqpoint{3.132171in}{0.994465in}}%
\pgfpathlineto{\pgfqpoint{3.174397in}{1.063636in}}%
\pgfpathlineto{\pgfqpoint{3.195510in}{1.087369in}}%
\pgfpathlineto{\pgfqpoint{3.216623in}{1.095508in}}%
\pgfpathlineto{\pgfqpoint{3.258849in}{1.092945in}}%
\pgfpathlineto{\pgfqpoint{3.279961in}{1.096790in}}%
\pgfpathlineto{\pgfqpoint{3.301074in}{1.099204in}}%
\pgfpathlineto{\pgfqpoint{3.322187in}{1.088843in}}%
\pgfpathlineto{\pgfqpoint{3.343300in}{1.062269in}}%
\pgfpathlineto{\pgfqpoint{3.364413in}{1.028474in}}%
\pgfpathlineto{\pgfqpoint{3.385526in}{0.997733in}}%
\pgfpathlineto{\pgfqpoint{3.406639in}{0.973167in}}%
\pgfpathlineto{\pgfqpoint{3.448865in}{0.928690in}}%
\pgfpathlineto{\pgfqpoint{3.469978in}{0.904166in}}%
\pgfpathlineto{\pgfqpoint{3.491091in}{0.885026in}}%
\pgfpathlineto{\pgfqpoint{3.512204in}{0.877698in}}%
\pgfpathlineto{\pgfqpoint{3.533317in}{0.881180in}}%
\pgfpathlineto{\pgfqpoint{3.554430in}{0.888721in}}%
\pgfpathlineto{\pgfqpoint{3.575542in}{0.898228in}}%
\pgfpathlineto{\pgfqpoint{3.596655in}{0.915018in}}%
\pgfpathlineto{\pgfqpoint{3.617768in}{0.940269in}}%
\pgfpathlineto{\pgfqpoint{3.638881in}{0.971351in}}%
\pgfpathlineto{\pgfqpoint{3.659994in}{1.005317in}}%
\pgfpathlineto{\pgfqpoint{3.681107in}{1.035309in}}%
\pgfpathlineto{\pgfqpoint{3.702220in}{1.058359in}}%
\pgfpathlineto{\pgfqpoint{3.744446in}{1.099738in}}%
\pgfpathlineto{\pgfqpoint{3.765559in}{1.115397in}}%
\pgfpathlineto{\pgfqpoint{3.786672in}{1.118195in}}%
\pgfpathlineto{\pgfqpoint{3.807785in}{1.108625in}}%
\pgfpathlineto{\pgfqpoint{3.828898in}{1.093949in}}%
\pgfpathlineto{\pgfqpoint{3.850011in}{1.076710in}}%
\pgfpathlineto{\pgfqpoint{3.871123in}{1.052720in}}%
\pgfpathlineto{\pgfqpoint{3.892236in}{1.021232in}}%
\pgfpathlineto{\pgfqpoint{3.913349in}{0.987052in}}%
\pgfpathlineto{\pgfqpoint{3.934462in}{0.955393in}}%
\pgfpathlineto{\pgfqpoint{3.955575in}{0.929374in}}%
\pgfpathlineto{\pgfqpoint{3.976688in}{0.910084in}}%
\pgfpathlineto{\pgfqpoint{3.997801in}{0.896732in}}%
\pgfpathlineto{\pgfqpoint{4.018914in}{0.887183in}}%
\pgfpathlineto{\pgfqpoint{4.040027in}{0.880198in}}%
\pgfpathlineto{\pgfqpoint{4.061140in}{0.878083in}}%
\pgfpathlineto{\pgfqpoint{4.082253in}{0.887397in}}%
\pgfpathlineto{\pgfqpoint{4.103366in}{0.908353in}}%
\pgfpathlineto{\pgfqpoint{4.145592in}{0.957252in}}%
\pgfpathlineto{\pgfqpoint{4.187817in}{1.020014in}}%
\pgfpathlineto{\pgfqpoint{4.208930in}{1.042872in}}%
\pgfpathlineto{\pgfqpoint{4.230043in}{1.054365in}}%
\pgfpathlineto{\pgfqpoint{4.272269in}{1.070408in}}%
\pgfpathlineto{\pgfqpoint{4.293382in}{1.072544in}}%
\pgfpathlineto{\pgfqpoint{4.314495in}{1.068079in}}%
\pgfpathlineto{\pgfqpoint{4.335608in}{1.059919in}}%
\pgfpathlineto{\pgfqpoint{4.356721in}{1.048191in}}%
\pgfpathlineto{\pgfqpoint{4.377834in}{1.034455in}}%
\pgfpathlineto{\pgfqpoint{4.398947in}{1.022385in}}%
\pgfpathlineto{\pgfqpoint{4.420060in}{1.015571in}}%
\pgfpathlineto{\pgfqpoint{4.441172in}{1.014631in}}%
\pgfpathlineto{\pgfqpoint{4.462285in}{1.015037in}}%
\pgfpathlineto{\pgfqpoint{4.483398in}{1.011982in}}%
\pgfpathlineto{\pgfqpoint{4.504511in}{1.003971in}}%
\pgfpathlineto{\pgfqpoint{4.525624in}{0.993610in}}%
\pgfpathlineto{\pgfqpoint{4.546737in}{0.986198in}}%
\pgfpathlineto{\pgfqpoint{4.567850in}{0.981006in}}%
\pgfpathlineto{\pgfqpoint{4.588963in}{0.973145in}}%
\pgfpathlineto{\pgfqpoint{4.631189in}{0.949454in}}%
\pgfpathlineto{\pgfqpoint{4.652302in}{0.941422in}}%
\pgfpathlineto{\pgfqpoint{4.694528in}{0.934565in}}%
\pgfpathlineto{\pgfqpoint{4.736753in}{0.920679in}}%
\pgfpathlineto{\pgfqpoint{4.757866in}{0.918415in}}%
\pgfpathlineto{\pgfqpoint{4.863431in}{0.933860in}}%
\pgfpathlineto{\pgfqpoint{4.884544in}{0.939329in}}%
\pgfpathlineto{\pgfqpoint{4.926770in}{0.945267in}}%
\pgfpathlineto{\pgfqpoint{4.947883in}{0.954261in}}%
\pgfpathlineto{\pgfqpoint{4.968996in}{0.964579in}}%
\pgfpathlineto{\pgfqpoint{4.990109in}{0.970859in}}%
\pgfpathlineto{\pgfqpoint{5.011222in}{0.973936in}}%
\pgfpathlineto{\pgfqpoint{5.032334in}{0.978144in}}%
\pgfpathlineto{\pgfqpoint{5.074560in}{0.992072in}}%
\pgfpathlineto{\pgfqpoint{5.095673in}{0.995853in}}%
\pgfpathlineto{\pgfqpoint{5.137899in}{1.001963in}}%
\pgfpathlineto{\pgfqpoint{5.201238in}{1.012623in}}%
\pgfpathlineto{\pgfqpoint{5.243464in}{1.025803in}}%
\pgfpathlineto{\pgfqpoint{5.264577in}{1.029243in}}%
\pgfpathlineto{\pgfqpoint{5.306803in}{1.030909in}}%
\pgfpathlineto{\pgfqpoint{5.327915in}{1.034775in}}%
\pgfpathlineto{\pgfqpoint{5.349028in}{1.040116in}}%
\pgfpathlineto{\pgfqpoint{5.370141in}{1.043726in}}%
\pgfpathlineto{\pgfqpoint{5.391254in}{1.043854in}}%
\pgfpathlineto{\pgfqpoint{5.433480in}{1.036848in}}%
\pgfpathlineto{\pgfqpoint{5.496819in}{1.024778in}}%
\pgfpathlineto{\pgfqpoint{5.517932in}{1.021808in}}%
\pgfpathlineto{\pgfqpoint{5.539045in}{1.015549in}}%
\pgfpathlineto{\pgfqpoint{5.581271in}{0.995298in}}%
\pgfpathlineto{\pgfqpoint{5.602384in}{0.987565in}}%
\pgfpathlineto{\pgfqpoint{5.644609in}{0.975217in}}%
\pgfpathlineto{\pgfqpoint{5.686835in}{0.957337in}}%
\pgfpathlineto{\pgfqpoint{5.729061in}{0.939521in}}%
\pgfpathlineto{\pgfqpoint{5.750174in}{0.933027in}}%
\pgfpathlineto{\pgfqpoint{5.771287in}{0.929587in}}%
\pgfpathlineto{\pgfqpoint{5.813513in}{0.929716in}}%
\pgfpathlineto{\pgfqpoint{5.855739in}{0.930805in}}%
\pgfpathlineto{\pgfqpoint{5.876852in}{0.932792in}}%
\pgfpathlineto{\pgfqpoint{5.897965in}{0.937855in}}%
\pgfpathlineto{\pgfqpoint{5.940190in}{0.953000in}}%
\pgfpathlineto{\pgfqpoint{6.003529in}{0.977503in}}%
\pgfpathlineto{\pgfqpoint{6.045755in}{0.986945in}}%
\pgfpathlineto{\pgfqpoint{6.109094in}{1.006684in}}%
\pgfpathlineto{\pgfqpoint{6.193546in}{1.024030in}}%
\pgfpathlineto{\pgfqpoint{6.193546in}{1.024030in}}%
\pgfusepath{stroke}%
\end{pgfscope}%
\begin{pgfscope}%
\pgfpathrectangle{\pgfqpoint{0.895256in}{0.220000in}}{\pgfqpoint{5.550589in}{1.540000in}}%
\pgfusepath{clip}%
\pgfsetbuttcap%
\pgfsetroundjoin%
\pgfsetlinewidth{1.505625pt}%
\definecolor{currentstroke}{rgb}{0.501961,0.501961,0.501961}%
\pgfsetstrokecolor{currentstroke}%
\pgfsetdash{{5.550000pt}{2.400000pt}}{0.000000pt}%
\pgfpathmoveto{\pgfqpoint{2.160976in}{0.220000in}}%
\pgfpathlineto{\pgfqpoint{2.160976in}{1.760000in}}%
\pgfusepath{stroke}%
\end{pgfscope}%
\begin{pgfscope}%
\pgfpathrectangle{\pgfqpoint{0.895256in}{0.220000in}}{\pgfqpoint{5.550589in}{1.540000in}}%
\pgfusepath{clip}%
\pgfsetbuttcap%
\pgfsetroundjoin%
\pgfsetlinewidth{1.505625pt}%
\definecolor{currentstroke}{rgb}{0.501961,0.501961,0.501961}%
\pgfsetstrokecolor{currentstroke}%
\pgfsetdash{{5.550000pt}{2.400000pt}}{0.000000pt}%
\pgfpathmoveto{\pgfqpoint{3.174397in}{0.220000in}}%
\pgfpathlineto{\pgfqpoint{3.174397in}{1.760000in}}%
\pgfusepath{stroke}%
\end{pgfscope}%
\begin{pgfscope}%
\pgfpathrectangle{\pgfqpoint{0.895256in}{0.220000in}}{\pgfqpoint{5.550589in}{1.540000in}}%
\pgfusepath{clip}%
\pgfsetbuttcap%
\pgfsetroundjoin%
\pgfsetlinewidth{1.505625pt}%
\definecolor{currentstroke}{rgb}{0.501961,0.501961,0.501961}%
\pgfsetstrokecolor{currentstroke}%
\pgfsetdash{{5.550000pt}{2.400000pt}}{0.000000pt}%
\pgfpathmoveto{\pgfqpoint{4.187817in}{0.220000in}}%
\pgfpathlineto{\pgfqpoint{4.187817in}{1.760000in}}%
\pgfusepath{stroke}%
\end{pgfscope}%
\begin{pgfscope}%
\pgfpathrectangle{\pgfqpoint{0.895256in}{0.220000in}}{\pgfqpoint{5.550589in}{1.540000in}}%
\pgfusepath{clip}%
\pgfsetbuttcap%
\pgfsetroundjoin%
\pgfsetlinewidth{1.505625pt}%
\definecolor{currentstroke}{rgb}{0.501961,0.501961,0.501961}%
\pgfsetstrokecolor{currentstroke}%
\pgfsetdash{{5.550000pt}{2.400000pt}}{0.000000pt}%
\pgfpathmoveto{\pgfqpoint{5.201238in}{0.220000in}}%
\pgfpathlineto{\pgfqpoint{5.201238in}{1.760000in}}%
\pgfusepath{stroke}%
\end{pgfscope}%
\begin{pgfscope}%
\pgfsetrectcap%
\pgfsetmiterjoin%
\pgfsetlinewidth{0.803000pt}%
\definecolor{currentstroke}{rgb}{0.000000,0.000000,0.000000}%
\pgfsetstrokecolor{currentstroke}%
\pgfsetdash{}{0pt}%
\pgfpathmoveto{\pgfqpoint{0.895256in}{0.220000in}}%
\pgfpathlineto{\pgfqpoint{0.895256in}{1.760000in}}%
\pgfusepath{stroke}%
\end{pgfscope}%
\begin{pgfscope}%
\pgfsetrectcap%
\pgfsetmiterjoin%
\pgfsetlinewidth{0.803000pt}%
\definecolor{currentstroke}{rgb}{0.000000,0.000000,0.000000}%
\pgfsetstrokecolor{currentstroke}%
\pgfsetdash{}{0pt}%
\pgfpathmoveto{\pgfqpoint{6.445845in}{0.220000in}}%
\pgfpathlineto{\pgfqpoint{6.445845in}{1.760000in}}%
\pgfusepath{stroke}%
\end{pgfscope}%
\begin{pgfscope}%
\pgfsetrectcap%
\pgfsetmiterjoin%
\pgfsetlinewidth{0.803000pt}%
\definecolor{currentstroke}{rgb}{0.000000,0.000000,0.000000}%
\pgfsetstrokecolor{currentstroke}%
\pgfsetdash{}{0pt}%
\pgfpathmoveto{\pgfqpoint{0.895256in}{0.220000in}}%
\pgfpathlineto{\pgfqpoint{6.445845in}{0.220000in}}%
\pgfusepath{stroke}%
\end{pgfscope}%
\begin{pgfscope}%
\pgfsetrectcap%
\pgfsetmiterjoin%
\pgfsetlinewidth{0.803000pt}%
\definecolor{currentstroke}{rgb}{0.000000,0.000000,0.000000}%
\pgfsetstrokecolor{currentstroke}%
\pgfsetdash{}{0pt}%
\pgfpathmoveto{\pgfqpoint{0.895256in}{1.760000in}}%
\pgfpathlineto{\pgfqpoint{6.445845in}{1.760000in}}%
\pgfusepath{stroke}%
\end{pgfscope}%
\begin{pgfscope}%
\definecolor{textcolor}{rgb}{0.000000,0.000000,0.000000}%
\pgfsetstrokecolor{textcolor}%
\pgfsetfillcolor{textcolor}%
\pgftext[x=3.670551in,y=1.843333in,,base]{\color{textcolor}{\rmfamily\fontsize{12.000000}{14.400000}\selectfont\catcode`\^=\active\def^{\ifmmode\sp\else\^{}\fi}\catcode`\%=\active\def%{\%}FSK via Air (1000Hz and 2000Hz)}}%
\end{pgfscope}%
\begin{pgfscope}%
\pgfsetbuttcap%
\pgfsetmiterjoin%
\definecolor{currentfill}{rgb}{1.000000,1.000000,1.000000}%
\pgfsetfillcolor{currentfill}%
\pgfsetfillopacity{0.800000}%
\pgfsetlinewidth{1.003750pt}%
\definecolor{currentstroke}{rgb}{0.800000,0.800000,0.800000}%
\pgfsetstrokecolor{currentstroke}%
\pgfsetstrokeopacity{0.800000}%
\pgfsetdash{}{0pt}%
\pgfpathmoveto{\pgfqpoint{0.992478in}{0.289444in}}%
\pgfpathlineto{\pgfqpoint{1.986693in}{0.289444in}}%
\pgfpathquadraticcurveto{\pgfqpoint{2.014471in}{0.289444in}}{\pgfqpoint{2.014471in}{0.317222in}}%
\pgfpathlineto{\pgfqpoint{2.014471in}{0.690679in}}%
\pgfpathquadraticcurveto{\pgfqpoint{2.014471in}{0.718457in}}{\pgfqpoint{1.986693in}{0.718457in}}%
\pgfpathlineto{\pgfqpoint{0.992478in}{0.718457in}}%
\pgfpathquadraticcurveto{\pgfqpoint{0.964701in}{0.718457in}}{\pgfqpoint{0.964701in}{0.690679in}}%
\pgfpathlineto{\pgfqpoint{0.964701in}{0.317222in}}%
\pgfpathquadraticcurveto{\pgfqpoint{0.964701in}{0.289444in}}{\pgfqpoint{0.992478in}{0.289444in}}%
\pgfpathlineto{\pgfqpoint{0.992478in}{0.289444in}}%
\pgfpathclose%
\pgfusepath{stroke,fill}%
\end{pgfscope}%
\begin{pgfscope}%
\pgfsetrectcap%
\pgfsetroundjoin%
\pgfsetlinewidth{1.505625pt}%
\definecolor{currentstroke}{rgb}{0.121569,0.466667,0.705882}%
\pgfsetstrokecolor{currentstroke}%
\pgfsetdash{}{0pt}%
\pgfpathmoveto{\pgfqpoint{1.020256in}{0.614290in}}%
\pgfpathlineto{\pgfqpoint{1.159145in}{0.614290in}}%
\pgfpathlineto{\pgfqpoint{1.298034in}{0.614290in}}%
\pgfusepath{stroke}%
\end{pgfscope}%
\begin{pgfscope}%
\definecolor{textcolor}{rgb}{0.000000,0.000000,0.000000}%
\pgfsetstrokecolor{textcolor}%
\pgfsetfillcolor{textcolor}%
\pgftext[x=1.409145in,y=0.565679in,left,base]{\color{textcolor}{\rmfamily\fontsize{10.000000}{12.000000}\selectfont\catcode`\^=\active\def^{\ifmmode\sp\else\^{}\fi}\catcode`\%=\active\def%{\%}Playback}}%
\end{pgfscope}%
\begin{pgfscope}%
\pgfsetrectcap%
\pgfsetroundjoin%
\pgfsetlinewidth{1.505625pt}%
\definecolor{currentstroke}{rgb}{1.000000,0.498039,0.054902}%
\pgfsetstrokecolor{currentstroke}%
\pgfsetdash{}{0pt}%
\pgfpathmoveto{\pgfqpoint{1.020256in}{0.420617in}}%
\pgfpathlineto{\pgfqpoint{1.159145in}{0.420617in}}%
\pgfpathlineto{\pgfqpoint{1.298034in}{0.420617in}}%
\pgfusepath{stroke}%
\end{pgfscope}%
\begin{pgfscope}%
\definecolor{textcolor}{rgb}{0.000000,0.000000,0.000000}%
\pgfsetstrokecolor{textcolor}%
\pgfsetfillcolor{textcolor}%
\pgftext[x=1.409145in,y=0.372006in,left,base]{\color{textcolor}{\rmfamily\fontsize{10.000000}{12.000000}\selectfont\catcode`\^=\active\def^{\ifmmode\sp\else\^{}\fi}\catcode`\%=\active\def%{\%}Capture}}%
\end{pgfscope}%
\end{pgfpicture}%
\makeatother%
\endgroup%
}
    \caption{The plot of FSK modulation.}
    \label{fig:fsk}
\end{figure}

The Figure \ref{fig:fsk} shows the plot of FSK modulation. The bit sequence is 1, 0, 0, 1, 1. The frequency of 1 is 1000 Hz, and the frequency of 0 is 2000 Hz.

In the evaluation, we set the frequency of 1 to be 4000 Hz, and the frequency of 0 to be 8000 Hz. \textbf{FSK can directly pass the distance challenge without any complicated trick.}

\subsubsection{PSK}

Due to the clock drift issue, we didn't use PSK in our project 1, although we have implemented the PSK modulation and demodulation with OFDM.

\subsection{Frame Structure}

\paragraph{Frame Structure in Project 1 and Project 2}
\begin{verbatim}
fielid | chirp | data |
bits   |       | 40   |
\end{verbatim}
Since in Project 1 and Project 2, we are transmitting a fixed length of data, we don't need to include the length of the data in the frame structure.
We use FEC to correct the error in the data. We do not have retranmission in Project 1 and Project 2 so there is no CRC.

\paragraph{Frame Structure in Project 3 and Project 4}
\begin{verbatim}
field | chirp | len | src | dest | type | seq | data | crc |
bits  |       | 14  | 2   | 2    | 2    | 4   |      | 16  |
\end{verbatim}

As we divide the network into layer, the frame structure is also divided into several fields. \texttt{chirp} and \texttt{len} is encoded and decoded by the physical layer. \texttt{src}, \texttt{dest}, \texttt{type}, \texttt{seq}, \texttt{data} and \texttt{crc} is encoded and decoded by the link layer.
Therefore, \texttt{crc = crc16(src + dest + type + seq + data)}.
To avoid the corruption in \texttt{len}, we limit the length of the frame.

\subsection{Forward Error Correction}

Forward Error Correction (FEC) is a technique used to detect and correct errors in transmitted data.
By adding redundant information to the transmitted data, the receiver can correct errors in the received data without the need for retransmission.

First, we need to investigate the error model of the physical layer.
The error model of the physical layer is burst error.
For example, if we fail to locate the chirp signal, the received signal will be a burst error, probably the entire frame is corrupted.
Therefore, our FEC should be able to correct burst errors.

We have tried two FEC schemes, Hamming code and Reed-Solomon code.

Hamming code is a simple FEC scheme that can correct single-bit errors and detect double-bit errors. But it is too weak to correct burst errors. Moreover, the overhead of Hamming code is too large.

Reed-Solomon code is a powerful FEC scheme that can correct burst errors.
More specifically, Reed-Solomon code can correct \(t\) errors in a codeword of length \(n\), where \(t = \frac{n-k}{2}\), and \(k\) is the number of information bits in the codeword.
\textbf{We adopt RS(255, 223) in our project}, which can correct 16 errors in a codeword of length 255.
RS code has a much larger block size (255 bytes) than Hamming code (7 bytes), so we can handle burst errors more effectively.

Although RS code has a large block size, it is still not enough to correct the burst errors in the physical layer. We use another technique to correct the burst errors, which is \textbf{randomization}. The sender and receiver has a pre-defined random seed. The sender will reorder the bits in the frame according to the random seed, and the receiver will reorder the bits back according to the random seed. This technique can effectively correct burst errors.

\section{Link Layer}

Link layer is the second layer of the network stack.
In contrast to the physical layer, which is responsible for transmitting raw bits, the link layer is responsible for transmitting frames, which are sequences of bits with a specific structure.
The structure includes the source address, destination address, type, sequence number, data, and CRC.

Link layer in our project is responsible for reliablly transmitting bits to the receiver.
It means that our link layer has meachanisms to retry the transmission if the transmission fails. We use ACK to confirm the successful transmission, and retransmit the frame if the ACK is not received.

\subsection{CSMA Protocol}

Carrier Sense Multiple Access (CSMA) is a protocol used to avoid collisions in the network.

We implement a CSMA protocol using a state machine.
If the sender wants to transmit a frame, it first senses the medium.
If the medium is idle, the sender transmits the frame.
If the medium is busy, the sender waits for a certain time and senses the medium again.

But because of the latency issue, CSMA is not very effective in our project.

\section{Network Layer}

Network layer is the third layer of the network stack.
It is responsible for routing packets from the source to the destination.

In our project, instead of implement the network layer protocol, we leverage the OS's network layer protocol implementation by using TUN.

\subsection{TUN}

TUN is a virtual network device that allows the user to send and receive IP packets.

In detail, we create a TUN device, and assign an IP address to the TUN device.
The operating system will route the packets to the TUN device, and we can read the packets from the TUN device in our program, and send the sound waves to the physical layer. For the received sound waves, we can write them to the TUN device, and the operating system will route the packets to the destination or the application.

\section{Above Network Layer}

Above the network layer, we can implement any network protocol, such as TCP, UDP, etc.

Since we use TUN, we can use any network protocol that is supported by the operating system.

However, there are issues in the evaluation.
Windows has many background applications that use the network, which overwhelms our acoustic network.
We use a whitelist to filter the packets, and only forward the packets that are sent by our application.

\section{Implementation}

We use C++20 to implement the project.
Our program launch two threads, one for audio framework, and one for the main program.

For the audio thread, which is almost a real-time program, we use a \textbf{lock-free queue} to communicate between the audio framework thread and the main program thread to avoid the unexpected latency.

For the rest of the program, we use \textbf{coroutines} to handle the asynchronous tasks.
Coroutines are a powerful feature in C++20, which can be used to write asynchronous code in a synchronous way.
It is much easier to write and understand than the traditional callback-based asynchronous code.

However, coroutines have some limitations. It is not designed for real-time programming, so we use a lock-free queue to communicate between the audio thread and the main program thread.

Theoretically, coroutines may have a performance bottleneck,
because the scheduler of the coroutine is not designed for real-time programming. However, in our project, the performance bottleneck is the sample rate. The CPU in our laptop is powerful enough to handle 48000 samples per second.

\section{Feedback}

\subsection{Interesting and Challenging}

The project is very interesting and challenging. It is a great project.
The documentation and evaluation is very clear and detailed.

The project introduces many interesting concepts that I have never learned before, such as the physical layer protocol, link layer protocol.

The setup of the project is pretty easy.
Since sound applications are in user space. User space programs are easy to write and debug.

The project is very challenging.
The physical layer is very hard to implement, especially with someone who has not learned \textit{Signals and Systems} before. Fourier Transform, Convolution, etc. are crucial in the physical layer design.

\subsection{Physical Layer Takes Too Much Time}

One concern is that the physical layer takes too much time to implement.
Most computer science students have no knowledge in communication systems.

Moreover, the physical layer is not very useful for computer science students.
I believe that most of the computer science students will work on link layer and above, not the physical layer.

I suggest that the project should less focus on the physical layer.
It takes me sometimes to finetune our implementation to archive the performance goal.

Futhermore, I hope we can add several advanced topics.
One is high speed network, such as DPU, RDMA, etc.
We can have a programming project using DPU, RDMA or DPDK.

\section{Conclusion}

This project successfully demonstrated the implementation of an acoustic communication network using sound waves as the physical medium.
The physical layer was designed to handle modulation, demodulation, and synchronization, while the link layer ensured reliable frame transmission through mechanisms such as CSMA and acknowledgment-based retransmission. The network layer was integrated using TUN devices, allowing the use of standard network protocols.
It is a great project, and we have learned a lot from it.


\end{document}